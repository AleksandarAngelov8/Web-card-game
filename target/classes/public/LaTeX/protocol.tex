\documentclass{article}
\usepackage[utf8]{inputenc}
\usepackage{graphicx}
\usepackage{etex}
\usepackage[T1]{fontenc}
\usepackage{listings}
\DeclareUnicodeCharacter{202F}{,}
\DeclareUnicodeCharacter{02BC}{,}
\title{Parliamentary Protocol}
\begin{document}
    \maketitle
    \section{Protocol Information}
    \begin{itemize}
        \item ID: 14+19
        \item Sitzungsnummer: 14
        \item Titel: Plenarprotokoll der 14. Sitzung
        \item Ort: 
        \item Datum: 2018-02-22
        \item Wahlperiode: 19
        \item Start: 09:00
        \item Ende: 21:58
    \end{itemize}
    \section{Reden}





	1. Kirsten Kappert-Gonther (BÜNDNIS 90/DIE GRÜNEN) Vielen Dank, Herr Präsident. Darauf antworte ich gerne. – Unter den Bedingungen des Schwarzmarktes, unter den Bedingungen der Prohibition kann Cannabis eine Einstiegsdroge sein. Der Witz ist ja gerade, dass wir das Cannabis aus dem Schwarzmarkt herauslösen und in entsprechenden Fachgeschäften unter den Bedingungen des Gesundheits- und Jugendschutzes freigeben wollen.Dann haben die Nutzerinnen und Nutzer nicht mehr das Risiko, dass das Cannabis wie auf dem Schwarzmarkt verunreinigt ist, und zwar nicht nur durch Streckmittel, sondern auch durch synthetische Substanzen, die von den Dealern beigemischt werden, um die Konsumentinnen und Konsumenten an andere Substanzen zu binden. Das, was Sie gerade vorgetragen haben, spricht für die kontrollierte Freigabe.




	2. Achim Post (SPD) Herr Präsident! Liebe Kolleginnen und Kollegen! Vielleicht eine Vorbemerkung: Wenn uns zu der Tragödie und Katastrophe in Syrien nur einfällt, abwechselnd auf Moskau, auf Washington, auf Ankara, auf Damaskus oder Teheran zu verweisen, wäre meine erste Frage: Warum gucken wir nicht zu uns – nach Brüssel, nach Berlin, nach Paris, nach Rom –, um zu fragen: Was können wir eigentlich mehr und besser machen? Deshalb, liebe Kolleginnen und Kollegen, ist diese Debatte heute – vor dem informellen Rat morgen in Brüssel – so wichtig; denn sie zeigt doch eines: Es geht um was in Europa, und zwar in diesen Tagen, in diesen Wochen und in diesem Jahr.Ich bin dafür, dass wir endlich aus dem Krisenmodus herauskommen – raus aus der Defensive, rein in die Offensive – und darüber reden, was Europa zusammenhält und wie die Herausforderungen der Zukunft gemeinsam gemeistert werden können.Es ist für mich keine Übertreibung: Die nächsten Monate werden maßgeblich darüber mitentscheiden, wie die Zukunft Europas aussieht. Es macht einen Unterschied, ob es uns gelingt, Europa in schwierigen Zeiten zu stärken – oder eben nicht. Es macht einen Unterschied, ob es uns gelingt, die Euro-Zone zu reformieren und krisenfester zu machen – oder eben nicht. Und es macht einen Unterschied, ob es möglich ist, die europäische Außenpolitik weiter zu stärken und mit einer Stimme in der Welt zu sprechen – oder eben nicht. Und, liebe Kolleginnen und Kollegen, es macht einen Unterschied, ob es uns gelingt, den beschlossenen sozialen Pfeiler der Europäischen Union mit Leben zu füllen und für mehr Gerechtigkeit zu sorgen – oder eben nicht.Was also passiert morgen beim informellen Rat der 27? Worum geht es? Was sollte passieren, und was sollte besser nicht passieren? Eine Sache sollte besser nicht passieren: Wir sollten in der Frage der europäischen Spitzenkandidaten keine Rolle rückwärts machen.Auch bei der Europawahl 2019 muss gelten: Kommissionspräsident oder Kommissionspräsidentin kann nur werden, wer sich vorher als Spitzenkandidat einer europäischen Parteienfamilie zur Wahl gestellt hat.Das sieht das Europaparlament so, das sieht meine Fraktion so – und ich hoffe, die Mehrheit dieses Hauses auch.Eines aber brauchen wir dringend: die Debatte über den nächsten mehrjährigen Finanzrahmen für die Jahre 2021 bis 2027; denn dabei geht es um nicht mehr und nicht weniger als die Handlungsfähigkeit der Europäischen Union. Ich habe klare Erwartungen an eine Haushaltsdebatte, an eine – wie ich hoffe – ehrliche Haushaltsdebatte:Erstens. Der künftige EU-Haushalt muss ein Zukunftshaushalt sein mit deutlichem Fokus auf Investitionen, auf Forschung und Entwicklung, auf Arbeit gerade für junge Menschen und auf eine stärkere europäische Außen- und Sicherheitspolitik.Zweitens. Dann, liebe Kolleginnen und Kollegen, sind wir zu höheren Beiträgen Deutschlands zum EU-Haushalt bereit. Das ist ein wichtiges Signal gerade auch an die Adresse jener Staaten, die in dieser Frage bisher auf der Bremse stehen. Denn eines geht nicht: Wir können der EU nicht immer neue Aufgaben aufbürden und dann abwarten, ob Sterntaler vom Himmel fallen. Das wird nicht passieren, schon gar nicht bei einem Haushalt für 500 Millionen Menschen, der gerade einmal doppelt so groß ist wie der Haushalt meines Heimatlandes Nordrhein-Westfalen, liebe Kolleginnen und Kollegen.Nein, wir sollten den nächsten mehrjährigen Finanzrahmen der EU dazu nutzen, um den Zusammenhalt und die Investitionskräfte der Euro-Zone zu stärken.Was ist dabei unser Interesse, was ist das Interesse der Bundesrepublik Deutschland? Kein Land profitiert mehr von der EU als unser Land, auch und gerade ökonomisch. Wir exportieren eben gerade nicht die meisten Güter nach China oder in die USA, sondern nach Europa. 60 Prozent gehen nach Europa, fast 45 Prozent in die Europäische Union.Was sind unsere Interessen bei diesem mehrjährigen Finanzrahmen? Wir wollen nicht, dass es bei der Strukturförderung durch die EU zu einem abrupten Bruch kommt, gerade auch mit Blick auf die Förderung strukturschwächerer Regionen in Deutschland. Wir wollen, dass künftige Haushalte nach dem Prinzip der wechselseitigen Solidarität aufgestellt werden; denn – die Bundeskanzlerin hat es gesagt – Solidarität darf keine Einbahnstraße sein.Zusammengefasst: Gelder, die in eine starke europäische Außen- und Sicherheitspolitik, in den europäischen Grenzschutz, in transeuropäische Netze, in eine moderne europäische Infrastruktur, in die europäische Spitzenforschung oder in Wachstum und Beschäftigung in Europa fließen, sind Gelder, die letztlich in europäische öffentliche Güter und in einen echten europäischen Mehrwert investiert werden.Deshalb, liebe Kolleginnen und Kollegen, werbe ich dafür: Wir brauchen einen neuen Aufbruch für Europa – nicht irgendwann, sondern jetzt – als Resultat praktischen politischen Handelns. Deutschland muss dabei in den nächsten Monaten wieder eine, nein, die treibende Kraft in Europa werden.Zum Schluss. Ich bin fest davon überzeugt: Mit den europapolitischen Vereinbarungen zwischen Union und SPD kann und wird das gelingen. Auf eines können sich alle hier im Hause verlassen: Die SPD-Bundestagsfraktion wird ihren Beitrag dazu leisten, und zwar beharrlich, stetig und mit Leidenschaft.Vielen Dank für die Aufmerksamkeit.




	3. Christoph Hoffmann (FDP) Sehr geehrter Herr Präsident! Sehr geehrte Kolleginnen und Kollegen! Als Bürgermeister habe ich mich noch vor wenigen Monaten um Flüchtlinge in meiner Ortschaft gekümmert. Dabei waren auch viele, die von den Barbaren des IS gefangen, gefoltert, vergewaltigt worden waren. Mit diesen Flüchtlingen teile ich gerne mein Paradies. Dies nur als Hinweis an die „Alternative Südkurve“ hier im Parlament.Die kurdischen Verbände haben maßgeblich dazu beigetragen, den IS zurückzudrängen, übrigens mit hohem Blutzoll und deutschen Waffen. Genau diese kurdischen Verbände werden nun von türkischen Truppen bekämpft. Die Türkei trägt damit völkerrechtswidrig den Krieg in ein Gebiet, das vom syrischen Bürgerkrieg bisher weitgehend verschont war und in das sich viele Flüchtlinge gerettet hatten. Der türkische Präsident produziert mit dem Angriff neues Elend, neuen Krieg, neue Flüchtlinge und unterstützt damit auch die islamistischen Kräfte.Nun droht mit dem Eingreifen der syrischen Armee in Afrin eine unkalkulierbare Konfliktausdehnung. Es kann nicht sein, dass ein NATO-Partner einen Belagerungsring um Afrin zieht, um die verbleibende Bevölkerung zu bombardieren und auszuhungern. Erdogan begibt sich damit fast in die Nähe der Ebene des Schlächters Assad, der selbiges gerade in Ost-Ghuta in grausiger Weise vormacht. Und die Bundesregierung? Sie schweigt weitgehend zu den Vorgängen rund um den türkischen Einmarsch. Statt einer klaren Haltung hat sie gestern mit einem dünnen allgemeinen Appell reagiert. Im Badischen nennt man das Trochehüüler.Das Außenministerium schwurbelt gar dieser Tage von einer fluiden Lage in Nordsyrien, um keine Haltung einzunehmen. So geht man nicht mit Despoten um. Gegenüber Despoten braucht man eine klare Haltung, Herr Gabriel.Ganz anders ist eine junge, wohltuend mutige Generation von Spitzenpolitikern in europäischen Nachbarländern, eine neue Generation, die man sich hierzulande auch wünscht. So hat der französische Präsident Macron die Türkei frühzeitig und direkt aufgefordert, alle Kriegshandlungen einzustellen und Hilfsgüterlieferungen nach Afrin zuzulassen. Es ist schwer erträglich, dass sich die Bundesregierung nicht im Sinne einer europäischen Achse diesen Worten angeschlossen hat.Das wäre ja genau das gewesen, was Frau Merkel heute Morgen in der Regierungserklärung gefordert hat: eine kohärente europäische Außenpolitik.Deutschland hat aus Mitteln der wirtschaftlichen Zusammenarbeit bereits 2 Milliarden Euro für Flüchtlinge rund um diese Region in Syrien gegeben. Da nun wieder neue Fluchtbewegungen erzeugt werden, ist es wichtig und nötig, hier weiterhin schnell und unkompliziert zu helfen. Die Bundesregierung darf sich nicht durch die Freude über die Freiheit von Deniz Yücel den Blick auf den fragwürdigen autokratischen Kurs der Türkei vernebeln lassen.




	4. Britta Haßelmann (BÜNDNIS 90/DIE GRÜNEN) Sehr geehrter Herr Präsident! Liebe Kolleginnen und Kollegen hier im Haus! Liebe Kollegin von der FDP, liebe Frau Kloke, nach ein paar Wochen haben wir gemerkt, dass die AfD das immer so nimmt, wie es ihr gerade passt. Aber diese Methode wird nicht aufgehen, meine Damen und Herren; da bin ich zuversichtlich. Bei aller Unterschiedlichkeit in der Sache werden die demokratischen Fraktionen hier im Hause an dieser Stelle zusammenstehen.Deshalb lohnt es nicht, auch nur ein weiteres Wort zu verschwenden, um auf die völkische Tirade des Kollegen Seitz einzugehen. Lassen Sie uns lieber über die Sache reden.Herr Schnieder, ich möchte Sie und Ihre Fraktion jetzt endlich überzeugen, dass es sinnvoll wäre, nach so vielen Jahren der Blockade in Sachen Transparenz und Lobbyismus endlich aufzuhören, gebetsmühlenartig die alten, platten Argumente für die Ablehnung eines Lobbyregisters vorzutragen. Würden Sie sich die Mühe machen, unseren Antrag einmal genau zu lesen, dann wüssten Sie, dass es in diesem Antrag weder eine Forderung nach Offenlegung aller Kontakte der Regierungsmitglieder gibt noch eine Forderung nach Offenlegung aller Kontakte der Mitglieder des Deutschen Bundestages; denn so schlau wie Sie sind wir schon lange.Wir wissen, dass das Abgeordnetenmandat, dass das freie Mandat einen bestimmten Rahmen vorgibt, der natürlich einzuhalten ist. Das machen wir in unserem Antrag deutlich.Eines können wir nicht negieren, meine Damen und Herren, und das finde ich interessant: In der vorangegangenen Debatte über ein Cannabiskontrollgesetz haben die Redner von der CDU/CSU in aller Breite von den Gefahren des Drogenkonsums berichtet, von den Gefahren der illegalen Drogen und der legalen Drogen. Ich frage Sie: Wo ist eigentlich das Tabakwerbeverbot geblieben?Jetzt fragen sich vielleicht manche: Warum spricht sie das im Zusammenhang mit Lobbyismus an? – Weil es in diesen Zusammenhang gehört, meine Damen und Herren.Wir haben nämlich jahrelang im Deutschen Bundestag darüber geredet, dass es endlich zum Schutz der Gesundheit, zur Gefahrenabwehr dieser gefährlichen Droge Tabak ein Tabakwerbeverbot geben muss. Und was ist während der Koalitionsverhandlungen zwischen Union und SPD passiert? Das Thema ist aus den Koalitionsverhandlungen entschwunden. Niemand vonseiten der SPD und der CDU/CSU ist mehr dafür, gegenüber den großen Lobbykonzernen der Tabakindustrie ein Tabakwerbeverbot durchzusetzen. Ist das nicht Beweis genug dafür, dass wir einmal über Einflussnahme und Lobbyismus reden sollten?Wir könnten weiter darüber reden im Hinblick auf die Frage Abgasskandal in der Automobilindustrie. Wir könnten über die Frage von Einflussnahmen von Monsanto und Bayer reden. Wir könnten auch über ganz normale Interessenvertretung von Verbänden, NGOs sowie Unternehmen reden, die wir zu Anhörungen einladen, die wir zu Fachgesprächen einladen, um mit den Leuten in der Sache über Gesetzentwürfe zu reden. Wir diffamieren nicht alle Interessenvertreterinnen und Interessenvertreter, sondern wir wollen, dass es mehr Transparenz gibt.Deshalb meine Botschaft an die CDU/CSU: Hören Sie endlich damit auf, einen Popanz aufzubauen im Hinblick auf das freie Mandat! Geben Sie Ihre Blockade auf, und lassen Sie uns endlich wie in den USA, in Kanada und in acht europäischen Ländern ein verbindliches Lobbyregister auflegen! Das ist heute State of the Art und nicht die alte Verbändeliste aus dem Jahr 1972.Da nützt es auch nichts, dass sie hier die ganzen Schwierigkeiten im Hinblick auf die Abgeordneten anführen.Frau Kollegin, es gibt den Wunsch zu einer Zwischenfrage.Meine Redezeit ist ja zu Ende. – An die SPD möchte ich noch Folgendes sagen: Liebe Kolleginnen und Kollegen, strapazieren Sie uns bitte nicht wieder vier Jahre mit der Aussage: Ich hätte gerne ein Lobbyregister, es kann aber leider nichts werden, weil der Koalitionspartner nicht mitmacht. – Verdammt noch mal, wenn wir in der Lage waren, mit der Union und der FDP ein Lobbyregister zu vereinbaren, dann frage ich mich: Warum haben Sie das dann nicht auf die Kette gekriegt?




	5. Frank Magnitz (AfD) Frau Präsidentin! Meine Damen und Herren! Bevor ich mich dem eigentlichen Thema zuwende, drängt es mich, zu den Redebeiträgen, die hier von links und von Grün gekommen sind, doch noch ein paar Extraworte zu sagen.Hier wird kollektiv und intensiv auf die deutsche Autoindustrie eingetreten und eingeschlagen. Ich habe das Gefühl, dass niemandem von Ihrer Seite so richtig klar ist, wie gerade diese Branche zum BIP Deutschlands beiträgt, von dem Sie intensiv profitieren. Seien Sie also bitte ein bisschen umgänglicher und gehen Sie vorsichtiger mit der deutschen Autoindustrie um!Aber jetzt zum eigentlichen Thema. Kostenloser ÖPNV – welch wunderbare Vision! Zu dem, was sich hinter dem Begriff der Kostenlosigkeit verbirgt, hat der Kollege Wiehle schon einiges gesagt, und er hat es, glaube ich, ganz gut erläutert. Aber es sei noch einmal darauf hingewiesen, dass wiederum diejenigen zur Ader gelassen werden sollen, die ohnehin schon die Hauptlast unseres Umverteilungsstaates schultern. Auch das richte ich intensiv und besonders an die linke Seite des Plenums. Nun sollen sie auch noch teilenteignet werden – über die eingeschränkte Nutzbarkeit der für teures Geld gekauften Fahrzeuge. Deshalb lohnt es sich, den Blick einmal auf das zu richten, was der ganzen unsinnigen Debatte zugrunde liegt.Da ist zuerst einmal der obskure Abmahnverein mit der wohlklingenden Bezeichnung „Deutsche Umwelthilfe“ zu nennen.Der Wohlklang bekommt sofort einen schalen Nachgeschmack, wenn man erfährt, dass zum Beispiel Toyota einer der großen Sponsoren dieses Vereins ist.Richten wir unser Augenmerk weiter auf die USA – bekanntermaßen Vorreiter im Energiesparen und bei der Emissionsvermeidung, unschwer erkennbar an schwebend leichten SUVs, zierlichen Pick-ups und anderen Energiesparmodellen.In diesen US-Amerikanern erwacht das grüne Gewissen passenderweise just in dem Moment, als VW sich anschickt, der größte Autobauer der Welt zu werden. Dabei genehmigen sich die Amerikaner einen Grenzwert, der um das 2,5-Fache über dem liegt, was man sich in Brüssel ausgedacht hat, geschweige denn, dass man über Fahrverbote – wohlgemerkt in den USA – überhaupt nur nachdenken würde.Es werden in den USA von Anwaltsfirmen milliardenschwere Vergleichsverhandlungen zulasten der deutschen Fahrzeugindustrie geführt.Ähnliche Fälle mit gleichgelagerten Verstößen der US-Industrie werden mit lächerlich geringen Sanktionen geahndet.Auch das ist eine Methode, sich Wettbewerbsvorteile gegenüber unserer Industrie zu verschaffen. Ein Schelm, wer Böses dabei denkt.Heute rächt sich die ungenügende Verhandlungsstrategie in Brüssel, bei der man leichtfertig die Interessen Deutschlands vernachlässigt hat. Das Kind ist nunmehr in den Brunnen gefallen, und mit wildem Aktionismus soll es geborgen werden. Glauben Sie im Ernst, liebe Vertreter der Bundesregierung, dass Sie damit unabhängige Richter in Leipzig oder gar in der EU-Kommission im Geringsten beeindrucken könnten? Ganz makaber wird es, wenn Sie in Ihrem Koalitionsvertrag schreiben: Wir setzen EU-Recht eins zu eins um. – Das war dann wohl nichts.Ein Blick über den Tellerrand – sprich: die Schweizer Grenze – könnte auch bei unseren Debatten zu mehr Sachlichkeit führen. Das exzellente ÖPNV-Angebot in der Schweiz ist natürlich nicht kostenlos, aber eine sehr gute Ergänzung zur persönlichen Freiheit des Bürgers, seine Mobilitätsentscheidung eigenständig zu treffen. Dass man Bürger nicht zur Benutzung anderer Verkehrsmittel zwingen kann, musste angesichts der aktuellen Debatte auch der grüne Verkehrsminister Hermann heute Morgen im Deutschlandfunk einräumen.Damit sind wir bei dem eigentlich entscheidenden Aspekt angekommen: den Grenzwerten in ihrer ganzen Beliebigkeit und Losgelöstheit – losgelöst nämlich von der Realität, von wissenschaftlichen Erkenntnissen, von Verhältnismäßigkeit und Vernunft,entsprungen einzig und allein links-grüner Ideologie der Kategorie: Ich wünsch mir meine Welt, wie sie mir gefällt.Wie sonst wäre es eigentlich zu erklären, dass wir kollektiv die 70er-Jahre mit Millionen von Fahrzeugen, die hochgiftige Abgase produzierten, überlebt haben? Mir fallen da noch ein paar Sachen ein, wie zum Beispiel Nematoden in Fischen, BSE und anderes. Alles Dinge, die uns längst umgebracht haben müssten. Warum ist die 29,5‑fache Überschreitung des Stickoxidgrenzwertes für den Straßenverkehr am Arbeitsplatz völlig bedenkenlos hinzunehmen? Oder werden jetzt die Arbeitgeber aufgrund der Vernachlässigung der Fürsorgepflicht angezeigt und in Haftung genommen?Alle diese Grenzwerte – 40 Mikrogramm Stickoxid, aktuell 130 Gramm CO 2 pro Kilometer mit starker Tendenz zur weiteren Absenkung auf 95 Gramm in zwei Jahren, Grenzwerte für Feinstaubbelastungen – sind beliebig, man möchte fast sagen: beliebig gewürfelt.Gänzlich unberücksichtigt bleibt hierbei, dass sich der Stickoxidanteil in der Umgebungsluft innerhalb der vergangenen 20 Jahre etwa halbiert hat. Ich füge hinzu: Durch die Erneuerung der Fahrzeugflotten wird er weiter absinken.Kollege Magnitz, ich unterbreche Sie ungerne, aber das Minus vor der Minutenanzeige bedeutet tatsächlich, dass Sie jetzt zum Schluss kommen müssen.Ja, ich bin sofort fertig. – Die Debatten in diesem Hohen Hause sollten sich durch intensiven Realitätsbezug auszeichnen.Diskussionen über Glaubensfragen gehören auf Kirchentage.Amen.Sie wissen, dass wir eine Verabredung haben, bei der ersten Rede von Abgeordneten im Bundestag nur im absoluten Notfall zu unterbrechen. Aber wenn nicht erkennbar ist, dass der Abgeordnete das Zeichen hier vorne am Rednerpult, mit dem wir ja erst einmal lautlos kommunizieren, auch wahrnimmt, dann müssen wir an irgendeiner Stelle eingreifen.




	6. Agnes Malczak (BÜNDNIS 90/DIE GRÜNEN) Sehr geehrter Herr Präsident! Liebe Kolleginnen und Kollegen! Ich glaube, man sollte mit Demut auf die schwierige Lage im Nahen und Mittleren Osten schauen, mit Demut im Rückblick auf die eigenen Fehler, die der Westen in der Nahostpolitik gemacht hat, aber auch mit Demut, weil sich, glaube ich, niemand anmaßen sollte, jetzt die Lösung parat zu haben, wie wir mit all diesen schwierigen Konflikten und Krisen umgehen sollten.Weil die Lage so brandgefährlich und so ernst ist, muss man sich eine Frage sehr klar stellen: Tut eigentlich die deutsche Außenpolitik, tut die europäische Außenpolitik, tut die Bundesregierung alles, was sie tun kann, um einen Beitrag zu mehr Sicherheit, Frieden und Stabilität in dieser schwierigen Region zu leisten? Meine Damen und Herren, ich finde, die Antwort fällt sehr ernüchternd aus: Es wird immer wieder zu wenig oder teilweise gar nichts getan, und es gibt Bereiche, in denen man sogar das absolut Falsche tut. Es ist höchste Zeit, dies zu ändern. Ich will das an drei Beispielen aufzeigen: an der humanitären Hilfe, an der Frage, wie die Mitgliedstaaten die Vereinten Nationen bei diesen schwierigen Themen unterstützen, und an den Waffenexporten.Es ist sicherlich eine gute Nachricht, dass sowohl im Irak als auch in Syrien die Terrorschergen von Daesh keine großen Gebiete mehr kontrollieren und die Menschen dort nicht mehr terrorisieren. Aber Terrorregime lassen sich nun einmal nicht militärisch besiegen, auch wenn sie unterlegen sind; denn der Terror lässt sich nicht mit Waffen bekämpfen, sondern am Ende nur mit politischen Antworten. Umso wichtiger ist es, dass wir den Irak nicht aus dem Blick verlieren und dass wir in befreiten Gebieten nicht nur dafür sorgen, dass Wasser und Elektrizität wieder fließen, sondern auch dafür, dass es dort politische Lösungen und Aussöhnung gibt; denn sonst droht hier ein nächster Konflikt.Vor diesem Hintergrund macht es mich wirklich immer noch fassungslos, dass es die reichen Länder dieser Welt nicht schaffen, die Hilfe bereitzustellen, die die fragilen Staaten im Nahen und Mittleren Osten brauchen, um mit den Fluchtkatastrophen klarzukommen. In den Flüchtlingslagern rund um Syrien hungern die Menschen nach wie vor. Es wird nicht genug für Gesundheit und Bildung getan. Meine Damen und Herren, das ist doch der nächste Nährboden für mehr Radikalisierung und neue Konflikte. Diese herzlos-geizige, sicherheitspolitisch brandgefährliche Ignoranz muss jetzt endlich aufhören.Jahr für Jahr ermordet Assad mit Unterstützung von Russland und Iran seine eigene Bevölkerung mit Giftgas und Fassbomben. Es wird ausgehungert, getötet und vergewaltigt. Einerseits gegen die Flüchtlinge in Deutschland zu hetzen und andererseits hier zu stehen und kein kritisches Wort zu diesem Massenmörder zu sagen, so verlogen kann nur die AfD sein.In dieser schwierigen Lage in Syrien engagieren sich die Vereinten Nationen immer wieder, um die Gewalt zu beenden und einen innersyrischen Friedensprozess für die Zukunft anzustoßen, einen politischen Prozess, und Jahr für Jahr scheitern sie dabei kläglich, weil sie von ihren Mitgliedstaaten im Stich gelassen werden. Auch die Bundesregierung hätte hier mehr Einsatz zeigen können. Statt sich für eine starke Rolle der Vereinten Nationen einzusetzen, hat man sich bereitwillig an der Koalition der Willigen beteiligt. Herr Gabriel und auch Frau von der Leyen, es reicht eben nicht aus, auf der Münchner Sicherheitskonferenz schöne, wohlfeile Reden zu halten, wie man die Rolle der Vereinten Nationen stärken will, sondern es kommt darauf an, wenn es ernst wird, und da haben Sie nicht alles getan, was Sie tun konnten.Diese Koalition der Willigen mit den USA, mit der Türkei, mit Saudi-Arabien müsste man eigentlich eine Koalition der eigenen Widersprüche nennen. Die Eigeninteressen kommen immer als Erstes zum Zuge; hinzu kommt noch das rücksichtslose Vorgehen des Iran und Russlands. Fast alle Staaten, die in dieser Region militärisch aktiv sind, stellen die eigenen Interessen über eine politische Lösung. Das ist auch Teil der Erklärung, warum die Lage in Syrien von Jahr zu Jahr schlimmer wird. Die Leidtragenden sind die Menschen dort. Eine Politik „USA first“, „Türkei first“, „Saudi-Arabien first“, „Russland first“, „Iran first“ wird nie zu einem guten Ergebnis in Syrien führen. Es ist höchste Zeit für eine Politik „VN first“ und „Menschenrechte first“.Liebe Kolleginnen und Kollegen, immer wenn man meint, es könne in Syrien gar nicht schlimmer werden, wird man von der Realität brutal eines Besseren belehrt. Nun rollen zu allem Übel noch Erdogans Panzer völkerrechtswidrig in Syrien ein, um die Kurden in Nordsyrien zu bekämpfen. Weder von der NATO noch von der Bundesregierung hat man zu diesem Völkerrechtsbruch, zu dieser Gewalteskalation bis heute ein klares Wort gehört. Gleichgültig werden einfach weitere Rüstungsexporte genehmigt. Die Bundesregierung schaut seit Monaten gewollt und, ich würde mittlerweile sogar sagen: billigend weg, wenn es darum geht, die Gesetzeslücke zu schließen, die es Rheinmetall ermöglicht, in der Türkei eine Panzerfabrik aufzubauen.Schließen Sie diese Gesetzeslücke, und beenden Sie die Rüstungsexporte in die Türkei!Leider ist dies nicht das einzige Beispiel dafür, dass Sie in dieser schwierigen Region die Gewinninteressen deutscher Rüstungsunternehmen über Frieden, Sicherheit und Menschenrechte stellen. In fast jeder Sitzung des Bundessicherheitsrates der letzten Jahre wurde ein Waffenexport an einen Staat behandelt, der für den brutalen Krieg im Jemen mitverantwortlich ist. Der Kollege Nouripour hat kürzlich nachgefragt. Im letzten Jahr gingen Waffenexporte im Umfang von 1,3 Milliarden Euro an Staaten, die diesen Krieg durchführen. Ich war erfreut, als ich gesehen habe, dass die SPD durchgesetzt hat, wofür wir als Grüne in den Jamaika-Verhandlungen gekämpft haben, nämlich einen Exportstopp gegenüber dieser Kriegsallianz. Ich war aber bitter enttäuscht, als ich in Ihrem Koalitionsvertrag gelesen habe, dass Sie sich drei Hintertüren eingebaut und beschlossen haben, dass die Patrouillenboote jetzt doch geliefert werden sollen. Ich muss sagen: Gerade wenn wir auf den Nahen und Mittleren Osten schauen, muss Schluss sein mit Waffenexporten in Krisengebiete, an Kriegsherren und an Menschenrechtsverletzer.Frau Kollegin, kommen Sie bitte zum Schluss.Ich bin beim letzten Satz. – Liebe Kolleginnen und Kollegen, gerade weil die Lage in der Region so schwierig ist, müssen wir für eine gemeinsame und engagierte europäische Antwort sorgen. Hier braucht es kein Wegducken und keine verheerenden Waffendeals, sondern eine klare Haltung und eine Stärkung der Vereinten Nationen. Hier braucht es keine nationalen Egoismen, sondern mehr Geld für humanitäre Hilfe und vor allem mehr Mut und neue Wege für die Diplomatie.Vielen Dank.




	7. Rita Hagl-Kehl (SPD) Herr Präsident! Liebe Kolleginnen und Kollegen! Vieles im Antrag der Grünen ist richtig und wichtig. Genau deshalb haben wir diese Punkte auch bereits in den Koalitionsverhandlungen festgeschrieben.Gerade wenn es um Glyphosat geht – das kommt mir manchmal vor wie unser Lieblingsthema in den letzten zwei Jahren –– das weiß ich nicht, dafür bin ich nicht verantwortlich –, haben wir festgeschrieben, dass wir eine systematische Minderungsstrategie entwickeln müssen und dass wir den Einsatz einschränken wollen, mit dem Ziel des schnellstmöglichen Ausstiegs; „schnellstmöglich“ heißt für uns: Noch in dieser Legislaturperiode muss der Ausstieg vonstattengehen.Wir sehen es leider nicht so, wie es der Kollege Ebner vorhin formuliert hat: Machen Sie sofort den Cut! – Ich kann nur darauf verweisen, dass auch die Grünen immer sehr gerne mit der Bahn fahren und sie als ökologisches Fortbewegungsmittel sehen. Aber dann fragen Sie doch einmal bei der Deutschen Bahn nach, was sie dort dazu sagen, wenn wir auf der Stelle den Einsatz von Glyphosat auf den ICE-Trassen verbieten. Mal sehen, wie es dann weiterläuft.Den Weg hin zum Ausstieg haben wir im Rahmen einer Ackerbaustrategie festgelegt, die wir bis zur Mitte der Legislaturperiode – bis dahin ist es nicht mehr lange hin – entwickeln werden. Im Rahmen dieser Ackerbau­strategie muss die Reduktion aller chemisch-synthetischen Pflanzenschutzmittel erfolgen. Gleichzeitig brauchen wir aber auch Innovationsprogramme für digital-mechanische Methoden; denn wir müssen die chemischen Mittel ersetzen. Dafür wollen wir Fördermittel bereitstellen. Ohne Geld bekommen wir natürlich keine alternativen Methoden.Des Weiteren müssen die Zulassungsverfahren für Wirkstoffe und Pflanzenschutzmittel – darauf wird im Antrag der Grünen ebenfalls verwiesen – überdacht und verändert werden. Wir wurden diesbezüglich bereits von der EU ermahnt. Es stimmt, wir brauchen mehr Transparenz im Zulassungsverfahren. Insbesondere die Behörden brauchen mehr Personal. Ich spreche hier explizit vom UBA.Wir brauchen viel mehr Forschung; der Kollege hat das bereits angesprochen. Dafür werden wir die notwendigen Mittel bereitstellen. Insbesondere wollen wir die Forschung im ökologischen Bereich fördern, damit entsprechende Mittel im Pflanzenschutzbereich vorhanden sind. Wir werden das BÖLN mithilfe von Geldmitteln sehr stark aufwerten. Das ist der richtige Weg; denn die zu entwickelnden ökologischen Mittel sind nicht nur für die ökologische, sondern auch für die konventionelle Landwirtschaft gedacht; auch hier können sie eingesetzt werden.Die Anstrengungen im Rahmen des Nationalen Aktionsplans, der bislang leider nur wenige Ergebnisse gezeitigt hat, müssen wir verstärken. Auch die Neonicotinoide spielen eine große Rolle. Die Grünen mahnen an, dass hier keine Ergebnisse gezeitigt wurden, weil die Menge durch das Verbot bestimmter Stoffe nicht verringert wurde. Wenn man aber mit den Imkern spricht, dann weiß man, dass sehr wohl Ergebnisse erzielt wurden und dass es den Bienen um einiges besser geht. Ich glaube, das zeigt auch die Praxis.– Auf der Grünen Woche waren diese Imker zuletzt vertreten. Vielleicht hätten Sie sie dort treffen können.Auf die Biodiversität wird mein Kollege Carsten Träger noch genauer eingehen.Lassen Sie mich kurz noch auf eine Aussage eingehen, die in den letzten Tagen in der Presse kursierte. Frau Klöckner, die als potenzielle Landwirtschaftsministerin gehandelt wird, hat gesagt, dass sie nun konventionelle Pflanzenschutzmittel für den Ökolandbau freigeben will. Ich muss ehrlich sagen: Damit würde man den Ökolandbau ad absurdum führen.Ich kann mir beim besten Willen nicht vorstellen, was sie sich bei dieser Aussage gedacht hat.– Nein, vermengen kann man das nicht.Der Verbraucher kauft Bioprodukte, weil er die Behandlung mit Pflanzenschutzmitteln nicht möchte.Vor diesem Hintergrund ist es absoluter Blödsinn, konventionelle Pflanzenschutzmittel für den Ökolandbau freizugeben. Stattdessen ist es notwendig, dass die konventionelle Landwirtschaft ökologischer wird. Wir machen die ökologische Landwirtschaft mit solchen Maßnahmen kaputt, weil sie dann für ihre Produkte keinen Absatz mehr findet.Wir wollen, dass alles nachhaltiger und ökologischer im Sinn unserer Natur und der Menschen wird.Herzlichen Dank für Ihre Aufmerksamkeit.




	8. Alexander Krauß (CDU) Herr Präsident! Meine sehr verehrten Damen und Herren! Wir haben viel darüber diskutiert, was wäre, wenn wir eine liberale Drogenpolitik hätten. Sie von der FDP haben ein Modellprojekt gefordert. Wir müssen es uns aber gar nicht so schwer machen, sondern können auf andere Länder schauen. Frau Kollegin Kappert-Gonther hatte den Bundesstaat Colorado angesprochen. Ich bin vor einigen Monaten in Colorado gewesen, einem Bundesstaat, der die Drogenpolitik vor vier Jahren liberalisiert hat, und mit zwei Erkenntnissen nach Hause gefahren: Der Drogenkonsum hat zugenommen, und Jugendliche kommen leichter an Drogen, weil es immer einen dummen Erwachsenen gibt, der ihnen das Zeug billig verkauft. – Das waren die Erkenntnisse.Ich war auch in San Francisco. Wenn wir an San Francisco denken, dann denken wir an die Straßenbahnen, die den Berg rauf- und runterfahren,an Palmen und an das Meer. Ich war von dieser Stadt geschockt. Bei einem Gang durch die Straßen hat es überall nach Cannabis und Urin gerochen. In dieser Stadt lagen Hunderte Obdachlose herum, die wie menschlicher Müll behandelt worden sind. Dieses Menschenbild und diese Auswirkungen der Drogenpolitik möchte ich in Deutschland nicht, meine sehr geehrten Damen und Herren.Jede Woche gibt es in den USA 1 000 Tote durch Opiate. Der Präsident hat den gesundheitlichen Notstand ausgerufen. Es gibt eine richtige Drogenepidemie. Das ist das Ergebnis der Verharmlosung von Drogen. Wer harte Drogen nimmt, hat seine Drogenkarriere meist mit Cannabis begonnen. Deswegen führt eine liberalisierte Drogenpolitik zu mehr Drogen, zu mehr Leid und zu mehr Drogentoten.Jetzt sind hier einige Argumente geäußert worden. Eines war, Verbote brächten nichts. Doch, Verbote bringen etwas.Sie verhindern Schlimmeres.Liebe Kolleginnen und Kollegen, wir bzw. die Bundesländer haben ein Nichtraucherschutzgesetz eingeführt. Die Zahl der Raucher, vor allem der jungen Raucher, ist deutlich gesunken. Gesetze zeigen also, was gut und was richtig ist, so wie auch Verkehrsschilder etwas anzeigen. Ein Stoppschild sagt „Halt!“, und ein junger Mensch, der in einem Gesetz liest: Cannabis ist schlecht, der weiß: Das ist ein Stoff, der seiner Gesundheit schadet.Es ist übrigens ganz klar: Gefährlich sind nicht die beigemischten Stoffe, sondern das Cannabis ist das Gefährliche. Das schädigt die Gesundheit.Man könnte jetzt sagen: Es halten sich aber nicht alle an die Gesetze, und dann kann man sie auch abschaffen. – Liebe Kolleginnen und Kollegen, mit der gleichen Logik könnte man fragen: Wieso darf ich eigentlich in der Stadt nur 50 km/h fahren? Es gibt doch so viele Menschen, die schneller als 50 km/h fahren. Es wäre doch eine völlig schizophrene Logik, deswegen das Gesetz abzuschaffen. Ein Staat muss klare Regeln aufstellen. Er muss Gesetze aufstellen, damit sich die Leute daran halten.Als Nächstes haben wir gehört: Der illegale Markt soll ausgetrocknet werden. Stellen wir uns einmal vor, liebe Kolleginnen und Kollegen, Cannabis würde in Fachgeschäften verkauft. Was machen die Drogendealer? Gehen sie zur Arbeitsagentur und beantragen eine Umschulung?Nein, was werden sie machen? Sie werden dann erstens Cannabis an die unter 18-Jährigen verkaufen; denn die bekommen es ja noch nicht im Geschäft. Als Zweites werden sie sich fragen, was es noch so an neuen Produkten gibt, die man auf den Markt bringen kann, und da ist Crystal das Nächste. Sie müssen also keine Angst haben: Die werden nicht arbeitslos.Lassen Sie mich zum Schluss kommen. Cannabis schadet der Gesundheit. Das ist ganz klar. Cannabis schadet jungen Menschen. Ihre Anträge schaden sowohl der Gesundheit als auch der Jugend. Deshalb werden wir sie guten Herzens ablehnen.Vielen Dank.




	9. Jens Koeppen (CDU) Vielen Dank. – Frau Präsidentin! Liebe Kolleginnen und Kollegen! Herr Kollege Köhler, ich konzentriere mich mit Ihrem Einverständnis einmal auf die beiden Anträge von Bündnis 90/Die Grünen, weil dazu vielleicht etwas mehr zu sagen ist.Liebe Kollegin Baerbock, wenn Sie Forderungen stellen und Wünsche äußern, dann müssen sie natürlich mit realistischen Lösungsansätzen verbunden sein.Ansonsten sind es politische Träumereien.Sie suggerieren doch, dass man nur ganz schnell aus der Braunkohle aussteigen und die Landschaft mit Windkraftanlagen zustellen muss, und schon ist die Energieversorgung gesichert, sind die Erfolge da und wird das Klima geschützt. Das ist aus meiner Sicht ein fataler Trugschluss.Wenn Sie sich zur Energiewende bekennen – das tun Sie ja – und den Erfolg der Energiewende haben wollen, dann müssen Sie sich auch zum Zieldreieck bekennen. Das haben Sie bisher ja immer gemacht, aber Sie haben sich dabei letztendlich irgendwo verlaufen.Zum Zieldreieck – das ist übrigens gleichwertig – gehören erstens die Versorgungssicherheit, zweitens die Wirtschaftlichkeit und drittens die Umweltverträglichkeit. Hier haben Sie starke Defizite.Erstens: die Versorgungssicherheit. Insbesondere bei der Versorgungssicherheit haben Sie starke Defizite; denn wenn Sie aus der Kohle so massiv ohne Alternativen aussteigen wollen, dann gibt es Versorgungsengpässe.Sie können nun einmal nicht gleichzeitig aus der Kernenergie und aus der Braunkohle aussteigen. Das funktioniert nicht.Zweitens: die Wirtschaftlichkeit. Hier haben wir jetzt mit den Ausschreibungen ein zartes Pflänzlein gesetzt. Wir wollen die erneuerbaren Energien mit marktorientierten Schritten in den Markt bringen. Wir haben da auch schon die ersten Erfolge zu verzeichnen; denn wir haben jetzt im Schnitt einen Preis von 3,82 Cent pro Kilowattstunde gegenüber 9,6 Cent pro Kilowattstunde im Jahre 2017. Nun muss man doch dieses Pflänzchen erst einmal wachsen lassen. Sie aber fordern einen massiven Ausbau der Windenergie mit zusätzlichen Gigawattleistungen, ohne zu wissen, wie der so erzeugte Strom abtransportiert werden kann.Drittens: die Umweltverträglichkeit. Zur Umweltverträglichkeit gehören natürlich der Klimaschutz, aber auch der Umweltschutz und der Naturschutz. Für mich gehört auch der Mensch dazu. Dazu kommen wir nachher noch. Der bloße Zubau sagt nichts über den Erfolg der Energiewende aus. Es wird kein einziges Gramm CO 2 eingespart, wenn wir die Leistungen nicht nutzen können, wenn wir die Energie nicht abtransportieren können.Mit diesem teilweise sehr teuren und nutzlosen Zubau der Windenergie bricht bei den Bürgern die Akzeptanz der Energiewende ein.Herr Koeppen, erlauben Sie eine Zwischenfrage oder -bemerkung der Kollegin Nestle?Gerne.Bitte schön.Können Sie sich vorstellen, dass man aus erneuerbarem Strom jede Menge sinnvolle Dinge machen kann, die CO 2 sparen, auch wenn man mit dem Netzausbau noch nicht fertig ist? Beispiele dafür sind der Wärmebereich oder der Verkehrsbereich.Es liegt ein Konzept der Landesregierung Schleswig-Holstein vor: Strom nutzen statt abschalten. Es ist also möglich, mit jeder zugebauten Kilowattstunde Windstrom tatsächlich Klimaschutz zu machen. Kennen Sie dieses Konzept? Und: Finden Sie es richtig?Ich finde das Konzept sehr gut und sehr richtig. Dafür gibt es auch in meinem Wahlkreis sehr gute Beispiele, etwa aus dem überschüssigen Strom Wasserstoff herzustellen, der für die Industrie sehr wertvoll ist und mit dem man zusätzliche Speicher füllen kann. Das ist sehr gut.Deswegen sage ich – und das ist auch unser Ansatz –: Wenn wir die Netze nicht haben, brauchen wir eine Änderung im EEG. Sie würde schlicht und ergreifend lauten – das ist eine Forderung an unseren möglichen zukünftigen Koalitionspartner –, dass nur die Energie vergütet werden soll, die genutzt wird.Nur der Strom, der mit dem Zähler gemessen wird oder der letztendlich in einen Speicher fließt, muss genutzt werden. Das ist eine Forderung, die wir schon lange stellen. Zu ihrer Umsetzung kommt es aber nicht. Deswegen bleiben Ihre Vorstellungen Träumereien. Deswegen werden wir mit der Energiewende keinen Erfolg haben, wenn wir blind zubauen, ohne zu wissen, was wir mit der Energie machen.Wir verfolgen noch einen weiteren Ansatz. Wir verfolgen den Ansatz, dass der Ausbau der erneuerbaren Energien und der Ausbau der Netze synchron verlaufen. Alles andere ginge in Richtung ungenutzte Energie.Herr Koeppen, es gibt den Wunsch, eine Frage zu stellen, von Frau Dr. Verlinden.Bitte schön.Herr Koeppen, vielen Dank, dass Sie die Frage zulassen. – Sie sagten: Der Windenergieausbau bringt Probleme mit sich. Man muss vorher verschiedene Voraussetzungen erfüllen usw. – Alles schön und gut. Darüber haben wir gestern im Ausschuss ausführlich gesprochen.Aber ich habe jetzt noch nicht verstanden, wie viel Gigawatt Windenergie onshore Sie nächstes Jahr ausbauen wollen. Es gibt ein Erneuerbare-Energien-Gesetz, in dem eine bestimmte Summe steht. Es gibt einen Koalitionsvertrag, in dem steht: Wir machen eine Sonderausschreibung von noch einmal 2 Gigawatt zusätzlich. Wie viel Gigawatt werden denn nun nächstes Jahr ausgeschrieben?Dazu haben wir im vergangenen Jahr ganz deutliche Festlegungen getroffen. Diese Zahlen sollten wir auch nicht erhöhen. Die durch die Ausschreibung festgeschriebene Leistung darf nicht erhöht werden. Noch einmal als Beispiel: Bei mir im Wahlkreis – das ist die Planungsgemeinschaft Uckermark-Barnim – stehen 800 Windkraftanlagen. Diese Anlagen stehen nicht nur da, sondern sie stehen teilweise auch still. Trotzdem wird eine bestimmte Leistung vergütet.Man kann doch niemandem in Deutschland erklären, dass zusätzliche Windkraftanlagen gebaut werden sollen, obwohl die Netze fehlen. Das Fehlen von Netzen wird übrigens von den gleichen Leuten beklagt, die den zusätzlichen Ausbau von Windkraftanlagen fordern. Deswegen ist es aus meiner Sicht nicht richtig, dass wir zusätzliche Windkraftanlagen bauen, ohne den Strom wirklich nutzen zu können.Gerade bei Bündnis 90/Die Grünen ist es noch ein bisschen einseitig mit dem Zieldreieck. Damit bin ich wieder bei Ihrer neuen Vorsitzenden, Frau Baerbock. Sie hat in der „Märkischen Allgemeinen Zeitung“ vom vergangenen Freitag auf die Frage, ob sie für die Bürger, die die Anlagen nicht vor der Haustür haben wollen, Verständnis habe, Folgendes geantwortet – ich zitiere –:Direkt vor dem Haus hat sie ja keiner, dafür gibt es Abstandsregeln.– Das ist in der Tat so: Vor dem Haus hat sie natürlich keiner.Aber ein Abstand von 800 Metern ist bei einer Windkraftanlage, die wie bei uns in der Planungsregion 200 Meter hoch ist, schon vor der Haustür.– Dazu kann ich Ihnen gerne Beispiele zeigen, und zwar nicht nur auf Fotos, sondern ich lade Sie gerne ein, sich das vor Ort anzusehen. – Das ist mit den Bürgern nicht zu machen.Ich zitiere weiter:Wenn sich eine Bürgerinitiative gegen den Bau von Windrädern im Wald engagiert, aber nichts dagegen hat, wenn der Wald für eine Autobahn abgeholzt wird, dann frage ich mich, um was es wirklich geht. Und ich frage: Wenn ihr den Windpark nicht wollt, wäre euch stattdessen ein Kohlekraftwerk lieber?Das ist an bösem Sarkasmus nicht zu überbieten.Das ist Realitätsverweigerung, und es ist Ignoranz gegenüber den Sorgen der Menschen, die keine weiteren Windkraftanlagen in ihrer Region wollen.Und es geht weiter – Zitat –:Natürlich gibt es viele lautstarke Bürgerinitiativen gegen Windräder. Aber das muss man ins Gesamtverhältnis setzen: In der Bevölkerung erreichen die Erneuerbaren Akzeptanzwerte von nahezu 90 Prozent.„Lautstarke Bürgerinitiativen“ – das erinnert mich an einen Brief, den ich von einem Windmüller bekommen habe, der in der Bundeswindlobby arbeitet. Er schrieb, ich solle mich doch nicht immer mit den Schreihälsen aufhalten, sondern lieber richtige Energiepolitik machen. – Wenn wir so mit den Bürgern umgehen, von „lautstarken Bürgerinitiativen“ reden und die Menschen als Schreihälse bezeichnen, dann geht die Energiewende völlig daneben.Herr Koeppen, wollen Sie eine Zwischenfrage oder -bemerkung zulassen?Ja, bitte.Doch, wir haben schon noch was vor. – Das ist dann die letzte Zwischenfrage oder -bemerkung. Kollege Krischer.Herzlichen Dank, Herr Koeppen, dass Sie die Zwischenfrage zulassen. – Ich habe Ihnen eben sehr aufmerksam zugehört.Das haben Sie nicht.Doch, ich habe sehr aufmerksam zugehört. – Ich habe jetzt schon seit vier Jahren diese Große Koalition im Blick. Die Union regiert übrigens schon seit zwölf Jahren. Alles das, was Sie hier an Problemen benennen, haben Sie politisch zu verantworten.Wir haben eben von der Bundesumweltministerin, die das auch bleiben möchte, gehört, dass es eine neue Entschiedenheit geben soll. Das kann man nur so interpretieren, dass beim Ausbau der Windenergie und vielen anderen Dingen, die Sie jetzt kritisieren, viel mehr passieren soll. Ich bitte Sie, mir zu erklären: Stehen Sie hinter der neuen Entschiedenheit von Frau Hendricks, oder erleben wir hier schon wieder, dass die Große Koalition, bevor sie überhaupt zustande gekommen ist, schon wieder alles kaputttritt, was es möglicherweise an positiven Ansätzen in diesem Bereich geben könnte?Sie haben mir offensichtlich nicht zugehört. Ich habe mich auf Ihre beiden Anträge fokussiert. Darin geht es um den sofortigen Braunkohleausstieg und um den Zubau von Windenergieanlagen. So wie Sie es in Ihrem Antrag vorsehen – mit zusätzlichen Ausschreibungen usw. – ist aus meiner Sicht ein Erfolg nicht gewährleistet. Wir werden keinen Erfolg haben, wenn wir ohne die wirkliche Nutzbarkeit der Energie, die wir dann produzieren, weitere Windkraftanlagen bauen. Darauf habe ich mich fokussiert.Dabei geht es weder um die Ziele, die wir in den Koalitionsvertrag aufgenommen haben,noch um die Ziele, die wir in Kyoto oder sonst wo vereinbart haben. Es geht vielmehr darum – Sie haben wirklich nicht zugehört –, dass ein reiner Zubau ohne Nutzbarkeit eben nicht zur Energiewende und zum Erfolg der Energiewende beiträgt.Die Zustimmung für die Windkraftanlagen mag ja in Berlin-Mitte oder in Potsdam bei 90 Prozent liegen. Aber ich lade Sie gerne ein, Frau Baerbock, in die Uckermark, in den Barnim, auf die Nauener Platte, nach Ostprignitz-Ruppin oder ins Potsdamer Umland. Dort wird die Ablehnung bald bei 90 Prozent liegen, wenn wir so weitermachen.Wir müssen die Energiewende wieder vom Kopf auf die Füße stellen. Der Anlagenzubau muss in erster Linie der bezahlbaren, sicheren und sauberen Energieversorgung dienen und nicht der Renditeversorgung der Windenergiebranche. Blinder, massiver Zubau, fehlende Regelungen und technogene Verriegelung – Windkraftanlagen im Wald sind übrigens der größte Unsinn – sind Gift für die Akzeptanz der Energiewende und schaden den erneuerbaren Energien.




	10. Gottfried Curio (AfD) Sehr geehrter Präsident! Geehrte Abgeordnete! Die AfD begrüßt die Freilassung von Herrn Yücel aus politischer Willkürhaft, begrüßt den Erfolg der außerordentlichen Bemühungen der Bundesregierung. Wir bedauern aber ebenso, dass eine Freilassung weiterer zu Unrecht Inhaftierter nicht mit gleicher Intensität betrieben wurde und aussteht.Die politische Bevorzugung von Herrn Yücel wurde nicht begründet, wirft aber die Frage auf, ob hier ohne Ansehung der Person gehandelt wurde. Zu vermeiden ist der mögliche Eindruck, dass mit seiner ganz außerordentlichen Vorzugsbehandlung eine stille Billigung seiner wohlbekannten deutschlandfeindlichen Äußerungen einhergeht. Geboten erscheint deshalb, dass die Bundesregierung eine Missbilligung dieser Äußerungen ausspricht.Selbstverständlich soll die Freiheit, sich so zu äußern, erhalten bleiben. Selbstverständlich soll dagegen jetzt nicht rechtlich vorgegangen werden. Natürlich soll die Bundesregierung sich nicht zu jeder Schmähung äußern. Aber wenn diese eine Person so herausgehoben wird, muss die ganze Geschichte erzählt werden. Das Hohelied, das auf Herrn Yücel angestimmt wird, kann den Missklang seiner Äußerungen nicht übertönen.Steigen wir also in den Morast des Tatsächlichen.Dem Autor, der den Niedergang Deutschlands beklagt,Thilo Sarrazin, wünscht Herr Yücel den Tod. Zitat: Möge der nächste Schlaganfall sein Werk gründlicher verrichten an dieser Menschenkarikatur. – Gemeint ist Sarrazin.Das ist unser geliebter Deniz. Überhaupt, diese uns aufgedrängte Vornamenvertraulichkeit – wir Deutschen und unser Deniz –:Wann sind wir nur zu dieser herzlichen Verbundenheit gekommen?Diese Ikone der Linkspresse also mit ihrer menschenverachtenden Sarrazin-Äußerung, so jemand ist für Frau Merkel ein Fall besonderer Dringlichkeit. Für Yücel ist der baldige Abgang der Deutschen Völkersterben von seiner schönsten Seite. Das ist für Herrn Gabriel ein deutscher Patriot!Yücel freut sich aufs Deutschensterben. Für ihn, betont Gabriel, habe er sich zweimal mit Präsident Erdogan getroffen. Yücel meint, etwas Besseres als Deutschland findet sich allemal. Gabriel, Chefkoch im Braten von Extrawürsten, betont, er habe mehrfach mit dem türkischen Außenminister Cavusoglu gesprochen. Yücel will, dass Deutschland zwischen Frankreich und Polen aufgeteilt oder in einen Rübenacker verwandelt wird.Das ist für Gabriel ein Brückenbauer. – Vielleicht zwischen Frankreich und Polen?Yücel freut sich, dass das unvollendete Werk von „Bomber-Harris“ und Ilja Ehrenburg – Massentötung und Massenvergewaltigung – endlich vollendet wird. Heiko Maas – ausgerechnet der Erfinder des Tatbestands Hass – twittert bei Freilassung: „Beste Nachricht, wo gibt.“Ähnlich besoffen vor Freude gibt sich ein ganzer Reigen von Politikern und Medienleuten bestimmter Couleur, während dieser Hassprediger gegen alle Deutschen hetzt, schwelgend in gruppenbezogener Menschenfeindlichkeit – wahrlich Rassismus pur!Und dann soll mit dem Verweis auf angebliche Satire, die alles dürfe – Übertreibung ist da aber nur eine Tarnkappe für wirklichen Hass –, eine Zwei-Klassen-Redefreiheit eingeführt werden, eine für establishmentnahe Antideutsche und eine für patriotische Normaldeutsche.Äußert der antideutsche Linke Hass, ist es Satire. Äußert der Normalbürger Kritik an der Regierung, ist es Hass. Maas bestimmt, was Satire ist, Gabriel, wer Patriot ist! Willkommen in der Welt der 16-Prozent-Partei!Übrigens gibt es für Herrn Yücel Hoffnung. Wenn er den Abgang der Deutschen beschleunigt sehen will, so kann er als Doppelstaater hierzu rein numerisch effektiv beitragen, indem er die ihm offenbar verhasste deutsche Staatsbürgerschaft abgibt.Das wäre doch ehrlich und überfällig.Und sonst? Die Presse lässt die Regierung gut aussehen. Frau Merkel hält ihre geheimen Pressetreffen ab und sagt, was dran ist. Man sorgt füreinander.Wir wissen nicht, wie viel Munition und Panzermaterial für diese Topgeisel gedealt werden wird.Wie viele Kurden werden sterben müssen? Jedenfalls können Politiker und Presse diese höchst wundersame Freilassung ihrer Ikone bejubeln. Die Bevölkerung aber muss nun vermuten, dass die GroKo-Politik Sonderklimmzüge für Gesinnungsfreunde macht und durch antideutsche Hetze nicht etwa abgeschreckt, sondern zu diplomatischen Höchstleistungen angespornt wird. Hier würde eine Missbilligung Klarheit schaffen.Und wenn man jetzt schaut, wie Herr Yücel Gabriels gedenkt, so ahnt man, dass er – als letztes Wort wollen wir es uns denken – mit Blick auf letzte Umfragen sagen würde – Deniz, der alte Satiriker –: Der baldige Abgang aber der SPD ist Parteiensterben von seiner schönsten Seite.Ich danke Ihnen.




	11. Angela Merkel (CDU) Herr Präsident! Liebe Kolleginnen und Kollegen! Meine Damen und Herren! Wir alle registrieren, was um uns herum in der Welt passiert. Wir alle sehen auch, wie bewährte Grundsätze in Zweifel gezogen und Partnerschaften auf die Probe gestellt werden. Wir alle sehen, wie sehr Europa durch seine geografische Lage exponiert ist; denn die Kriege und Konflikte in Syrien, Libyen oder der Ukraine finden nicht irgendwo auf der Welt statt, sondern nur wenige Flugstunden von Berlin entfernt. Wir alle sehen, dass die Verletzung völkerrechtlich anerkannter Grenzen in Europa kein Tabu mehr ist. Und wir alle sehen auch, wie sich der Schwerpunkt der Weltwirtschaft zunehmend verlagert, ganz besonders nach Asien, nach China. Europäische Unternehmen sind nicht mehr in allen Bereichen an der Weltspitze. Gerade im digitalen Bereich entwickeln sich andere Regionen deutlich schneller.Ich bin deshalb überzeugt: Erstens wartet die Welt nicht auf uns – weder auf uns in Deutschland noch auf uns in Europa –, und zweitens brauchen wir mehr denn je europäische Antworten auf die drängenden großen Fragen unserer Zeit.Das ist genau der Geist, in dem wir auch morgen beim informellen Treffen der EU-Staats- und Regierungschefs die Debatten über Europas Zukunft führen.Noch einmal ein kurzer Rückblick: Wir haben 2016 in Bratislava begonnen – einmal als Erkenntnis daraus, dass wir viele Entscheidungen zu langsam treffen, aber auch als Antwort auf die Entscheidung der Briten, die Europäische Union zu verlassen –, einen Prozess aufzusetzen, und wir haben Schwerpunkte für das gemeinsame Handeln als Staats- und Regierungschefs entwickelt in den Bereichen Migration, innerer und äußerer Sicherheit sowie nicht zuletzt wirtschaftlicher und sozialer Entwicklungen. In diesen Bereichen wollen wir die Menschen mit konkreten Antworten überzeugen und ihnen zeigen, dass Europa etwas für die Menschen in den einzelnen Ländern erreicht. Dies haben wir anlässlich der Feier zu 50 Jahren Römische Verträge im vergangenen Jahr noch einmal bekräftigt. Im Oktober letzten Jahres hat sich der Europäische Rat dazu ein ehrgeiziges Arbeitsprogramm gegeben, mit dem wir die einzelnen Arbeitsetappen bis zur Europawahl 2019 bestimmt haben. Darin ordnet sich auch der morgige informelle Rat ein.Deshalb ist es auch alles andere als ein Zufall, dass das erste Kapitel des neuen Koalitionsvertrages von CDU, CSU und SPD Europa gewidmet ist. Vielmehr ist das unsere nationale Antwort auf die europäische Agenda. Und ich übertreibe sicherlich nicht, wenn ich feststelle: Prominenter stand Europa bisher in keinem Koalitionsvertrag. Damit betonen wir das, was ich schon so oft mit dem Satz zum Ausdruck gebracht habe, dass es Deutschland auf Dauer nur gutgehen kann, wenn es auch Europa gutgeht.Damit geben wir ein klares Bekenntnis zum europäischen Arbeitsprogramm ab, denn wir brauchen einen neuen Aufbruch für Europa.Drei Bereiche sind dabei ganz besonders wichtig – auch in meinen Gesprächen heute und morgen in Brüssel –: erstens das uns alle so sehr beschäftigende Thema der Migration. Hier haben wir in unserem Koalitionsvertrag vereinbart, weiterhin konsequent an den Fluchtursachen anzusetzen.Niemand – flieht er nun vor Krieg, Verfolgung oder Perspektivlosigkeit – verlässt seine Heimat leichtfertig. Im Umkehrschluss heißt das, dass der Kampf gegen Fluchtursachen ein Kampf für Lebensperspektiven in der Heimat oder zumindest nahe der Heimat zu sein hat.Am Freitagvormittag – also morgen direkt vor Beginn des informellen Rates – findet ein Treffen der EU-Kommission, der Afrikanischen Union und der Vereinten Nationen mit den fünf Sahelstaaten, also Tschad, Mali, Niger, Burkina Faso und Mauretanien, statt, an dem auch Frankreich, Spanien, Italien und Deutschland teilnehmen werden. Hierbei werden wir uns mit der Kooperation beim Kampf gegen illegale Migration beschäftigen. Es wird um Entwicklungshilfe sowie um den Kampf gegen den Terrorismus in der Region gehen. Dank der EU-Kommission, die dort bei der Koordination eine große Rolle spielt, sind wir hier vorangekommen, und Deutschland hat sich in diese Kooperation in den letzten Monaten intensiv eingebracht. Insbesondere in Niger unterstützen wir die Regierung beim Kampf gegen skrupellose und menschenverachtende Schlepper und Schleuser und schaffen – das ist ja, wie wir immer wieder sehen, die Voraussetzung für einen nachhaltigen Erfolg – gemeinsam mit der nigrischen Regierung Perspektiven für Arbeit und Bildung für die Menschen in den betroffenen Regionen. Außerdem unterstützen wir die wirklich sehr segensreiche Arbeit der Internationalen Organisation für Migration, IOM, sehr intensiv.Im Koalitionsvertrag haben CDU, CSU und SPD darüber hinaus die Notwendigkeit eines wirkungsvollen Schutzes der europäischen Außengrenzen unterstrichen.Und nicht zuletzt halten wir an der Notwendigkeit eines gemeinsamen europäischen Asylsystems fest, das krisenfest und endlich auch solidarisch sein muss, gerade auch was die faire Verteilung von Flüchtlingen innerhalb der EU angeht.Dies, meine Damen und Herren, ist das bei weitem unbefriedigendste Kapitel der europäischen Flüchtlingspolitik; das steht völlig außer Frage. Aber ebenso völlig außer Frage steht, dass es mit Zähigkeit und mit Geduld gelingen wird, eine nachhaltig solidarische Lösung zu finden. Unser Ziel ist es, hier bis Juni dieses Jahres wesentliche Schritte erreicht zu haben.Zweitens geht es um die Wirtschaft. Die Stärkung der Wettbewerbsfähigkeit der Europäischen Union ist und bleibt zentrale Aufgabe. Gerade in wirtschaftlich guten Zeiten ist es deshalb von gar nicht zu unterschätzender Bedeutung, dass es den Menschen in der Europäischen Union nach den schweren Jahren der europäischen Staatsschuldenkrise inzwischen wirtschaftlich zunehmend besser geht. Alle EU-Mitgliedstaaten verzeichnen ein stabiles Wachstum. Die Aussichten für dieses und für das kommende Jahr sind positiv. Europaweit haben heute so viele Menschen Arbeit wie seit neun Jahren nicht mehr.Aber: Wir dürfen uns auf diesen Erfolgen keineswegs ausruhen – im Gegenteil! Es geht zum einen darum, schnelle Fortschritte bei der Gestaltung des digitalen Wandels zu machen. Gelingt es uns Europäern, die neuen technologischen Möglichkeiten so zu nutzen, dass wir international wettbewerbsfähig bleiben? Diese Frage ist offen, und davon hängt für die Zukunft der Wohlstand ab. Deshalb kommt der Schaffung des digitalen Binnenmarktes eine entscheidende – ich würde sagen: die entscheidende – Bedeutung zu. Die estnische Präsidentschaft hat hierzu auf dem Sondergipfel in Tallinn im vergangenen Herbst einen bedeutenden Beitrag geleistet und uns im Übrigen vor Augen geführt, wie weit man auch national sein könnte.Es geht hier um den Ausbau der Infrastruktur, um Forschung im Chipbereich, um Forschung vor allem im Bereich der künstlichen Intelligenz, um Fragen der Besteuerung von internationalen Internetkonzernen, um die Umsetzung der Datenschutz-Grundverordnung, die im Mai dieses Jahres in Kraft tritt und die unsere Wirtschaft vor große Herausforderungen stellen wird, und es geht natürlich auch darum, die Menschen mitzunehmen, ihre Sorgen und Ängste ernst zu nehmen und durch Weiterbildung und lebenslanges Lernen auch die Voraussetzungen dafür zu schaffen, dass alle von dem digitalen Fortschritt profitieren können.Zum anderen sind wir bei der Ausgestaltung der Wirtschafts- und Währungsunion keinesfalls am Ende angelangt. Deshalb müssen wir die Verbesserung der Wettbewerbsfähigkeit im Euro-Raum wirksamer verankern. Dazu gehört natürlich die Eigenverantwortung jedes Mitgliedstaates, durch eigene ehrgeizige Reformen Wachstum und Stabilität zu stärken. Dabei bleibt auch der Stabilitäts- und Wachstumspakt in Zukunft der Kompass unseres Handelns. In Europa muss für uns dabei weiterhin gelten, dass Haftung und Kontrolle stets zusammengehen. Dieses Thema werden wir auf dem Rat im März dieses Jahres vertieft erörtern. Parallel arbeitet die Euro-Gruppe intensiv an der Fortentwicklung der Bankenunion. Natürlich werden wir uns im März dieses Jahres dann auch mit Fragen der Ausgestaltung der Wirtschafts- und Währungsunion insgesamt beschäftigen.Drittens, meine Damen und Herren, ist uns die gemeinsame Außen- und Sicherheitspolitik wichtig. Wir wollen, dass Europa nach außen geschlossen auftritt. Ein wesentlicher Beitrag hierzu ist die weitere Stärkung der gemeinsamen Zusammenarbeit in der Sicherheits- und Verteidigungspolitik. Wir haben nach bemerkenswert kurzer Vorbereitungszeit – Deutschland und Frankreich haben hier eine entscheidende Rolle gespielt – auf dem Rat im Dezember letzten Jahres den Startschuss für die Ständige Strukturierte Zusammenarbeit im Verteidigungsbereich gegeben – ein Projekt, das jahrzehntelang in Europa nicht zustande gekommen ist. In den nächsten Tagen geht es darum, die ersten konkreten Projekte wirklich auf die Reihe zu bringen.Meine Damen und Herren, lassen Sie mich kurze Bemerkungen zu der nationalen Diskussion machen.Erstens. Unsere Bundeswehr leistet herausragende Arbeit in den internationalen Einsätzen. Wir sind in der NATO der zweitgrößte Truppensteller, inklusive der truppenstellungsgleichen Einsätze. Das heißt, wir spielen eine entscheidende Rolle. Lassen Sie mich an dieser Stelle unseren Soldatinnen und Soldaten hier zu Hause und im Ausland herzlich danken.Zweitens. Wir haben Mängel in der Bundeswehr. Diese Mängel beruhen auf Entwicklungen, die in vielen Jahren zuvor stattgefunden haben. Es ist richtig und wichtig, dass der Wehrbeauftragte diese Mängel benennt. Es ist aber auch richtig und wichtig, dass wir darüber nicht vergessen, was geleistet wurde. Wir müssen aufpassen, dass wir international nicht in eine etwas zwiespältige Rolle kommen: auf der einen Seite zu beklagen, was bei uns alles nicht passt und klappt, und auf der anderen Seite immer wieder – ich würde sagen, das passiert in keinem anderen Mitgliedsstaat der NATO – den Zielkorridor für Ausgaben infrage zu stellen, dem wir zugestimmt haben und zu dem wir uns selbst verpflichtet haben. Das passt nicht zusammen, und damit wird man kein verlässlicher Verbündeter, meine Damen und Herren.Neben der verstärkten Sicherheitskooperation ist ein einheitliches europäisches Auftreten in der Außenpolitik Voraussetzung dafür, dass wir unser europäisches Gewicht in die Lösung von Konflikten und in die internationale Kooperation einbringen können. Bei der gemeinsamen Außenpolitik gibt es in der Europäischen Union noch viel zu tun. Das gilt für unsere Beziehungen in Richtung Russland, das gilt für unsere Beziehungen in Richtung China, und das gilt auch für unsere Rolle bei der Lösung regionaler Konflikte.Meine Damen und Herren, liebe Kolleginnen und Kollegen, Ihnen geht es wahrscheinlich so wie mir: Das, was wir im Augenblick sehen, die schrecklichen Ereignisse in Syrien, den Kampf eines Regimes nicht gegen Terroristen, sondern gegen seine eigene Bevölkerung, die Tötung von Kindern, das Zerstören von Krankenhäusern,all das ist ein Massaker, das es zu verurteilen gilt und dem wir ein klares Nein entgegensetzen. Wir sind aber auch aufgefordert, zu versuchen, eine größere Rolle dabei zu spielen, um ein solches Massaker beenden zu können. Darum müssen wir uns als Europäer bemühen, meine Damen und Herren.Diese Aufforderung gilt auch für die Verbündeten des Assad-Regimes, ganz besonders für den Iran und Russland. Hier gibt es eine Verantwortung. Unser Bundesaußenminister hat heute früh noch mit Herrn Maurer vom Internationalen Komitee vom Roten Kreuz telefoniert, und er wird mit dem russischen Außenminister sprechen. Wir müssen alles, was in unserer Kraft steht, tun, damit dieses Massaker ein Ende findet.Meine Damen und Herren, CDU, CSU und SPD geben mit ihrer Koalitionsvereinbarung also wesentliche Impulse für eine deutsche Europapolitik, die erfolgreich sein kann. In ihr drückt sich aus, dass Deutschland bereit ist, seine Verantwortung zu übernehmen. Das leitet mich auch auf den Beratungen des morgigen informellen Europäischen Rates im Kreis der 27 Staats- und Regierungschefs – Großbritannien wird morgen nicht dabei sein –, auf dem wir uns mit der Zukunft unserer europäischen Institutionen und dem künftigen Finanzrahmen der Europäischen Union befassen werden.Dabei wollen wir nicht vergessen, dass 2019 ein Jahr mit vielen europapolitischen Umbrüchen sein wird. Deshalb müssen 2018 die Weichen richtig gestellt werden. Ende März nächsten Jahres wird mit Großbritannien ein großes und wichtiges Land die Europäische Union verlassen,zwei Monate später wählen die Europäerinnen und Europäer ein neues gemeinsames Parlament, und im Herbst wird eine neue Europäische Kommission ernannt. Im November 2019 läuft die zweite und letzte Amtszeit von Donald Tusk aus, des Präsidenten des Europäischen Rates. Auf all diese Entwicklungen müssen wir uns jetzt vorbereiten.Das betrifft zum einen die Vorbereitung der Institutionen – allen voran des Europäischen Parlaments – auf die nächste Europawahl. In diesem Zusammenhang werden wir auch über das Thema Spitzenkandidaten sprechen, was ja ein Thema der Parteien ist. Die Parteienfamilie, der ich angehöre, die Europäische Volkspartei, hat diese Institution des Spitzenkandidaten inzwischen in ihr Statut übernommen. Durch den Austritt Großbritanniens werden ja 73 Sitze im Europäischen Parlament frei, und das Europäische Parlament schlägt vor, 27 von diesen 73 Sitzen neu zu verteilen und die übrigen Sitze zunächst einzusparen, auch als Reserve für die Zukunft. Ich erwarte für diesen Vorschlag breite Zustimmung im Kreis der Staats- und Regierungschefs.Wir dürfen und können zum anderen die Debatte über die Zukunft Europas natürlich nicht von der Debatte über den neuen europäischen Finanzrahmen ab dem Jahr 2021 trennen. Deshalb werden wir genau darüber am Freitag auch zum ersten Mal diskutieren. Dieser Rahmen ist ein Bestandteil der Antwort auf die Frage, was für ein Europa wir in Zukunft haben wollen. Der neue Haushalt soll Europa unterstützen, die anstehenden Herausforderungen dann wirklich meistern zu können. Die vereinbarten Prioritäten müssen sich deshalb in einem modernen Haushalt angemessen widerspiegeln.Dabei betrachte ich den Einschnitt, den der Austritt Großbritanniens ohne jeden Zweifel für die Europäische Union bedeutet, auch als Chance, die EU-Finanzen insgesamt auf den Prüfstand zu stellen. Wir müssen den Blick für das Wesentliche schärfen und unsere Entscheidungen danach ausrichten, in welchen Bereichen der europäische Mehrwert für uns Mitgliedstaaten, aber auch für Europa als Ganzes am größten ist. Ich will ein handlungsfähiges, ein solidarisches, ein selbstbewusstes Europa. Dafür müssen wir bereit sein, Europa da zu stärken, wo europäische Lösungen besser als nationale oder regionale sind.Ein Beispiel ist der Aufbau eines europäischen Grenzschutzes.Inzwischen erfüllt dieser europäische Grenzschutz seine Aufgaben viel wirkungsvoller, als das noch vor zwei Jahren der Fall war. Wir haben den Schutz der Außengrenzen 2016 auf eine neue Grundlage gestelltund die europäische Grenz- und Küstenschutzwache Frontex mit neuen Befugnissen ausgestattet. Aber es bleibt in diesem Bereich noch sehr viel zu tun. Die Europäische Union hat 14 000 Kilometer Außengrenze. Deshalb muss die Personalausstattung von Frontex massiv verbessert werden.Bei der Neuverteilung der Strukturfondsmittel müssen wir darauf achten, dass die Verteilungskriterien künftig auch das Engagement vieler Regionen und Kommunen bei der Aufnahme und Integration von Migranten widerspiegeln.Meine Damen und Herren, übergeordnetes Ziel eines modernisierten EU-Haushalts muss natürlich ein wettbewerbsfähiges Europa sein. Dies umfasst zum einen die Stärkung von Forschung, Innovation und Digitalisierung sowie Infrastruktur, wie wir uns das auch in unserem Koalitionsvertrag vorgenommen haben. Zum anderen sollten EU-Mittel aber auch stärker dafür eingesetzt werden, die Umsetzung von Strukturreformen voranzubringen. Strukturreformen sind ja eine Daueraufgabe für ein wettbewerbsfähiges und stabiles Europa, und der EU-Haushalt muss dazu einen wichtigen Beitrag leisten.Mit Blick auf die Regionen wollen wir sicherstellen, dass auch im neuen EU-Haushalt eine starke Kohäsionspolitik gewährleistet ist. Weniger entwickelte Mitgliedstaaten brauchen weiter Unterstützung. Zugleich sollten die EU-Strukturfonds weiterhin allen Regionen zukommen, damit sie jeweils ihre Herausforderungen angehen können. Das ist natürlich auch eine Frage der Solidarität. Dabei ist unser Verständnis, dass Solidarität keine Einbahnstraße ist.Es obliegt allen Mitgliedstaaten, die Verantwortung für das Ganze nie aus dem Blick zu verlieren. Dazu gehört selbstverständlich auch die Wahrung unserer gemeinsamen europäischen Werte, auf denen die Europäische Union überhaupt nur beruhen kann. Deshalb ist es von großer Bedeutung, dass wir auf dem morgigen informellen Treffen der 27 EU-Staats- und Regierungschefs auch über den Vorschlag europaweiter Bürgerdialoge sprechen, mit denen wir in diesem Jahr Europa wieder stärker in das Bewusstsein unserer Bürgerinnen und Bürger bringen wollen. Ich halte das für ein ganz wichtiges Anliegen.Wir haben als frühere Bundesregierung einen Dialog mit unseren Bürgerinnen und Bürgern geführt. Das war sehr aufschlussreich. Wir müssen genauso für die europäische Idee werben, sie erläutern und versuchen, Bürgerinnen und Bürger dafür zu begeistern. Das heißt auch, dass wir den Menschen zuhören, dass wir uns um ihre Wünsche, Sorgen und Anliegen besser kümmern. Das gelingt nach meiner festen Überzeugung am besten im persönlichen Gespräch. Deshalb wollen wir in diesen Dialog mit der Zivilgesellschaft, den Bürgerinnen und Bürgern eintreten.Gemeinsam mit dem französischen Präsidenten Emmanuel Macron werde ich beim informellen Europäischen Rat dafür werben, dass sich möglichst viele Mitgliedstaaten an diesen Dialogen beteiligen. Die Ergebnisse der nationalen Dialoge sollten dann Ende des Jahres zusammengeführt werden und unsere weiteren europäischen Überlegungen unterstützen; denn die Bürgerinnen und Bürger erwarten, dass wir nicht einfach nur sprechen, sondern dass das, was sie uns mit auf den Weg geben, auch umgesetzt wird. Ich bin sehr dankbar, dass der Kommissionspräsident Jean-Claude Juncker angeboten hat, den Prozess der Bürgerdialoge mit den Mitteln der Kommission zu unterstützen.Herr Präsident, liebe Kolleginnen und Kollegen, das sind unsere Leitlinien für das bevorstehende Treffen der europäischen Staats- und Regierungschefs. Der nächste reguläre Rat wird im März dieses Jahres stattfinden. Unser Treffen soll ein gemeinsames Verständnis darüber schaffen, wie die zukünftig 27 EU-Mitgliedstaaten vorangehen wollen und wie sie die gemeinsamen Herausforderungen bewältigen wollen. Wir werden Beharrlichkeit brauchen, Verantwortungsbewusstsein und die Bereitschaft, der europäischen Sache zu dienen. Dann können wir erfolgreich sein. Deutschland wird sich so einbringen, dass wir erfolgreich werden.Herzlichen Dank.




	12. Barbara Hendricks (SPD) Frau Präsidentin! Liebe Kolleginnen und Kollegen! Die Bundesregierung ist sich darüber im Klaren, dass das Klimaziel einer Minderung der Treibhausgasemissionen um mindestens 40 Prozent bis 2020 im Verhältnis zu 1990 nicht einzuhalten ist bzw. ohne zusätzliche Maßnahmen verfehlt wird. Das Umweltministerium geht auf Basis aktueller Daten von Ende vergangenen Jahres von einer größeren Lücke aus, als bisher gedacht. Das hat zum Beispiel mit dem Wirtschaftswachstum und der Entwicklung der Bevölkerungszahl zu tun.Wir müssen nun zügig wirksame Maßnahmen ergreifen, um diese Lücke so schnell wie möglich zu schließen. Wir werden noch in diesem Jahr unter Einbindung aller Akteure an einem sozialverträglichen und für die Regionen verkraftbaren Kohleausstiegsfahrplan arbeiten.Der Ausstieg aus der Kohle ist aber nur ein Teil der Lösung. Zum Beispiel müssen auch im Gebäudebereich die Emissionen weiter gesenkt werden. Vor allem aber ist es allerhöchste Zeit, dass die Bereiche Verkehr und Landwirtschaft ihre Beiträge leisten und endlich wirksam umsteuern.Die Treibhausgasemissionen des Verkehrs lagen zuletzt sogar wieder über dem Niveau von 1990. Das ist einfach nicht akzeptabel, und da müssen auch durchaus die Bürgerinnen und Bürger ein Bewusstsein für ihr eigenes Verkehrsverhalten entwickeln.– Nein, ich will nicht nur die Bürgerinnen und Bürger beschimpfen.– Nein. – Also, dass wir auf jeden Fall eine Verkehrswende brauchen, ist gar nicht zu bestreiten. Dass man persönlich zur Verkehrswende beiträgt, wenn man zum Beispiel nicht ein großes, dickes, fettes Auto kauft, ist aber auch nicht zu bestreiten. Es liegt insbesondere in den Händen des künftigen Bundesverkehrsministers, diesen Trend so schnell wie möglich umzukehren und tatsächlich eine Verkehrswende herbeizuführen.Mit dem Klimaschutzplan 2050 und den Sektorzielen für 2030 haben wir einen klaren Fahrplan, um unser Ziel einer Minderung von mindestens 55 Prozent bis 2030 zu erreichen. Dazu werden wir noch in diesem Jahr ein Maßnahmenprogramm mit konkreten Schritten für alle Sektoren erarbeiten. Dabei dürfen wir uns auch vor schwierigen Debatten nicht wegducken. Der Entwurf des Koalitionsvertrages zeigt ganz in diesem Sinne eine neue Entschiedenheit, mit der die Bundesregierung für die Erreichung wirklich anspruchsvoller Klimaschutzziele kämpfen will.– Doch, das ist mit der Union verabredet, und ich verlasse mich darauf, dass wenn wir zu einer Koalitionsbildung kommen, wir das auch entsprechend umsetzen werden.Der Vertrag bekräftigt nicht nur die bestehenden Klimaziele, sondern er bekennt sich auch unmissverständlich zur Umsetzung des Klimaschutzplans 2050 und seiner Sektorziele für 2030. Das gibt uns eine belastbare Grundlage für die durchaus anspruchsvolle Arbeit, die vor uns liegt. Am anschaulichsten wird das anhand des vereinbarten Klimaschutzgesetzes, das einen starken und verlässlichen rechtlichen Rahmen für den Klimaschutz bilden wird. Das ist neu. Das haben wir in der letzten Legislaturperiode noch nicht verabreden können. Dies wird die Verbindlichkeit von Klimaschutz und die Investitionssicherheit weiter erhöhen.Liebe Kolleginnen und Kollegen, auf den Deutschen Bundestag kommen in diesem Jahr und auch in den folgenden Jahren wichtige Debatten zu, die über den Erfolg der Klimapolitik in Deutschland und letztlich über die ökonomische Zukunftsfähigkeit unseres Landes entscheiden. Es ist ein Thema, das uns alle angeht und dessen Folgen uns alle betreffen. Lassen Sie uns deshalb den Weg in eine treibhausgasneutrale Zukunft gemeinsam und verantwortungsvoll gehen. Die nächsten Jahre sind entscheidend.Herzlichen Dank.Vielen Dank, Dr. Barbara Hendricks. – Ich habe da oben gerade jemanden erkannt, den wir kennen. Ich begrüße unseren ehemaligen Kollegen Alexander Bonde auf der Tribüne.




	13. Uwe Feiler (CDU) Sehr geehrter Herr Präsident! Meine Damen und Herren! Liebe Kolleginnen und Kollegen! Im Koalitionsvertrag steht Europa an vorderster Stelle. Das ist angesichts der Herausforderungen, vor denen wir in Europa, aber auch in der Welt stehen, richtig und gut. Dabei liegt es vor allem an Deutschland selbst, bei den wichtigen Weichenstellungen, die die Europäische Union in den kommenden Monaten zu treffen hat, seine Interessen und die europäischen Interessen nicht nur zu formulieren, sondern entschlossen dafür zu werben und Mehrheiten zu gewinnen. Deshalb bin ich unserer Bundeskanzlerin äußerst dankbar, dass sie morgen in Brüssel genau das macht, und zwar unaufgeregt, aber zugleich zielstrebig und erfolgreich, so wie wir sie kennen.Mit der Entscheidung des Vereinigten Königreichs, aus der Europäischen Union auszuscheiden, ändern sich für Deutschland und die anderen Mitgliedstaaten wesentliche Rahmenbedingungen der Zusammenarbeit. Neben den rechtlichen Fragen fällt der Zeitpunkt der Austrittsverhandlungen in die Beratungen über den mehrjährigen Finanzrahmen der Europäischen Union. Allen hartnäckigen Gerüchten über den Britenrabatt zum Trotz verlieren wir neben einer großen Volkswirtschaft einen großen Nettozahler für den Haushalt der Europäischen Union. Der Entschluss der Briten, der Europäischen Union den Rücken zu kehren, reißt ein beträchtliches Loch von circa 12 Milliarden bis 14 Milliarden Euro pro Jahr in den EU-Haushalt. In diesem Zusammenhang begrüße ich es deshalb ausdrücklich, dass die Kommission diese Lücke zumindest zur Hälfte durch Einsparungen schließen möchte, anstatt ausschließlich nach immer mehr Geld aus den Mitgliedstaaten zu rufen.Jedoch werden wir in den kommenden Jahren nicht nur mit einem Defizit auf der Einnahmenseite zu rechnen haben. Die heutige Zeit stellt uns vor eine Vielzahl neuer Herausforderungen und Aufgaben. Die Bekämpfung des Terrorismus, die Gewährleistung der inneren und äußeren Sicherheit, der Grenzschutz, Verteidigungsinvestitionen sowie Ausgaben für große Forschungsprojekte müssen finanziert werden. In den vergangenen Jahren ist von uns zu Recht immer wieder gefordert worden, dass sich die Europäische Union hier stärker engagiert. Den Vorschlag der Kommission, diese Mehrausgaben bis zu 80 Prozent aus neuen Mitteln zu decken, halte ich persönlich vor diesem Hintergrund für nachvollziehbar. Nach den Vorstellungen der Kommission sollen die angesprochenen Einsparungen durch maßvolle Kürzungen in der europäischen Agrar- und Kohäsionspolitik erfolgen. Da diese beiden Positionen zusammen mit einem Anteil von 73 Prozent einen Großteil der Ausgabenpositionen ausmachen, ist es verständlich, hier als Erstes anzusetzen.Als Brandenburger Abgeordneter sei mir hier aber der Einwand erlaubt, dass neben den Direktzahlungen für unsere Landwirtschaft insbesondere die Strukturfonds einen unersetzlichen Beitrag zur Stärkung des ländlichen Raums leisten.Im Unterschied zum Vorschlag von Haushaltskommissar Oettinger geht der Kommissionspräsident Jean-­Claude Juncker sogar einen Schritt weiter. Durch den Ausschluss aller Länder mit einem überdurchschnittlichen Pro-Kopf-Einkommen von Strukturfondsmittel würden zwar 100 Milliarden Euro im nächsten MFR zusätzlich eingespart werden. Jedoch fielen neben Deutschland die Beneluxstaaten, Frankreich, Schweden, Dänemark und Österreich aus der Förderung heraus, obwohl es sich bei diesen Staaten um Nettozahler handelt. Ich kann mir nicht vorstellen, dass die Akzeptanz der EU in diesen Staaten zunimmt, wenn von den starken Volkswirtschaften zu Recht mehr Mittel gefordert werden,dann aber ein Ausschluss von großen Teilen der Förderungspolitik erfolgt.Auch in Deutschland gibt es nach wie vor Regionen, die einer Unterstützung durch die Europäische Union bedürfen. Deshalb haben alle Mitgliedstaaten einen Beitrag zur Konsolidierung des Haushaltes der EU zu leisten. Die Mitgliedstaaten sind hier gefragt, ihre Kreativität nicht nur auf zusätzliche Programme und Projekte zu beschränken, sondern auch mit konkreten Einsparvorschlägen aufzuwarten. So warb die Kommission erst kürzlich für mehr Einsparungen, aber eben auch um die Bereitschaft der Mitgliedstaaten, zusätzliche Mittel für neue Aufgaben und Projekte bereitzustellen.Dies darf meiner Meinung nach aber nur geschehen, solange die Investitionen einen europäischen Mehrwert erzielen, das heißt, einen Zweck für die europäische Gemeinschaft erfüllen, ökonomischen und technischen Fortschritt bringen sowie für Stabilität und Ordnung in den Mitgliedstaaten sorgen. Aus meiner Sicht ist es unerlässlich, diesen Maßstab auch bei bestehenden Projekten anzulegen. Wir brauchen also eine Ausgabenkritik auch für bestehende Projekte.Mehrausgaben könnten beispielsweise zum Teil mit neuen Einnahmen bestritten werden. Der Kommission schweben eine neue EU-weite Plastiksteuer oder die Aufnahme des EU-Emissionshandelssystems in den EU-Haushalt vor. Darüber wird zu diskutieren sein, wenn wir uns ausführlich mit den Eigenmitteln der Europäischen Union auseinandersetzen. Für mich ist aber wichtig, dass neue Einnahmequellen eine Lenkungswirkung aufweisen.Vielleicht bietet aber auch der Brexit eine Chance, seit langer Zeit strittige Vorhaben voranzutreiben. Ich nenne als Stichworte die Gemeinsame konsolidierte Körperschaftsteuer-Bemessungsgrundlage oder eine EU-weite Finanztransaktionsteuer.Nun ist es in der EU wie in jeder großen Familie: Sobald man sich mit finanziellen Forderungen auseinandersetzen muss, werden die meisten erst einmal ganz still. So sind Österreich, die Niederlande und weitere Mitgliedstaaten nicht bereit, Mehrkosten zu übernehmen, bleiben aber Vorschläge schuldig, wo genau Einsparungen stattfinden sollen. Ich erwarte deshalb mit großer Spannung, wie sich Österreich im zweiten Halbjahr 2018 präsentiert, wenn es die Ratspräsidentschaft innehat. Mehr, zum Teil auch berechtigte Ausgabenwünsche beim Austritt eines großen Nettozahlers und gleichbleibenden Zahlungen der anderen Mitgliedstaaten werden keine Lösung darstellen.In Richtung der AfD sei gesagt: Hinsichtlich der von Ihnen angesprochenen angeblichen Unsummen, die wir aus Berlin nach Brüssel schicken, bin ich der Meinung, dass wir damit vollkommen richtig handeln; denn 1 Prozent des Steueraufkommens des Bundes für ein Europa in Frieden, Freiheit und Wohlstand auszugeben, halte ich für einen angemessenen Beitrag.Danke für Ihre Aufmerksamkeit.




	14. Marc Henrichmann (CDU) Herr Präsident! Meine lieben Kolleginnen und Kollegen! Liebe Besucher! Die erste Rede ist immer ein bisschen mit Freude darüber verbunden, jetzt endlich in die Sacharbeit intensiv einzusteigen. Aber die Ernüchterung machte sich gestern Mittag schnell breit; der Antrag war bereits angekündigt. Da wir wissen, dass in Verfassungsrechte eingegriffen werden soll, erwartet man tiefgehende Begründungen. Die Begründungen waren doch relativ dürftig.Wenn man in Verfassungsrechte eingreift, dann hätte man zumindest dahin gehend Begründungen erwartet, welche konkreten Probleme denn bestehen, wie viele Burkaträgerinnen man irgendwann – wo auch immer – gesehen hat, gerade weil es in diesem Bereich Verschärfungen im letzten Jahr gab; das wurde angesprochen. Vielleicht bin ich kommunalpolitisch romantisiert, aber ich kenne das Formulieren von Anträgen so, dass man eine Problemlage schildert und dann dazu den Lösungsweg bietet. Die Frage ist: Geht es hier um Lösungen oder eben nicht? Das müssen Sie beantworten.In der vergangenen Woche bin ich zum Kommunalen Integrationszentrum in meinem Heimatkreis gefahren und habe die Mitarbeiter gefragt: Wie viele vollverschleierte Personen habt ihr hier im letzten Jahr gesehen? Die Reaktion waren erstaunte Blicke, und die Antwort lautete: gar keine. Wie gesagt, wir reden hier über Verfassungsrechte, eine entsprechende Begründung hätte man zumindest erwarten dürfen. Neben Ernüchterung löste der Antrag auch Erstaunen aus, weil im ersten Satz der Begründung dieses Antrags der Schutz der Freiheitsrechte muslimischer Frauen gefordert wird. So etwas bei der AfD zu lesen, ist schon allerhand. Donnerwetter! Da hat ein Umdenken offensichtlich schon stattgefunden. Hut ab vor dem zum Islam konvertierten Kollegen Ihrer Fraktion aus Brandenburg.Zur Sache: Es hat im Jahr 2017 – wir haben es gehört – ein partielles Vollverschleierungsverbot gegeben mit weitreichenden Folgen in den Bereichen öffentlicher Dienst, Gerichte, Bundeswehr und bei Wahlen. Und wir haben die Grenzen des verfassungsrechtlich Möglichen damit ausgeschöpft. Ich kann ehrlich gesagt nicht erkennen, wo sich hier die Rechtslage oder die Rechtsauffassung geändert haben soll. Dazu lese ich auch im Antrag überhaupt nichts. Gesetze aber bis an den Rand des verfassungsmäßig Möglichen verschärfen zu wollen, um vermeintliche Probleme zu lösen, die im Antragstext gar nicht konkret benannt werden, halte ich für absurd.Die Union hat mit der Verschärfung im letzten Jahr ein eindeutiges und richtiges Signal gesetzt. Die Vollverschleierung widerspricht unserer Kultur. Ich gehe sogar noch weiter: Sie steht für ein falsches Gesellschaftsbild, und sie behindert jegliche Integration.Aber ich glaube auch, dass wir nicht alles verbieten können und nicht alles verbieten müssen, was wir ablehnen. Um es mit Montesquieu zu sagen:Wenn es nicht unbedingt erforderlich ist, ein Gesetz zu machen, dann ist es unbedingt erforderlich, kein Gesetz zu machen.Das ist für mich übrigens auch der Unterschied zwischen selbstbewusstem Konservatismus und rückwärtsgewandter Willkür.Unsere Werte basieren auf dem christlichen Menschenbild, auf Gleichberechtigung und Freiheit. Wir sollten selbstbewusst für unsere Werte einstehen und sie verteidigen. Ich glaube, nur so können wir auch andere von unseren Werten überzeugen. Als Familienvater ist es mir wichtig, zu fragen: Was geben wir eigentlich unseren Kindern mit auf den Weg? Angst oder Selbstbewusstsein? Haben wir Angst vor einer Handvoll – das bestätigen alle Studien zu diesem Thema – ausländischer Touristinnen aus den Golfstaaten, die Burka tragen? Ich glaube, diese Angst müssen wir nicht haben.Der Antrag, der heute vorliegt, ist angstgesteuert und setzt auf Willkür. Aber Willkür – das wurde schon angesprochen – kann schnell jeden treffen. Deswegen sollten wir auf der Grundlage unseres Menschenbildes Politik machen und nicht auf dem Rücken Einzelner.Ich will gar nicht verhehlen, dass es Probleme und Handlungsbedarf gibt: bei Sprache, bei Integrationskursen. In Deutschland muss der Rechtsstaat durchgesetzt werden. Deswegen treten wir für ein Musterpolizeigesetz, für mehr Polizei und eine starke Justiz ein. Wir müssen auch den Dialog mit den Islamverbänden anders und offensiver führen, als wir das bislang getan haben. Ich rufe die gemäßigten Vertreter ausdrücklich auf, sich an der Debatte zu beteiligen. Wir können keinen isolierten und keinen radikalen Islam in den Hinterhöfen unserer Städte akzeptieren. Aber Ihnen geht es hier nicht um den kritischen Dialog oder um Lösungen. Sie schüren Ängste; das ist Ihre Existenzberechtigung.Schlussbemerkung: Ich bin, ehrlich gesagt, dieses Pingpong ein bisschen leid, das wir in den letzten Tagen in diesem Hohen Haus erlebt haben. Immer wenn es um Islam und Integration geht, wird im linken politischen Spektrum schnell die rosarote Brille aufgesetzt, und auf der anderen Seite gibt es Verbitterung und Schaum vor dem Mund. Ich glaube, es bringt keinem was, wenn wir die Bracken hochklappen und Mauern hochziehen und uns gegenseitig mit Schmutz beschmeißen. Die Wahrheit liegt, wie so oft, in der Mitte. Wir stehen für die Partei der Mitte. Uns treibt das Vertrauen in die Stärke und das Selbstbewusstsein Deutschlands und seiner Menschen an.Vielen Dank.Herr Kollege Henrichmann, herzlichen Dank für Ihren Beitrag. Die Zeit haben Sie etwas überschritten. – Ich möchte darauf hinweisen, dass bei einer ersten Rede parlamentsüblich keine Zwischenfragen zugelassen werden, um den Redner nicht aus dem Konzept zu bringen.Herr Kollege Baumann, ist an mir etwas vorbeigegangen? Ich habe nur gesehen, dass mehrere Personen Ihrer Fraktion schlagartig den Saal verlassen haben. Ich hoffe nicht, dass das Präsidium da etwas nicht mitbekommen hat.– Gut.




	15. Lorenz Gösta Beutin (DIE LINKE.) Frau Präsidentin! Liebe Kolleginnen und Kollegen! Liebe Zuhörerinnen und Zuhörer! Es ist immer so: Ich setze mich hin, schreibe mir meine Rede, und dann sitze ich hier, höre die Debatte – und vor mir spricht ein Redner der AfD – und denke: Das ist so ein hanebüchener Unsinn, das kannst du nicht einfach stehen lassen.Ich will nur einen Punkt sagen: Laut einer aktuellen Studie teilen 8 Prozent der Menschen in der Bundesrepublik Deutschland die klimaleugnerischen Positionen der AfD. Nicht einmal die gesamte Klientel der AfD teilt diese klimaleugnerischen Positionen.Aber fast 80 Prozent der Menschen in der Bundesrepublik Deutschland sagen: Wir müssen viel entschiedenere Maßnahmen zum Klimaschutz ergreifen.Und genau die liegen richtig, und genau auf deren Seite stehen wir hier.Nun will ich kurz zum Antrag der FDP kommen. Meine Damen und Herren, Sie beweisen mit Ihrem Antrag, dass Sie letztlich doch nicht mehr sind als die Partei der Lobbyisten. Das hat sich nicht geändert.Sie treten einseitig für die Interessen der deutschen Industrie und nicht für die Interessen der Menschen in diesem Land ein.Mit Ihrem Antrag – dazu führe ich gleich aus – sägen Sie an den Grundpfeilern der deutschen Klimapolitik. Das ist ein verheerendes Signal.Was Sie wollen, ist nichts anderes als die alte Leier der FDP: Der Markt richtet alles. – Nichts anderes bedeutet es, wenn Sie sagen, der Emissionshandel solle auf alle Wirtschaftssektoren ausgeweitet werden.Das wäre in der Konsequenz der Versuch, wirksame Klimaschutzmaßnahmen grundlegend auszubremsen. Es wäre eine irrwitzige Maßnahme, würden wir diesem Antrag zustimmen.Denn wenn wir den Emissionshandel auf alle Wirtschaftssektoren ausweiteten, hieße das beispielsweise, dass der Wärmesektor eingeschlossen wäre, dass die Autoindustrie eingeschlossen wäre, dass die Agrarindustrie mit ihren Emissionen eingeschlossen wäre. Das würde bedeuten, dass sich all diese Industrien mit Emissionszertifikaten freikaufen könnten. Das wäre nichts anderes als eine moderne Form des Ablasshandels. Deswegen lehnen wir das ab.Aber lassen Sie mich noch ein Wort sagen: Ich bin immer wieder sehr über den Gleichklang von AfD und FDP in Fragen des Klimaschutzes erstaunt, und zwar an der Stelle, dass beide fordern, das Erneuerbare-Energien-Gesetz abzuschaffen.Das Erneuerbare-Energien-Gesetz abzuschaffen, würde die Axt an die Grundlagen der deutschen Klimapolitik legen.Und noch viel mehr: Es würde 350 000 Arbeitsplätze auf dem Sektor der erneuerbaren Energien kosten, denn dort haben wir jetzt schon 350 000 Arbeitsplätze, und das Bundeswirtschaftsministerium prognostiziert, dass wir in den nächsten Jahren weitere 250 000 Arbeitsplätze hinzugewinnen werden.Das Wasser steht uns längst bis zum Hals. Das hat auch die Studie ergeben, die besagt: Der Meeresspiegelanstieg hat sich verdoppelt. – Wir müssen jetzt handeln.Wir müssen jetzt in den Kohleaussieg einsteigen, wir müssen jetzt die 20 dreckigsten Braunkohlemeiler lahmlegen. Wir müssen jetzt Ökostromausschreibungen forcieren. Wir müssen jetzt in die ökologische Verkehrswende einsteigen, und dazu gehört auch ein kostenfreier öffentlicher Personennahverkehr.Die Zeit zu handeln, ist jetzt. Es geht um die Zukunft der gesamten Menschheit. Es geht um unser aller Lebensgrundlagen!Vielen Dank.




	16. Andreas Lenz (CSU) Sehr geehrte Frau Präsidentin! Sehr geehrte Damen und Herren! Liebe Kolleginnen und Kollegen! Es ist interessant, über was man alles sprechen kann, wenn es eigentlich um das ERP-Wirtschaftsplangesetz 2018 geht.Liebe Kollegen, nur weil man bei gewissen Entscheidungen hier im Parlament und im Rahmen von Regierungshandeln nicht dabei war, heißt das nicht, dass sie nicht stattgefunden haben, wie in einigen Wortbeiträgen anklang.Die Ursprünge des ERP-Sondervermögens liegen mittlerweile 70 Jahre zurück. Das muss man jetzt nicht jedes Jahr wiederholen. Am 3. April 1948 wurde der Marshallplan vom Kongress der Vereinigten Staaten verabschiedet. Ich glaube, angesichts dieses 70-jährigen Jubiläums kann man schon einmal sagen, dass wir dankbar sind, dass der Marshallplan damals initiiert wurde und dadurch das deutsche Wirtschaftswunder erst überhaupt möglich wurde. Es ist, wie gesagt, ein Beitrag für das deutsche Wirtschaftswunder gewesen. Wir haben auch schon gehört, dass es wichtig ist, dass wir das transatlantische Austauschprogramm, das ebenfalls mit ERP-Mitteln finanziert wird, gerade jetzt weiterbetreiben.Für das Wirtschaftsplangesetz 2018 werden Mittel in Höhe von rund 790 Millionen Euro bereitgestellt. Diese Mittel ermöglichen wiederum ein Fördervolumen von insgesamt 6,75 Milliarden Euro. Die Schwerpunkte wurden schon genannt; sie liegen zum einen nach wie vor in der regionalen Wirtschaftsförderung, aber zum anderen auch in der Wachstumsfinanzierung, in den Existenzgründungen, in der Förderung von Innovationen sowie in der Exportfinanzierung, wobei es übrigens in erster Linie um Entwicklungshilfe mit dem Schwerpunkt Afrika geht.Es ist zwar immer noch so, dass gerade die Existenzgründer und die mittelständischen Unternehmen in ihrer Finanzierungsstruktur oftmals gegenüber Großunternehmen benachteiligt sind; trotzdem konnte das Förderpotenzial in der Vergangenheit aufgrund des Niedrigzinsumfeldes – wir haben es gehört – nicht immer gänzlich ausgeschöpft werden. Gleichzeitig besteht hoher Bedarf an Wagnisfinanzierungen, beispielsweise in der frühen Seed-Phase, in der Gründungsphase generell und ebenso im Bereich der Wachstumfinanzierungsmöglichkeiten für schnell wachsende Unternehmen. Beide Punkte wurden bereits in den letzten vier Jahren deutlich adressiert.Mit der Gründung der KfW-Beteiligungstochter werden die Bereiche Wagnis- und Beteiligungskapitalfinanzierung nun weiter ausgebaut. Dabei werden die erfolgreichen Instrumente des High-Tech Gründerfonds, des ERP-Venture-Capital-Fondsinvestments und von Coparion gestärkt. Diese Instrumente unterstützen gerade die jungen Hightechunternehmen. Ziel ist es, das Zusagevolumen dieser drei Säulen bis zum Jahr 2020 auf 200 Millionen Euro jährlich zu verdoppeln. Wir achten bei der neuen Beteiligungsgesellschaft natürlich auf die Informations- und Kontrollrechte des Parlaments. Das Gebot des Substanzerhalts des Sondervermögens steht dabei an erster Stelle. Und wir wissen natürlich auch, dass der Staat nicht der bessere Investor ist. Der Europäische Investitionsfonds kann hier durchaus Vorbild sein. Dieser erzielt mittlerweile übrigens sogar ganz ordentliche Renditen.An die 90 Prozent von Venture-Capital-Finanzierungen werden nie Gewinne machen, andere hingegen sehr wohl. Das müssen wir aushalten und sozusagen von vorneherein einpreisen. Öffentliche Mittel können die Angebotslücke allerdings nur teilweise schließen; wir haben es gehört. Bedeutsamer ist die Hebelwirkung zur Mobilisierung von privaten Investitionen. Die Verfügbarkeit von Wagniskapital im Verhältnis zur Größe der Volkswirtschaft ist in Deutschland in der Tat eher gering. Der Anteil des investierten Wagniskapitals am Bruttoinlandsprodukt liegt etwa in den USA bei 0,33 Prozent, bei uns im Gegensatz dazu bei 0,03 Prozent. Auch in einigen europäischen Ländern ist dieser Anteil höher. Fest steht also: Den Gründern und jungen aufstrebenden Unternehmen steht in Deutschland häufig zu wenig Wagniskapital zur Verfügung.Antworten geben wir übrigens im Koalitionsvertrag. Wir bekennen uns hier ausdrücklich dazu, die Investitionsbereitschaft in Wachstumsunternehmen zu erhöhen. Ideen aus Deutschland müssen auch mit heimischem Kapital finanziert werden. Deshalb ist es wichtig, dass wir mehr institutionelle Anleger, aber vor allem mehr privates Kapital für Investitionen in Start-ups akquirieren. Es ist eben schon entscheidend, woher das Kapital kommt.Entschuldigung, Herr Dr. Lenz. – Liebe Kolleginnen und Kollegen, könnten Sie dem Redner ein bisschen mehr Aufmerksamkeit schenken? Es ist zwar spannend, dass Sie miteinander reden, aber er hat es verdient, dass Sie ihm zuhören. Sonst können Sie ihm ja nicht qualifiziert antworten.Herr Lenz, Sie sind dran.Es ist, wie gesagt, schon entscheidend, woher das Kapital kommt, weil natürlich auch die Eigentumsverhältnisse dementsprechend gestaltet sind. Ich will es an einem Beispiel festmachen: Ein Studienkollege, ein Studienfreund von mir hat vor circa acht Jahren ein Unternehmen gegründet. Er wurde gefördert von EXIST und auch vom High-Tech Gründerfonds. Er wurde dann eingeladen ins Silicon Valley, auch durch den German Silicon Valley Accelerator. Er lernte dort seine Venture-Capital-Geber kennen. Das war natürlich besser, als wenn er gar keine Kapitalgeber gefunden hätte; aber lieber wäre es mir natürlich schon, wenn er auch dementsprechende heimische Kapitalgeber gefunden hätte, weil natürlich immer die Gefahr droht, dass die potenziellen Global Player von morgen abwandern und in anderen Gebieten ihre Tätigkeit aufnehmen.Der Wohlstand von morgen sind die Unternehmensgründungen von heute. Deshalb müssen wir die Voraussetzungen schaffen, dass Gründungen von der Idee bis zum Börsengang, bis zum Global Player auf ein bestmögliches Umfeld treffen. Einen weiteren Schritt hierzu gehen wir mit den neuen Möglichkeiten durch das ERP-Sondervermögen. Gleichzeitig erhalten wir die bewährten Instrumente. Ich bitte deshalb um Zustimmung und bedanke mich für die Aufmerksamkeit.




	17. Alexander Radwan (CSU) Herr Präsident! Meine sehr verehrten Damen und Herren! Unsere heutige Aussprache widmet sich der aktuellen Entwicklung im Nahen Osten. Eigentlich müssten wir uns regelmäßig über die dramatischen Entwicklungen und die aktuellen Erscheinungen dort austauschen. Die Sicherheitskonferenz wurde schon mehrfach erwähnt. Wenn man sich die Sicherheitskonferenzen in den letzten Jahren anschaut, dann kann man sicherlich die Überschrift wählen: Instabilität in der Welt nimmt zu. – Die Unberechenbarkeit ist eine gewisse Konstante geworden. Das beginnt beim Verhältnis der USA zu Russland. Ich nehme keine Bewertung vor, wer hier wem nähersteht. Vielmehr geht es schlicht und ergreifend darum, dass sich das Vertrauen zwischen diesen beiden Mächten nicht vergrößert.Es hat natürlich Auswirkungen auf die zur Diskussion stehende Region, wenn sich die Mächte, die bisher dort ordnende Hand waren und Einfluss genommen haben, zurückziehen. Dort, wo man sich zurückzieht, entsteht ein Vakuum. In dieses Vakuum gehen nun regionale Mächte, die selbst ordnende Hand sein möchten und Strukturen schaffen wollen. Das ging mit dem IS los. Wenn man sich die momentan medial sehr stark dargestellte Situation in Syrien anschaut, dann kann man zu Syrien aktuell nur sagen: Dort bilden sich momentan aus dem Konglomerat aus Türkei, Iran, Kurdenthematik und Saudi-Arabien Koalitionen, die zumindest ich – es ist fraglich, ob das jemand anderes vorausgesehen hätte – für nicht möglich gehalten hätte. Diese Koalitionen haben eigene Interessen und schaffen Strukturen vor Ort. Es ist richtig, dass wir alles daransetzen müssen, den Staatsterror in Syrien zu stoppen. Herr Annen, das von Ihnen angesprochene Verhältnis zwischen der NATO, der Türkei und den USA und die dort entstandenen Widersprüche – ich weiß nicht, ob Sie das vor einem halben Jahr vorausgesehen hätten – muss die NATO auf die Agenda setzen. Aber aufgrund der Aktualität des Konflikts in Syrien stehen andere Konflikte nicht mehr im Vordergrund, obwohl es dabei um Entwicklungen geht, die genauso gefährlich sind.Der Iran wurde bereits angesprochen. Natürlich wollen wir das Atomabkommen beibehalten. Gleichzeitig müssen wir den Iran ermahnen, das falsche Spiel gegenüber Israel und den Golfstaaten nicht weiter zu betreiben. Gleiches gilt für Saudi-Arabien, das momentan durch Verhandlung versucht, selbst in den Besitz von Atomkraftwerken zu kommen und so auf Augenhöhe mit dem Iran zu sein. Die Situation, die sich dort vorfinden lässt, wird sich in den nächsten ein, zwei oder drei Jahren nicht deeskalieren. Vielmehr bauscht es sich weiter auf. Das Gleiche gilt für das Existenzrecht Israels. Das steht auf der deutschen Agenda ganz oben. Gleichzeitig müssen wir alle in dieser Region ermahnen, alles daranzusetzen, dass es nicht zu einer weiteren Eskalation kommt. Das gilt für Russland genauso wie für die USA.Die NATO hatte ich bereits angesprochen. Das Gleiche gilt für die Europäische Union. Europa sollte endlich dort, wo ein Vakuum entstanden ist, mit einer Stimme sprechen. Dort, wo Konflikte zunehmen, ist mehr Diplomatie notwendig und nicht weniger, ist mehr Einflussnahme notwendig und nicht weniger. Meine Bitte lautet: Man kann Erdogan und anderen zu Recht vorwerfen, dass sie Außenpolitik aus Sicht der Innenpolitik machen. Wenn wir unsere Verantwortung ernst nehmen, sollten wir das in Deutschland lassen.Lassen Sie mich mit Blick auf die Äußerungen des Kollegen der AfD nur eines sagen: Ich weiß nicht, ob die Situation im Nahen Osten dadurch besser wird, dass, wie gefordert, alle Türken aus Deutschland ausgewiesen werden und in meinem Wahlkreis während einer Wahlkampfveranstaltung die AfD gefordert hat – so berichtete es eine Zeitung –: „Wir müssen den Islam vernichten“, und ob das ein Beitrag dazu ist, diese Region zu befrieden.Besten Dank.Herzlichen Dank, Herr Radwan. – Als Nächstes hat der Kollege Dr. Christoph Hoffmann das Wort für die Freien Demokraten. Ich weise darauf hin: Es ist seine erste Rede in diesem Parlament.




	18. Astrid Grotelüschen (CDU) Meine sehr geehrte Frau Präsidentin! Verehrte Kolleginnen und Kollegen! Liebe Zuhörer! Ich möchte meine Rede mit einem Zitat des Mannes eröffnen, der mit seiner politischen Weitsicht nicht nur zum Wiedererstarken der deutschen und der europäischen Wirtschaft nach dem Zweiten Weltkrieg beigetragen hat, sondern auch heute noch einen erheblichen Beitrag zur Förderung des deutschen Mittelstandes leistet. Der Politiker und General George C. Marshall soll einmal gesagt haben: „Kleine Taten, die man ausführt, sind besser als große, die man plant.“Mit dem heute zu beschließenden ERP-Wirtschaftsplangesetz wollen wir es nicht bei Planungen belassen, sondern wir wollen deutschlandweit Impulse setzen: für Investitionen, für Modernisierung, für das Wachstum mittelständischer Unternehmen, wie es als Ziel im Gesetz kurz und prägnant dargestellt wird. Wir fokussieren uns auf die Förderung der mittelständischen deutschen Wirtschaft.Für diejenigen, die sich zum ersten Mal mit der Thematik auseinandersetzen oder auch für die Zuhörer auf den Rängen mag sich unser heutiger Tagesordnungspunkt vielleicht etwas abstrakt geben. Ich muss zugeben: Anfangs ist es mir ähnlich gegangen. Aber ich finde, je länger man sich mit dieser Thematik beschäftigt, desto schneller wird einem klar, welch einzigartiges und auch vom finanziellen Volumen her interessantes Förderinstru­mentarium wir hier in den Händen halten.Dabei, meine Damen und Herren, kann die KfW auf eine mittlerweile 70-jährige erfolgreiche Geschichte von gezieltem Wiederaufbau, der 1948 mit Mitteln des berühmten Marshallplanes begann, zurückblicken. Heute sind die Kredite, die Zuschüsse oder auch die Beteiligungs- und Risikofinanzierungen nach wie vor ein wichtiges und auch ein gutes Instrument, das wir über die KfW in den Händen halten, um vor allen Dingen kleinere und mittlere Unternehmen zu unterstützen. Ich versichere Ihnen: Die Stärkung unserer mittelständischen Wirtschaft ist und bleibt ein Kernanliegen der Union.Meine verehrten Damen und Herren, ich denke, es ist in diesem Plenum unstrittig: Nur mit gesunden Firmen, die mit Kreativität, mit guten Mitarbeitern, Produkten und Dienstleistungen made in Germany ihre Firma, den Handwerksbetrieb oder auch eine andere Tätigkeit in einem freien Beruf gestalten, können wir diese Kennziffern, die wir laut dem aktuellen Wirtschaftsplan aufzeigen, halten. Nur mit diesem unternehmerischen Engagement ist es möglich, dass Deutschland seine Spitzenposition in Bezug auf die wirtschaftliche Entwicklung – wir werden ja derzeit nur noch von den USA, China und Japan übertroffen – behält und ausbaut.Dabei, liebe Kollegen, bedarf es eines guten Zusammenspiels. Wir als Politiker sind natürlich gefordert, verlässliche und unbürokratische Rahmenbedingungen zu schaffen, auf Veränderungen zu reagieren, aktiv für Weiterentwicklungen zu sorgen und neue Akzente zu setzen. Ein Beispiel für das Setzen von neuen Akzenten war zum Beispiel der Antritt im Jahre 2007, bei dem mit der Neuordnung des ERP-Wirtschaftsförderungsgesetzes die Grundlage für eine erweiterte Gründer- und Förderstruktur, nämlich ein Stück weit weg von dieser klassischen Programmförderung hin zu mehr Beteiligungsfinanzierung, geschaffen wurde.Verehrte Kollegen, ich weiß auch, einzelne Haushaltsansätze im Gesetz und auch der uns vorliegende Bericht der Bundesregierung über die Inanspruchnahme der Fördermittel zeigen auf, dass die Potenziale der Wirtschaftsförderung durch die KfW-Bank in den letzten Jahren nicht voll ausgeschöpft wurden. Das ist sicherlich im Zusammenhang mit der schon langanhaltenden Niedrigzinsphase zu sehen. Es liegt also aus meiner Sicht an uns, nachzusteuern und auch die Investitionsförderung an die Bedarfe der Unternehmen und an sich verändernde Wirtschaftsrahmendaten anzupassen. Darüber haben wir in der letzten Legislatur diskutiert, und das Ergebnis ist, dass sich die KfW auf Optimierung und Veränderung konzentrieren wird. Sie wird erstens für eine qualitative Verbesserung sorgen, also die bessere Ausnutzung von schon bestehenden Investitionskapazitäten. Zweitens geht es darum, quantitative Verbesserungen zu generieren.Zusammengefasst kann man sagen: Es gilt, das Gute zu bewahren und zu verstetigen sowie Neues zu wagen.Daher – das hebt auch der aktuelle Jahreswirtschaftsbericht des Bundesministeriums für Wirtschaft und Energie hervor – haben wir Ende März vergangenen Jahres hier in diesem Plenum mit fraktionsübergreifender Zustimmung zum Anpassungsvertrag über die ERP-Förderrücklage den Weg für eine „Intensivierung des KfW-Engagements im Bereich Wagniskapital- und Beteiligungsfinanzierung“ freigemacht.Mit Blick auf die rund 781 Millionen Euro, die in diesem Jahr laut Plan aus dem Sondervermögen für die Förderung der gewerblichen Wirtschaft bereitgestellt werden, bedeutet das, dass wir unsere Anstrengungen – mein Vorredner hat es schon erwähnt – besonders im Bereich Wagniskapital- und Beteiligungsfinanzierung verstärken werden. Hierzu haben wir konkret die Gründung der KfW-Beteiligungstochter beschlossen – ich brauche das nicht zu wiederholen, Herr Kollege Mohrs –, die bis 2020 mit 200 Millionen Euro Investitionsvolumen entsprechend agieren soll. Wir hoffen, damit bestehende Kapitalangebotslücken von geschätzt derzeit 500 Millionen bis 600 Millionen Euro jährlich zu schließen und damit – und ich denke, das ist das Entscheidende – den Weg zum Aufholen in Richtung agilerer Märkte, wie sie in den USA und in Israel zu finden sind, zu ebnen. Ich finde, das ist ein starkes Signal, meine Damen und Herren.Mit der Arbeit im Unterausschuss Regionale Wirtschaftspolitik und ERP-Wirtschaftspläne haben wir genau diese Entwicklung ganz eng begleitet. Ich wünsche mir, dass wir das auch in diesem Rahmen hoffentlich weiter tun werden. Dabei muss unser Anspruch sein, dass wir, ohne privaten Akteuren Konkurrenz zu machen – abgesehen davon, dass die KfW als staatliche Bank das auch gar nicht leisten könnte –, weitere Hürden für Gründungs- und Wachstumsfinanzierungen abbauen.Bei diesem Engagement, meine Damen und Herren, geht es aber nicht nur darum, Unternehmen in der Phase der Konsolidierung und auch der Markterweiterung die Möglichkeit zu geben, das notwendige Kapital für Investitionen in Technologie, Vertriebswege und Markt­erschließung zu akquirieren, sondern es gilt gleichzeitig auch, einem Trend entgegenzusteuern. Und diesen Trend möchte ich zum Schluss meiner Rede ansprechen. Denn während wir von Beschäftigungsrekord zu Beschäftigungsrekord eilen und mehr und mehr Menschen gute Arbeit und ein sicheres Einkommen haben, zeichnet sich ab, dass die Bereitschaft, die relative Sicherheit einer Anstellung zugunsten einer risikoreicheren Selbstständigkeit aufzugeben, um eine eigene Geschäftsidee zu realisieren, im Allgemeinen sinkt. Das ist auch nachvollziehbar.Ein Blick auf die Zahlen des KfW-Gründungsmonitors zeigt, dass die Gründungsquote in Deutschland seit 2002 stark gefallen ist. Die gute Nachricht dabei ist: Der Anteil der sogenannten Chancengründer, also der Menschen, die explizite Geschäftsideen umsetzen und damit überdurchschnittlich häufig Marktneuheiten entwickeln, stellt mit 310 000 von derzeit 672 000 Gründungen die größte Gruppe dar. Hier gilt es eben, entsprechende Akzente zu setzen und KfW-Anreize für den Gang in die Selbstständigkeit zu setzen.Frau Kollegin.Ich komme zum Schluss meiner Rede auf den Ursprungsgedanken von George Marshall zurück: Planen ist gut, Machen ist besser. In diesem Sinne: Lassen Sie uns das gemeinsam anpacken, auch gerne in gemeinsamer Arbeit in einem Unterausschuss. Vor allen Dingen bringen wir heute erst einmal das Gesetz auf den Weg.Ich bedanke mich bei Ihnen ganz herzlich für Ihre Aufmerksamkeit.




	19. Sabine Leidig (DIE LINKE.) Frau Präsidentin! Werte Kolleginnen und Kollegen! Liebe Gäste! Im Unterschied zu allen anderen hier will Die Linke den Nulltarif im öffentlichen Nahverkehr, also Fahren ohne Fahrschein für alleals Antrieb für eine soziale und ökologische Verkehrswende.Dafür werben wir seit Jahren. Wir haben Fachleute zurate gezogen, Konzepte entwickelt, Beispiele gesammelt, von denen man lernen kann – in Frankreich, in Belgien, vor der Haustür hier um die Ecke im brandenburgischen Templin. Bisher sind wir hier im Bundestag dafür belächelt und manchmal auch beschimpft worden.Jetzt aber überrascht die Bundesregierung mit der Idee, den Nulltarif in einigen konkret benannten Städten einzuführen, um den Autoverkehr und die Abgasbelastung zu reduzieren.„Prima“, sage ich, aber das war, wie ich jetzt gerade höre, wohl doch nicht so ernst gemeint: kein Anschluss unter dieser Nummer, kein Konzept, kein konkreter Plan und kein wirklicher Wille, glaube ich. Das überrascht mich bei dieser Regierung nicht wirklich.Dass aber auch die Grünen abwinken, hat mich schon überrascht, und das verstehe ich überhaupt nicht. Dass ausgerechnet der hessische Verkehrsminister Tarek Al‑Wazir gegen die Einführung des Nulltarifes auftritt, verstehe ich schon gar nicht. Langsam glaube ich, dass eine ökologische Verkehrspolitik wirklich nur mit Links geht.Es ist doch super, dass mit diesem Brief jetzt die Tür einen Spalt breit offen ist, und die Leute wollen es, sehen es ein. 71 Prozent der Bevölkerung finden die Idee gut, dass der öffentliche Nahverkehr ab 2020 kostenfrei genutzt werden kann. Das geht aus einer ganz aktuellen Umfrage von Infratest dimap hervor, und das liegt übrigens wesentlich höher als der Zustimmungswert zur Großen Koalition. Jetzt kommt es darauf an, diesen Rückenwind, auch der Öffentlichkeit, zu nutzen – natürlich nicht ohne Gesamtstrategie. Hau ruck irgendetwas irgendwie zu machen, führt natürlich zum Chaos und würde als Beweis dafür dienen, dass es nicht funktioniert.Der Verband Deutscher Verkehrsunternehmen, VDV, hat natürlich recht: Die Kommunen und ÖPNV-Verbünde, die mit einem solchen Nulltarif starten, brauchen erhebliche Unterstützung. Mit Sicherheit – das ist ja genau das, was wir wollen – werden Fahrgastzuwächse von 25 bis 40 Prozent erwartet. Dafür muss Vorsorge getroffen werden; das ist doch klar. Außerdem muss man natürlich auch die Bedingungen für Fuß und Fahrrad deutlich verbessern; denn auch das sind wichtige Alternativen zum Autofahren.Mittelfristige Ziele sollen aber ein bundesweiter ÖPNV-­Nulltarif und die Halbierung des Autoverkehrs sein.So könnten wir die klimapolitischen Versprechen einhalten und die Lebensqualität in den Städten wirklich verbessern.Die Linke hat dafür einen Stufenplan vorgelegt, und wir starten mit drei sehr konkreten Paketen:Erstens. Los geht es mit Modellprojekten für kostenfreien ÖPNV in 15 Städten, und zwar in den 15 Städten, in denen die Belastung mit Abgasen besonders groß ist. Diese Städte erhalten vom Bund eine Förderung in Höhe von 90 Prozent, damit sie diesen kostenlosen ÖPNV ab 2019 unter der Überschrift „Gesundheitsschutz geht vor“ einführen können.Zweitens werden 8 Milliarden Euro jährlich als bundesweite Sofortmaßnahme in den Ausbau und für die Verbesserung der Qualität von Bus und Bahn investiert. Da gibt es ja wirklich viel zu tun, vor allem in den ländlichen Räumen; das wäre der eine Schwerpunkt. Der andere Schwerpunkt sind die Metropolregionen, in denen besonders viele Pendlerinnen und Pendler unterwegs sind. Dichtere Takte, mehr und neue Fahrzeuge, mehr Personal, Qualifizierung und bessere Bezahlung – das sind die wichtigsten Rahmenbedingungen, die man schaffen muss, damit es wirklich gelingt, mehr Menschen zu motivieren, einen guten öffentlichen Nahverkehr zu nutzen.Drittens wollen wir ein Bundesprogramm „Freie Fahrt für Kinder und Jugendliche mit Bus und Bahn“. Alle Menschen unter 18 Jahren, Schülerinnen und Schüler, Auszubildende und Menschen, die auf Hartz IV angewiesen sind, fahren ab 2019 zum Nulltarif, und der Bund übernimmt die Kosten.Wir können das bezahlen; das wissen Sie alle.Das ist eine Frage des politischen Willens und der Umverteilung. Allein die Subventionen für den Dieseltreibstoff belaufen sich auf 8 Milliarden Euro jährlich.Subventionen in Höhe von 4 Milliarden Euro gibt es für das Dienstwagenprivileg. Und: Wir fordern eine Sonderabgabe der Automobilindustrie, zweckgebunden als Abgabe zur Verbesserung der Luftqualität, fünf Jahre lang jeweils 4 Milliarden Euro. Wir können das gut rechtfertigen. Im Straßenverkehrsgesetz kann das Feilbieten nicht genehmigter Fahrzeuge mit einer Geldbuße in Höhe von 5 000 Euro geahndet werden. Wir gehen davon aus, dass mindestens 5 Millionen nicht genehmigungsfähige illegale Diesel-Pkw verkauft wurden; also eine Geldstrafe in Höhe von rund 25 Milliarden Euro, die rechtmäßig bei der Automobilindustrie eingetrieben werden müsste.Kollegin Leidig, achten Sie bitte auf Ihre Redezeit.Diese Geldbuße könnte in eine Abgabe umgewandelt werden, die klar zielgerichtet wäre und wirklich etwas für alle Menschen verbessern würde. Das ist das Ziel, das die Linke verfolgt: Mobilität für alle mit weniger Verkehr.Danke.




	20. Bijan Djir-Sarai (FDP) Vielen Dank. – Herr Präsident! Meine Damen und Herren! Es ist grundsätzlich gut, dass der Deutsche Bundestag sich mit der aktuellen Entwicklung im Nahen und Mittleren Osten beschäftigt. In nur vier Minuten die zahlreichen Konflikte in dieser Region darzustellen, ist allerdings eine außerordentlich große Herausforderung.Die Münchner Sicherheitskonferenz am vergangenen Wochenende machte einmal mehr deutlich, wie fragil und gefährlich die politische Lage im Nahen und Mittleren Osten ist. Es finden derzeit verschiedene Stellvertreterkriege in Syrien statt. Die Bedrohung Israels durch das iranische Regime wächst erschreckend. Der Jemen versinkt im Bürgerkrieg. Afghanistan ist alles andere als sicher. Die Stabilität des Libanons bröckelt. Der Konflikt zwischen Israelis und Palästinensern ist nach wie vor ungelöst. Der Machtkampf zwischen Saudi-Arabien und Iran verschärft sich. – Diese Aufzählung ließe sich beliebig weiterführen. Leider sind weitere Konflikte am Horizont erkennbar.Jahrzehntelang hat es keine deutsche bzw. europäische Strategie für den Nahen und Mittleren Osten gegeben. Vielmehr wurde gewartet und darauf vertraut, dass die USA eine Entscheidung treffen oder eine Strategie entwickeln, und dann wurde geschaut, welche Rolle im Rahmen dieser US-Parameter übernommen werden kann. Zwei Faktoren haben diese Vorgehensweise aus meiner Sicht verändert: die Flüchtlingskrise im Jahr 2015 und die Wahl des US-Präsidenten Trump.Die Flüchtlingskrise aus dem Jahr 2015 hat uns erschreckend deutlich gezeigt, dass die Probleme dieser Region nicht weit weg, irgendwo auf der Welt, sondern unmittelbar vor der Haustür Europas sind. Wenn wir wollen, dass diese Probleme nicht zu uns kommen, sondern vor Ort gelöst werden, dann müssen wir Europäer uns in dieser Region politisch mehr engagieren.Die Wahl des US-Präsidenten Trump und seine politische Ausrichtung verdeutlichen uns, dass es notwendiger denn je ist, die Gemeinsame Außen- und Sicherheitspolitik Europas zu stärken und sich speziell im Nahen und Mittleren Osten politisch zu engagieren.Gelegentlich wird in den Hauptstädten Europas oder auch der USA so diskutiert, als ob wir diejenigen seien, die die Zukunft dieser Region gestalten müssten. Ja, wir müssen eine europäisch-transatlantische Strategie für diese Region entwickeln. Was jedoch am Ende des Tages politisch vor Ort passiert, müssen die Menschen selbst bestimmen. Ob Saudis, Iraker, Iraner oder Syrer, am Ende des Tages müssen die Menschen selbst Verantwortung für ihre Zukunft übernehmen und entscheiden, in welcher Ordnung und in welchem System sie leben wollen.Meine Damen und Herren, der Nahe und Mittlere Osten leidet seit Ende der 70er-Jahre an einer Krankheit. Diese Krankheit nennt sich Fundamentalismus. Soziale Ungerechtigkeit, Armut, Bildungsferne und Perspektivlosigkeit sind Elemente, die die Gesellschaften der Region prägen und zu Radikalität führen. Erst wenn diese gesellschaftspolitischen Probleme nachhaltig angepackt werden, wird die Region eine Zukunft und eine Chance haben.Die Problemfelder im Nahen und Mittleren Osten sind uns hinreichend bekannt. Wir erleben trotzdem derzeit eine Situation, die gefährlicher ist als je zuvor. Türken gegen Kurden, Iran gegen Israel, Saudi-Arabien gegen Iran, Palästinenser gegen Israelis, USA gegen Russland, Schiiten gegen Sunniten, NATO gegen NATO – noch nie war die Region so ein gefährliches Pulverfass wie heute. Ich bedaure, dass die Türkei, die einst ein Stabilisationsfaktor in dieser Region war, kein zuverlässiger Partner mehr ist.Meine Damen und Herren, die Suche nach verbündeten Partnern im Nahen und Mittleren Osten ist außerordentlich schwierig. In diesem Zusammenhang warne ich davor, Saudi-Arabien als strategischen Partner zu betrachten.Ich bin mir über die wirtschaftliche Bedeutung Saudi-Arabiens im Klaren. Wer aber über Jahrzehnte weltweit Terrorismus finanziert, Extremismus fördert und die Menschenrechte im eigenen Land mit Füßen tritt, kann niemals ein zuverlässiger und seriöser Partner sein, auch wenn er ein Bollwerk gegen iranische Interessen in dieser Region ist.Herr Kollege, kommen Sie zum Schluss, bitte.Ein letzter Satz, Herr Präsident. – Die Situation im Nahen und Mittleren Osten ist komplex. Wir dürfen trotzdem nicht aufhören, uns in dieser Region für Personen, Gruppen oder Bewegungen einzusetzen, die für Menschenrechte und Bürgerrechte eintreten; denn diese Werte sind auch in der Region universell und unteilbar.Vielen Dank für Ihre Aufmerksamkeit.




	21. Susanne Ferschl (DIE LINKE.) Sehr geehrte Frau Präsidentin! Sehr verehrte Kolleginnen und Kollegen! Liebe Besucherinnen und Besucher auf den Tribünen! Die Fraktion Die Linke hat das Thema der sachgrundlosen Befristung in der Vergangenheit schon mehrmals auf die Tagesordnung gesetzt.Warum? Die Antwort ist ganz einfach: Weil Sie das Problem bislang nicht gelöst haben –im Übrigen auch nicht im Koalitionsvertrag. Unser Maßstab ist nicht, ob Ihnen unsere Anträge gefallen, sondern ob sich für die Mehrheit der Beschäftigten im Land endlich etwas verbessert.Die Zahlen sind bekannt. Die Zahl der Befristungen hat sich in den letzten 20 Jahren verdreifacht. Jede zweite Neueinstellung erfolgt befristet. Diese Zahlen sind konstant geblieben – trotz der wirtschaftlichen Lage, die ständig so bejubelt wird.Wenn man sich die sachgrundlosen Befristungen anschaut, stellt man fest: Das ist knapp die Hälfte der Befristungen. Das Problem ist: Das sind nicht nur Zahlen, sondern hinter diesen Zahlen verbergen sich Lebensgeschichten. Ich bin seit über 20 Jahren Betriebsrätin, und ich habe viele dieser Geschichten kennengelernt: Menschen, die einem gegenübersitzen und nicht wissen, ob sie am Ende des Monats noch eine Arbeit haben, die nicht wissen, ob sie künftig noch ihre Miete bezahlen können, die nicht wissen, ob sie sich das Studium ihrer Kinder in Zukunft noch leisten können. Sie haben offensichtlich keine Vorstellung davon, was das für diese Menschen bedeutet; denn ansonsten hätten Sie diesen Wahnsinn schon lange beendet.Für Unternehmer ist es natürlich toll. Sie wälzen damit das wirtschaftliche Risiko auf die Beschäftigten ab. Befristungen sind eine permanente Probezeit, und für die Betroffenen bedeutet das: ein Leben im Wartestand, ständig zwischen Hoffnung und Angst. Durch den Druck und die Angst, den Arbeitsplatz zu verlieren, sind die Kolleginnen und Kollegen letztlich bereit, schlechtere Arbeitsbedingungen und schlechtere Löhne zu akzeptieren. Der Anteil der Niedriglöhne ist bei den Befristeten dreimal so hoch wie bei den Unbefristeten. Das darf nicht so weitergehen.Nun zum Koalitionsvertrag. Sie, liebe Kolleginnen und Kollegen der SPD, wollten für die Abschaffung der sachgrundlosen Befristung kämpfen.Davon ist wenig übrig geblieben. Sie, liebe Kolleginnen und Kollegen von der CDU/CSU, haben auf Teufel komm raus blockiert.Jetzt geht es im Ergebnis nur noch um die Einschränkung der sachgrundlosen Befristung. Das ist viel zu wenig. Ihr Gesetz wird für die Mehrheit der Menschen in diesem Land überhaupt nichts ändern.Die Mehrheit arbeitet nämlich in Betrieben mit weniger als 75 Mitarbeitern,und da soll alles beim Alten bleiben. Das sind 96 Prozent der Betriebe. Da, wo das Gesetz greifen soll, in den größeren Betrieben, beschränken Sie es noch mal: Da dürfen weiterhin 2,5 Prozent der Verträge befristet werden. Sie schaffen damit einen Flickenteppich aus unterschiedlichen Regelungen in ein und demselben Betrieb statt einer klaren und einheitlichen Regelung. Damit spalten Sie Belegschaften von neuem.Ich frage mich: Warum beschränken, warum nicht abschaffen? Kommen Sie jetzt bitte nicht wieder mit dem Argument der Flexibilität! Die Menschen haben die Nase voll davon. Sie brauchen endlich mehr soziale Sicherheit.In der Öffentlichkeit gibt es längst eine Mehrheit für die Abschaffung der sachgrundlosen Befristung, und es müsste sich eigentlich auch hier eine finden lassen. Wir Linke fordern das seit unserer Gründung. Die SPD ist eigentlich dafür, der Arbeitnehmerflügel der CDU/CSU auch. Wir hätten heute hier die Chance, endlich eine Entscheidung im Sinne der Menschen zu treffen. Deswegen: Schaffen wir die sachgrundlose Befristung endgültig ab! Stimmen Sie unserem Antrag zu!Vielen Dank.




	22. Jürgen Hardt (CDU) Herr Präsident! Liebe Kolleginnen und Kollegen! Die Eindrücke von der Münchner Sicherheitskonferenz sind in der Tat ernüchternd. Diese Erkenntnis verbindet uns alle, die wir teilgenommen haben. Die Zahl der Probleme – auch im Nahen und Mittleren Osten – ist größer geworden. Die Ideen, wie diese Konflikte bewältigt oder zumindest eingehegt werden können, sind leider dürftig, und es ist tatsächlich mehr übereinander als miteinander gesprochen worden. Insofern ist das ein Warnsignal für uns alle, dass die nächsten zwölf Monate doch einen Wandel in wichtigen Punkten bringen müssen.Wenn wir den Nahostkonflikt, die Situation im Nahen Osten analysieren, kommen wir auf mindestens fünf Hauptkonfliktthemen, die dort eine Rolle spielen. Das ist erstens die Bedrohungssituation für Israel. Das ist zweitens der Krieg in Syrien. Das ist drittens der Krieg im Jemen. Das ist viertens die Konfrontation Saudi-Arabiens mit dem Iran, die religiös begründet ist und seit langem – seit Jahrzehnten, vielleicht seit Jahrhunderten – besteht. Es ist nicht zuletzt fünftens das iranische Atomprogramm bzw. das Aufrüstungsprogramm, das der Iran betreibt. Ich möchte hier zu dem einen oder anderen Aspekt einige Gedanken beisteuern.Zunächst zur Situation Israels. Israel ist durch die Kämpfe, durch die Auseinandersetzungen in der Region, insbesondere durch den Bürgerkrieg und den Kampf ­Assads gegen sein Volk in Syrien, konkret bedroht. Bereits seit langer Zeit nimmt der Zufluss an Waffen aus ebendiesem syrischen Kriegsgebiet für Terrorgruppen, etwa im Libanon, die gegen Israel operieren, zu. Wir haben jetzt Erkenntnisse darüber, dass tatsächlich vom Iran gestützte Kräfte von syrischem Boden aus ganz konkret an die Grenze Israels herangerückt sind. Es ist völlig verständlich, dass das Volk Israels, auch sein Ministerpräsident, darauf mit Härte, aber auch mit Emotionen reagiert. Wir haben das am Sonntagmorgen auf der Münchner Sicherheitskonferenz erlebt.Die Sicherheit Israels ist für uns in Deutschland Staatsräson. Deswegen können wir den israelischen Bürgerinnen und Bürgern nur versichern, dass wir sie bei ihrem Versuch unterstützen, die Sicherheit des Landes zu gewährleisten und Frieden in der Region zu erreichen. Das ist das erste Ziel.Ministerpräsident Netanjahu hat in seiner Rede eine kleine Passage gehabt, die mich hat aufhorchen lassen. Er hat gesagt, dass in der wachsenden Aggressivität Irans auch eine Bedrohung für andere in der Region, für andere arabische Staaten, aber auch für die Palästinenser, zu sehen ist und dass darin vielleicht eine neue Chance auf eine Verständigung Israels mit der arabischen Welt und eine Verständigung Israels mit den Palästinensern liegt. Ich wünsche mir – dazu kann ich die israelische Regierung nur ermutigen –, dass diese sich möglicherweise eröffnende Chance genutzt wird. Ich kann für die CDU/CSU-Fraktion sagen, dass wir unsere Freunde in Israel bei diesem vielleicht neuen Weg in eine friedliche Zukunft mit den Nachbarn massiv unterstützen.Der zweite Punkt ist die Situation in Syrien. Das, was wir in Syrien derzeit sehen und erleben, hat seine Ursache darin, dass Assad durch die Unterstützung Russlands vor zwei Jahren wieder Luft zum Atmen bekommen hat. Wir waren auf der Konferenz in Wien und dann bei dem Friedensprozess in Genf mit Blick auf eine Lösung für die Zukunft Syriens schon viel weiter als heute.Putin, der diesen Wiener und Genfer Friedensprozess durch seinen Außenminister hat moderieren lassen, hat sich dann entschlossen, dafür zu sorgen, dass Assad weiter die Macht im Land anstreben kann. Es sieht auch so aus, als würde er dabei vorankommen.Denjenigen, die sagen: „Es wird keine Zukunft Syriens ohne Assad geben“, sage ich umgekehrt: Angesichts der Bilder, die wir gerade in diesen Tagen sehen: Wie sollen denn die Menschen in diesem Land an eine Zukunft mit Assad glauben? Wie sollen die Menschen daran glauben, dass ein Diktator, der Fassbomben auf die Bevölkerung geworfen hat und der nach meiner Auffassung auch Chemiewaffen gegen die Bevölkerung eingesetzt hat, diesem Land eine gute Zukunft brächte? Ich glaube, das geht nicht. – Herr Präsident, es blinkt hier wegen der Redezeit, nicht, weil Sie etwas von mir wollen, ja?Herr Kollege, genau das will ich von Ihnen: dass Sie die Redezeit beachten.Ja, in Ordnung. – Ich sage einen letzten Satz zur Türkei. Ich halte den Einsatz der Türkei im Norden Syriens für extrem kritisch, auch aus völkerrechtlicher Sicht.Ich bitte die Türken, diesen Kampfeinsatz doch einzustellen und sich auf die Kontrolle der türkisch-syrischen Grenze zu konzentrieren.Ich glaube, dies wäre auch militärisch der richtige Ratschlag an den türkischen Präsidenten, den ihm vielleicht auch seine Generale bald geben werden.Herzlichen Dank.




	23. Florian Hahn (CSU) Sehr geehrter Herr Präsident! Liebe Kolleginnen und Kollegen! Europa ist eine der größten Errungenschaften der Neuzeit: mit Frieden, mit Freiheit, mit Wohlstand für über 500 Millionen Menschen auf diesem Kontinent. Wir wollen, dass das so bleibt. Aber dafür muss sich einiges ändern. Wir müssen bereit sein, Europa weiterzuentwickeln; denn die Rahmenbedingungen in unserer unmittelbaren Nachbarschaft und in der Welt haben sich verändert.Die bisherigen Triebfedern der europäischen Einigung waren die Versöhnung nach dem Krieg, die dauerhafte Befriedung dieses Kontinents und die Schaffung von Wohlstand für alle. Heute kommen Europas Herausforderungen maßgeblich auch von außen: mit neuen internationalen Krisenherden, mit dem Zerfall von Staatlichkeit bis an unsere Grenzen, mit neuen Playern in einer globalisierten, digitalisierten Welt und mit einem massiven Migrationsdruck aus dem Süden und aus dem Osten. Das verändert die Vorzeichen für das Friedens-, Freiheits- und Wohlstandsprojekt Europa.Außerdem steckt Europa in einer Vertrauenskrise. Mit Großbritannien verlässt die drittgrößte Volkswirtschaft Europas die EU. Gut jeder dritte EU-Abgeordnete ist mittlerweile nicht mehr Proeuropäer, sondern EU-Skeptiker. Für viele Menschen ist die EU kein Symbol für Zukunft, sondern für Zentralismus, überflüssige Regulierung und Umverteilung. Der Frieden wird leider als selbstverständlich hingenommen.Als überzeugte Proeuropäer dürfen wir uns damit nicht abfinden. Deshalb sagen wir ganz klar: Ein Weiter-so darf es in Europa nicht geben. Wir wollen mehr Europa in den großen Fragen – bei der Verteidigung, beim Außengrenzenschutz, in der Forschung – und weniger Europa im Kleinen. Wir wollen ein Europa, in dem die Menschen den Mehrwert der Europäischen Union spüren. Wir wollen ein Europa, in dem vieles besser, aber nicht alles gleich ist. Das ist unser Ziel.Der mehrjährige Finanzrahmen 2021 bis 2027, mit dem sich die Staats- und Regierungschefs auf der kommenden Tagung beschäftigen werden, ist ein wichtiger und richtiger Schritt auf dem Weg, diese Ziele zu erreichen. Für den EU-Haushalt muss das Gleiche gelten, was wir in Deutschland seit 2005, seit CDU und CSU die Bundesregierung führen, praktizieren: Prioritäten setzen, Zukunftsaufgaben angehen, maßhalten und Ausgaben begrenzen. Eines ist klar: Wenn Europa kleiner wird, sollte das Budget nicht größer werden.Die EU kann maßhalten. Das ist das Signal, das die Menschen von Europa erwarten.In einem zweiten Schritt müssen wir dann im EU-Haushalt die richtigen Prioritäten setzen und die Probleme adressieren, die die Menschen wirklich bewegen.Wenn wir die EU-Bürger fragen, was die größten Probleme in Europa sind, dann sagen sie mit weitem Abstand „Einwanderung“ – 39 Prozent – und „Terrorismus“ – 38 Prozent. Auf den Plätzen drei und vier folgen mit 17 Prozent „Wirtschaftliche Lage“ und mit 16 Prozent „Öffentliche Verschuldung in den eigenen Mitgliedstaaten“. Das sind die größten Sorgen der Menschen. Diese Sorgen müssen wir ernst nehmen, die Agenda in der EU mitbestimmen lassen und im Finanzrahmen adressieren.Das heißt, wir müssen einen klaren Fokus auf öffentliche Sicherheit, Terrorismusbekämpfung, Verteidigung, die Stärkung der Wettbewerbsfähigkeit und solides Haushalten legen. So machen wir die Europäische Union wieder stark; so sorgen wir dafür, dass die EU wieder an Akzeptanz bei den Menschen vor Ort gewinnt. Das war übrigens auch unser Ziel in den Koalitionsverhandlungen, und das haben wir auch erreicht.Unsere Europapolitik steht dabei auf drei Säulen: Stabilität, Sicherheit und Subsidiarität.Erste Säule: Stabilität. Der von SPD, CDU und CSU verhandelte Koalitionsvertrag enthält ein klares Bekenntnis zum soliden Haushalten und zum Stabilitäts- und Wachstumspakt. Unsere Haltung ist dabei sehr klar: Wir wollen eine Stabilitätsunion und keine Schuldenunion. Wir wollen Stärke durch Eigenverantwortung und keine Abhängigkeiten von Umverteilungen. Deshalb gibt es in diesem Koalitionsvertrag keine Euro-Bonds, keine Mechanismen für einen europäischen Finanzausgleich, keinen europäischen Finanzminister und keine europäische Wirtschaftsregierung, und das ist auch richtig so.Zweite Säule: Sicherheit. Das Sicherheitsbedürfnis in Deutschland und in ganz Europa war noch nie so groß wie heute. Deshalb brauchen wir eine aktive Sicherheitsoffensive für die Europäische Union. Das hat die Bundeskanzlerin in ihrer Regierungserklärung auch zu Recht deutlich gemacht.Deshalb müssen wir auch unsere Wehrfähigkeit verbessern und den Weg zu einer europäischen Verteidigungsunion konsequent weitergehen. Die jetzt vereinbarte Ständige Strukturierte Zusammenarbeit – ­PESCO – war überfällig. Die europäische Verteidigung ergänzt und stärkt die NATO, sie macht Europa strategisch autonomer und schafft mehr Sicherheit.Die PESCO-Teilnehmer erklären regelmäßig, ihre Verteidigungsbudgets real erhöhen zu wollen. Wir als CSU und die Union insgesamt halten das für richtig und unbedingt erforderlich. Das 2-Prozent-Ziel ist dabei der richtige Orientierungspunkt.Sehr geehrter Herr Lindner, wenn wir über die letzten zwölf Jahre sprechen, dann sollten wir nicht vergessen, dass sich erstens die außenpolitischen Rahmenbedingungen in diesen zwölf Jahren massiv verändert haben und dass Sie zweitens vier Jahre von diesen zwölf Jahren mit an der Regierung waren.Ich glaube, wir sind uns einig, dass wir jetzt wissen, dass eine gute Ausrüstung und eine europäische Verteidigung notwendig sind und dass das nur mit entsprechenden echten Investitionen geht. Die Trendwenden dahin haben wir in den letzten vier Jahren eingeleitet, und wir müssen jetzt Gas geben und diese konsequent verfolgen.In der EU muss schließlich gelten: Ohne sichere Grenzen nach außen gibt es keine offenen Grenzen nach innen. – Bis der Schutz der Außengrenzen gewährleistet ist, müssen Binnengrenzkontrollen weiterhin möglich sein. Wir müssen Frontex weiter stärken und zu einer echten europäischen Grenzschutzpolizei ausbauen. Auch das hat die Bundeskanzlerin dankenswerterweise ganz besonders deutlich gemacht.Auch unsere Integrations- und Aufnahmefähigkeit hat eine Grenze, und diese Grenze müssen wir auch politisch abbilden. Es ist nicht akzeptabel, dass Deutschland mehr Flüchtlinge aufnimmt als alle anderen 27 Mitgliedstaaten zusammen. Eine Reform der Dublin-Verordnung darf deshalb nicht dazu führen, dass sich diese ungleichen Lastenverteilungen noch verschärfen.Verteilung allein ist dabei keine Lösung für den Flüchtlingsstrom. Wir brauchen harmonisierte Standards bei der Versorgung und Unterbringung von Asylbewerbern, einen wirksamen Schutz der Außengrenzen und effiziente Rückführungen direkt vor Ort. Der Grundsatz muss sein: Wer wirklich bedroht ist, dem wird in Europa geholfen; wer aber aus rein wirtschaftlichen Gründen kommt oder wessen Fluchtgründe nachhaltig entfallen sind, der muss auch wieder zurück.Dritte Säule: Subsidiarität. Wer in Europa unsere Zukunft sieht, muss die Menschen mitnehmen und darf deren Bereitschaft zur Solidarität nicht überdehnen. Die EU als reines Projekt der mobilen, kosmopoliten Eliten ist nicht mehrheitsfähig. Europa muss mit den Menschen auf der Straße und nicht nur mit den Menschen in den Konferenzräumen geeint werden. Die Menschen sind stolz, Bayern, Franken, Schwaben, Hanseaten, Deutsche, Österreicher, Franzosen, Spanier zu sein. Das muss respektiert und wertgeschätzt werden. Dem europäischen Zentralstaat erteilen wir deshalb eine klare Absage. Aber eine klare Absage erteilen wir auch denen, die glauben, dass die Zukunft darin besteht, Europa rückabzuwickeln. Das führt ins Verderben, meine Damen und Herren. Das dürfen wir nicht zulassen.Wir wollen ein Europa, das sich auf die Kernaufgaben konzentriert, die wir gemeinsam besser lösen können als jeder nationale Mitgliedstaat alleine. Wir wollen ein Europa der Bürger, der Nationen, der Regionen und der Vielfalt. Wir wollen ein schlankes Europa der Stärke. Dafür kämpfen wir heute und in den nächsten Jahren.Herzlichen Dank.




	24. Mario Mieruch (LKR) Sehr geehrter Herr Präsident! Sehr geehrte Damen und Herren! Liebe Gäste! Es wäre super, wenn alle Unternehmer die Sicherheit hätten, über viele Jahre hinweg verlässlich planen zu können, und es wäre genauso super, wenn wir auch jedem Arbeitnehmer eine entsprechende Planungssicherheit angedeihen lassen könnten. Leider leben wir aber nicht in der Planwirtschaft.Wir müssen uns auf die sich wandelnden Anforderungen des Arbeitsmarkes im Rahmen der Globalisierung einstellen. Die Flexibilität ist angesprochen worden. Ich will aber nicht all das wiederholen, was in dieser doch schon fortgeschrittenen Debatte alles erwähnt wurde.Was wird passieren, wenn wir die sachgrundlose Befristung einfach abschaffen? Ein paar Beispiele sind schon genannt worden: Es wird mehr Zeitarbeitsverträge geben. Die Produktion wird ins Ausland verlagert. Es wird vielleicht auch mehr Scheinselbstständige geben, um diese Risiken abzufedern. All das müssen wir berücksichtigen, wenn wir die Befristungsthematik behandeln wollen. Natürlich müssen wir vernünftige Regelungen schaffen, die Missbrauch verhindern und ihn auch wirkungsvoll sanktionieren.Was in der gesamten Debatte bisher komplett gefehlt hat, womit wir aber nachhaltig etwas verändern können, ist die Qualifikation unserer Arbeitnehmer und Arbeitnehmerinnen. Wir müssen die Anforderungen und vor allen Dingen auch die Leistungsfähigkeit unseres Schulsystems wieder erhöhen. Wir müssen unsere Bürger einfach fitter machen, damit sie mit ihren Qualifikationen den sich ändernden Bedingungen im globalen Wettbewerb besser entgegentreten können. Verantwortungsvolle Unternehmer machen das von ganz allein. Sie versuchen, ihre qualifizierten Mitarbeiter möglichst lange zu halten, und lagern nicht einfach alles in Billigjobs aus. Auf diese Weise haben wir eine echte Chance, das Problem an der Wurzel zu packen und nicht immer nur über Symptome zu reden.Vielen Dank.




	25. Falko Mohrs (SPD) Meine sehr geehrte Frau Präsidentin! Meine Damen und Herren! Wir kommen jetzt in der Tat zum eher nüchternen Teil des späten Mittags, deswegen aber nicht zu einem weniger wichtigen.Leistungsvermögen, Qualität, Wettbewerb, Stärke und Innovationen sind Erfolgsfaktoren für die deutsche Wirtschaft und natürlich auch für den Mittelstand und für Existenzgründer.In 2017 sind in dem Bereich Venture-Capital Investitionen in Höhe von 4,3 Milliarden Euro getätigt worden. Darüber kann man sich freuen. Das ist eine Verdoppelung gegenüber 2016. Wir müssen aber eben auch zur Kenntnis nehmen, dass mehr als die Hälfte dieser Investitionen aus dem Ausland getätigt wurde. Im Vergleich zum größten Markt in diesem Bereich, den USA, haben wir eine Finanzierungslücke von 60 Milliarden Euro. Es ist allen völlig klar, dass wir diese Lücke natürlich nicht durch staatliche Förderprogramme schließen können, sondern dass es darum gehen muss, das Kapital der Industrie und des Mittelstandes zu heben.Das ERP-Sondervermögen, das Sondervermögen aus dem European Recovery Program, dessen Wirtschaftsplangesetz für 2018 wir heute beraten, tut genau das: Es fördert Unternehmen in dem volkswirtschaftlich wichtigen und bedeutsamen Bereich der Gründungen und der Innovationen und ist somit ein wichtiges Element für die Schaffung und den Erhalt von Arbeitsplätzen und unserer internationalen Wettbewerbsfähigkeit.Im Wirtschaftsplan 2018 sind 790 Millionen Euro dafür vorgesehen. Damit erhalten insbesondere mittelständische Unternehmen, Existenzgründer und freie Berufe eine zinsgünstige Finanzierung über die KfW in einem Volumen von ungefähr 6,8 Milliarden Euro.Neu ist in diesem Jahr, dass eine KfW-Tochtergesellschaft mit dem besonderen Ziel gegründet werden soll, Venture-Capital und Venture-Debts zur Verfügung zu stellen, um mit dieser direkten Kapitalbeteiligung Existenzgründern zu helfen.Es geht hier also um die Finanzierung von Gründungs- und Wachstumsphasen dieser jungen Unternehmen. Mit den 120 Millionen Euro, die wir in 2018 dafür zur Verfügung stellen, erwarten wir dank eines Hebels von fünf ungefähr 600 Millionen Euro für den Venture-Capital-Markt. Von der Zeitplanung her sehen wir vor, dass dieses Tochterunternehmen der KfW im ersten Quartal gegründet wird und dann im dritten Quartal, weil es leider noch den einen oder anderen Abstimmungsbedarf, auch mit der Europäischen Union, gibt, seine Geschäftstätigkeit aufnimmt.Bei all dem Positiven muss uns aber auch klar sein, dass wir allein dadurch die Finanzierungslücke nicht schließen können, sondern weiterdenken müssen.In Deutschland und in Europa ist es Kapitalsammelstellen, Versicherungen oder Pensionsfonds untersagt, im Bereich des Wagniskapitals zu investieren. Wir können jetzt natürlich anfangen, den europäischen Rechtsrahmen dafür zu verändern. Das dauert aber sehr lange und ist natürlich politisch sehr schwierig. Deswegen haben wir in unserem Koalitionsvertrag einen anderen Weg vorgesehen, und zwar mit einem Dachfonds, der die Gewinne, aber auch die Verluste an dieser Stelle kappt, um es so auch Versicherungen und Pensionsfonds zu ermöglichen, in diesen Digitalfonds zu investieren. Wenn wir einmal zu unserem nördlichen Nachbarn Dänemark schauen, dann sehen wir, dass sich das Ökosystem der Start-ups dort massiv verbessert hat.Meine Damen und Herren, festzuhalten ist also, dass wir in Deutschland mehr tun können und müssen, um vielversprechenden Gründern und jungen Unternehmen im Wachstum finanziellen Spielraum zu ermöglichen; denn so sichern wir unsere Wirtschaftskraft und unsere Arbeitsplätze in der globalen Welt und erwirtschaften wir das Geld, das wir für unseren Sozialstaat benötigen. Es ist der Mittelstand von morgen, den es hier zu fördern gilt.Herzlichen Dank.




	26. Friedrich Straetmanns (DIE LINKE.) Sehr geehrter Herr Präsident! Liebe Kolleginnen und Kollegen! Sehr geehrte Zuhörerinnen und Zuhörer! Ich war 26 Jahre Sozialrichter und weiß: Zur Demokratie gehört ein reger Austausch mit den Bürgerinnen und Bürgern dieses Landes wie auch mit außerparlamentarischen Initiativen und Organisationen. So betrachtet ist vieles, was als Lobbyismus gefasst wird, nicht unbedingt schlecht. Auch ich rede sehr bewusst zum Beispiel mit Gewerkschaften, um die Interessen von arbeitslosen und arbeitenden Menschen hinreichend zu berücksichtigen.Viele Bürgerinnen und Bürger in meinem Wahlkreis Bielefeld verstehen jedoch nicht mehr, wie Gesetze letztlich entstehen und wer wie Einfluss nimmt. Das muss uns als Abgeordnete doch Sorgen machen.Lobbyismus ist ein milliardenschweres Geschäft. Mindestens 40 Milliarden Dollar gaben die 30 im DAX gelisteten Konzerne 2017 allein in den USA für Lobbyarbeit aus – so jedenfalls wurde es bei abgeordnetenwatch.de recherchiert. Woher stammen diese Zahlen? Sie stammen aus dem in den USA verpflichtenden Lobbyregister. Ein solches Lobbyregister gibt es hier in Deutschland nicht. Wie viel Geld die DAX-Konzerne in Berlin in Lobbytätigkeit stecken, ist nirgends aufgeführt.Wir haben ein Verbandsregister, in dem nur rudimentäre Daten stehen. Daten zum Umfang der Tätigkeit und zur Einflussnahme auf die Politik lassen sich daraus nicht entnehmen. Das ist im Sinne der Transparenz nicht länger hinnehmbar.Genau deswegen haben wir uns entschlossen, nicht die Bundesregierung zur Vorlage eines Gesetzentwurfes aufzufordern, sondern selber einen – dringend notwendigen – Gesetzentwurf vorzulegen. Denn: Wenn man sich den immer weiter ausbreitenden Glaubwürdigkeitsverlust der parlamentarischen Demokratie in diesem Land in Erinnerung ruft, stellt man fest, dass es notwendig ist, ein Lobbyregistergesetz zu erlassen. Das sollten wir hier beraten.Ich möchte zum Beleg für meine These auf das Ergebnis der Bundestagswahl 2017 verweisen, die für mich einen Vertrauensverlust dokumentiert.Selbst die Deutsche Gesellschaft für Politikberatung – quasi eine Lobbyorganisation für den Lobbyismus – fordert ein Lobbyregister. Die SPD verspricht seit Jahren, ein Gesetz vorzulegen. In den Jamaika-Verhandlungen hatte man sich auf ein Lobbyregister geeinigt. Es ist jetzt an der Zeit, ein solches vorzulegen. Daran werden die Menschen Ihre Glaubwürdigkeit messen.Belege für die Notwendigkeit lassen sich schnell finden.Meine Damen und Herren, der Abgasskandal in der Automobilindustrie ist noch nicht lange her. Vergangenes Jahr wurde der Abschlussbericht des Untersuchungsausschusses hier im Bundestag vorgelegt. Wir als Linke hatten ein Sondervotum erstellt und Aufklärung gefordert. Wir wollten wissen, wie die unheilvolle Verstrickung von Politik und Konzerninteressen funktioniert.Damals hatten sich die Parteien der Regierung auf eine Position geeinigt, die sie auf EU-Ebene vertreten sollten. Diese Position passte dem Verband der Automobilindustrie aber nicht. Daher kontaktierte dessen Vertreter politisch Verantwortliche per E-Mail und Telefon, um die Position in ihrem Sinne zu verändern. Bei Herrn Dobrindt fanden sie ein ganz großes offenes Ohr. Schnell wurde die Position der Bundesregierung auf die Schnelle auf Linie des Verbandes gebracht, freilich ohne dies mit den anderen Verantwortlichen im Kabinett abzustimmen. Zu solchen Freunden kann man der Automobilindustrie nur gratulieren.Alle Vorurteile über Gemauschel in der Politik werden hierdurch bestätigt. Nur Transparenz schafft mehr Vertrauen in die Demokratie. Unser Gesetzentwurf ist ein Beitrag hierzu.Meine Damen und Herren, wir wissen: Es gibt Bedenken, dass auch Grundrechte durch ein solches Gesetz berührt sind, wie zum Beispiel die Berufsfreiheit. Sollten sich diese auf Einzelheiten, wie etwa die Höhe von Bußgeldern bei Verstößen, beziehen, so lassen sich diese in der weiteren Debatte hoffentlich auflösen. Die Linke ist hier gesprächsbereit. Arbeiten Sie im Ausschuss an unserem Gesetzentwurf mit, und stimmen Sie ihm dann schlussendlich zu.Vielen Dank.




	27. Christian Petry (SPD) Herr Präsident! Meine sehr verehrten Damen und Herren! Europa – ein wichtiges Thema. Wir behandeln es fast jede Sitzungswoche, und das ist auch angemessen.Frau Weidel – sie ist nicht da; sie ist, glaube ich, nach Diktat verreist; na ja –, was angeblich hochbezahlte EU-Bürokraten angeht: 33 000 Beschäftigte hat die EU-Verwaltung; 38 000 Beschäftigte hat die Stadtverwaltung München. Mit diesem Vergleich möchte ich nur einmal andeuten, was diese 33 000 Beschäftigten alles leisten. Das sollte man schon sehen.Herr Lindner, ich habe eins nicht verstanden. Sie haben ausgeführt: Die sozialen Standards führen zur Euroskepsis.Ich bin da überhaupt nicht Ihrer Meinung. Ich habe eher das Gefühl, dass es eine verfehlte liberale Finanzpolitik war, die in die Krise und auch zur Euroskepsis geführt hat.Insoweit bin ich sehr dafür, dass wir die sozialen Standards hier mit im Blick haben.Herr Bartsch, Sie haben gesagt, das sei alles zu unkonkret. – Da gebe ich Ihnen recht. Aber Ihre Rede war auch sehr unkonkret. Ich unterschreibe dreimal, dass im Kampf gegen die Jugendarbeitslosigkeit entschlossen gehandelt werden muss. Ein Satz ihrerseits dazu, wie man das machen soll, täte uns gut. Darüber können wir natürlich auch in Zukunft sprechen.Frau Göring-Eckardt, herzlichen Dank für das Lob an Martin Schulz. Auch das können wir dreimal unterschreiben. Es ist wichtig, dass Europa hier im Mittelpunkt steht und dass dies entsprechend behandelt wird.Der mehrjährige Finanzrahmen ist Hauptgegenstand der Debatte. Die Vorteile der Integration in den Binnenmarkt für Deutschland sind klar: Wir sind einer der größten Nutznießer von Europa. Unser Wohlstand basiert auf Europa. Wenn wir den Wohlstandsindex 2014 zugrunde legen, erkennen wir: Er ist in Deutschland um knapp 100 Punkte gestiegen ist, im europäischen Schnitt aber nur um 30 Punkte. Das heißt, wir leben mit unserem Wohlstand von Europa. Wir brauchen Europa. Europa ist ein Garant, nicht nur für Frieden, sondern auch für Beschäftigung und Wohlstand in unserem Land. Ich denke, es lohnt sich, dafür zu kämpfen, dass dies erhalten bleibt. Da brauchen wir nicht Parolen von der rechten Seite auf den Leim zu gehen.Was soll Europa mehr machen? Herr Oettinger hat es genannt: Die Terrorismusbekämpfung soll zusätzlich und stärker stattfinden. Die innere Sicherheit soll gestärkt werden. Der Grenzschutz soll ausgebaut werden. Investitionen in die gemeinsame Verteidigung soll es geben. Die Forschung soll ausgebaut werden. Das digitale Zeitalter – eine europäische Aufgabe. Das alles bedeutet zusätzliche Ausgaben, die wir finanzieren müssen. Natürlich ist auch die Stärkung der sozialen Säule zu nennen – ein ganz zentrales Anliegen der Sozialdemokratie. Das alles sind zusätzliche Aufgaben, die finanziert werden müssen. Dafür treten wir ein.Für das Ganze gibt es einen Haushalt von 160 Milliarden Euro – zweimal so groß wie der von Nordrhein-Westfalen, zweieinhalbmal so groß wie der von Bayern. Das ist die Größenordnung. Damit soll das gestemmt werden. Es fallen durch den Austritt der Briten 14 Milliarden Euro weg. Trotzdem ist das durchaus zu finanzieren.Im Zusammenhang mit dem mehrjährigen Finanzrahmen sind Vorschläge gemacht worden. Es ist von der Körperschaftsteuer und der Finanztransaktionsteuer als Möglichkeiten die Rede. Die Plastiksteuer ist dabei; die Eigenmittel auf Basis der Mehrwertsteuer und des Bruttonationaleinkommens werden genannt. Diese Instrumentarien dafür, wie wir das lösen können, werden wir diskutieren müssen. Natürlich ist auch der Beitrag zu nennen, den wir leisten. Wir sind bereit, da mehr zu tun.Das sind wichtige Dinge; denn wir wollen, dass etwa 150 Milliarden Euro mehr für die Sicherung der Außengrenzen zur Verfügung stehen. Es sind 100 000 Personen notwendig, wenn wir es so machen wollen wie Kanada oder die Vereinigten Staaten. Erasmus+ – für die Jugend ganz wichtig –: plus 30 Milliarden Euro. Digitale Wirtschaft: 70 Milliarden Euro zusätzlich. Horizon 2020: bis zu 160 Milliarden Euro zusätzlich. Das muss finanziert werden. Der Kohäsionsfonds macht 34 Prozent, die Landwirtschaft 40 Prozent der Ausgaben aus. Hierfür werden 400 Milliarden Euro verausgabt. Dies wollen wir erhalten. Das geht nur, wenn wir die Einnahmeseite auf Vordermann bringen. Dazu sind wir bereit. Das muss in Europa solidarisch passieren; dies muss in Europa gleichmäßig passieren. Das ist ein guter Ansatz. Vorschläge hierzu sind gemacht worden. Die Sozialdemokratie ist dazu bereit.Ein Letztes; ich komme zum Schluss. Ein weiterer Schwerpunkt: die institutionellen Fragen. Das Thema Spitzenkandidat ist genannt worden. Ich bin froh, dass wir Einigkeit darüber haben. Ich hätte mich noch mehr gefreut, wenn die Konservativen in Europa im Parlament den Weg frei gemacht hätten für eine Wahlrechtsreform, für transeuropäische Listen, für das, was tatsächlich eine weitere Stärkung der Demokratie in Europa und des Europäischen Parlaments ausgemacht hätte. Aber das ist nicht vorbei; die Diskussion wird weitergehen. Wir werden uns dafür einsetzen, und wir werden mit Leidenschaft für ein gutes Europa kämpfen.In diesem Sinne: Glück auf!




	28. Thomas L. Kemmerich (FDP) Frau Präsidentin! Liebe Kollegen! Liebe Kolleginnen! Liebe Besucher auf der Tribüne! Liebe Zuschauer, die die Debatte über das Internet verfolgen! Selbstverständlich werden wir zustimmen. Schließlich wird hier ein Förderprogramm für den Mittelstand aufgelegt, das in erster Linie dazu dient, Wachstum, private Kapitalgesellschaften, Innovation und Export zu fördern. Es ist keine Regionalförderung. So weit zu meinem Vorredner. Die Sache mit den Eiern lasse ich einmal weg.Ich halte es allerdings für zu wenig, wenn wir dies als Erfolg für den Mittelstand feiern und damit innehalten. Blicken wir zunächst auf die Instrumente des ERP-Programms. Ist es denn noch zeitgemäß, oder handelt es sich nur um eine immerwährende Fortführung der Maßnahmen aus den Vorjahren? Wir wissen schon heute, dass die Fördermittel insgesamt nicht abgerufen werden.Welche Gründe wird das wohl haben? Im Umfeld der Niedrigzinspolitik ist es für viele Unternehmen trotz der Tatsache, dass Unternehmenskredite noch relativ teuer sind, wenig attraktiv, ein dermaßen bürokratisch überladenes Instrument in Anspruch zu nehmen, das von der Programmatik her unübersichtlich ist und deshalb in Durchführung und Umsetzung sehr schwierig ist.Auch Banker lehnen es ab. Man bedient sich lieber herkömmlicher Instrumente, der Annuitätenkredite der Sparkassen und Volksbanken, die Kapital im Überfluss haben und händeringend nach Anlagemöglichkeiten suchen. Natürlich dürfen wir nicht vergessen, dass die Nettoinvestitionen des deutschen Mittelstandes seit Jahren aufgrund konjunktureller Einflüsse rückläufig sind. Der Mittelstand denkt nicht wie der Wirtschaftsbericht von 2018 bis 2019, sondern in längeren Perioden. Wenn man einmal auf einen Horizont von zehn Jahren schaut, erkennt man durchaus dunkle Wolken in Form von Überhitzungstendenzen, in Form von Bürokratie – das wurde genannt –, aber auch in Form eines Fachkräftemangels.Kurz etwas zum Thema Wagnisfinanzierung. In unseren Augen ist es besser, hier einen privaten Rechtsrahmen zu schaffen, um Wettbewerbsgleichheit mit den angelsächsischen Räumen und mit anderen Strukturen, die viel erfolgreicher sind, herzustellen. Ich denke, wir sind viel zu spät und Deutschland ist das Schlusslicht. Ein Mytaxi könnte Weltmarktführer sein; es ist aber Uber. Das sollte uns zu denken geben. Wir sollten mehr der Frage nachgehen, was wir in Deutschland, in Europa wirklich dafür tun können, damit sich solche Unternehmen hier ansiedeln.Ich möchte noch kurz auf den KMU-Begriff eingehen. Er ist seit 2005 unverändert. Ich war die Tage bei einem mittelständischen Unternehmen. Es hat jetzt 350 Mitarbeiter und einen Umsatz von 80 Millionen Euro. Dort sagt man: Wir gehörten zu den KMU; ab heute befinden wir uns auf Augenhöhe mit BASF und anderen Global Playern. Auch da sollten wir uns fragen: Tun wir da dem unternehmerisch handelnden Mittelstand nicht einen Bärendienst? Wir sollten alles als Mittelstand definieren, wo Eigentum und Verantwortung in einer Hand liegen. Ich denke, das sollte das Hauptkriterium sein und nicht nur schiere Größe.Schauen wir zurück auf das, was in den letzten Jahren für den Mittelstand passiert ist. Ist das bürokratische Chaos entrümpelt worden? Jeder trägt „Bürokratieabbau“ wie ein Mantra vor sich her; passieren tut nichts. Wir haben Mindestlohndokumentationspflichten. Wir haben das Lohnentgeldgleichheitsgesetz, das Teilzeit- und Befristungsgesetz, die Arbeitsstättenverordnung, Statistiken, die zum Himmel schreien. Ich könnte das Ganze endlos ausführen. Meine Redezeit gibt es nicht her.Wir fordern die Bundesregierung und auch die noch zu bildende GroKo dazu auf, endlich zuzupacken und etwas für den Mittelstand zu tun, und zwar nicht wie bisher: Entbürokratisierung – Fehlanzeige! Digitalisierungsstrategie für den Mittelstand – Fehlanzeige! Ein entsprechender Rechtsrahmen für die Mittelstandsfinanzierung – Fehlanzeige! Einwanderungsgesetz, das dem Mittelstand nützt – Fehlanzeige! Hier besteht ein großer Bedarf, etwas zu tun. Einen weiteren großen Bedarf sehe ich darin, etwas gegen die insgesamt schwache Akzeptanz mittelständischer Unternehmen in dieser Gesellschaft zu tun. Diese schwache Akzeptanz merkt man an Ihren Äußerungen und leider auch an dem nicht so prall gefüllten Saal.Wir mittelständische Unternehmer tun eben nicht, was immer behauptet wird: dass wir das Finanzamt hintergehen, dass wir Mitarbeiter ausbeuten oder Kunden hintergehen. Nein, weit gefehlt. Ihr Misstrauen sollten Sie endlich überwinden. Mittelständler sind vielmehr Leute, die für sich, ihr Umfeld, ihre Familien und vieles andere eine große Verantwortung tragen und sie täglich leben.Wir sollten in Deutschland endlich dazu übergehen, dem Mittelstand eher etwas zuzutrauen, als ihm zu misstrauen.Wir sollten mehr mit Vertrauen als mit Verboten arbeiten. Auch das ist eine Aufgabe der Gesellschaft. Es ist nicht selbstverständlich, was im Wirtschaftsbericht niedergeschrieben ist. Es ist wichtig, dass der Mittelstand geachtet wird. Der Aufschwung ist nicht selbsttragend und keine Selbstverständlichkeit. Der Mittelstand finanziert all die Programme, die wir hier in diesem Parlament gern auflegen. Deshalb: Mehr Respekt und Hochachtung vor dem Mittelstand.Vielen Dank.




	29. Gabriele Hiller-Ohm (SPD) Herr Präsident! Meine Damen und Herren! Liebe Kolleginnen und Kollegen! Eines ist klar: Die SPD will sachgrundlos befristete Arbeitsverträge abschaffen. Sie sind unfair und gehören deshalb in den Papierkorb.Das haben Sie, liebe Kolleginnen und Kollegen der Linken, richtig in Ihrem Antrag zitiert. Ja, SPD, Linke und Grüne hätten rein rechnerisch eine Mehrheit hier im Parlament für die Abschaffung der sachgrundlosen Befristung gehabt. Tatsache ist aber auch, dass CDU, CSU und SPD sich vor vier Jahren in einem Koalitionsvertrag auf bestimmte Ziele und eine gemeinsame Regierung verständigt haben. Da auch Sie, also die Linkspartei, schon Koalitionen mit anderen Parteien eingegangen sind, wissen Sie sehr genau, dass man niemals und schon gar nicht als kleinerer Partner alles zu 100 Prozent durchsetzen kann. Auch die Linkspartei geht in Regierungsverantwortung natürlich Kompromisse ein. Hören Sie also auf, den Menschen weismachen zu wollen, dass diese Regeln hier im Bundestag bei der sachgrundlosen Befristung plötzlich keine Gültigkeit mehr hätten!Leider haben Sie in Ihrem Antrag mit keinem Wort erwähnt, dass es inzwischen den Entwurf eines Koalitionsvertrages von CDU, CSU und SPD gibt, der die sachgrundlosen Befristungen aufgreift. Wir befinden uns also keineswegs im luftleeren Raum. Jetzt könnte Bewegung in die Sache kommen – und das ganz ohne die unzähligen Anträge der Linken.In einer Großen Koalition würden wir die sachgrundlosen Befristungen zwar nicht gänzlich abschaffen – dazu waren CDU und CSU nicht bereit –, aber es ist der SPD gelungen, endlich deren Blockadehaltung aufzubrechen.Meine Damen und Herren, mit der gefundenen Einigung würden diese Arbeitsverhältnisse deutlich eingeschränkt werden. Heute gibt es etwa 1,3 Millionen Menschen mit solchen Verträgen. In größeren Betrieben, in Betrieben mit mehr als 75 Beschäftigten, werden befristete Verträge besonders häufig eingesetzt. 830 000 Arbeitnehmerinnen und Arbeitnehmer sind davon betroffen. Das sind im Schnitt etwas über 5 Prozent der Belegschaft. Diese Zahl wollen wir halbieren.Jede sachgrundlos befristete Neueinstellung, die die 2,5‑Prozent-Grenze übersteigt, würde dann automatisch als unbefristet gelten. Das, meine Damen und Herren, wäre ein scharfes Schwert.Von der Vereinbarung im Koalitionsvertrag könnten 400 000 Menschen profitieren. Sie hätten die Chance auf ein unbefristetes Arbeitsverhältnis.400 000 Beschäftigte – das wäre fast zweimal die Einwohnerzahl meiner Heimatstadt Lübeck und ein Schritt in die richtige Richtung.Außerdem wollen wir die Dauer der sachgrundlosen Befristungen von 24 auf 18 Monate verkürzen und dadurch zusätzlich einschränken sowie Missbrauch mit solchen Verträgen in der Leiharbeit verhindern. Leiharbeiter dürfen längstens 18 Monate in einem Betrieb arbeiten. Leider kommt es immer wieder vor, dass heute eigentliche Leiharbeit mit sachgrundlos befristeten Verträgen auf 24 Monate verlängert wird. Das wäre dann nicht mehr möglich.Meine Damen und Herren, Sie sehen, die SPD hat ein konkretes Ergebnis ausgehandelt.Das war bei Jamaika, also bei den Verhandlungen von Grünen, FDP und CDU/CSU übrigens nicht der Fall.Dort hieß es zum Thema Befristungen lediglich – ich zitiere –:Wir wollen befristete Arbeitsverträge mit und ohne Sachgrund erhalten, aber ihren Missbrauch bekämpfen.Wir, liebe Kolleginnen und Kollegen der Grünen, haben uns nicht mit Wischiwaschi von der CDU/CSU abfrühstücken lassen. Nein, wir verringern die sachgrundlosen Befristungen ganz konkret um 400 000. Das ist ein Verhandlungserfolg der SPD, der sich sehen lassen kann und den Menschen vor Ort direkt helfen wird.Frau Hiller-Ohm, gestatten Sie eine Zwischenfrage des Kollegen Pascal Meiser von den Linken?Oh, ich bin schon am Ende meiner Redezeit, Herr Präsident.Es ist aber immer noch möglich, diese zu beantworten. – Es hat sich erledigt. Danke.




	30. Volker Ullrich (CSU) Herr Präsident! Meine sehr verehrten Damen und Herren! Diese Debatte über ein Lobbyregister beginnt eigentlich mit einer bemerkenswerten Pointe: Die Fraktion Die Linke bringt einen Gesetzentwurf zum Thema Lobbyismus ein und lässt sich diesen Gesetzentwurf selbst von einer Lobbyorganisation schreiben. Sie schreiben in dem Gesetzentwurf, dass Sie die Vorlage von LobbyControl aufgegriffen haben. Tatsache ist aber, dass Sie sie wortwörtlich abgeschrieben haben. Sie hätten den Gesetzentwurf doch wenigstens ein Stück weit selber erarbeiten und ihn einfach von einer Lobbyorganisation abschreiben lassen sollen.Frau Kollegin Haßelmann, Sie beklagen hier, dass ein Tabakwerbeverbot nicht vereinbart worden ist.Nun kann man über Tabakwerbeverbote in der Tat trefflich streiten. Ich verstehe aber nicht, wieso Sie auf der einen Seite für ein Tabakwerbeverbot eintreten und in der folgenden Debatte die Werbung für Schwangerschaftsabbrüche entkriminalisieren wollen. In beiden Fällen wird Leben geschützt.Meine Damen und Herren, die parlamentarische Demokratie lebt vom Interessenausgleich. Zum Interessenausgleich gehören auch die Abwägung von Meinungen und die kluge Suche nach den besten Lösungen für unsere Gesetzentwürfe. Deswegen haben wir selbstverständlich Anhörungen im Deutschen Bundestag, auch mit externen Sachverständigen. Abgeordnete informieren sich nicht nur im Wahlkreis, sondern ganz konkret auch bei denjenigen, die von Gesetzgebungsvorhaben betroffen sind. Das ist Normalität in einer parlamentarischen Demokratie.Ein Parlament, das sich abschottet und keine Meinungen von außen zulässt, wird seinem Auftrag nicht gerecht. Wir sollten uns davor hüten, jede Vertretung von Interessen gleich unter einen Generalverdacht zu stellen. Vielmehr gehört Interessenvertretung in einer Demokratie dazu.Die Grenze ist da überschritten, wo die Interessenvertretung unlautere Ziele verfolgt. Das müssen wir eindämmen; gar keine Frage. Aber Ihr Gesetzentwurf wird dieser Aufgabe nicht gerecht. Ganz im Gegenteil: Der Gesetzentwurf der Fraktion der Linken ist verfassungsrechtlich höchst problematisch. Er beginnt damit, dass ein Bundesbeauftragter für Interessenvertretung installiert werden soll. Gut; das sei so hingenommen. Auch die Breite der Daten, die Sie einfordern, mag noch im Rahmen dessen liegen, was zulässig ist. Aber dass Sie in § 11 Absatz 5 Ihres Gesetzentwurfes die Möglichkeit zulassen, dass der Bundesbeauftragte Wohnungen und Geschäftsräume von Verbänden und Unternehmen jederzeit betreten darf, und zwar ohne richterlichen Beschluss, das verstößt gegen das Recht auf Schutz der Wohnungs- und Geschäftsräume; das ist schlichtweg ein verfassungsfeindlicher Vorschlag.Ich möchte daran erinnern, dass es in unserem Rechtsstaat den sogenannten Richtervorbehalt gibt und dass Sie den Beschluss eines Richters oder im Eilfall einer Staatsanwaltschaft brauchen, wenn Sie Geschäftsräume durchsuchen wollen. Dass Sie das hier völlig vergessen oder negieren, zeigt, dass es Ihnen nicht auf rechtsstaatliche Sorgfalt ankam, sondern auf einen schnell und mit ziemlich heißer Nadel gestrickten Gesetzentwurf.Ich muss auch kritisieren, dass die Vorteilsabschöpfung und die Geldbußen ziemlich außerhalb jeglichen Verhältnisses stehen. In Ihrem Gesetzentwurf fordern Sie, dass bis zu 10 Prozent des weltweiten Umsatzes eines Unternehmens als Geldbuße für die fehlerhafte Registrierung zugrunde gelegt werden können. Da muss ich fragen: Ist da die Verhältnismäßigkeit noch gewahrt? Möglicherweise Geldbußen in Milliardenhöhe für eine fehlerhafte Angabe?Aber abgesehen davon: Viel verwerflicher ist, dass Sie in diesen Gesetzentwurf keine einzige Rechtsschutzmöglichkeit für diejenigen einbauen, die von dieser Geldbuße betroffen sind, und dass diese Geldbuße in der Tat nicht irgendein unabhängiges Gericht festsetzt, sondern der Bundesbeauftragte selber. Auch das hat mit Gewaltenteilung und Rechtsstaatlichkeit nichts zu tun, und auch da ist der Gesetzentwurf ziemlich schlecht.Meine Damen und Herren, natürlich gibt es im Bereich der Interessenvertretungen noch Verbesserungsbedarf.Wir können darüber sprechen, ob wir die Verbänderegistrierung, die es seit 1992 gibt, nicht verbessern, sie durch weitere Möglichkeiten der Information anreichern. Wir müssen auch über die Finanzierung von Lobbyorganisationen oder auch von NGOs sprechen. Die Vorkommnisse um Oxfam beispielsweise haben ganz klar deutlich gemacht, dass wir einen Nachholbedarf haben, auch darauf zu schauen, wie sich sogenannte NGOs finanzieren und was sie mit ihrem Geld machen.Wir sollten auch bei NGOs ähnlich transparent sein, wie wir es bei Parteispenden sind.Die Transparenz in diesem Bereich weist sicherlich noch Nachholbedarf auf.Insgesamt, meine Damen und Herren, dürfen wir aber eines nicht vergessen: Der Wert des freien Mandates und die freie Interessenvertretung auch der Bürger sind in Einklang zu bringen. Und letzten Endes rechtfertigen wir uns vor den Medien und vor den Bürgern. Aber das sollte nicht Anlass sein, einen schlechten und verfassungswidrigen Gesetzentwurf vorzulegen. Deswegen werden wir Ihren Entwurf ablehnen.Herzlichen Dank.




	31. Wolfgang Kubicki (FDP) Frau Präsidentin! Liebe Kolleginnen und Kollegen! Zunächst einmal möchte ich mich bei meiner Fraktion bedanken, dass sie mir die Gelegenheit gibt, frei von präsidialer Zurückhaltung in die politische Debatte einzugreifen, und das auch noch zu einem AfD-Antrag. Aber ich bin böse mit euch, weil es sich um einen Antrag von intellektueller Erbärmlichkeit handelt.Wir lesen, dass die Bundesregierung den Fall Yücel sonderbehandelt haben soll.– Ja. Ich komme gleich darauf zurück, Herr Brandner. – Das sei empörend. Aber statt die Bundesregierung dafür zu rügen – das wäre ja eine Maßnahme –, lenken Sie sofort auf den davon Betroffenen und erklären, er sei es nicht wert, dass man sich um ihn gekümmert habe.Sie belegen das mit Zitaten aus Kolumnen des Jahres 2011 und 2012. Alleine das ist schon erbärmlich, weil wir ja erwarten könnten, dass Sie das mit neueren Zitaten belegen würden.– Herr Brandner, das ist Ihre Meinung. Ich komme gleich darauf zurück. – „Die Welt“ hat heute dankenswerterweise dokumentiert, und zwar in unnachahmlicher Art, dass die Behauptung der Sonderbehandlung falsch ist. Das kann man auch nachvollziehen.Ich sage Ihnen mal: Ich sehe Ihnen nach, dass Sie Satire nicht verstehen und deshalb solche Anträge stellen.Das sehe ich Ihnen nach. Aber was ich Ihnen nicht nachsehe, Herr Brandner – ich habe gehört, Sie seien studierter Jurist –, ist, dass Sie die Bundesregierung auffordern, etwas Rechtswidriges und Verfassungswidriges zu tun. Sie fordern die Bundesregierung auf, die Missbilligung der Äußerungen eines Journalisten auszusprechen.Sie wissen aus eigener Geschichte – Sie haben ja ein Organstreitverfahren in Sachen Wanka geführt –, dass das Verfassungsgericht festgestellt hat, dass die Bundesregierung dazu gar nicht befugt ist, weil nicht der Eindruck entstehen darf, dass hier direkt oder indirekt eine Zensur ausgeübt wird.Sie haben sich hinsichtlich Ihrer eigenen Person über Maßnahmen der Bundesregierung beschwert. Jetzt aber kommen Sie und fordern die Bundesregierung auf, genau das zu tun, worüber Sie sich beschwert haben. Das nenne ich intellektuell ziemlich erbärmlich.Kollege Kubicki, gestatten Sie eine Frage oder Bemerkung?Besonders gerne, weil das meine Redezeit verlängert. Ich möchte zur Fortbildung beitragen.Bitte.Geehrter Herr Kollege Kubicki, die Sonderbehandlung des Herrn Yücel ist die eine Sache. Aber etwas ganz anderes ist der Umstand, dass es in diesem Hohen Hause scheinbar nicht mehr möglich ist, Äußerungen, die direkt oder indirekt den Volkstod unseres Volkes verlangen, zu rügen.Herr Kubicki, sind Sie nicht vielleicht mit mir gemeinsam der Meinung, dass das für dieses Hohe Haus ein wirklich erbärmliches Schauspiel ist?Sie müssen bitte stehen bleiben, wenn Sie eine Antwort bekommen möchten.Frau Präsidentin, darf ich antworten? – Zunächst einmal, Herr Kollege, hat Sie niemand daran gehindert, hier Ihre Reden zu halten und zu rügen, was Sie für rügenswert halten.Das muss man im Rahmen der Meinungsfreiheit ertragen, und das schafft dieses Hohe Haus.Aber Sie haben keinen Anspruch darauf, dass irgendein anderer Abgeordneter Ihrer Auffassung folgt. Das sieht die Verfassung nicht vor, und das werden wir auch nicht tun.Das war die Antwort. Sie können sich jetzt gerne wieder setzen oder auch stehen bleiben. Mir ist das relativ egal.Die Tatsache, dass Herr Yücel sich im Hinblick auf Herrn Sarrazin danebenbenommen hat, ist gerichtsfest festgestellt worden. Wir alle haben ja mal schlechte Tage – die AfD mehr als andere –,aber jedenfalls ist die Persönlichkeitsrechtsverletzung nicht wegzudiskutieren.Dafür muss man sich auch verantworten, und dafür durfte die „taz“ dankenswerterweise 20 000 Euro Schadensersatz an Herrn Sarrazin leisten.– Das mag sein, Herr Kollege Gauland. Und jetzt komme ich zu Ihnen. Persönlichkeitsrechtsverletzungen sind übrigens auch bei Ihnen Standard. Bei Ihnen persönlich möglicherweise nicht – ich komme gleich darauf zurück –, aber bei Ihnen als AfD insgesamt ist das Standard.Dass der Kollege Curio hier zum Doppelpass erklärt hat, dass die Integrationsbeauftragte der Bundesregierung ein lebendes Beispiel dafür sei, dass die Integration gescheitert sei,nehme ich noch hin. Das ist für mich grenzwertig, vielleicht auch unhöflich und unanständig, aber hinnehmbar – Banane.Aber dass Herr Gauland außerhalb dieses Hauses erklärt, diese Frau müsse man entsorgen– wegen dieser Satire dürft ihr jetzt gleich 20 000 Euro auf den Tisch legen, die auch Herr Yücel bzw. die „taz“ auf den Tisch legen durften, dann sind wir damit durch –,ist genau die Persönlichkeitsrechtsverletzung, die wir hier nicht wollen. Denn wer von „Entsorgen“ spricht, insinuiert dabei immer, dass es darum geht, Reste oder Abfall zu entsorgen, und ein Mensch ist kein Abfall.Herr Gauland, die „taz“ hat sich bei Herrn Sarrazin entschuldigt. Das ist nobel und anständig. Dass Sie sich bei der Integrationsbeauftragten der Bundesregierung nicht entschuldigen, zeigt, wie unanständig Sie eigentlich sind. Das ist keine Politik, die wir in Deutschland wollen.Ich bin in diesem Land geboren worden. Ich bin hier aufgewachsen und habe alle Möglichkeiten gehabt. Ich bin stolz auf dieses Land, und ich möchte nicht dauernd in Situationen kommen, in denen ich mich dafür schämen muss, dass politische Entscheidungsträger in diesem Land wieder so reden wie Sie.Herzlichen Dank.




	32. Stefan Gelbhaar (BÜNDNIS 90/DIE GRÜNEN) Sehr geehrte Damen und Herren! Sehr geehrte Frau Präsidentin! Es ist schon bemerkenswert, dass die FDP die Aktuelle Stunde dazu nutzt, ein Loblied auf den Diesel zu singen, dass die CDU die Aktuelle Stunde nutzt, um sich von ihren Ministern zu distanzieren,und dass die SPD versucht, zu beweisen, dass die Bundesregierung bei diesem Thema nur denkt, aber nicht handelt. Das finde ich schade. Ich hingegen möchte der Bundesregierung zu ihrer ersten positiven Schlagzeile nach Monaten gratulieren: Bus und Bahn kostenlos –das hat viele Menschen inspiriert, das hat viele Diskussionen ausgelöst. Das heißt, diesen Ankündigungen von immerhin drei Bundesministerien und der damit losgetretenen Debatte müssen nun auch Taten folgen.Natürlich wird Ihnen von der Bundesregierung jetzt mitgeteilt, warum das alles nicht gehen soll. Die Fachverbände merken dabei durchaus gewichtige Punkte an. Und Sie? Sie in der Bundesregierung sind sogleich dabei, zurückzurudern. Warum so verzagt? Seien Sie sich gewiss: Wir sind an Ihrer Seite, wenn Sie Bus und Bahn besser und günstiger machen wollen.Ja, bislang ist da die Bundesregierung nicht in Erscheinung getreten. Sie haben tatenlos zugesehen, wie die Fahrpreise Jahr für Jahr gestiegen sind, wie sich die Verkehrsverbünde einen Preiserhöhungsmechanismus gebaut haben. Da kam von Ihnen nur eine Art stiller Applaus. Das muss ein Ende haben. Die Preise müssen runter und nicht rauf.Darum lassen Sie uns diese Aufgaben endlich angehen.Ja, die Lösung wird kompliziert sein. Natürlich ist der Preis nur ein Baustein, aber eben ein wichtiger. Das Angebot von Bus und Bahn muss gut sein, also bequem, schnell und sicher, aber eben auch bezahlbar. Das ist für viele Menschen nicht mehr gegeben. Deswegen gibt es so viele Schwarzfahrer. Deswegen müssen so viele Eltern rechnen, ob das Geld für ein Schülerticket der Kinder da ist. Das müssen und das können wir ändern. Das Land Berlin, rot-rot-grün regiert, zeigt, dass das geht.Ja, das kann und muss ein Teil der viel beschworenen Mobilitätswende sein. Saubere und bezahlbare Mobilität darf nicht exklusiv sein, sondern muss für alle verfügbar sein. Ansätze dafür gibt es genug. Nehmen wir das Wiener Modell als Beispiel. Fahrgäste mit Jahreskarte zahlen dort 1 Euro pro Tag.Das hatte eine Verdopplung der Zahl der Fahrgäste seit der Einführung im Jahr 2012 zur Folge. Oder nehmen Sie das durchgerechnete Modell einer Berliner Bärenkarte, das eine Lösung für die Rushhour aufzeigt. All das kann die Menschen aus dem Auto locken. Aber auch das ist nur der Anfang einer breiten Palette von Instrumenten.Wir müssen natürlich in das Fahrrad investieren. Warum fehlt dazu eigentlich jegliche Aussage in diesem ominösen Brief? Wann bekennen Sie sich endlich zur technischen Nachrüstung von Dieselautos und dazu, dass das die Automobilindustrie zu zahlen hat und niemand anderes?Dass das wirkt und bezahlbar ist, hat gerade der ADAC bewiesen. Deswegen: Ran ans Werk!Bei alledem dürfen Sie nicht den ländlichen Raum vergessen. Wo kein Bus fährt, hilft es nichts, wenn der ÖPNV kostenlos ist. Das heißt, die Investitionen in das Angebot von Bus und Bahn müssen auf dem Land und in der Stadt deutlich erhöht werden.Damit komme ich zum zweiten Punkt in dieser Debatte. Sehr geehrte Frau Hendricks – aber auch Herrn Altmaier und Herrn Schmidt sei gesagt –, wenn Sie den Vorschlag eines kostenlosen ÖPNV nur in die Zeitung gebracht haben, um endlich nicht mehr über Fahrverbote, Nachrüstung und Dieselsubventionen reden zu müssen, dann seien Sie sich gewiss: Das lassen wir Ihnen nicht durchgehen.Sie haben auf breiter Front Erwartungen geweckt. Jetzt müssen Sie liefern. Sagen Sie nicht, dass Sie das Geld dafür nicht finden. Dabei rede ich noch nicht einmal von den zig Milliarden Folgekosten des Straßenverkehrs, sondern davon, dass Sie den Diesel jedes Jahr mit über 8 Milliarden Euro subventionieren. Sie haben also – das sage ich auch Ihnen, Herr Donth – Geld wie Heu. Wir werden Ihnen jeden Tag ins Stammbuch schreiben: Geben Sie dieses Geld richtig aus!Da, wo Sie heute den Diesel subventionieren, müssen wir in Zukunft den ÖPNV sowie den Fuß- und Radverkehr unterstützen.Es wäre wirklich perfide, wenn diese ganze Nummer nur ein Ablenkungsmanöver, nur Fake News der Bundesregierung wären. Dann würden Sie das Vertrauen in die Bundesregierung untergraben, noch bevor sie wieder im Amt ist.Meine Damen und Herren, es geht um saubere Luft, es geht um Gesundheitsschutz, es geht einfach darum, dass die Menschen in diesem Land nicht mehr jeden Tag vergiftete Luft einatmen müssen. Dafür zu sorgen, ist Ihre verdammte Aufgabe als Bundesregierung. Da können Sie sich nicht herausreden und sagen, das sei Aufgabe der Kommunen. Dieser Verantwortung müssen Sie gerecht werden. Machen Sie Bus und Bahn endlich besser und günstiger! Fördern Sie den Radverkehr stärker als bisher! Zwingen Sie die Automobilindustrie, für ihre Fehler einzustehen und zumindest die Nachrüstung rasch zu bezahlen! Das sind Ihre Aufgaben! Briefeschreiben war gestern. Heute gilt: Ran ans Werk!Vielen Dank.




	33. Bernd Rützel (SPD) Sehr geehrter Herr Präsident! Liebe Kolleginnen! Liebe Kollegen! Sehr geehrte Damen und Herren! Ich jedenfalls danke den Linken. Ich bin froh, dass heute noch einmal über diesen Antrag diskutiert wird.Wir haben das oft getan. Ich glaube, es ist gut, dass wir so die Möglichkeit haben, über Ergebnisse unserer Koalitionsverhandlungen zu sprechen. Meine Kollegin Gabi Hiller-Ohm hat das schon wunderbar erläutert.Ich will anhand dreier Punkte sagen, warum das für uns im Moment so wichtig ist: Wir begrenzen die erlaubten Befristungen. Wir verkürzen die Dauer. Wir schränken Kettenbefristungen drastisch ein.Das Institut für Arbeitsmarkt- und Berufsforschung, IAB, hat in seinem aktuellen Heft bescheinigt, dass wir damit den stärksten Eingriff in das Befristungsrecht seit 1985 vornehmen. An dieses Jahr kann ich mich gut erinnern. 1985 hat die Kohl-Regierung die sachgrundlose Befristung eingeführt. Ich war damals 16 Jahre alt, war Jugendvertreter, bin auf die Straße gegangen und habe dagegen demonstriert, weil das ungerecht war. Deswegen bin ich froh, wenn wir jetzt wieder stark eingreifen, um hier zu regulieren, und 400 000 Menschen – so viele profitieren davon, wahrscheinlich sogar noch mehr – deutlich besser stellen.Manche mögen das wenig finden. Die sagen vielleicht: Ja, aber es gibt 1,3 Millionen sachgrundlos befristete Verträge, und nur 400 000 profitieren von der Vereinbarung. – Aber, Kolleginnen und Kollegen, ist das nichts? Ist das keine Verbesserung? Ist das nicht ein großer Fortschritt? Ich glaube schon, dass wir den Betroffenen schuldig sind, dass wir das tun. Wir haben jetzt die Möglichkeit dazu. Wir wollen das und werden das auch tun.Ich will ein Beispiel nennen. Ein Unternehmen mit 200 Beschäftigten darf künftig nur noch fünf Arbeitsverträge sachgrundlos befristen. Das ist eine deutliche Verbesserung gegenüber heute. Außerdem – ich habe es gesagt – werden wir die Zahl der unsäglichen Kettenbefristungen begrenzen. Das muss doch Ihre Zustimmung finden, liebe Kolleginnen und Kollegen, die den Antrag eingebracht haben.Ich bin froh, dass die Union am Ende der Koalitionsverhandlungen doch noch ihren Widerstand dagegen aufgegeben hat. Das ist gut für unser Land, und irgendwann finden auch Sie gut, dass das unbefristete Arbeitsverhältnis wieder die Regel wird.Ich bin nicht der Einzige, der das so sieht. Reiner Hoffmann, der DGB-Vorsitzende, schreibt, unser Verhandlungsergebnis bringe eine wichtige strukturelle Verbesserung. Auch Frank Bsirske, der Verdi-Vorsitzende, lobt die geplante Neuregelung und sagt, wir würden damit das Leben vieler Menschen verbessern.Und die beiden müssen es wissen.Wir könnten uns noch mehr vorstellen, aber dafür braucht man Mehrheiten. Das ist der Unterschied zwischen realer Politik und Wünschen. Von daher lasst uns das jetzt anpacken.Ich will noch eines sagen: Wer befristet angestellt ist, bildet seltener einen Betriebsrat und geht seltener in eine Gewerkschaft. Wer befristet beschäftigt ist, bekommt auch seltener einen Kredit bei seiner Bank und kann seine Zukunft weniger planen. Aber diese Bedingungen, dieses Miteinander haben uns doch insgesamt stark gemacht. Von daher brauchen Menschen nicht nur einen Job; sie brauchen einen sicheren Job. Das ist ein Unterschied.Kollege Rützel, gestatten Sie eine Zwischenfrage?Jawohl.Bitte sehr.Lieber Kollege, vielen Dank für die Gelegenheit, eine Zwischenfrage zu stellen und kurz zu intervenieren. – Sie haben von Mehrheiten gesprochen. Deswegen will ich vorwegsagen: In der letzten Legislatur hätte es die Mehrheit gegeben. Hätten Sie so viel Mut gehabt wie bei der Ehe für alle, dann hätten wir in der letzten Legislaturperiode die sachgrundlose Befristung komplett abschaffen können. Das wäre eine wunderbare Sache gewesen. Es ist schade, dass Sie den Mut nicht hatten.Jetzt kümmern Sie sich um 400 000 Menschen; das ist schön. Aber 900 000 Menschen lassen Sie im Regen stehen. Und es gibt viele praktische Fragen, die auch die Kolleginnen und Kollegen in den Gewerkschaften – Sie haben ja einige zitiert – umtreiben. Mir stellt sich schon die Frage, wie Sie sich das eigentlich genau vorstellen. Wer soll die Schwellenwerte in den Betrieben kontrollieren? Was machen wir mit Betrieben, wo es keinen Betriebsrat gibt? Wie soll das dort geregelt werden? Das sind doch komplizierte Details, und aus unserer Sicht – wir können uns das nicht vorstellen – gibt es nur halbseidene Regelungen. Warum haben Sie nicht den Mut, die sachgrundlose Befristung komplett abzuschaffen? Stimmen Sie unserem Antrag zu!Erstens, lieber Kollege, bin ich Ihnen für diese Frage sehr dankbar; denn sie eröffnet mir die Gelegenheit, noch einmal zu betonen, dass wir wahrscheinlich viel mehr hätten machen können in der letzten Legislatur. Aber eine Regierung braucht Verlässlichkeit; das gilt für beide Seiten. Wenn man sich verabredet, zu regieren, dann hält man auch zusammen. Jeden Tag eine andere Mehrheit in diesem Hause zu suchen, bringt keine Kontinuität. Das wissen Sie – das haben Sie in die Frage hineingelegt –; darüber haben wir schon oft gesprochen.Zweitens. Wir lassen niemanden im Regen stehen. Wir verbessern konkret das Leben vieler Menschen in Deutschland, nicht nur in diesem Bereich, auch in vielen anderen Bereichen.Wenn wir etwas verbessern, dann ist das auf jeden Fall besser, als nichts zu tun und niemandem zu helfen.Von daher will ich noch einmal betonen, dass durch die Vereinbarung eine deutliche Verbesserung eintritt. Ich hoffe, dass wir das, was wir in unseren Koalitionsverhandlungen vereinbart haben, bald in Regierungsverantwortung zum Wohle von mindestens 400 000 Beschäftigten umsetzen können.Vielen Dank.




	34. Volker Ullrich (CSU) Frau Präsidentin! Meine sehr verehrten Damen und Herren! Freude und Erleichterung sind die Gefühlslagen, die wir über die Haftentlassung von Deniz Yücel empfinden. Aber wir tragen auch Sorge in uns in Bezug auf all diejenigen Journalisten, die noch in der Türkei inhaftiert sindund in jüngster Zeit auch zu langen Freiheitsstrafen verurteilt wurden.Von dieser Stelle sei erinnert: Die Türkei ist Vertragspartner der Europäischen Menschenrechtskonvention. Presse- und Meinungsfreiheit sind wesentliche Bestandteile dieses Vertragswerkes. Auch daran muss sich die Türkei halten.Wenn wir über die Bewertung der Arbeit von Deniz Yücel durch Politiker der AfD sprechen, dann kommt man nicht daran vorbei, dass die Fraktionsvorsitzende Alice Weidel gesagt hat:Yücel ist weder Journalist noch Deutscher.Das ist deswegen besonders niederträchtig und empörend, weil die Staatsangehörigkeit dazu führt, dass jemand in die Werte- und Verantwortungsgemeinschaft unseres Landes eintritt. Wer allerdings jemandem wegen seiner Herkunft oder wegen seines Aussehens die Staatsangehörigkeit abspricht, der muss wissen, dass er sich außerhalb des Verfassungsbogens bewegt.Die Äußerungen von Deniz Yücel in vielen Jahren journalistischer Arbeit kenne ich persönlich größtenteils nicht. Die Äußerungen, die Sie in dem Antrag zusammengefasst haben, sind teilweise aus dem Kontext gerissen.Aber es kommt gar nicht darauf an, ob diese Äußerungen mir persönlich gefallen. Es wäre weder meine Sprache noch mein Inhalt. Entscheidend ist etwas anderes. Es geht um die Frage: Welche Reichweite hat Meinungsfreiheit in diesem Land und in dieser Gesellschaft? Ich bitte Sie, zur Kenntnis zu nehmen, dass die Meinungsfreiheit schlichtweg konstituierend für eine freiheitliche demokratische Gesellschaft istund dass ohne Meinungs- und Pressefreiheit Demokratie nicht möglich ist.Wenn Sie also sagen, dass der Deutsche Bundestag die Bundesregierung auffordern solle, eine einzelne Meinungsäußerung zu missbilligen, dann hat das noch eine ganz andere Note, nämlich eine, bei der Sie sich ebenfalls außerhalb des Verfassungsbogens bewegen. Es ist nicht Aufgabe staatlicher Organe, Meinungen zu kommentieren, zu bewerten oder gar zu zensieren.Kollege Ullrich, gestatten Sie eine Frage oder Bemerkung?Nein. Es ist alles gesagt.Jemand, der – zum Beispiel als Journalist – eine Meinung äußert und damit gegen die Rechte anderer verstößt, der wird von den Gerichten, wenn er die Grenzen der Meinungsfreiheit überschritten hat, auch entsprechend verfolgt. Beleidigung und Volksverhetzung sind in diesem Land zu Recht strafbar.Das sollten auch einige von Ihnen bitte zur Kenntnis nehmen.Aber es ist nicht unser Konzept eines freiheitlichen Staates, dass wir Vorschriften machen, wie jemand zu denken und zu reden hat.Wer das möchte, der findet sich in einem anderen Staat wieder – in einem Staat ohne Meinungsfreiheit, in einem Staat ohne Demokratie,in einem autoritären Gemeinwesen, in dem die freie Rede, das Miteinander gefährdet sind und das natürlich auch keine Demokratie mehr ist.Deswegen sage ich Ihnen ehrlich: Wehret den Anfängen! Ein solcher Antrag ist verfassungsfeindlich – ich bitte Sie, das zur Kenntnis zu nehmen –, und deswegen lehnen wir ihn ab.Ich schließe die Aussprache.Wir kommen zur Ab stimmung über den Antrag der Fraktion der AfD auf Drucksache 19/846 mit dem Titel „Verhalten der Bundesregierung im Fall Deniz Yücel“. Die Fraktion der AfD hat namentliche Abstimmung verlangt.Ich bitte die Schriftführerinnen und Schriftführer, die vorgesehenen Plätze einzunehmen. – Von mir aus gesehen links fehlen noch Schriftführer. – Hier vorne rechts fehlen ebenso Schriftführer. – Sind jetzt alle Schriftführerinnen und Schriftführer an ihrem Platz? – Hier vorn fehlt noch aus der CDU/CSU-Fraktion ein Schriftführer. Vielleicht kann jemand für den Kollegen Whittaker übernehmen. – Danke schön. Sind jetzt alle Schriftführerinnen und Schriftführer am Platz? – Das ist der Fall. Ich eröffne die Abstimmung über den Antrag auf Drucksache 19/846.Ist ein Mitglied des Hauses anwesend, das noch nicht abgestimmt hat? – Dann ist jetzt die letzte Gelegenheit. – Haben alle abgestimmt? – Das ist der Fall.




	35. Andreas Jung (CDU) Frau Präsidentin! Liebe Kolleginnen und Kollegen! Lieber Herr Kollege Gelbhaar, Glückwunsch zur ersten Rede! Ich will gleich dort anfangen, wo Sie aufgehört haben. Es geht auf gar keinen Fall darum, sich herauszureden. Wenn wir uns zu einer gemeinsamen Verantwortung in diesem Land bekennen, dann ist es richtig, dass wir als Bund in Partnerschaft mit den Kommunen und Ländern gemeinsam vorangehen.Genau so sind die in Rede stehenden Initiativen zu verstehen. Es geht hier nicht um Wegdrücken, sondern darum, dass man sich dieser Aufgabe gemeinsam stellt.Ohne Zweifel haben wir Anlass, über das Thema Umwelt und Verkehr zu sprechen. Wir haben Handlungsdruck wegen der Stickstoffemissionen. Wir haben uns im Klimaschutz ehrgeizige Ziele gesetzt. Die Entwicklung der CO 2 -Emissionen im Verkehr war in den letzten Jahren nicht zufriedenstellend, im Gegenteil. Wir haben uns nun ehrgeizige Ziele für die nächsten Jahre gesteckt. Wir haben einen Klimaschutzplan verabschiedet und werden ihn in Gesetzesform gießen und die darin enthaltenen Maßnahmen umsetzen. Wegen Stickstoffemissionen und Klimaschutz müssen wir uns dieser Aufgabe widmen.Zur konkreten Frage der Förderung des ÖPNV hat mein Kollege Michael Donth schon ausführlich Stellung genommen. Ich will nur Folgendes aufgreifen: Selbstverständlich ist es richtig – diese Aufgabe hat sich die Koalition gestellt –, den ÖPNV zu stärken. Wir sehen das aber im Gesamtkontext. Wir wollen den ÖPNV im Sinne nachhaltiger Mobilität in der Stadt stärken, darüber hinaus Schiene und Bahn ausbauen sowie alternative Antriebe voranbringen. Hier voranzukommen, gelingt aber nur im Rahmen eines Gesamtkonzepts.Ich möchte darauf verweisen: Dieser Brief wird manchmal auf diesen einen Vorschlag reduziert. Vorher ist aber schon richtiggestellt worden, dass es ein ganzes Bündel an Maßnahmen ist.Ich will eins noch etwas stärker herausgreifen: Das ist das konkrete Vorhaben, den Kommunen Emissionslimits bei Bussen und Lieferwagen zu ermöglichen. Ich halte das für ein richtiges Instrument. Wenn dieses Instrument angewendet wird, dann steht das selbstverständlich im Zusammenhang damit, dass man auf diesem Weg auch unterstützt und hilft. Es muss und es gibt Förderung, um auf Elektrobusse umzustellen, die selbstverständlich verfügbar und vorhanden sind. Es muss Unterstützung geben, um Dieselbusse umzustellen und nachzurüsten. Ich finde, auch das ist ein Punkt, den man in dieser Diskussion sehen muss. Wir wollen ihn in dieses Gesamtkonzept einbringen.Darum geht es: dass wir uns vornehmen, nachhaltige Mobilität in Deutschland voranzubringen.Wir haben die Nationale Plattform Elektromobilität. Es gibt aus der Bundesregierung den Vorschlag, diese Plattform zu einer Nationalen Plattform Zukunft der Mobilität weiterzuentwickeln. In der Zukunft muss die Mobilität nachhaltig sein. Wir haben im Entwurf des Koalitionsvertrags ein ganzes Bündel an Maßnahmen vereinbart, für Elektromobilität mehr Anreize zu schaffen, beispielsweise bei der Dienstwagenförderung, aber auch bei anderen Arten der Förderung. Es geht auch um die Fortsetzung der Förderung des Aufbaus von Infrastruktur: 100 000 Ladesäulen sollen in den nächsten Jahren entstehen, um Elektromobilität tatsächlich auf die Straße zu bringen.Die Themen Wasserstoff- und Brennstoffzelle, alternative Kraftstoffe, synthetische Kraftstoffe, Gas und erneuerbare Energien, sie alle gehören zum Gesamtkontext. Nötig wird nicht eine Maßnahme sein; vielmehr muss eine Vielzahl von Maßnahmen mit Entschiedenheit und mit Nachdruck vorangebracht werden. Dafür müssen wir als Politiker sorgen, und wir müssen dabei auch die Wirtschaft in die Pflicht nehmen. Unser Ziel muss sein, dass das effiziente Ökoauto der Zukunft bei uns in Deutschland gebaut wird.Nur dann werden wir auch den wirtschaftlichen Wettbewerb gewinnen und auf den Märkten bestehen. Dann werden wir unsere Ziele im Bereich Umwelt und Klimaschutz erreichen. Das ist unsere Aufgabe. Sie zu erfüllen, das ist unsere Verantwortung, und der stellen wir uns.Herzlichen Dank.




	36. Stephan Stracke (CSU) Grüß Gott, Herr Präsident! Meine sehr verehrten Damen und Herren! Die Verabredungen von Union und SPD zum Befristungsrecht gehen dahin, das Befristungsrecht umfassend zu reformieren. Wir sorgen für mehr Rechtssicherheit für die Arbeitnehmerinnen und Arbeitnehmer, die befristet beschäftigt sind, indem wir den Interessenausgleich zwischen dem Flexibilitätsbedürfnis der Arbeitgeber auf der einen Seite und dem Sicherheitsbedürfnis der Arbeitnehmer auf der anderen Seite noch einmal deutlich zugunsten der Arbeitnehmer justieren. Das ist das zentrale Ergebnis unserer Verhandlungen in diesem Bereich, und ich finde das auch gut.Im Mittelpunkt der Unionspolitik steht dabei nicht die sachgrundlose Befristung. Im Grunde haben wir hier durch die Regelungen in diesem Bereich, zum Beispiel durch die Höchstfristen, ein schon relativ stark reguliertes Instrument. Das, was uns bedrängt, ist doch die Lage derjenigen, die sich von Befristung zu Befristung hangeln, wo über viele Jahre hinweg ein Sachgrund nach dem anderen aneinandergehängt wird. Deshalb haben die Betroffenen keine Planungssicherheit, keine verlässliche Perspektive für ihre Lebensplanung,wenn es beispielsweise darum geht, eine Familie zu gründen, ein Haus zu bauen. Das führt zum Teil auch zu gesundheitlichen Einschränkungen. Genau diese Lebensperspektive geben wir ein Stück weit zurück, indem wir die Kettenbefristungen reformieren. Hier gibt es bislang keine zeitlichen Begrenzungen durch den Gesetzgeber, und das ändern wir. Wir ändern dies, weil die Rechtsprechung des Bundesarbeitsgerichtes zwar feinziseliert ist, aber wenig mit Rechtssicherheit und Rechtsklarheit für den Rechtsanwender zu tun hat. Wir sagen ganz klar: Nach fünf Jahren Befristung ist Schluss.Dann gibt es keine weitere Befristung. Damit sorgen wir für Rechtssicherheit und Rechtsklarheit.Es ist das zentrale Anliegen der Unionspolitik, hier für mehr Rechtssicherheit für die Arbeitnehmerinnen und Arbeitnehmer zu sorgen und gerade ihr Sicherheitsbedürfnis in diesem Bereich zu stärken.Herr Kollege, gestatten Sie eine Zwischenfrage?Selbstverständlich gerne, Herr Präsident. – Frau Müller-Gemmeke, bitte.– Ja, selbstverständlich.Vielen Dank. – Ja, ich habe vorhin geredet, aber gerade deswegen habe ich jetzt noch eine Frage.Es ist gut, dass bei den Kettenbefristungen etwas gemacht wird; das unterstützen wir. Das habe ich vorhin zwar nicht gesagt; aber das sage ich jetzt.Ja, das ist doch gut. Sehr gut!Ich habe vorhin ein Problem bei der sachgrundlosen Befristung geschildert, das ja wohl unsäglich ist. Was wollen Sie gegen diese üblen Formen der sachgrundlosen Befristung machen?Schauen Sie, liebe Frau Kollegin Müller-Gemmeke: Wir erkennen das Flexibilitätsbedürfnis der Unternehmen im Hinblick auf die sachgrundlose Befristung an. Sie ist das einzige Instrument, das in diesem Bereich relativ unbürokratisch ist. Aber wir verändern dieses Instrument auch, indem wir beispielsweise die Höchstfristen senken. Wir verändern auch die Regelungen zu den Verlängerungsmöglichkeiten. Bisher war eine dreimalige Verlängerung in maximal zwei Jahren möglich. In Zukunft wird nur noch eine einmalige Verlängerung möglich sein.Gleichzeitig führen wir bei der sachgrundlosen Befristung eine Quotierung innerhalb des jeweiligen Unternehmens ein. Dies führt dazu, dass wir einen massiven Eingriff in das Recht der sachgrundlosen Befristung vornehmen, auch zugunsten der Arbeitnehmerinnen und Arbeitnehmer. Ich finde, es ist in Ordnung, dass wir das tun. Aber wir dürfen das Instrument der sachgrundlosen Befristung nicht einfach streichen, weil es eben auch das Bedürfnis der Arbeitgeberseite nach Flexibilisierung gibt.Wenn es um das Thema Missbrauch geht, sind wir uns einig. Natürlich wollen auch wir keinen Missbrauch von befristeter Beschäftigung. Deswegen werden wir gemeinsam überlegen, wie wir das insgesamt gestalten können. Ich glaube, die Verabredungen, die Union und SPD im Rahmen der Koalitionsverhandlungen gemeinsam getroffen haben, sind gut und tragfähig, und sie verbessern vor allem die Rechtssicherheit für die Arbeitnehmerinnen und Arbeitnehmer.Nun wünsche ich uns, dass wir diese gemeinsamen Verabredungen auch umsetzen können.




	37. Alexander Throm (CDU) Sehr geehrte Frau Präsidentin! Werte Kolleginnen und Kollegen! Nachdem hier jetzt so viel Gift verspritzt wurde,will ich zunächst damit beginnen, dass es für uns, für die CDU/CSU-Fraktion, zuallererst ein Grund zur Freude und zur Dankbarkeit ist, dass Herr Yücel, der über ein Jahr aus rein politischen Gründen inhaftiert war, endlich freigekommen ist.Da die Überschrift Ihres Antrags „Verhalten der Bundesregierung im Fall Deniz Yücel“ lautet, gibt mir dies die Gelegenheit, ebendieses Verhalten zu loben. Es war vorbildlich.Eine klare, hartnäckige und unnachgiebige Haltung auf allen Ebenen gegenüber der Türkei hat sich durchgesetzt. Unser Dank gilt allen in der Bundesregierung, die daran beteiligt waren, zuallererst der Bundeskanzlerin und dem Vizekanzler. Unser Dank geht aber auch an alle anderen Beteiligten in der Zivilgesellschaft: an die journalistischen Kollegen, an die Anwälte und an alle, die dazu beigetragen haben, dass der Druck auf die türkische Regierung so groß wurde, dass sie letztlich nachgeben musste.Liebe Kolleginnen und Kollegen, der Fall Yücel und der Umgang mit ihm sind geradezu ein Paradebeispiel für unseren freiheitlich-demokratischen Rechtsstaat, für unsere Haltung zu Meinungs- und Pressefreiheit auf der einen und den Schutz unserer Staatsbürger auf der anderen Seite. Wir lassen es nicht zu, dass jemand in seiner Meinungs- und Pressefreiheit eingeschränkt wird, und zwar unabhängig davon, ob wir dessen Auffassung teilen oder nicht.Schon gar nicht akzeptieren wir, dass ein Journalist wegen seiner politischen Meinungsäußerungen ohne Anklage und ohne rechtsstaatliches Verfahren inhaftiert wird. Freiheit und Unabhängigkeit der Presse sind ein hohes Gut, das wir schützen. Dabei gibt es auch keine Vorzugsbehandlung, wie behauptet wird. Wer die deutsche Staatsangehörigkeit hat, erhält den Schutz des deutschen Staates uneingeschränkt.Das galt für Mesale Tolu, Peter Steudtner und selbstverständlich für Herrn Yücel, und das gilt auch für die anderen weiterhin inhaftierten Deutschen in türkischen Gefängnissen.Und: Ja, der Grat zwischen freier Meinungsäußerung und der Verletzung der Persönlichkeit eines anderen ist schmal. Es muss niemandem gefallen, was Deniz Yücel 2011/2012 in seinen als Satire erkennbaren Texten über Herrn Sarrazin und zuvor über dessen Buch geschrieben hat.Man muss das wirklich nicht gut finden. Doch dafür gab es teilweise Sanktionen des Presserates und die gerichtliche Untersagung der weiteren Verbreitung.Die AfD will heute, dass die Bundesregierung aufgefordert wird, eine Missbilligung auszusprechen. Doch der Bundestag ist der falsche Ort für die Bewertung einer Satire.Wir sind kein literarisches Quartett.Der schmale Grat, von dem ich sprach, gilt auch für die AfD und für Frau Weidel. Ich bedaure, dass sie dieser Debatte heute nicht beiwohnt. Sie hat das letzte Wochenende fast nichts anderes getan, als sich über Twitter und andere Medien zu dem Fall Yücel zu äußern.In einer Twitter-Meldung hat sie ihn am Tag nach seiner Entlassung als „antideutschen Hassprediger“ beschimpft.Da hat auch sie diesen Grat überschritten. Herr Yücel hatte immerhin die Größe, sich nach seinem kritikwürdigen Text zu entschuldigen. Ich wage zu bezweifeln, dass Frau Weidel diese Größe ebenfalls hat.Sie hat in dieser Twitter-Meldung auch geschrieben: Einer – ich zitiere –, „der nicht nur einmal die Grenzen des guten Geschmacks verließ, sollte eigentlich keine deutsche Staatsbürgerschaft besitzen.“Doch die deutsche Staatsbürgerschaft ist keine Frage des guten Geschmacks.Die AfD will sich hier als Patriotin darstellen. So eine Aussage ist aber alles andere als patriotisch. Sie ist schlicht und einfach erbärmlich.Nun zu Ihrem Kollegen Poggenburg. Er hat am Aschermittwoch die Türken in Deutschland aufs Übelste beleidigt und damit ganz sicher die Grenzen des guten Geschmacks verlassen.– Darüber sind wir uns doch, Herr Dr. Gauland, einig. Sie haben ihn ja schließlich abgemahnt.– Warten Sie doch einmal ab! – Ich habe dann genau dasselbe getan wie Frau Weidel im Hinblick auf Herrn Yücel.– Herr Dr. Gauland, regen Sie sich doch nicht so auf.Ich habe lange gesucht, um bei Twitter oder Facebook irgendein Zitat zu finden, in welchem Sie Herrn Poggenburg ebenfalls seine deutsche Staatsbürgerschaft absprechen. Nichts dergleichen konnte ich finden.Ein letzter Gedanke zum Thema „guter Geschmack“: Guten Geschmack kann man nicht immer in Gesetze und Normen fassen. Manchmal geht es ganz banal um Anstand, Menschlichkeit und Mitgefühl.Hier kam ein Mensch nach über einem Jahr Haft – größtenteils in Isolation verbracht – in einem Land frei, in dem rechtsstaatliche Grundsätze nicht in dem Maße gelten, wie wir sie kennen und erwarten,und Ihre Frau Weidel hat nichts anderes zu tun, als diesen Menschen aus ihrem warmen und sicheren Abgeordnetenbüro heraus per Twitter zu beschimpfen und zu diskreditierenund dann einen derart schrägen Antrag zu stellen, wie Sie dies heute tun. Das ist kollektive Unanständigkeit.Das Verhalten der Bundesregierung – darum geht es im Titel der Debatte – war vorbildlich. Ihre Kritik ist genau das Gegenteil davon. Deshalb lehnen wir Ihren Antrag ab.




	38. Annalena Baerbock (BÜNDNIS 90/DIE GRÜNEN) Meine sehr verehrten Damen und Herren! Liebe Kollegen und Kolleginnen! Liebe Frau Bundesumweltministerin! Liebe Union und SPD! Ich kann ja verstehen, dass es Ihnen schwerfällt, über Klimaschutz zu diskutieren, wenn die Grünen nicht mit am Tisch sitzen. Aber dann geben Sie doch einfach zu: Das war uns nicht so wichtig; wir haben uns um andere Themen gekümmert.Stattdessen reden Sie hier Dinge schön – auch Frau Hendricks; Sie wissen es doch besser –, die einfach nicht schönzureden sind. Sie verkaufen uns hier als vollen Erfolg der letzten Legislaturperiode und des Koalitionsvertrages, dass das 2020-Ziel leider nicht zu erreichen ist. Supererfolg in der Klimapolitik, kann ich da nur sagen!Der nächste Erfolg – und ich hoffe, Sie verstehen folgende Ironie –, den Sie uns verkaufen wollen, ist der Emissionshandel. Sie sagen, er sei eine totale Erfolgsgeschichte der letzten Bundesregierung, und es gebe Superantworten im Koalitionsvertrag. Da steht aber nichts dazu drin. Ihre Politik führte dazu, dass wir im Nicht­emissionshandelsbereich bis 2020 leider das nächste Ziel verfehlen werden. Wir schaffen es noch nicht einmal, die vorgegebenen 14 Prozent CO2-Minderung zu erreichen. Das nächste Vertragsverletzungsverfahren droht bereits; das kann ich Ihnen schon sagen.Was kommt stattdessen? Wir kaufen jetzt von Bulgarien Emissionszertifikate. Superidee! Ein totaler Erfolg dieser Bundesregierung!Als gestern im Ausschuss das Wirtschaftsministerium, das die Zertifikate ja einkaufen muss, gefragt wurde: „Wie viele Zertifikate sind das denn, wie viel wird das kosten, und wer bezahlt das eigentlich?“, lautete die Antwort: Keine Ahnung. – Es war genau so eine Antwort wie beim letzten Vertragsverletzungsverfahren im Automobilbereich.Das ist wirklich peinlich, meine sehr verehrten Damen und Herren.Der nächste Kracher – Achtung wieder Ironie –: Als wir den Élysée-Vertrag gefeiert haben, haben Sie sich ganz stolz auf die Brust geklopft und gesagt: Die Festsetzung des CO 2 -Preises machen wir mit Macron; da kommt eine richtige Ansage. – Schaut man in den Koalitionsvertrag, stellt man fest, dass Sie dort schreiben: Ach nein, so haben wir das aber nicht gemeint. Der Macron könnte uns ja noch beim Wort nehmen. Oh, verstärkte Zusammenarbeit: Dann können wir es ja gar nicht darauf schieben, dass die Polen nicht mitmachen. – Was lesen wir jetzt also im Koalitionsvertrag?Unser Ziel ist ein CO 2 -Bepreisungssystem, das nach Möglichkeit global ausgerichtet ist– da ist nichts mehr mit verstärkter Zusammenarbeit der europäischen Nachbarländer –,jedenfalls aber die G20-Staaten umfasst.Ach, Herr Trump soll jetzt auch noch mitmachen? Dann kommt der Mindestpreis ja wirklich nächstes Jahr, sehr verehrte Damen und Herren!Zum Kohleausstieg. Frau Hendricks, jetzt haben Sie auch noch angekündigt: Ja, das wird auf jeden Fall etwas werden. – Sorry, ich weiß, Sie hören nicht auf uns Grüne; das ist vollkommen klar.Aber dann nehmen Sie bitte Ihre eigenen Worte ernst.Im November letzten Jahres, nach dem Scheitern von Jamaika, haben wir hier über einen Abbau von 7 Gigawatt bei der Kohle diskutiert. Frau Weisgerber, da hatten auch Sie noch zugestimmt, wie alle hier im Haus; na gut, Sie von der AfD diskutieren da ja gar nicht mit. Herr Jung sagte damals zu den Vorschlägen, die zum Thema Kohle – Klammer auf: 7 Gigawatt; Klammer zu – auf dem Tisch lagen: Egal wer Verantwortung tragen wird, hinter diesen Vorschlag dürfen wir nicht zurück. – Das war die Position der Union hier im Deutschen Bundestag.Dann kam Herr Westphal von der SPD. Er ist nicht der größte Klimafreund, und wir haben immer heftig diskutiert. Nichtsdestotrotz: Auch Herr Westphal wollte damals, nach dem Scheitern von Jamaika, nicht hinter uns oder die Union zurückfallen. Also sagte er: Eigentlich ist der Vorschlag der Grünen – es war ja unser Antrag, diese Kohleblöcke abzuschalten – ganz gut; es fehlt nur der Strukturwandelfonds. – Komisch! Der steht jetzt in Ihrem Koalitionsvertrag drin. Nur von den 7 Gigawatt ist gar nichts mehr zu lesen.Wo sind die denn, bitte, hingekommen?Dann kommen Sie und sagen: Jetzt haben wir uns eine Kohlekommission ausgedacht. – Superidee! Darüber diskutieren wir im Wirtschaftsausschuss schon seit vier Jahren. Das ist ja wirklich etwas Neues, sehr verehrte Damen und Herren! Das ganze Klimakapitel ist ein Desaster. Das müssen Sie endlich anerkennen.Ganz kurz noch zur FDP. Bitte klären Sie mit sich selbst, ob Sie gerne auf dieser Seite des Hauses sitzen wollen oder auf der Seite, die etwas von Klimapolitik versteht.Es kann doch nicht sein, dass Sie in Ihrem Antrag schreiben, der Deutsche Bundestag wolle beschließen, dass wir das 2020-Ziel nicht mehr erreichen. Sie wollen also das Kyoto-Protokoll aufkündigen, und das sollen wir hier auch noch beschließen. Das ist doch absurd, liebe FDP!Der nächste Gag, der da drinsteht, ist: Die ärmsten Länder der Welt sollen bei der Bewältigung des Klimawandels jetzt nur noch dann Hilfe bekommen, wenn sie den Emissionshandel einführen; denn das ist ja Ihr Allheilmittel. Jetzt hören Sie mir mal zu!Aber nur noch ganz kurz.Auf der Klimakonferenz war auch der Staat Kiribati vertreten, der demnächst im Meer untergehen wird. Sie sagen jetzt: Wir siedeln euch, Bewohner von Kiribati, die ihr bald im Meer untergeht, erst um, wenn ihr den Emissionshandel eingeführt habt.Erst dann kriegt ihr Geld dafür. – Wissen Sie, was sie dort für eine Industrie haben? Deren einzige Industrie ist die Fischerei, und aus Kokosnüssen wird Kokosnussöl hergestellt.Das ist absurd; das geht nicht. Das ist nicht nur eine schlechte Klimapolitik,liebe FDP, das ist eine menschenrechtsverachtende Politik.Herzlichen Dank.




	39. Danyal Bayaz (BÜNDNIS 90/DIE GRÜNEN) Frau Präsidentin! Liebe Kolleginnen und Kollegen! Was hat das Ganze hier mit Donald Trump zu tun? Was hat das Ganze mit Populismus zu tun, der unsere Zukunft infrage stellt? Ich finde: eine ganze Menge.Das ERP fördert nicht nur unsere Wirtschaft; es ist auch ein Zeichen der Verbundenheit, der Verbundenheit von Deutschen und Amerikanern. Wir finanzieren damit zum Beispiel USA-Stipendien, unterstützen gerade in Zeiten wie diesen eine junge Generation, weil wir mit Völkerverständigung an einer guten, einer gemeinsamen Zukunft arbeiten wollen.Aber wir unterstützen das ERP auch, weil es unserem Ansatz von Nachhaltigkeit und Innovation entspricht. Mit dem Gesetz legen wir endlich den Grundstein für mehr Risikokapital durch die KfW. Das ist ein wichtiger Schritt.Aber – jetzt kommen zwei Aber –: Erstens. Wir müssen sicherstellen, dass die KfW transparent über ihre Beteiligungen berichtet, nicht weil wir der bessere Unternehmer sind, aber weil wir unserer Aufgabe der demokratischen Kontrolle nachkommen müssen. Zweitens frage ich mich, warum wir die vollen 200 Millionen Euro Venture-Capital erst ab dem Jahr 2020 bereitstellen wollen. Kein Gründer da draußen sagt: Das ist okay; jetzt warte ich zwei Jahre mit meiner Idee. – Das muss schneller gehen, meine Damen und Herren.In Ihrem Koalitionsvertrag steht – ich zitiere –:Wir wollen, dass Ideen aus Deutschland auch mit Kapital aus Deutschland finanziert werden können.Mehr Risikokapital, das ist gut; aber das hier klingt für mich, ehrlich gesagt, ein bisschen zu sehr nach „­Germany first“. Fragen Sie mal die Gründerszene da draußen! Woher das Kapital kommt, ist am Ende nicht so wichtig.Wichtig ist, dass wir endlich mehr Innovation auf die Straße kriegen. Da muss ich ganz ehrlich sagen: Die abgewählte Große Koalition hat in den letzten vier Jahren insgesamt viel zu wenig für die Gründerszene gemacht. Schauen Sie einmal in den Doing Business Report der Weltbank. Da steht Deutschland im Bereich der Gründungsfreundlichkeit wo? Auf Rang 113; ganz im Ernst. Das ist nicht Champions League, das ist nicht zweite Liga, das ist Kreisklasse-Süd. Das ist zu wenig für ein Land wie die Bundesrepublik Deutschland, meine Damen und Herren.Ja, unserer Wirtschaft geht es gut; aber damit es so bleibt, müssen wir endlich ein paar Dinge anstoßen, wie die Gestaltung der Digitalisierung und die ökologische Modernisierung. Das sind Riesenaufgaben. Dafür brauchen wir Gründer, die mutig sind, die neue Wege gehen, die sich eben nicht mit der GroKo-Methode zufriedengeben, die da heißt: business as usual.Ganz ehrlich: Wir hatten bei Jamaika wirklich gute Dinge verhandelt:Stipendien und Darlehen für Gründer, Bürokratieabbau, Crowdfunding, Einstieg in die digitale Bildung.Aber das mit der Kultur des Scheiterns für Start-ups haben einige bei den Sondierungen bekanntlich zu wörtlich genommen. Schade, gerade für die Gründerszene.Die haben Sie wirklich hängen lassen, liebe Kollegen von der FDP.Herr Kollege Kemmerich, ich beglückwünsche Sie zu Ihrer ersten Rede, aber ich hätte mir gewünscht, dass Sie Ihre erste Rede hier als Regierungspolitiker halten und sagen würden, was Sie alles Gutes für den Mittelstand gemacht haben. Ich hätte mir gewünscht, dass Sie auch darüber mit dem Mittelständler gesprochen hätten, den Sie letzte Woche besucht haben.Wir kriegen gerade viele neue Ideen aus Frankreich und müssen erst einmal überlegen: Wie finden wir eigentlich diese Ideen? Ich frage mich: Warum wagen wir eigentlich nicht mal etwas Neues? Warum schlagen eigentlich wir Macron nicht einmal eine Idee vor? Warum nicht zum Beispiel so etwas wie einen deutsch-französischen Start-up-Fonds, der unsere Gründerszene hier mit der in Paris, wo gerade unheimlich viel passiert, zusammenbringt?Das wäre doch mal was Neues. Frankreich ruft die Start-up-Nation aus, Sie wollen lieber das Heimatministerium. Heimat finde ich auch wichtig; verstehen Sie mich nicht falsch. Aber um Heimat zu bewahren, braucht es eben nicht nur Folklore, es braucht auch Fortschritt.Bis 2025 wollen Sie 3,5 Prozent vom BIP für Forschung und Entwicklung investieren. Heimat und Hightech – das geht zusammen. Ich sage Ihnen einmal, wo: In Baden-Württemberg, in meiner Heimat, liegen wir heute schon dank einer klugen Politik und dank eines innovativen Mittelstandes bei Investitionen in Höhe von fast 5 Prozent. In Bayern sind es gerade gute 3 Prozent. Von der grünen Regierung kann man in dem Bereich noch richtig was lernen, liebe Kollegen von der CSU.Deswegen: Mehr Einsatz für Forschung, Entwicklung und Start-ups. Viele kluge Köpfe da draußen warten auf Sie. Das ERP-Vermögen ist dafür eine richtig gute Basis. Machen Sie was draus!Danke schön.




	40. Eva Högl (SPD) Einen schönen guten Abend! Sehr geehrter Herr Präsident! Liebe Kolleginnen und Kollegen! Meine sehr geehrten Damen und Herren! Schwangerschaftsabbruch ist ein wirklich sensibles und durchaus schwieriges Thema. Ungewollte Schwangerschaften führen nicht selten zu Konflikten, für die betroffenen Frauen auf jeden Fall zu schwierigen Entscheidungen. Deshalb, liebe Kolleginnen und Kollegen, ist es so wichtig, dass sich schwangere Frauen gut und ausführlich informieren können, bevor sie eine solche Entscheidung treffen.Dabei spielen Ärztinnen und Ärzte eine ganz entscheidende Rolle.Ärztinnen und Ärzte sind diejenigen, die mit Sachverstand, mit Erfahrung kompetent über Schwangerschaftsabbrüche informieren und aufklären können. Nichts anderes hat die Gießener Ärztin Kristina Hänel getan.Sie hat auf ihrer Internetseite eine PDF-Datei mit allgemeinen Informationen über Schwangerschaftsabbrüche eingestellt und auf die Möglichkeit hingewiesen, diese in der Praxis vornehmen zu lassen. Sie wurde deswegen nach § 219a StGB vom Amtsgericht Gießen am 24. November letzten Jahres zu 6 000 Euro Geldstrafe verurteilt.Lassen Sie eine Zwischenfrage zu?Nein. – Und zwar deswegen, weil das Amtsgericht der Meinung war, dass schon diese objektive Information den Tatbestand Werbung erfülltund dass bereits die Tatsache, dass Ärztinnen und Ärzte ein Honorar bekommen, einen Vermögensvorteil darstellt.Die Folgen dieses Urteils sind gravierend, liebe Kolleginnen und Kollegen. Sie führen einerseits zu Rechtsunsicherheit bei Ärztinnen und Ärzten. Denn wo ist die Abgrenzung zwischen objektiver Information, auf die alle Frauen angewiesen sind, und Werbung? Ich persönlich bin der Auffassung, diese Entscheidung greift auch in die Berufsfreiheit der Ärztinnen und Ärzte nach Artikel 12 Absatz 1 Grundgesetz ein; denn sie sind nicht mehr in der Lage, objektiv zu informieren.Die gravierendsten Auswirkungen hat es für die betroffenen schwangeren Frauen; denn sie werden unzumutbar in ihrer Möglichkeit beschränkt, sich einen Arzt oder eine Ärztin frei zu wählen und sich informieren zu lassen.Deshalb ist die Position der SPD-Bundestagsfraktion ganz klar: § 219a muss gestrichen werden.Wir haben das bereits am 11. Dezember in einem Gesetzentwurf formuliert.Liebe Kolleginnen und Kollegen, ich will in Richtung Union ganz deutlich sagen: Dieses eine Urteil zeigt, dass wir hier als Gesetzgeber im Deutschen Bundestag Handlungsbedarf haben.Wir können das nicht den Gerichten überlassen, und wir können auch nicht warten, bis es 100 weitere Urteile gibt. Es gibt massiv Anzeigen und auch schon Klagen. Wir müssen vielmehr erkennen, dass es hier Handlungsbedarf gibt.§ 219a, liebe Kolleginnen und Kollegen, ist nicht mehr zeitgemäß.Denn Schwangerschaftsabbrüche sind noch unter Strafe gestellt, aber sie sind straffrei. Das ist das Gesamtkonstrukt des §§ 218 ff., und § 219a bewirkt, indem er zu einer Strafbarkeit für objektive Information führt, genau das Gegenteil dessen, was eigentlich mit dem Gesamtkonstrukt gemeint ist.Lassen Sie eine Zwischenfrage zu?Nein.Wir dürfen uns hier im Bundestag nicht wegducken. Wir haben Handlungsbedarf. Wir müssen den § 219a streichen oder ändern.Ich werbe noch einmal ganz ausdrücklich dafür, dass wir das fraktionsübergreifend machen.Wir haben ja gute Erfahrungen damit gemacht, liebe Kolleginnen und Kollegen – bei der Reform des § 218, bei der Quote und zuletzt in der letzten Legislaturperiode bei der Reform des Sexualstrafrechts –, dass wir versuchen, bei diesen sensiblen Fragen im Deutschen Bundestag eine breite Mehrheit zu erzielen. Es ist eine Gewissensentscheidung. Es ist ein ganz sensibles Thema, wie ich schon sagte, und ich hoffe sehr, dass wir gemeinsam im Deutschen Bundestag erstens den Handlungsbedarf erkennen und zweitens eine wirklich gute Lösung finden: für das Selbstbestimmungsrecht der Frauen, für die Rechte der Ärztinnen und Ärzte und natürlich letztendlich auch für den Schutz des ungeborenen Lebens.Vielen Dank.




	41. Alexander Ulrich (BSW) Frau Präsidentin! Liebe Kolleginnen und Kollegen! Selbstverständlich ist es richtig und wichtig, dass wir mittelständische Unternehmen bei Gründungen und Innovationen fördern, gerade auch dann, wenn sie aus strukturschwachen Regionen kommen. Deshalb kann ich schon zu Beginn sagen: Auch unsere Fraktion wird diesem Gesetz zustimmen.Wenn man sich die einzelnen Redner anschaut, stellt man fest: Wir haben, was dieses Gesetz angeht, wirklich eine große Mehrheit. Trotz alledem will ich auf Herrn Komning von der AfD-Fraktion eingehen. Manchmal ist es ganz gut, dass man sich hier im Bundestag klar äußert. Sie haben hier für eine Sonderwirtschaftszone für strukturschwache Regionen plädiert, und Sie haben damit auch Ostdeutschland gemeint. Damit jedem klar ist, was damit gemeint ist: „Sonderwirtschaftszonen“ heißt schlechteres Arbeitsrecht, weniger Arbeitnehmerrechte, noch geringere Löhne, weniger Steuern und Abgaben. Gerade das lehnt die große Mehrheit dieses Hauses ab. Wir brauchen keine Sonderwirtschaftszone Ost.Wir brauchen im 28. Jahr der Wiedervereinigung eine Angleichung Ost an West. Dafür streiten wir als Fraktion Die Linke.Die Mittel in diesem Programm sind begrenzt. Bekanntlich soll das Sondervermögen stabil gehalten werden. Wie es entstanden ist, haben die Vorredner schon erwähnt. Umso wichtiger ist es natürlich, diese Gelder effektiv und zielgerichtet einzusetzen. Da haben wir als Fraktion aber schon ein paar Bedenken und Zweifel. Ein Großteil der Förderung beruht weiterhin auf Zinserleichterungen. Wir bezweifeln sehr, dass das angesichts der Niedrigzinsphase wirklich sinnvoll ist. Die Zinsbelastung ist gerade nicht das Problem. Hier lösen wir eher Mitnahmeeffekte aus, als dass wir Wachstumsimpulse in schwachen Regionen setzen.Dann müssen die ausgeuferten Verwaltungskosten dringend reduziert werden – ich habe es auch im Ausschuss schon angesprochen –,damit die Fördermittel auch wirklich bei den Unternehmen ankommen. Die Förderkosten des Kreditneugeschäfts sollen 2018  42,7 Millionen Euro betragen. Laut Bundesrechnungshof entfallen 16,6 Millionen Euro auf die KfW-Bearbeitungsgebühr. Das sind sage und schreibe 39 Prozent. Das geht so nicht. Das muss vom Bundestag und vom Ministerium nochmals dringend überprüft werden.Wir müssen bei den wirtschaftspolitischen Maßnahmen immer auch sagen: Inwiefern dient das insgesamt der deutschen Wirtschaft oder der europäischen Wirtschaft? Was wir angesichts der riesigen Außenhandelsüberschüsse, die wir in Deutschland haben, kritisieren, ist, dass auch über dieses Programm Exporte unterstützt werden. Das lehnen wir ab. Diese Förderung sollte aufgehoben werden.Kurzum: So wichtig die Wirtschaftsförderung aus diesem Programm ist, so wichtig ist es, dass wir es besser mit dem vernetzen, was in Deutschland wirtschaftspolitisch notwendig und sinnvoll ist. Da haben wir das Hauptproblem, dass die Effekte der Förderung verpuffen, solange wir die Sparpolitik der letzten 20 Jahre in Deutschland so fortführen. Gerade in den schwachen Regionen fehlt es in erster Linie an öffentlicher und privater Nachfrage. Solange die Kaufkraft und die Investitionstätigkeit zu schwach sind und am Boden liegen, kann auch die beste Wirtschaftsförderung nicht nachhaltig wirken. Deshalb müssen die Fördermaßnahmen um eine nachfrageorientierte Politik ergänzt werden.Wir brauchen öffentliche Investitionen in Bereichen wie dem ökologischen Umbau, dem sozialen Wohnungsbau und gerade in ländlichen Regionen im Gesundheitssystem. Statt einer Schuldenbremse und der schwarzen Null brauchen wir einen Privatisierungsstopp und eine finanzielle Besserstellung der Kommunen.Wir brauchen eine Reform der Gemeindefinanzen, mit der unter anderem die Gewerbesteuer zu einer Gemeindewirtschaftssteuer ausgebaut wird. Gerade davon würden KMU richtig profitieren.Wir müssen endlich aufhören, weiter am Niedriglohnsektor festzuhalten. Dazu muss man leider feststellen: Mit der Großen Koalition, die jetzt wieder geplant ist, wird sich an den prekären Arbeitsverhältnissen, an den Millionen von Beschäftigten im Niedriglohnsektor nichts verändern.Abschließend, Frau Präsidentin: Die SPD hat versucht, weiszumachen, dass man in der langen Nacht der Koalitionsverhandlungen noch einmal stundenlang und hart um prekäre Beschäftigung, Leiharbeit und Bürgerversicherung gerungen habe.Herr Seehofer hat es gesagt: Es ging euch gar nicht darum. Es ging euch nur um Ministerposten.Eine Schande, was aus dieser SPD geworden ist!




	42. Mariana Iris Harder-Kühnel (AfD) Sehr geehrter Herr Präsident! Sehr geehrte Damen und Herren!Die Würde des Menschen ist unantastbar. Sie zu achten und zu schützen ist Verpflichtung aller staatlichen Gewalt.So steht es in Artikel 1 unseres Grundgesetzes.Jeder hat das Recht auf Leben …So steht es in Artikel 2 unseres Grundgesetzes.Menschenwürde und das Recht auf Leben kommen natürlich auch dem ungeborenen menschlichen Leben zu.Das hat das Bundesverfassungsgericht in zahlreichen Entscheidungen bestätigt.Die AfD ist die Partei der Rechtsstaatlichkeit.Wir treten dafür ein, dass der Staat dieser verfassungsrechtlichen Verpflichtung nachkommt und auch das ungeborene Leben schützt, meine Damen und Herren.Daher werden wir am Werbeverbot für Schwangerschaftsabbrüche nach § 219a festhalten.Die vorliegenden Anträge der Grünen und der Linken sind voll von Fehlinformationen. Sie wollen den Bürgern weismachen, dass die bloße Information über Schwangerschaftsabbrüche durch Ärzte nach § 219a strafbar ist. Sie wollen den Bürgern weismachen, dass Schwangeren nach § 219a der Zugang zu diesen Informationen verwehrt wird. Dies, meine Damen und Herren, ist schlicht und ergreifend falsch. Auch hier gilt die alte Juristenweisheit: Ein Blick ins Gesetz erspart viel Geschwätz.Die ausführliche Information und Beratung von Schwangeren ist in den §§ 218 ff. ausdrücklich vorgeschrieben. Sie ist sogar zwingende Voraussetzung dafür, dass überhaupt ein Schwangerschaftsabbruch straffrei vorgenommen werden darf. §§ 218 ff. sind das Ergebnis einer jahrzehntelangen politischen Diskussion,der Abwägung zwischen dem Lebensrecht des Ungeborenen einerseits und dem Selbstbestimmungsrecht der Frau andererseits.Lässt sich eine Frau beraten und will sie auch nach der Beratung noch abtreiben, dann erhält sie eine Liste mit Namen von zur Abtreibung bereiten Ärzten.Offene Werbung dafür, dass ein Arzt eine solche Leistung anbietet, ist also gar nicht nötig, was im Übrigen durch die hohe Zahl von noch immer 100 000 Abtreibungen bei nur 700 000 Geburten pro Jahr bestätigt wird – 100 000 Menschen, 100 000 Kinder, die auch leben wollten.Tatsächlich ist die Information sogar nur einen Klick entfernt. Machen Sie den Selbstversuch! Geben Sie das Wort Schwangerschaftsabbruch einmal bei Google ein. In wenigen Sekunden finden Sie im Internet alles, was Sie jemals über Schwangerschaftsabbrüche wissen bzw. eigentlich niemals erfahren wollten. Nein, § 218a stellt nicht die Information über Schwangerschaftsabbrüche unter Strafe,sondern die Kommerzialisierung, das heißt, das Werben. Es soll verhindert werden, dass Schwangerschaftsabbrüche – zur Erinnerung, wir reden hier immerhin über die Tötung menschlichen Lebens –in der Öffentlichkeit als etwas Normales dargestellt werden, für das Reklame gemacht werden darf.Wohin, liebe Genossinnen und Genossen, soll denn eine ersatzlose Streichung des § 219a führen?Es muss dann mit offener Werbung im Fernsehen, im Internet, im Radio gerechnet werden. Wo, liebe FDP, beginnt grob anstößige Werbung? Ist Werbung für Abbrüche an Litfaßsäulen anstößig oder erst dann, wenn man drei Abbrüche zum Preis von zweien anbietet?Wie verstörend wäre es, wenn gerade Ärzte, die sich dem Schutz des Lebens verpflichtet haben, Maßnahmen zur Tötung von Leben aktiv bewerben!Streicht man § 219a und lässt die Werbung auch noch durch den Arzt zu, der die Abtreibung selbst vornimmt, dann wird die gesamte Systematik der Konfliktberatung konterkariert; denn die Beratung soll nach dem Gesetz so erfolgen, dass die Frau zur Fortsetzung der Schwangerschaft ermutigt wird. Ihr soll bewusst werden, dass das Ungeborene in jedem Stadium der Schwangerschaft auch ihr gegenüber ein eigenes Recht auf Leben hat. Gerade in dieser schwierigen Entscheidungsphase soll eine ergebnisoffene Beratung stattfinden. Zu glauben, dass eine ergebnisoffene Beratung durch einen Arzt erfolgt, der mit dem Abbruch wirbt und damit sein Geld verdient,ist schlicht und ergreifend naiv, meine Damen und Herren! Man würde den Bock zum Gärtner machen.Übrigens hat die Vorschrift des § 219a in der Praxis kaum Bedeutung. Im Jahr 2016 gab es lediglich eine einzige Verurteilung. Da frage ich Sie: Haben wir keine dringenderen Probleme in Deutschland?Sie konstruieren hier anhand eines Einzelfalls, nämlich der Verurteilung einer Ärztin in Gießen, ein Scheinproblem. Sie suggerieren, dass es Ihnen um die Freiheit der Frau, deren Informationsfreiheit und das Selbstbestimmungsrecht über ihren Körper geht, nach dem Motto „Mein Bauch gehört mir“. Ich aber sage Ihnen: Die Freiheit der Frau und ihre körperliche Unversehrtheit sind in Deutschland und millionenfach weltweit durch ganz andere Dinge gefährdet. Setzen Sie sich damit einmal ideologiefrei auseinander!Die AfD lehnt sowohl die ersatzlose Streichung als auch die Aufweichung des Werbeverbots bei Schwangerschaftsabbrüchen ab. Lassen Sie uns lieber dafür werben, dass sich werdende Eltern und alleinstehende Frauen für das Leben entscheidenund hierfür jede erdenkliche staatliche Hilfe erhalten.Denn die Alternative für Deutschland steht für eine Willkommenskultur für Kinder,für eine Kultur des Lebens und nicht für die Kommerzialisierung des Tötens.Vielen Dank.




	43. Enrico Komning (AfD) Frau Präsidentin! Meine Damen und Herren! Wir beraten heute abschließend über die diesjährige Verwendung des ERP-Sondervermögens, ein Förderinstrument, das uns als Marshallplan noch in bester Erinnerung ist, das aber im heillosen Durcheinander von Fördertöpfen untergeht und gerade in Mittel- und Ostdeutschland kaum Wirkung entfaltet. Sonst gäbe es hier nämlich schon lange die blühenden Landschaften, die sie nach der Wende nie geworden sind. Es fragt sich, ob die Gesamtstruktur der Wirtschaftsförderung überhaupt noch zukunftsfähig ist; denn fette Hennen gibt es nur noch in der Politik. Der Mittelstand und das Handwerk magern immer mehr zum Suppenhuhn ab.Die Förderpolitik der Großen Koalition ist in meinem Wahlkreis, im östlichen Mecklenburg und in Vorpommern, schlicht unsichtbar. In Wolgast verfallen die Häuser, in Pasewalk gibt es löchrige Straßen, in Neubrandenburg lebt jeder Sechste von Hartz IV, und Anklam ist nach wie vor eine Internetwüste. Was glauben Sie eigentlich, warum uns die Unternehmer und die ganzen jungen Menschen verlassen? Mit diesem Förderchaos werden Sie sie nicht halten und erst recht nicht zurückholen können.Meine Damen und Herren, ja, ERP wäre ein wichtiges Förderinstrument,vor allem auch für die strukturschwachen Gebiete in Mittel- und Ostdeutschland, und Förderpolitik ist eine Kernaufgabe des Staates, und zwar dann, wenn der freie Markt nicht funktioniert. Dies tut er in weiten Teilen Deutschlands eben nicht. Um das weitere wirtschaftliche und demografische Ausbluten aufzuhalten, brauchen wir aber eine gesamtheitliche, kluge Standortpolitik sowie Förderbaukästen, und zwar zugeschnitten auf die konkret notwendigen Förderbedürfnisse vor Ort. Absichtsbekundungen der hier schon länger Regierenden gibt es ja viele; einzig der Umsetzungswille fehlt.Über vieles wurde in der letzten Legislaturperiode gesprochen. Aber über Absichtserklärungen sind Sie leider nicht hinausgekommen. Sie gackern laut, legen aber keine Eier. Wer keine Eier hat, sollte nicht regieren.Wie schon der Bundesrechnungshof in seinem Bericht an den Hauptausschuss kritisiert, bleibt die ERP-Förderung hinter ihren Möglichkeiten zurück. Die 2007 vereinbarte Förderzielgröße von heute inflationsbereinigten 345,4 Millionen Euro wird mit lediglich geplanten 287,4 Millionen Euro, also 60 Millionen Euro weniger, bei weitem nicht ausgeschöpft. Und was die Transparenz der neu zu gründenden KfW-Tochter angeht, ist strittig, ob es nur eine gesetzliche Prüfberechtigung des Bundesrechnungshofes oder eine weiter gehende vertragliche Prüfvereinbarung geben soll. Laut Staatssekretär Beckmeyer – er ist anwesend – soll der Bundesrechnungshof dort „auch mal die Lampe reinhalten“ dürfen. Dem Bundeswirtschaftsministerium ist zum Thema „zukunftsorientierte Wirtschaftsförderung“ offensichtlich noch nicht einmal ein Licht aufgegangen. Sie laufen mit der Kerze durch die Gegend. Der deutsche Mittelstand benötigt aber Flutlicht.Ich frage Sie: Wie ist es denn zu erklären, dass die bereitgestellten Fördermittel für Unternehmensgründer zu lediglich einem Drittel in Anspruch genommen wurden? Ich sage es Ihnen: Die Rahmenbedingungen für Neugründungen in Deutschland sind trotz ERP-Förderung und anderer Förderungen einfach schlecht. Auf mageren Böden vermag selbst der wärmste Regen die beste Saat nicht aufgehen zu lassen.Wir brauchen niedrige Steuern, endlich schnelles Internet, günstigen Strom und wenig Bürokratie. Besonders arme Regionen müssen besonders gut behandelt werden. Da muss man sich die Frage stellen, ob man nicht für abgrenzbare strukturarme Gebiete handels- und unternehmensrechtliche Sonderregelungen zulässt, um so Standortnachteile auszugleichen. Das Nachdenken über spezielle Steuerregelungen sowie über die Abschaffung oder zumindest die drastische Reduzierung behördlicher Genehmigungsverfahren in solchen Gebieten darf kein Tabu sein. Wenn Sie meinen, dass rechtliche Sonderregelungen für einzelne Regionen in Deutschland nicht möglich seien, sage ich Ihnen: Wir sind der Gesetzgeber. Machen wir es möglich, für unser Land, für unsere Landsleute und für unseren deutschen Mittelstand.In der Hoffnung, dass das ERP-Vermögen endlich in ein ganzheitlich abgestimmtes Fördersystem Eingang findet, stimmen wir heute dem Entwurf eines ERP-Wirtschaftsplangesetzes noch zu; denn wenig ist immer noch besser als nichts.Vielen Dank.Der nächste Redner, den ich aufrufen möchte – es ist seine erste Rede im Deutschen Bundestag –, ist Thomas Kemmerich für die FDP-Fraktion. – Ich bin noch immer etwas irritiert durch den Ausflug in die Welt der Eier.




	44. Cem Özdemir (BÜNDNIS 90/DIE GRÜNEN) Frau Präsidentin! Sehr geehrte Kolleginnen und Kollegen!Man muss sich vergegenwärtigen, worüber wir heute tatsächlich reden. Wir reden über die Arbeit und die Artikel eines deutschen Journalisten. So etwas kennen wir sonst nur aus autoritären Ländern. Der Deutsche Bundestag hingegen benotet nicht die Arbeit von Journalisten und Journalistinnen. Bei uns in der Bundesrepublik Deutschland ist das Parlament keine oberste Zensurbehörde. So etwas gibt es nur in den Ländern, die Sie bewundern. Deutschland gehört nicht dazu.In unserem Land, der Bundesrepublik Deutschland, gibt es nicht die Gleichschaltung, von der Sie nachts träumen. Bei uns gibt es Pressefreiheit, ein Wort, das in Ihrem Wortschatz ganz offensichtlich nicht vorhanden ist.Die Pressefreiheit werden wir Ihnen gegenüber genauso verteidigen wie gegenüber Ihren Genossen in der Türkei, die Deniz Yücel ein Jahr seines Lebens geklaut haben.Wir sind froh, dass Deniz Yücel frei ist. Damit kein Missverständnis entsteht: Genauso froh wären wir, wenn er Gustav Müller oder sonst wie heißen würde; denn jeder Bürger dieses Landes hat es verdient, dass sich dieses Land für ihn einsetzt; das ist doch wohl eine Selbstverständlichkeit. Jeder weiß es, außer Ihnen.Wir alle, der demokratische Teil dieses Hauses,setzen uns dafür ein, dass die anderen Journalisten, die ebenfalls in Haft sind, aber keinen deutschen Pass haben, freigelassen werden – sie haben es genauso verdient –; denn Journalismus ist kein Verbrechen.Aber zur Wahrheit gehört leider auch: Das Land hat sich in dem einen Jahr, in dem Deniz Yücel im Gefängnis war, dramatisch verändert, und davon zeugt diese Debatte. Denn mittlerweile sitzen Abgeordnete in diesem Haus, die ich nicht anders als Rassisten bezeichnen kann. Wer sich so gebiert, ist ein Rassist.Ich meine diese Damen und Herren hier ganz rechts. Ich stehe am Mikrofon und Gott sei Dank können Sie es mir nicht abstellen. Ich weiß, in dem Regime, von dem Sie träumen, könnte man das Mikrofon abstellen; aber das kann man hier Gott sei Dank nicht. Sie werden es nicht schaffen, das zu ändern. Glauben Sie es mir!Sie wollen bestimmen, wer Deutscher ist und wer nicht.Wie kann jemand, der Deutschland, der unsere gemeinsame Heimat so verachtet, wie Sie es tun, darüber bestimmen, wer Deutscher ist und wer nicht Deutscher ist?Ich sage Ihnen mal eins: Wenn Sie darüber bestimmen würden, wer Deutscher ist und wer nicht Deutscher ist, dann wäre das ungefähr so, als wenn man Rassisten an das Ausstiegstelefon für Neonazis setzen würde.Übrigens, wenn Sie die Nummer des Ausstiegstelefons für Neonazis brauchen: Ich habe sie. Ich kann sie Ihnen gern zur Verfügung stellen.Kollege Özdemir, gestatten Sie eine Zwischenfrage?Nein, ich gestatte keine Zwischenfrage.Sie alle von der AfD, wie Sie da sitzen, würden, wenn Sie ehrlich wären, zugeben, dass Sie dieses Land verachten.Sie verachten alles, wofür dieses Land in der ganzen Welt geachtet und respektiert wird. Dazu gehört beispielsweise unsere Erinnerungskultur, auf die ich als Bürger dieses Landes stolz bin.Dazu gehört die Vielfalt in diesem Land, auf die ich genauso stolz bin. Dazu gehören Bayern, Schwaben, dazu gehören aber auch Menschen, deren Vorfahren aus Russland kommen, und dazu gehören Menschen, deren Vorfahren aus Anatolien kommen und die jetzt genauso stolz darauf sind, Bürger dieses Landes zu sein.Dazu gehört – das muss ich schon einmal sagen; da fühle ich mich auch als Fußballfan persönlich angesprochen – unsere großartige Nationalmannschaft. Wenn Sie ehrlich sind: Sie drücken doch den Russen die Daumen und nicht unserer deutschen Nationalmannschaft. Geben Sie es doch zu!Dieses Hohe Haus verachten Sie genauso, wie Sie die Werte der Aufklärung verachten. Sie sind aus demselben faulen Holz geschnitzt wie diejenigen, die Deniz Yücel verhaften ließen. Sie sind aus demselben faulen Holz geschnitzt wie Erdogan, der Deniz Yücel für ein Jahr seines Lebens verhaftet ließ. Ich sage es einmal in einem Satz: Die AKP hat einen Ableger in Deutschland. Er heißt AfD, und er sitzt hier.Lassen Sie mich zum Schluss sagen: Sie hatten ja vor kurzem einen politischen Aschermittwoch. Mich hat das eher an eine Rede im Sportpalast erinnert. Ich will Ihnen zurufen: Unser Deutschland, dieses Deutschland, ist stärker, als es Ihr Hass jemals sein wird.Ihr tobender Mob wollte am Aschermittwoch, dass ich abgeschoben werde. Das geht leichter, als Sie sich das vorstellen. Am kommenden Samstag bin ich wieder in meiner Heimat. Ich fliege nach Stuttgart. Dort nehme ich die S-Bahn, und ich steige am Endbahnhof Bad Urach aus. Da ist meine schwäbische Heimat, und die lasse ich mir von Ihnen nicht kaputtmachen.Ein kurzer geschäftsleitender Hinweis: Es kann durchaus passieren, ganz egal, wer aus dem Präsidium des Deutschen Bundestages hier vorn gerade Dienst hat, dass es Entscheidungen oder auch Handlungen von amtierenden Präsidentinnen und Präsidenten gibt, die einen Fehler enthalten oder aber nachträglich in irgendeiner Weise zu rügen sind. Dafür haben wir Regeln. Das hat einen guten Grund. Jedenfalls tragen wir das nicht im Plenum des Bundestags miteinander aus, sondern der Ort für diese ganze Geschichte ist der Ältestenrat.Wenn einzelne Abgeordnete, und zwar wechselseitig – ich nehme hier Aufforderungen sehr wohl zur Kenntnis, trotz des hohen Lärmpegels –, mit meiner Sitzungsleitung unzufrieden sind, dann werden wir das am nächsten Donnerstag, 14 Uhr, in der nächsten regulären Sitzung des Ältestenrats behandeln; ansonsten wissen, denke ich, die Geschäftsführerinnen und Geschäftsführer, wie wir, wenn es notwendig werden sollte, zu einer Sondersitzung des Ältestenrats kommen, um diese Dinge zu regeln.Im Übrigen – das will ich dann auch gleich sagen, Herr Baumann –: Weder konnte ich hier vorn alle Zwischenrufe – wieder: aus allen Fraktionen – verstehen, aufgrund des sehr hohen Lärmpegels– jetzt bin ich dranund nur ich –,noch war es mir möglich, in dieser hitzigen Debatte alle Äußerungen, die hier vorn gemacht wurden – das gilt nicht nur für den letzten Beitrag –, in ihrer Tragweite und Wirkung bis zuletzt zu überschauen. Aber auch dafür haben wir Regeln. Sollte es in dieser Debatte Dinge gegeben haben, die die Regeln überschritten haben, haben wir den Ort, über den ich gerade sprach, um gegebenenfalls auch nachträglich etwas in irgendeiner Weise zu sanktionieren oder einzugreifen.Ich habe im Moment keine Veranlassung, von hier vorn die Debatte weiter aufzuhalten. Ich nehme die Dinge, die hier angemeldet wurden, zur Kenntnis – wir werden sie einer entsprechenden Beratung zuführen –, und ich bitte jetzt, den geschäftsleitenden Hinweis in der weiteren Debatte zu befolgen.




	45. Christine Buchholz (DIE LINKE.) Herr Präsident! Meine Damen und Herren! Die AfD fordern nun also ein Verbot der Vollverschleierung im öffentlichen Raum. Lassen Sie mich es vorneweg sagen: Wir lehnen den Zwang, eine religiöse Bekleidung zu tragen, wie den Zwang, sie abzulegen, ab.Die Abgeordneten der AfD wissen höchstwahrscheinlich, dass das Verbot gegen Artikel 4 des Grundgesetzes verstieße. Der vorliegende Antrag der AfD ist damit für den Papierkorb. Wieder eröffnet die AfD eine rassistische Scheindebatte.Eines ist klar: Es geht der AfD weder um die betroffenen Frauen noch um weibliche Selbstbestimmung.Das Programm der AfD ist frauenfeindlich.Sie will das Antidiskriminierungsgesetz abschaffen und Geschlechterquoten streichen. Sie ist sogar gegen Aktionen für gleichen Lohn für gleiche Arbeit wie den Equal Pay Day. Auch zahlreiche Abgeordnete sind schon durch frauenfeindliche Äußerungen aufgefallen; Herr Boehringer ist da nicht alleine.Schauen Sie einmal die sexistischen Hasskommentare Ihrer Anhänger an. Das ist einfach nur widerlich.Den Frauen, die gegen ihren Willen Burka oder Nikab tragen, hilft das Verbot überhaupt nicht.Frau Kollegin, erlauben Sie eine Zwischenfrage des Kollegen Brandner, AfD?Nein, die haben genug Redezeit. – Die AfD will offenbar, dass Frauen, die gegen ihren Willen Burka oder Nikab tragen, nicht einmal das Haus verlassen können, um sich öffentlich zu bewegen, einen Arzt zu besuchen, an einem Elternabend teilzunehmen oder um Hilfe in Anspruch zu nehmen. Es ist absolut lächerlich, dass sich die AfD zur Anwältin der weiblichen Selbstbestimmung aufspielt.Worum geht es der AfD also? Es geht darum, weiter rassistische Vorurteile gegen Muslime und den Islam zu verbreiten. Die AfD behauptet, unter Muslimen würde ohne Verbot der Vollverschleierung ein Gruppendruck zur Vollverschleierung entstehen.Hier entsteht vor Ihrem geistigen Auge das Bild einer „kulturellen Landnahme“, das eben auch Herr Curio bemüht hat.Das ist absoluter Bullshit.Die übergroße Mehrheit der muslimischen Frauen in Deutschland ist nicht vollverschleiert und sieht die Vollverschleierung als nicht geboten an. Das weiß übrigens auch jeder, der mit muslimischen Frauen spricht, anstatt sie zu verfolgen.Die AfD stellt hier eine Behauptung auf, die nichts mit der Realität zu tun hat, sondern ausschließlich ihrer rassistischen Paranoia entspringt.Die AfD schreibt in ihren Antrag, dass der Islam an sich unverschleierte Frauen als ehrlos markiere. Auch das ist eine reine Behauptung, die durch nichts belegt ist.In Wirklichkeit nehmen hier in Deutschland Hass und Gewalt gegen Muslime zu, vor allem gegen muslimische Frauen. Gerade vor zwei Tagen wurde eine Frau in einem Berliner Supermarkt verletzt, als ihr gewaltsam der Schleier vom Kopf gerissen wurde. Das ist das Resultat Ihrer Hetze, meine Damen und Herren von der AfD.Aber auch die CDU/CSU befördert diese Scheindebatte seit Jahren, wenn sie immer wieder über das Thema Burka redet. Herr Mayer hat das eben noch einmal vorexerziert. Leider geht auch der vorliegende Koalitionsvertrag der AfD auf den Leim, wenn in ihm ein Vollverschleierungsverbot in Gerichten angekündigt wird; denn hier gibt es ausreichend Regelungen.Die Linke fordert Hilfe und Unterstützung für alle Frauen, denen Zwang angetan wird, ob dieser Zwang nun religiös begründet wird oder nicht. Wir stehen an der Seite aller Frauen, die sich gegen ihre Unterdrückung und für ihre Rechte einsetzen. Was diese Frauen am allerwenigsten brauchen, ist der vorliegende Antrag der AfD.Herzlichen Dank, Frau Kollegin Buchholz. – Ich muss sagen, die Frauen treffen die Redezeit immer ganz genau, während die Männer deutlich überziehen. Das werden wir demnächst ändern.




	46. Philipp Amthor (CDU) Frau Präsidentin! Liebe Kolleginnen und Kollegen! Um es noch einmal ganz deutlich zu sagen: Die CDU/CSU-Bundestagsfraktion hat sich dafür eingesetzt, die Vollverschleierung im öffentlichen Raum zu begrenzen, und wir werden das auch fortsetzen. Das ist mit den Sozialdemokraten vereinbart, und dafür setzen wir uns ein.Wir wollen das alles aber verfassungskonform tun, und diesem Anliegen wird der Antrag der AfD einfach nicht gerecht. Bevor ich dazu komme, will ich aber, weil das in der Debatte bisweilen unterging, sagen: Überall, wo es rechtlich möglich ist, werden wir uns der Vollverschleierung entgegenstellen,und zwar deshalb, weil wir fest davon überzeugt sind, dass Burka und Nikab in keiner Weise unserer Vorstellung von einem Rechtsstaat und einer deutschen Wertekultur entsprechen.Ich sage hier ausdrücklich: Die Religionsfreiheit gehört zu Deutschland, aber der politische Islam gehört nicht zu Deutschland, und gegen ihn werden wir uns wenden.Und überall dort, wo der politische Islam versucht, unsere offene Lebensweise zu beschränken, werden wir ihm mit der Härte des Rechtsstaates begegnen. Da sind mir auch die Zahlen, wie wenige das sind, egal – das wird es mit uns nicht geben.Nun aber in Richtung AfD, bevor Sie sich zu früh freuen: Wissen Sie, wenn wir von den Moslems verlangen, dass sie sich an unsere Regeln halten,dann tun wir selbst gut daran, uns auch an unsere eigenen Regeln zu halten. Sie wissen vielleicht: Mit Ihrem Vorschlag operieren Sie ganz deutlich im grundrechtssensiblen Bereich der Religionsfreiheit. Ich kann Ihnen nur die Empfehlung geben: Wenn Sie da operieren, sollten Sie auch Ihr OP-Besteck kennen!Es ist ja nicht nur so, dass Sie den Vorsitzenden des Rechtsausschusses stellen, sondern es ist auch ein Faktum, dass ein Viertel Ihrer Fraktion Juristen sind. Diese Expertise findet sich in dem Antrag aber in keiner Weise wieder.Vielmehr ist der Antrag mit heißer Nadel gestrickt und strotzt – ganz ehrlich – vor falschen Behauptungen. Das fängt schon mit dem Schutzbereich der Religionsfreiheit an. Sie behaupten, Burka-Tragen unterfalle nicht dem Schutzbereich der Religionsfreiheit. Wissen Sie, ich finde das auch nicht toll, aber die ständige Rechtsprechung des Bundesverfassungsgerichts und sogar das EGMR-Urteil, das Sie zitieren, besagen: Die Religionsfreiheit schützt auch das Burka-Tragen im Schutzbereich. – Erster Fehler.Zweiter Fehler: Natürlich kann man Eingriffe in den Schutzbereich der Religionsfreiheit rechtfertigen, aber nach ständiger Rechtsprechung eben nur durch kollidierendes Verfassungsrecht. Dazu führen Sie substanziell gar nichts aus, sondern das, was Sie hier behaupten, und das, was Herr Curio hier behauptet hat, ist grober Unfug.Nehmen wir einmal das Beispiel Menschenwürde. Die Menschenwürde ist eben kein Zwangskorsett, um das ganz deutlich zu sagen. Ich finde eine Burka ziemlich unwürdig, und ich finde es unwürdig, dass sich die Frauen dadurch auch zum Objekt machen lassen.Herr Kollege, erlauben Sie eine Zwischenfrage oder Bemerkung des Kollegen Nolte von der AfD-Fraktion?Aber sehr gern, Herr Nolte. Herzliche Einladung!Verehrter Herr Kollege, Sie reden hier viel über die Verfassung. War die Aufgabe der Grenzkontrolle durch die CDU verfassungskonform, und war die Homo-Ehe verfassungskonform?Herr Nolte, ich finde es schön, dass Sie diese Frage stellen. Ich tendiere dazu, Ihnen zu sagen: Ja. Denn es gibt zu beidem noch kein gegenläufiges Urteil des Bundesverfassungsgerichts.Ich sage Ihnen ausdrücklich: Bei der Homo-Ehe habe auch ich damals verfassungsrechtliche Bedenken gesehen. Darüber muss man reden; da wird es Entscheidungen geben. Aber machen Sie sich eines klar: Hier jetzt so zu argumentieren, als seien wir die Rechtsbrecher – Sie haben nicht einmal das Gutachten des Wissenschaftlichen Dienstes gelesen, worin steht, dass Ihr hier vorgelegter Entwurf verfassungswidrig ist –,ist völliger Unsinn. Machen Sie lieber Ihre Arbeit richtig, und hören Sie mir einmal zu; dann können Sie nämlich noch etwas über die Verfassung lernen.Die Menschenwürde hat Herr Curio als Stichwort angeführt und gesagt, die Menschenwürde sei jetzt hier das dem Entgegenstehende. Ich habe gesagt, ich finde die Burka für die Menschenwürde auch befremdlich; das ist nicht würdig. Aber Fakt ist: Die Menschenwürde schützt gleichzeitig nicht vor der autonomen Entscheidung, eine Burka zu tragen – so sinnlos wir das auch finden. Ich mache Ihnen das an einem Beispiel deutlich, das Sie vielleicht verstehen: Sie nehmen ja hier die Abgeordnetenfreiheit aus Artikel 38 des Grundgesetzes in Anspruch. Die zwingt Sie auch nicht dazu, richtige Anträge vorzulegen, sondern sie erlaubt es Ihnen, hier so einen Quatsch vorzulegen. Ganz ehrlich!Dann führen Sie noch das Recht auf Kommunikation an. Auch das ist absoluter Blödsinn. Wenn Sie sich da auf den EGMR stützen, kann ich Ihnen nur empfehlen, das Urteil zu lesen – und wenn Sie nicht, hätte das wenigstens einmal einer Ihrer Referenten tun sollen. Am Ende des Urteils gibt es nämlich ein Sondervotum der Richterin Angelika Nußberger – sie ist die deutsche ­EGMR-Richterin –, die sagt, dass dieses Recht auf Kommunikation verfassungsdogmatisch schief ist. Das hätten Sie einmal zur Kenntnis nehmen sollen. Aber offene Kommunikation ist ja nicht Ihre Stärke, da Sie sogar Ihre Parteitage verhüllen und da keine Journalisten zulassen. Ganz ehrlich!Sekunde!Ja?Es gibt noch eine Zwischenfrage. Wollen Sie sie zulassen?Gerne. Es gibt so viele Punkte, die ich hier aufarbeiten kann. Machen Sie weiter!Trotzdem gilt auch für Sie die Redezeit. – Gut, jetzt haben Sie die Zwischenfrage noch zugelassen.Vielen Dank, Herr Amthor, für das Zulassen der Zwischenfrage. – Ich möchte Ihnen ein Zitat vorlesen, und zwar von CSU-Generalsekretär Andreas Scheuer, und Sie bitten, dazu eine politische Bewertung vorzunehmen, –Gerne.– auch hinsichtlich der in diesem Jahr stattfindenden Landtagswahl in Bayern. Der CSU-Generalsekretär Andreas Scheuer hat im Jahr 2017 Folgendes gesagt:Ein Verbot– gemeint ist hier die Vollverschleierung –ist möglich und notwendig. Das deutsche Verbötchen zur Vollverschleierung muss so wie in anderen Ländern Europas ausgeweitet werden.Weiter:Wir geben unsere Identität nicht auf, sondern sind bereit, dafür zu kämpfen. Die Burka gehört nicht zu Deutschland.Ja, super!Wie ist das in Einklang zu bringen mit den Reden, –Total!– die ich aus Ihrer Fraktion zu diesem Punkt bis jetzt gehört habe?So, jetzt bitte.Ganz ehrlich, hätten Sie mal zugehört! Das ist genau das, was ich gesagt habe.Ich unterschreibe das, was Andi Scheuer gesagt hat, eins zu eins. Schauen Sie nach Bayern! Da kann man das nämlich kompetenziell beschränken. Das hat man in Bayern gemacht. In Bayern gibt es die Vollverschleierung in keiner Schule und in keiner Universität. Das wollen auch wir weiter ausweiten.Ich sage Ihnen zum Abschluss eines: Sie stellen sich hier hin und wollen die Retter des christlichen Abendlandes sein. Wir müssten für Ihr Anliegen aber die Verfassung ändern. Wenn man das machen würde, würden Sie die Axt an die Wurzel der Religionsfreiheit legen. Das Schrankensystem, das ich Ihnen gerade erklärt habe, hat das Bundesverfassungsgericht entwickelt, um christliche Minderheiten zu schützen. Das sollte doch für Sie das Entscheidende sein. Sie sind so weder die Retter des christlichen Abendlandes noch eine Rechtsstaatspartei. Machen Sie Ihre Arbeit ordentlich! Dann können wir im Innenausschuss darüber diskutieren.Herzlichen Dank.Vielen Dank, Philipp Amthor. – In Bayern gibt es auch noch das Dirndl, die Lederhos’n und noch viel mehr. Sie sind herzlich eingeladen, einmal nach Bayern zu kommen.Ich schließe damit die Aussprache.Interfraktionell wird Überweisung der Vorlage – –– Wollen Sie nicht hören, was damit passiert? – Gut.




	47. Wolfgang Wiehle (AfD) Sehr geehrte Frau Präsidentin! Liebe Kolleginnen und Kollegen! Kostenloser ÖPNV – das klingt sympathisch. Aber allein das Wort ist irreführend. Es ist nämlich nichts kostenlos.Der berühmte amerikanische Ökonom Milton Friedman, Nobelpreisträger für Wirtschaftswissenschaften, sagte – Zitat –: „There ain’t no such thing as a free lunch.“Sie alle wissen, was er damit meinte. Und alle wissen auch, wie die Öffentlichkeit die vollmundige Ankündigung der Bundesregierung in dem Brief nach Brüssel verstanden hat, in dem von dem kostenlosen öffentlichen Nahverkehr die Rede ist. Meine beiden Vorredner haben versucht, im Sinne der Bundesregierung die Zahnpasta ein wenig zurück in die Tube zu bekommen. Ich will die Sache jetzt doch einmal in Gänze betrachten.Wer wird das Ganze bezahlen, meine Damen und Herren? Finanzieren wird es natürlich der Bürger über Steuern und Abgaben.Wir sprechen hier nicht von Peanuts, sondern von einem Verzicht auf über 12 Milliarden Euro Fahrgeldeinnahmen jedes Jahr. Bei der Idee eines kostenlosen ÖPNV handelt es sich also um eine politische Utopie bar jeglicher Rückkopplung zur Wirklichkeit.Wer heute den ÖPNV nutzt, stellt fest, dass Busse und Bahnen zu vielen Zeiten überfüllt und oftmals leider auch unpünktlich sind. Sie agieren im Berufsverkehr schon jetzt an der Kapazitätsgrenze. Noch dazu leidet vielerorts das Wohlbefinden der Fahrgäste, weil Verkehrsmittel und Bahnhöfe zunehmend unsauber und vor allem auch unsicher daherkommen. Denken Sie nur an die Vorfälle der letzten Monate alleine hier in der Bundeshauptstadt.Seit Jahren – inzwischen seit Jahren, meine Damen und Herren! – fordern wir von der AfD eine Verbesserung der Sicherheit im ÖPNV. Wenn die Finanzierung noch prekärer wird, wird auch die Chance auf Verbesserungen in diesem Feld immer kleiner.Boris Palmer, Oberbürgermeister der Stadt Tübingen, seines Zeichens ein Grüner, möchte den ÖPNV ersatzweise über eine Bürgerabgabe finanzieren. Dann bekommen wir also eine Art GEZ für Busse und Bahnen.Auch das, meine Damen und Herren, ist politischer Unsinn. Selbst bei einer gewöhnlichen Nebenkostenabrechnung für Mieter – so hat der Bundesgerichtshof entschieden – müssen Erdgeschossbewohner nicht die Kosten für den Aufzug tragen, weil sie von ihm ja auch keinen Nutzen haben. Nach dem Palmer’schen Modell müssten auch diejenigen die Nahverkehrs-GEZ bezahlen, die davon nichts oder fast nichts haben: die Handwerker, die immer das Auto brauchen, weil sie dicke Werkzeugkoffer zu transportieren haben, viele ältere Herrschaften, die leider nicht mehr gut zu Fuß sind und kaum mehr Fahrten unternehmen, und, wenn man auf das Umland schaut, auch die Bevölkerung der kleinen Gemeinden, in denen nur drei- oder fünfmal am Tag überhaupt ein Bus hält.Es gibt Städte, die den kostenlosen Nahverkehr schon ausprobiert haben. Viele davon, zum Beispiel Templin – der Kollege Donth erwähnte es schon; 100 Kilometer nördlich von hier –, mussten ihn wieder abschaffen, weil der Zuschussbedarf um ein Mehrfaches gestiegen war.Um in Großstädten – und von diesen sprechen wir hier in allererster Linie – hohe Fahrgastzuwächse zu bewältigen, müssen zuerst massive Ausbaumaßnahmen umgesetzt werden. Dazu braucht es viele Jahre und – das wissen Sie alle, meine Damen und Herren – Planung, Planfeststellungsverfahren, jahrelange Baustellen, ganz zu schweigen von der Realisierbarkeit überhaupt. Genau an dieser Stelle ist der Vorschlag eines kostenlosen ÖPNV als Reaktion auf einen Drohbrief aus Brüssel und ein vielleicht kommendes Urteil des Bundesverwaltungsgerichts vollends absurd.Denn die so heftig diskutierten Probleme mit NO X und Feinstaub bestehen jetzt, im Jahre 2018. Jedes Jahr sinken aber die Schadstoffwerte. Das zeigen auch die Messungen des Umweltbundesamtes. Die Probleme werden also dann, wenn man endlich rein praktisch in der Lage sein wird, einen kostenlosen Nahverkehr zur Verfügung zu stellen, gar nicht mehr so gravierend sein. Da kann man nur sagen – das richtet sich jetzt an die Bundesregierung –: Thema verfehlt, Note sechs!Kostenloser Nahverkehr ist also wieder einmal ein Thema aus der Rubrik „Villa Kunterbunt“. Es ist lächerlich, zu meinen, damit ein Stückchen weit die Welt retten zu können. Stellen wir uns also der Wirklichkeit, meine Damen und Herren: Es gibt kein „free lunch“.Die AfD lehnt diesen Vorschlag deshalb ab.




	48. Jan Korte (DIE LINKE.) Frau Präsidentin! Meine sehr geehrten Damen und Herren! Es gibt in der Tat zurzeit extrem wichtige Themen, die wir hier diskutieren müssten: Armutsrenten, Kinderarmut, Kriege und vieles andere mehr. Was ist der AfD wichtig? In dieser Woche ist ihr Folgendes wichtig: Sie hat eine Aktuelle Stunde zu ihrer eigenen Demo beantragt, die irgendwie nicht geklappt hat. Und heute möchte sie unter anderem, dass der Bundestag Äußerungen eines Journalisten missbilligt.Der wahre Titel Ihres Antrags müsste eigentlich lauten – wenn er denn sachlich begründbar wäre –: „Vollcrash der AfD mit der Pressefreiheit und den Grundrechten“. Das ist der Kern Ihres Antrags.Ich will auf etwas anderes zu sprechen kommen. Wir alle in diesem Land haben es gesehen: das Bild von Deniz Yücel, als er seine Frau nach einem Jahr, in dem er unschuldig im Knast gesessen hat, in den Arm genommen hat. Es ist eigentlich so, dass Menschen einen inneren Kompass haben, mit Mitgefühl, Menschlichkeit und Freude, wenn sich Menschen wiederfinden, vor allem dann, wenn Menschen unschuldig im Knast gesessen haben.Dieser Kompass ist bei Ihnen vollständig im Eimer. So viel Niedertracht muss man erst mal zustande bringen.Nun zu Ihrer Fraktionsvorsitzenden. Sie hat gesagt: Yücel ist weder Journalist noch Deutscher. – Fällt Ihnen etwas auf?– Genau das ist der Klassiker: sich zu früh zu freuen. Fällt Ihnen eigentlich etwas auf? Ihre Kritik an Deniz Yücel gleicht exakt der Kritik des türkischen Präsidenten Erdogan. Sie sind der verlängerte Arm von Erdogan im Kampf gegen Yücel und die Pressefreiheit. So sieht es aus.Auch der türkische Präsident spricht Yücel ab, Journalist zu sein.Des Weiteren sagt Ihre Fraktionsvorsitzende – ich zitiere –:Ein ... „Journalist“,– wie Yücel –der nicht nur einmal die Grenzen des guten Geschmacks verließ, sollte eigentlich keine deutsche Staatsbürgerschaft besitzen.Zum einen kennen Sie sich mit dem Verlassen des guten Geschmacks wie kein anderer aus; das ist wahr.Zum anderen – das ist der ernste Teil – steckt dahinter ein Motto. Das Motto lautet: Ich entscheide, wer Deutscher ist und wer nicht.Auch das ist exakt die gleiche Denkweise wie die des türkischen Präsidenten Erdogan, der das Blut deutscher Abgeordneter untersuchen wollte. Beides ist völkisches Denken, das dieses Land und Europa in den Abgrund geführt hat. So einfach ist das.Interessant ist, dass gerade Sie, die Sie in jeder Woche austeilen bis zum Get-No, dann, wenn Kritik zurückkommt, wie Mimosen herumheulen und sich darüber beschweren, wie ungerecht das alles ist.– Ja, Sie weinen immer herum. Ich kenne das doch aus den Runden, Herr Baumann, wenn wir zusammensitzen.Als Linker bin ich beim Einstecken sehr ausgebildet.Ausgerechnet Sie, die Sie Überschreitungen zum Geschäftsmodell gemacht haben – ansonsten kämen Sie nicht vor, weil Sie keine Inhalte vorzuweisen haben –,stellen heute die Pressefreiheit grundlegend infrage, auf die Sie sich sonst immer berufen. Das ist ein gewisser Widerspruch.Es ist doch relativ bizarr, dass ausgerechnet Sie, die Sie ununterbrochen von einer herbeihalluzinierten Islamisierung Deutschlands reden, mit dem Islamisten Erdogan gegen den Journalisten Yücel und die Pressefreiheit vorgehen. Das muss Ihnen doch einmal auffallen. Ich bin so freundlich, Sie auf diesen Widerspruch aufmerksam zu machen.Zum Schluss eine kleine Denksportaufgabe. Es gibt in Deutschland nicht nur viele Partei- und Vereinsstatute, sondern auch so etwas wie Ehrenmitgliedschaften. Ich hätte einen Tipp für einen heißen Kandidaten für eine Ehrenmitgliedschaft bei Ihnen: den türkischen Präsidenten Erdogan.Ihr Antrag wird selbstverständlich volle Kanne abgelehnt.




	49. Elisabeth Motschmann (CDU) Frau Präsidentin! Meine sehr verehrten Damen und Herren! Liebe Kolleginnen und Kollegen! Deniz Yücel ist frei, und darüber freuen wir uns ohne Wenn und Aber.Im letzten Jahr sagte Erdogan noch: Er wird niemals freigelassen. – Er ist frei. Angela Merkel hat sich bemüht. Der Außenminister hat sich bemüht. Das ist ihr Job. Sie hatten Erfolg. Dafür sagen wir Dank.Er ist frei – und das, so Gabriel, ohne Verabredung, Gegenleistung oder Deals. Das ist wichtig.Viele weitere deutsche und türkische Journalisten sind nach wie vor in Haft, sechs türkische Journalisten lebenslang. Das ist schlimm. Natürlich müssen wir uns um jeden Einzelnen kümmern und versuchen, für die Freiheit zu kämpfen.Die AfD betont in ihrem Antrag, dass Deniz Yücel eine Vorzugsbehandlung bekommen habe; viermal taucht das auf.Sie unterstellen, dass weniger prominente und öffentliche Fälle weniger intensiv behandelt werden. Falsch! Ich habe die Familie eines Inhaftierten persönlich betreut, den keiner kannte, ein Pilger, der an einem Friedensmarsch teilgenommen hat. Er ist frei – dank der Bemühungen der Bundesregierung.Also: Ihre Aussage stimmt nicht.Alice Weidel kommentiert die Freilassung von Yücel – es wurde hier schon angesprochen; ich zitiere –:… ein unser Land regelrecht hassender „Journalist“, der nicht nur einmal die Grenzen des guten Geschmacks verließ, sollte eigentlich keine deutsche Staatsbürgerschaft besitzen.Eine unglaubliche Äußerung, meine Damen und Herren!Meinungsfreiheit und Pressefreiheit sind nicht verhandelbar in unserem Land.Sie sind vom Grundgesetz geschützt; das sollte Alice Weidel eigentlich wissen.Auch mir gefallen manche Positionen von Deniz Yücel überhaupt nicht. Dazu gehört, was er 2011/2012 in einer „taz“-Kolumne über den Geburtenschwund geschrieben hat. Aber das kann doch niemals ein Grund dafür sein, seine deutsche Staatsbürgerschaft infrage zu stellen, meine Damen und Herren.Welch ein Unsinn!Es wäre übrigens sinnvoller – das sage ich der AfD –, wenn Sie sich um die menschenverachtenden Aussagen Ihrer eigenen Parteifreunde kümmern würden, bevor Sie mit dem Finger auf andere zeigen.Der Aschermittwoch ist noch nicht so lange her. Was hat Herr Poggenburg gesagt? Kümmelhändler und Kümmeltreibersollen sich dahin scheren, wo sie hingehören: weit, weit, weit hinter den Bosporus, zu den Lehmhütten und Vielweibern.Zitat Ende. – Das ist widerlich, meine Damen und Herren.Herr Curio, Sie hatten von „Morast“ gesprochen. Das ist Morast, was wir da in Ihren Reihen erleben.Herr Gauland,es hat mich schon gewundert, dass Sie dazu sagen, es gäbe keinen Bedarf, eine innerparteiliche Debatte anzustrengen. Dann kommt der Satz: „Das bewegt mich nicht.“ Das ist eine ethische Bankrotterklärung, Herr Gauland.Wenn Sie das nicht mehr bewegt, dann haben Sie Ihren Beruf verfehlt.Mich bewegt es sehr.Kollegin Motschmann, Sie haben die Chance, wenn Sie eine Frage oder Bemerkung zulassen, die ablaufende Redezeit noch zu verlängern.Ja. Alles gut. Bitte schön.Frau Kollegin, Sie haben gerade, wie ich meine, nicht auf das eigentliche Thema in Ihrer Rede reagiert. Sie haben gerade von einer Bankrotterklärung gesprochen. Wenn dieses Hohe Haus nicht bereit ist, eine solche Äußerung wie die von Herrn Yücel zu verurteilen,frage ich Sie, ob es nach Ihrer Meinung nicht auch eine Bankrotterklärung dieses Hohen Hauses ist,ob wir uns nicht alle gemein machen mit denen, die auf Demonstrationszügen lauthals rufen: Deutschland verrecke! – Das ist doch hier die Frage.Ich weiß nicht, Herr Kollege, ob Sie Probleme mit den Ohren haben. Ich habe doch eben gesagt, dass mir auch nicht alles gefällt, was Deniz Yücel gesagt und geschrieben hat.Ich habe genauso gesagt, dass es eine ethische Bankrotterklärung ist, wenn man die eigenen miesen Äußerungen in der AfD nicht entsprechend zurückweist.Schließlich komme ich zu meiner Rede zurück.Wenn die AfD zu allem Überfluss in ihrem Wahlprogramm, das ich sehr genau studiert habe, meint, dass ihre Grundsätze auf den Werten des Christentums fußen, dann kann ich Ihnen nur bescheinigen, dass Ihre Werte angesichts solcher Äußerungen und solcher dümmlichen Anträge nichts, aber auch gar nichts mit dem Christentum zu tun haben.Kollegin Motschmann, achten Sie jetzt bitte auf die Zeit.Ja, ich bin am Ende.– Nicht am Ende, sondern ich komme zum Ende.– Keine Sorge, freuen Sie sich nicht zu früh.Ich will nur noch am Ende Bischöfin Ilse Junkermann zitieren, die über Herrn Poggenburg gesagt hat:Die Rede hat erneut den unverblümten Hass gezeigt, mit dem die AfD Menschen diffamiert.Sie fährt fort:Unsere Gesellschaft braucht Verantwortliche mit Umsicht und Weitblick und keine vor Hass und Menschenverachtung blinden Abgeordneten.Danke, Frau Bischöfin. Dieser Weitblick und diese Umsicht fehlen der AfD; daran sollten Sie arbeiten.Vielen Dank.Ich mache darauf aufmerksam, dass wir zu diesem Tagesordnungspunkt noch zwei Debattenbeiträge hören werden. Ich bitte also auch all diejenigen, die schon zur bevorstehenden Abstimmung herbeigeeilt sind, Platz zu nehmen. Diese Bitte gilt wiederum fraktionsübergreifend für die Mitglieder aller sechs Fraktionen hier im Hohen Hause.Sollte es notwendig sein: Abgeordnete, die gleichzeitig Mitglieder der Bundesregierung sind – Sie haben auch hier vorn Ihren festen Platz.Es kann ja sein, dass meine Stimme nicht ganz bis nach hinten durchdringt. Dann bitte ich Sie doch einfach einmal, Ihren Kollegen in den letzten Reihen einen Tipp zu geben, nämlich dass wir auf sie warten.




	50. Cornelia Möhring (DIE LINKE.) Herr Präsident! Liebe Kolleginnen und Kollegen! Für die, die es noch nicht verstanden haben: In § 219a wird mitnichten die Zulässigkeit oder das Verbot von Schwangerschaftsabbrüchen geregelt, sondern es ist das Werbungsverbot, was dort geregelt wird.Worum geht es also in § 219a? Es geht darum, dass Ärztinnen und Ärzte in Deutschland auf ihrer Webseite nicht darüber informieren dürfen, dass Schwangerschaftsabbrüche zu ihrem medizinischen Leistungsspektrum gehören. Werbung und Information werden durch die aktuelle Rechtsprechung gleichgesetzt. Sie dürfen zwar Abtreibungen vornehmen – unter bestimmten Umständen –, aber sie dürfen nicht darüber informieren. Zugespitzt heißt das, liebe Kolleginnen und Kollegen: Wir erwarten von diesen Ärztinnen und Ärzten, dass sie es heimlich tun. – Das ist doch völlig absurd.In der Folge werden den Frauen, die sich in so einer Notlage befinden, völlig unnötige Hürden in den Weg gestellt.Ungewollte Schwangerschaften wird es immer geben. Jede vierte Frau hat diese Situation einmal in ihrem Leben. Das Verhütungsmittel kann versagen. Die Lebenssituation kann sich unbeeinflussbar und plötzlich verändern.Weil das für mich bisher wirklich zu kurz gekommen ist, will ich den Blick noch einmal auf die betroffenen Frauen lenken. Eine Frau, die ungewollt schwanger geworden ist, ist in einem Ausnahmezustand. Sie ist in einer persönlichen und seelischen Krise. Das geht schon los mit dem Warten auf die Menstruation und der Sorge, dass die gar nicht kommt. Dann folgt der Gang zur Arztpraxis, und mit Glück trifft die Frau auf eine Ärztin oder einen Arzt, die oder der sie in ihrem Ausnahmezustand annimmt und sich einfühlt. Bis zu diesem Zeitpunkt hat die betroffene Frau mindestens schon tausendmal im Kopf hin und her bewegt, ob sie die Schwangerschaft abbrechen wird.Frau Kollegin, lassen Sie eine Zwischenfrage zu?Ehrlich gestanden, ist mein Bedarf an AfD-Aussagen schon irgendwie voll gedeckt.Wir hätten auch eine Zwischenfrage von der CDU/CSU zur Auswahl.Auch nicht. – Kommen wir zurück zum Thema und zu der Situation der Frau. Wenn die Frau die Frage mit Ja beantwortet – das tun im Übrigen bei weitem nicht alle Frauen, die ungewollt schwanger geworden sind –, dann muss sie zur Pflichtberatung. Wenn sie nach dieser Pflichtberatung immer noch den Abbruch machen lassen will, wird ihr, je nach Bundesland, keine Ärztin oder Klinik genannt, die den Eingriff vornehmen kann. Dann beginnt die Frau zu recherchieren, wo der Eingriff gemacht werden könnte, und landet schlimmstenfalls auf so widerlichen Seiten wie babycaust.de .Zurück zu der Frau. Wenn sie dann endlich einen Arzt gefunden hat, muss sie dort noch vor Ende der zwölften Schwangerschaftswoche einen Termin erhalten. Sie weiß zu diesem Zeitpunkt übrigens immer noch nicht, wie der Abbruch vorgenommen wird, ob mit Medikamenten oder durch eine Kürettage. Sie weiß auch nicht, wie der Ablauf in der jeweiligen Praxis ist, ob sie jemanden mitbringen kann, welche Risiken welche Methode hat; denn das alles kommt in der Pflichtberatung überhaupt nicht vor; das können nur die Ärzte machen.Frauen, die auf dem Land leben, haben ein noch größeres Problem. Die Beratungsstellen sind weiter entfernt, und für den Eingriff müssen die Frauen bis zu 150 Kilometer – einige sogar noch mehr – zurücklegen.Liebe Kolleginnen und Kollegen, wir sind Gesetzgeber, und wir sind auch dafür verantwortlich, dass das Recht auf Gesundheit durchgesetzt wird.Es muss doch in unser aller Interesse sein, dass eine Frau sich so umfänglich wie möglich informieren kann.Auf der Fachkonferenz der FDP hat mich die Aussage einer jungen Frau sehr beeindruckt. Ich will sie hier kurz zitieren: Wie absurd ist es eigentlich, dass alle möglichen Falschinformationen im Netz verbreitet werden können und nur diejenigen, die am meisten Ahnung haben, die Einzigen, die konkrete Infos zu Eingriffen und Ablauf geben können, bestraft werden? – Recht hat sie!Ich frage auch: Welches Frauenbild verbirgt sich eigentlich dahinter, wenn behauptet wird, dass eine Frau sich aufgrund von Werbung für einen Schwangerschaftsabbruch entscheidet? Das ist doch komplett irre. Ich kenne keine Frau, die sagen würde: Was für eine coole Werbung. Jetzt mache ich mal einen Schwangerschaftsabbruch.Wir Frauen wollen selbstbestimmt entscheiden, ob wir eine Schwangerschaft austragen oder nicht. Wir brauchen dafür gute Informationen. Die Voraussetzung dafür ist die Streichung des § 219a. Packen wir das endlich an!




	51. Kirsten Kappert-Gonther (BÜNDNIS 90/DIE GRÜNEN) Herr Präsident! Liebe Kolleginnen und Kollegen! Es ist völlig klar: Die Verbotspolitik ist auf ganzer Linie gescheitert. Der Schwarzmarkt blüht, und auf dem Schwarzmarkt gibt es weder Gesundheits- noch Jugendschutz.Die Vernunft und die Evidenz gebieten endlich die kontrol­lierte Freigabe von Cannabis.Dazu legen wir Grünen heute einen umfassenden Gesetzentwurf vor. Unser Gesetzentwurf regelt vom Anbau bis zum Verkauf alles, und zwar immer mit dem Gesundheits- und Jugendschutz fest im Blick.Ich finde es gut, Herr Dr. Schinnenburg, dass die FDP hier eine Initiative eingereicht hat. In den Sondierungsgesprächen haben Sie das Thema ja noch vollkommen uns überlassen.– Ja, das habe ich so gehört. – Als Sie in der vorletzten Legislaturperiode noch im Bundestag waren, hat Ihre Kollegin Initiativen zur Entkriminalisierung als Rauschsozialismus bezeichnet.Ich finde es ausdrücklich gut, dass Sie das jetzt anders sehen. Wir sind offen für den Schulterschluss mit allen hier im Haus, die eine vernünftige, progressive Drogenpolitik wollen.Es stimmt mich sehr zuversichtlich, Kollegin Dittmar, dass Sie sich heute hier so vernünftig, progressiv und komplett richtig eingelassen haben. Wenn wir Sie auch dauerhaft an unserer Seite haben, dann kann das wirklich noch etwas werden.Frau Kollegin, erlauben Sie eine Zwischenfrage von Herrn Hilse von der AfD-Fraktion?Nein. – Die FDP und die Linken schlagen in ihren Anträgen richtige Schritte vor, aber kleine Schritte. Meiner und unserer Meinung nach sind wir schon viel weiter. Die Linken fordern ja auch ein Gesetz, das die kontrollierte Freigabe regelt. Einen solchen Gesetzentwurf legen wir heute vor. Tun wir doch endlich Butter bei die Fische!Es war hier ja viel von Gesundheitsschäden die Rede. Warum setze ich mich als Fachärztin für Psychiatrie und Suchtmedizinerin mit immerhin über 20‑jähriger praktischer Erfahrung so vehement für die kontrollierte Freigabe ein? Das tue ich, weil es unter den Bedingungen der Prohibition – das ist doch der Witz – zu den Problemen kommt, die Sie hier beschreiben.Sie müssen endlich einmal der Tatsache ins Auge sehen, dass jede mögliche Gefahr – nicht jeder Cannabiskonsum ist automatisch schädlich – durch die Prohibition verschlimmert wird.Auf dem Schwarzmarkt Cannabis zu kaufen, ist, als würden Sie in eine Kneipe gehen und sagen: „Einmal Alkohol, bitte!“, ohne zu wissen, ob Sie Bier oder ­Wodka bekommen, dafür mit Streckmitteln inklusive. Auf dem Schwarzmarkt fragt auch niemand nach dem Ausweis. Der Jugendschutz existiert dort gar nicht. Ja, der Schwarzmarkt lebt ganz überwiegend von Cannabis. Wenn man dem Schwarzmarkt das Cannabis entzieht, dann trocknet das den Schwarzmarkt aus. So ist die Situation!Wir fordern von der Bundesregierung: Beenden Sie endlich die Gesundheitsgefährdung durch die Prohibition! Die Evidenz ist doch da. Wir brauchen keine neuen Modellprojekte. Schauen wir doch auf die guten Erfahrungen in Colorado und ganz aktuell in Kalifornien! Hören Sie auf die Strafrechtlerinnen und Strafrechtler des Schildower Kreises! Auch André Schulz, der Vorsitzende des Bundes Deutscher Kriminalbeamter, hat erst kürzlich gesagt: Unsere Polizei und Justiz hat wahrlich Wichtigeres zu tun, als Nutzerinnen und Nutzer zu verfolgen.Es ist doch absurd, dass wir derzeit etwa neunmal mehr Geld für die Strafverfolgung der Nutzerinnen und Nutzer ausgeben als für die Prävention.Übrigens, Kollege Pilsinger, auch über die Hälfte der Hausärztinnen und Hausärzte sprach sich in einer ganz aktuellen Umfrage für die Freigabe von Cannabis aus. Auch in der Psychiatrie ist schon länger bekannt, dass die Gesundheitsgefahren durch die Prohibition schlimmer werden.Liebe Kolleginnen und Kollegen, es ist an der Zeit, die Prohibition endlich zu beenden. Ich freue mich auf die Debatte im Gesundheitsausschuss.Vielen Dank.




	52. Alexander Gauland (AfD) Herr Präsident! Liebe Kolleginnen und Kollegen! Meine Damen und Herren! Frau Bundeskanzlerin, Ihre Erklärung enthält wenig Konkretes, und da, wo sie Konkretes enthält, stimmen wir zum großen Teil nicht zu. Aber das wird Sie nicht wundern.Sie sagen in Ihrer Erklärung: Wir brauchen ein europäisches Asylsystem, das krisenfest und solidarisch ist. – Sie sprechen von einer fairen Verteilung der Flüchtlinge innerhalb der EU. Aber Sie wissen, dass die Osteuropäer zu Recht niemals bereit sein werden, eine Verteilung der Flüchtlinge innerhalb der EU vorzunehmen. Herr Tusk hat uns das ganz klar und freundlich erklärt.Ich kann das auch deutlicher oder drastischer ausdrücken: Die Nationen wollen selbst bestimmen, wen sie in ihre Gemeinschaft aufnehmen. Es gibt keine nationale Pflicht zur Buntheit; so muss man, besonders an die Grünen gerichtet, sagen.Meine Damen und Herren, die Osteuropäer haben historische Erfahrungen, die sie zu dieser Haltung bringen. Sie waren Jahrhunderte von den Osmanen beherrscht,wurden von den Deutschen im Zweiten Weltkrieg leider versklavt und danach von der Sowjetunion 40 Jahre unterdrückt. Das sind Lebenserfahrungen von Völkern, die jetzt auf Selbstbestimmung und nationale Identität setzen. Da können sie mit einem EU-Zwangsverteilungssystem überhaupt nichts anfangen.Sie werden sich nicht vorschreiben lassen, wen sie aufnehmen. Deshalb ist es auch falsch, die Neuverteilung der Strukturfondsmittel – auch das kam vor – davon abhängig zu machen, ob Länder Migranten aufnehmen.Herr Kollege Gauland, gestatten Sie eine Zwischenfrage?Nein, jetzt nicht, danke. – Diese Idee ist übrigens typisch deutsch im schlechten Sinne, so wie wir eigentlich nach dem Zweiten Weltkrieg nie mehr sein wollten.Es ist – Entschuldigung, wenn ich das deutlich ausspreche – politische Erpressung, wenn Sie die Mittelvergabe davon abhängig machen. Deutschland entwickelt sich vom Lehrmeister über den Zuchtmeister zum Zahlmeister:Ihr bekommt Geld, wenn ihr euch wohl verhaltet. – Auch das wollten wir nie mehr in der deutschen Politik haben.Aber lassen Sie mich noch auf ein anderes Thema eingehen. Durch den Austritt Großbritanniens werden im Europaparlament 73 Sitze frei. Nun schlägt das Europaparlament vor, 27 Sitze neu zu verteilen und die übrigen zunächst einzusparen – als Reserve für die Zukunft. Der Kollege Bartsch hat hier zu Recht gefragt: Warum wird nicht alles eingespart? Es gibt weniger Mitglieder, es gibt weniger Aufgaben. Also ist es doch sinnvoll, dass es auch weniger Parlamentarier gibt.Aber ich ahne schon, was kommt – Frau Nahles und auch Herr Lindner haben davon gesprochen –: Diese Reserve soll für die transnationalen Listen da sein, wenn man auch noch nicht ganz so weit ist. Aber das, meine Damen und Herren, ist der Einstieg in die Vereinigten Staaten von Europa,die Herr Schulz für 2025 ausgerufen hat.Das wollen wir nicht. Wir wollen eine Zusammenarbeit der nationalen Staaten.Das Gefäß, in dem sich Freiheit und Selbstbestimmung manifestieren, ist – ich habe das schon einmal gesagt – der Nationalstaat. Dieser Gedanke stammt übrigens nicht von mir, sondern von dem großen Liberalen Lord ­Dahrendorf.Sie mögen hier im Hause mehrheitlich immer noch der Meinung sein, dass Europa eine deutsche Ersatzidentität ist, sozusagen eine weitere Ausprägung des deutschen Sonderwegs. Die anderen Europäer sehen das nicht so und für sich sehr viel praktischer.Die Österreicher und die Niederländer haben schon erklärt, dass sie nicht mehr Geld für transnationale Träume ausgeben wollen, und unser Weg sollte das auch nicht sein. Die AfD wird deutlich machen, dass es dafür in dieser Gesellschaft auch Mehrheiten gibt.Danke.




	53. Karsten Hilse (AfD) Vielen Dank, Herr Präsident. – Die meisten von Ihnen wissen – ich hatte es schon erwähnt –, dass ich über 30 Jahre Polizeibeamter war, und zwar im Streifendienst. Bei über 80 Prozent aller Feststellungen von Cannabis haben wir gleichzeitig Crystal Meth gefunden. Crystal Meth ist Ihnen sicherlich bekannt. Es führt sehr schnell zu einem drastischen geistigen und körperlichen Abbau.Unter Polizisten ist inzwischen klar, dass Cannabis eine Einstiegsdroge für Crystal Meth ist.Ich möchte bitte, dass Sie das beachten.Vielen Dank.




	54. Katrin Helling-Plahr (FDP) Sehr geehrter Herr Präsident! Meine Damen und Herren!Frauen sind dazu da, ihre Pflichten im Haus zu erfüllen. Nur in dringenden Notfällen dürfen sie das Haus verlassen. Dann aber müssen sie verschleiert sein.So zitiert Friedensnobelpreisträgerin Malala in ihrem Buch aus einer der Radiopredigten des pakistanischen Talibanchefs.Meine Damen und Herren von der AfD, Sie geben vor, mit Ihrem Antrag zuvorderst muslimischen Frauen helfen zu wollen. Sie wollen uns aber doch wohl nicht weismachen, dass die Männer, die ihre Frauen und Töchter bisher dazu zwingen, das Haus nur vollverschleiert zu verlassen, künftig plötzlich der Auffassung wären, ihre bisherige Auslegung des Koran ginge fehl, sie und ihre Familie hätten über Jahrhunderte geirrt. Oder wollen Sie uns glauben machen, dass diese Männer aufgrund eines entsprechenden Verbots bereit wären, gegen die jedenfalls von ihnen so verstandenen Gebote ihrer Religion zu verstoßen? Es ist doch offenkundig, was passieren würde, wenn Ihr Antrag Erfolg hätte: Frauen, die bisher von ihren Männern und Familien gezwungen werden, das Haus nur vollverschleiert zu verlassen, dürften das Haus künftig gar nicht mehr verlassen.Sie würden den Frauen also nicht helfen, sondern ihnen das letzte bisschen Freiheit und das letzte bisschen Teilhabe an unserer Gesellschaft rauben.Sie verstellten ihnen auch noch den durch ihren Schleierschlitz möglichen kleinen Einblick in eine Welt, in der Frauen anders leben.Das wissen Sie von der AfD-Fraktion auch. Also lassen Sie uns das Deckmäntelchen des Gutmenschentums, das Sie über ihr Vorhaben zu stülpen versuchen, abnehmen. In Wahrheit ist es Ihnen doch ganz recht, wenn diese Frauen künftig zu Hause bleiben. Immerhin führte das in jedem Fall dazu, dass Sie sie nicht sehen müssen.Im Ansatz kann ich das sogar verstehen: Auch mich befremdet der Anblick einer Frau in Burka oder Nikab zutiefst.Aber wenn wir nicht wollen, dass sich unsere Gesellschaftsordnung an die von unfreien Ländern annähert, wenn wir weiter in einer modernen, offenen und toleranten Gesellschaft leben wollen, dann müssen wir an der Stelle, an der das Gefühl der Befremdung in uns aufsteigt, unseren Verstand einschalten.Nicht alles, was befremdlich ist, gehört verboten. Kant hat gesagt: Die Freiheit des Einzelnen endet dort, wo die Freiheit des anderen beginnt. – Auch die Vollverschleierung muss in einer offenen Gesellschaft möglich sein.Das gebietet im Übrigen bereits unser Grundgesetz. Ein Verbot der Vollverschleierung in der Öffentlichkeit würde einen verfassungsrechtlich nicht zu rechtfertigenden Eingriff in das schrankenlos gewährte Grundrecht der Religionsfreiheit bedeuten, wäre in Deutschland also nicht verfassungsgemäß.Und wenn Sie in Ihrer Begründung auf den Europäischen Gerichtshof für Menschenrechte, der entschieden hat, dass Verschleierungsverbote nicht gegen die Europäische Menschenrechtskonvention verstoßen, Bezug nehmen wollen, habe ich für Sie, die ja sonst an jeder Stelle höchst europakritisch unterwegs sind, eine gute Nachricht: Die Europäische Menschenrechtskonvention steht innerstaatlich nur im Rang eines einfachen Gesetzes, also in einem Rang unter dem Grundgesetz.Sehr geehrte Damen und Herren von der AfD-Fraktion, ich unterstelle, dass auch jemand in Ihrer Fraktion diese juristische Expertise hat.Schließlich stellen Sie ja sogar den Vorsitzenden des Rechtsausschusses.Sie wissen, dass es gar nicht möglich wäre, Ihr Vorhaben verfassungskonform umzusetzen. Aber das ist Ihnen egal; denn es geht Ihnen nicht darum, hier in der Sache etwas zu erreichen, auch nicht für die innere Sicherheit.Schätzungen gehen von einzelnen bis maximal 200 bis 300 Burkaträgerinnen in Deutschland aus. Die Burka ist in unserem Land zum Glück weder Alltags- noch Massenphänomen.Ihnen geht es nur darum, die nächste rechtspopulistische Kampagne zu fahren, und da machen wir nicht mit.




	55. Timon Gremmels (SPD) Sehr geehrte Frau Präsidentin! Liebe Kolleginnen und Kollegen! Deutschland ist ein Land der Erfinder. Allein im letzten Jahr gab es knapp 70 000 Patentanmeldungen. „Der Wert einer Idee liegt in ihrer Umsetzung“, sagte einst schon Edison. Dafür brauchen Existenzgründer Kapital. Bisher sind wir in Deutschland bei der Bereitstellung von Wagniskapital suboptimal aufgestellt.Mit dem heute vorliegenden Gesetzentwurf ändern wir das. Wir geben der KfW weitere Instrumente für die Eigenkapitalgenerierung junger Start-ups aus dem sogenannten Sondervermögen des ERP in die Hand.Herr Kollege, Sie haben hier eine echt sexistische Rede gehalten.Ich verstehe ehrlich gesagt nicht, dass das an einem solchen Punkt nötig ist. Aber gut, ich verstehe die AfD in vielen Fragen nicht.Jetzt moniert die AfD, dass der Förderzielwert in Höhe von etwa 330 Millionen Euro bei einer tatsächlich geleisteten Förderung von circa 250 Millionen Euro im Jahre 2016 unterschritten wurde. Aber gerade weil die Programmförderung auch aufgrund der Niedrigzinsphase den Zielwert nicht erreicht, soll das Geschäft im Bereich des Eigenkapitals ausgebaut werden. Das ist doch Sinn und Zweck des Gesetzes, das wir heute hier beraten. Aber scheinbar haben Sie das nicht verstanden.Wer dem Kollegen von den Grünen gerade zugehört hat, dem kam es vor wie eine Jamaika-Traumatherapie.Das Gespräch zwischen den Kollegen der Grünen und der FDP hat mit den Inhalten wenig zu tun.Sie haben moniert, dass das alles nicht schnell genug vorangeht. Herr Kollege, Sie sollten sich einmal daran erinnern, wer den durchaus erfolgreichen High-Tech Gründerfonds, der ebenfalls aus dem ERP-Sondervermögen finanziert wurde, gegründet hat. Wir Sozialdemokraten und Grüne haben den High-Tech Gründerfonds damals im Jahre 2005 gemeinsam gegründet. Es ist ein guter Fonds, der junge, innovative Unternehmen fördert, zum Beispiel hinsichtlich Ausstattung mit modernster Technik und hinsichtlich nachhaltiger Lösungen im Bereich der Energiewende und des Klimaschutzes. Das haben wir damals gemeinsam gemacht.Aber das haben Sie von den Grünen wahrscheinlich verdrängt.Natürlich ist das eine trockene Thematik. Aber ich habe mir in der letzten Woche einmal angeschaut, welches Unternehmen es in meinem Wahlkreis Kassel gibt, das aus dem High-Tech Gründerfonds unterstützt wurde. Ich habe dort das Unternehmen enercast GmbH, ein Spezialist für die Prognose von Leistungen der Solar- und Windkraftanlagen, gefunden. Dieses Unternehmen ist mit dem High-Tech Gründerfonds frühzeitig gefördert worden und expandiert mittlerweile in den USA.Jetzt gibt es Forderungen der Linken – neben ihren Beschimpfungen der Sozialdemokratie; das machen sie auch immer gerne –, dass es keine Exportförderung wegen der Außenhandelsüberschüsse geben dürfe. Ich sage Ihnen: Das sieht die SPD-Bundestagsfraktion anders. Deutschland braucht innovative Unternehmen, die im Globalen wettbewerbsfähig sind. Deswegen ist das der richtige Weg.Ich sage Ihnen an dieser Stelle auch deutlich: Das Gespräch bei enercast hat mir gezeigt, wie wichtig die Rolle des High-Tech Gründerfonds ist, und zwar nicht nur als Geldgeber. Die Ausstattung mit Kapital ist zwar ein wichtiger Schritt in der ersten Runde. Aber das Netzwerk, das dahintersteht, und die Kontakte, die geknüpft werden können, sind entscheidend. Der neue Geschäftsführer ist auch aus dem Pool gekommen, den der High-Tech Gründerfonds führt. Außerdem ist es ein Gütesiegel für weitere Investitionen und für weitere Kapitalgeber.Frau Grotelüschen, Sie haben gerade gesagt, es gebe eine Art Konkurrenz zwischen der öffentlichen und privaten Kapitalförderung. Das ist doch totaler Humbug. Es ist vielmehr so, dass staatliche Fonds private Investoren nachziehen. Das war bei dem Unternehmen in meinem Wahlkreis auch so. Es hat wie ein Katalysator gewirkt. Mittlerweile ist noch Innogy Venture Capital in die Finanzierung eingetreten. Das zeigt: Nachhaltig ist staatliches Engagement erst dann, wenn es private Investitionen anreizt.Frau Präsidentin, ich komme zum Schluss. Die Bereitschaft, etwas zu riskieren, gilt nicht nur für Gründer, sondern auch für Investoren und auch für die Politik.In diesem Sinne: Danke und Glück auf!Vielen Dank, Herr Kollege Gremmels. – Damit schließe ich die Aussprache.Wir kommen zur Abstimmung über den von der Bundesregierung eingebrachten Gesetzentwurf über die Feststellung des Wirtschaftsplans des ERP-Sondervermögens für das Jahr 2018. Der Ausschuss für Wirtschaft und Energie empfiehlt in seiner Beschlussempfehlung auf Drucksache 19/855, den Gesetzentwurf der Bundesregierung auf Drucksache 19/164 (neu) anzunehmen. Ich bitte diejenigen, die dem Gesetzentwurf zustimmen wollen, um das Handzeichen. – Wer stimmt dagegen? – Wer enthält sich? – Der Gesetzentwurf ist damit in zweiter Beratung mit Zustimmung aller Fraktionen einstimmig angenommen.Dritte Beratung




	56. Petra Sitte (DIE LINKE.) Frau Kluckert, für den Fall nehme ich immer mein Fahrrad mit. Damit geht man gar kein Risiko ein.Frau Präsidentin! Meine Damen und Herren! Mein Wahlkreis ist Halle an der Saale, eine wirklich schöne Stadt. Ihr ist sogar die historische Stadtstruktur erhalten geblieben. Was einerseits charmant ist, bedeutet andererseits täglichen Staustress, nicht nur weil immer irgendwo gebaut wird, sondern weil auch zu viele Autos in und durch die Stadt fahren. Nicht genug, dass die Nerven blank liegen, auch ein gesundheitliches und ökologisches Dilemma hat sich daraus entwickelt. Laut Deutscher Umwelthilfe überschreiten wir die Stickstoffgrenzwerte um über 10 Prozent. Es läuft nun auch eine Klage der Umweltorganisation gegen die Stadt.Die Klageandrohung der EU-Kommission gegen Deutschland macht es uns auch nicht leichter. Zur Klageabwendung hat die Bundesregierung – hier schon thematisiert – der EU-Kommission einen Brief geschrieben. Hui, dachte ich: Die Bundesregierung entdeckt den ÖPNV. Na super.In Halle hat Die Linke schon vor Jahren Konzepte dafür öffentlich zur Diskussion gestellt. Derzeit wird intensiv über eine Strategie zum „Mitteldeutschen Verkehrsverbund 2025“ diskutiert. Im Stadtrat von Halle wurden allein sechs Modelle vorgestellt, wie es gehen könnte. Gespräche mit der Landesregierung und den Landtagsfraktionen laufen. Aber alle wissen: Ohne bundesgesetzliche Regelung wird es nicht gehen.Das alles ist unendlich mühsam. Es bedarf auf Bundesebene endlich politischer Grundsatzentscheidungen.Ich kann nur sagen: Wenn sich die Kanzlerin nur halb so oft mit Unterstützerinnen und Unterstützern des ÖPNV getroffen hätte, wie sie es mit den Chefs der Automobilkonzerne getan hat, wären wir deutlich weiter.Dabei gibt es Vorbilder in der EU und in Deutschland. Tallin hat schon eine Rolle gespielt. Diese Stadt hat, sozial motiviert, als einzige europäische Hauptstadt einen Gratisfahrschein eingeführt.Und jetzt sagen Sie mir bitte nicht, Tallin sei ja nicht so groß. Tallin und Estland sind nicht annähernd so reich wie Deutschland, und doch geht es.Sie können aber auch in Deutschland lernen. Das hat auch vor kurzem eine Rolle gespielt. Templin, eine hübsche Stadt in der Uckermark,mit einem Bürgermeister der Linken, hat ÖPNV umgestellt. Die Nutzerzahlen sind in dieser Zeit um ein Vielfaches gestiegen. Zunächst – stimmt – war alles gratis. Dann kam die Finanzkrise. Seitdem gibt es allerdings eine Art Jahreskarte à 44 Euro. Da sage ich: Das ist ein absolut akzeptabler Preis für eine Jahreskarte.Aber es bleibt ein hartes Stück Arbeit der Kommunalpolitik, die Subventionen zu sichern. Insellösungen, meine Damen und Herren, werden nicht reichen, um die Situation in Deutschland zu entspannen. Natürlich reicht auch Gratis-ÖPNV nicht – das ist doch völlig klar –; aber man könnte ja wenigstens mal entschlossen anfangen, statt nur darüber nachzudenken.Die Bundesregierung sollte einsehen, dass es eines neuen, komplexen Mobilitätskonzeptes bedarf. Statt hier im Bundestag intensiver darüber zu diskutieren, haben wir wertvolle Zeit mit Debatten um E-Autos, Kaufzuschüsse und dergleichen mehr verloren. Dabei weiß jeder, dass man auch mit einem E‑Auto hervorragend im Stau stehen kann.Wer sagt – das ist ja hier schon mehrfach gesagt worden –, dass kostenloser ÖPNV nicht finanzierbar wäre, dem will ich erwidern, dass Autoverkehr und damit verbundene Infrastrukturkosten dreimal höher liegen. Gratis-ÖPNV ist also kein Einzelprojekt von irgendwelchen Kommunalhelden. Das muss volkswirtschaftlich gedacht werden.Nun schlagen Sie in dem besagten Brief auch Modellstädte vor. Toll, habe ich gedacht, Modellstädte, großartig – da müssten ja im Osten mindestens Dresden und Halle dabei sein. Aber nichts da! Ostdeutschland? Mal wieder totale Fehlanzeige. Ich sage Ihnen: So, wie es in Ostdeutschland nie blühende Landschaften gegeben hat, sind wir auch kein Luftkurort geworden.Was meinen Sie denn, was ich deswegen für Kommentare zu hören und zu lesen bekommen habe? In den letzten 20 Jahren sind im Osten zuhauf Bahn- und Busverbindungen gestrichen worden. Jetzt zeigt sich eben, wie falsch das gewesen ist. Tausende sind aufs Auto angewiesen.Es muss im wahrsten Sinne des Wortes – gerade auch in Ostdeutschland – umgesteuert werden. Nutzen wir den politischen Druck, der sich jetzt aufbaut, für eine neue Mobilitätsidee – eine, die ökologisch, sozial und schließlich auch für alle Nutzerinnen und Nutzer verlässlich ist.




	57. Frauke Petry (Plos) Sehr geehrter Herr Präsident! Meine sehr geehrten Damen und Herren! Frau Bayram, Sie sollten besser recherchieren. Ich habe im vergangenen Jahr im Sächsischen Landtag den Entwurf eines Gesetzes zum Verbot der Vollverschleierung vorgestellt. Damals haben nicht nur Ihre Kollegen, sondern die Kollegen aller anderen im Landtag vertretenen Fraktionen massiv dagegen argumentiert. Schön, dass Sie sich im Juni dann doch zu dem kleinen, vorsichtigen Schritt entschieden haben, zumindest teilweise ein Vollverschleierungsverbot zu verhängen. Das ist ein vorsichtiger, aber nicht ausreichender Schritt; denn zur Lebenswirklichkeit in Deutschland gehört, dass nicht nur junge Migrantenfrauen vollverschleiert sind. Sie sollten wissen, dass es Frauen gibt, die in Deutschland geboren und aufgewachsen sind, die unter einem Schleier leben müssen, wenn sie sich in einer Beziehung mit Männern aus dem arabischen Kulturkreis befinden. Ich empfehle Ihnen allen das Buch „Gefangen in Deutschland“. Lesen Sie es, dann wissen Sie, dass wir nicht nur über 100 bis 300 Burkaträgerinnen reden, sondern über ein massives kulturelles Problem.Sie wollen die Argumente nicht hören, weil der Antrag aus der falschen Fraktion kommt. Ich gehöre dieser Fraktion ganz bewusst nicht mehr an. Aber es besteht nach wie vor Handlungsbedarf. Es geht in der Tat um eine Frage des kulturellen Verständnisses: Wollen wir diese kulturellen Unterschiede relativieren? Sind wir der Meinung, dass es in unserem Land kulturelle Errungenschaften gibt, die es zu schützen gilt, oder tun wir so, als sei alles gleich? Der Kulturrelativismus ist auf der linken Seite dieses Parlaments ja sehr stark vertreten.Meine Damen und Herren, es gibt Staaten – das wissen Sie genauso gut wie wir –, die eine islamische Staatsreligion haben und dennoch keine Vollverschleierung verlangen. Dass dies in den letzten 30 Jahren immer rückläufiger ist, ist auch dem Vormarsch des orthodoxen Islam mit politischem Anspruch zu verdanken, der spätestens an den europäischen Grenzen haltmachen sollte. Daher ist die Forderung eines vollständigen Verbots der Vollverschleierung im öffentlichen Raum eine Selbstverständlichkeit und ein Schutz unseres europäischen kulturellen Selbstbewusstseins, unserer Identität und unserer Werteordnung, die wir hier in diesem Hohen Haus eigentlich gemeinsam verteidigen müssten.




	58. Dirk Heidenblut (SPD) Sehr geehrter Herr Präsident! Meine lieben Kolleginnen und Kollegen! Sehr geehrte Damen und Herren! Ich muss zugeben: Ich habe schon viele Gründe gefunden, warum ich dankbar dafür bin, dass die FDP, die Grünen und die Linken diese Anträge vorgelegt haben. Ich bin dankbar, weil wir uns auf den Weg zu einer vernünftigen neuen Drogenpolitik begeben können. Ich bin dankbar, weil wir das Thema diskutieren können. Und jetzt bin ich erst recht dankbar, weil das, was wir jetzt machen können, obendrein eine Fortbildung für die Kolleginnen und Kollegen von CDU/CSU und AfD in Drogenpolitik und Suchtfragen ist.Das wird dann vielleicht auch dazu beitragen, dass in beiden Fraktionen der eine oder andere dem einen oder anderen Mythos nicht mehr so hinterherrennt und vielleicht auch nicht dermaßen abwertend über Menschen redet, die einmal Cannabis zu sich genommen haben. Das allein ist schon erschreckend.Was mich zugegebenermaßen auch ein bisschen irritiert – von der AfD ist, wie ich es verstanden habe, der Antrag schon angekündigt; aber möglicherweise kommt die CDU/CSU auch noch darauf zurück –, ist das vehemente Eintreten dieser beiden Fraktionen dafür, dass wir demnächst auch ein Alkohol- und Tabakverbot bekommen. Denn wenn ich die genannten Argumente richtig verstehe – Legalisierung fordert meines Wissens keiner dieser Anträge; das wurde hier sehr schön und vorsichtig zusammengefasst –, geht es um kontrollierte Abgabe. Das ist etwas völlig anderes als Legalisierung. Aber das kann der Kollege der AfD noch einmal im Duden oder in anderen Nachschlagewerken nachschauen.Alle Argumente, die gegen eine kontrollierte Freigabe vorgebracht wurden, sind im Umkehrschluss Argumente, die man hervorragend für ein Verbot von Alkohol und Tabak heranziehen kann. Alleine das Gesundheitsargument, Herr Kollege – Sie haben ja tolle Statistiken genannt und beschrieben, was alles im Kopf der armen Cannabisnutzer abläuft –: Sie wissen sicherlich, wie viele Tote es in Deutschland durch den Tabakkonsum gibt. Sie wissen sicherlich, wie es mit den Menschen auf unseren Straßen aussieht, die dem Alkohol übermäßig zuneigen. Insofern kann ich Ihre Einlassungen sozusagen nur als vehemente Unterstützung für einen Antrag der AfD, der sicher kommen wird – sie sind ja bekannt für interessante Anträge – verstehen.Herr Kollege, gestatten Sie eine Zwischenfrage des Kollegen Krauß?Ja, bitte schön.Ich möchte Ihre Frage gerne fortsetzen. Sie wissen auch, dass es aufgrund von Cannabiskonsum 19 500 Krankenhausbehandlungen pro Jahr gibt, obwohl Cannabis nach Ihrer Ansicht, wie Sie es geschildert haben, vollkommen unschädlich ist. Wie kommen dann 19 500 Krankenhausbehandlungen auf der Grundlage einer Cannabisindikation zustande?Sie möchten die Antwort schon hören, oder?Ich bin mir nicht ganz sicher, welcher Rede Sie bislang zugehört haben.Ich habe nicht gesagt, dass Cannabis völlig ungefährlich ist; das würde ich nie tun. Wir wissen um die Probleme im Jugendbereich.– Sie sagen jetzt nichts mehr. Ich sage ja jetzt etwas. – Ich habe auch nicht gesagt, dass es keine Probleme gibt und dass es keine Gesundheitskosten verursacht, wenn versucht wird, Menschen mit unterschiedlichem Suchtkonsum – die Zahlen können wir später zum Beispiel mit denen aus dem Alkoholbereich vergleichen – bei der Bewältigung der Sucht zu helfen. Aber Sie haben offensichtlich nicht nur meiner Rede, sondern auch den vielen Vorträgen zuvor nicht zugehört.Weil wir Cannabiskonsumenten in die Illegalität treibenund weil wir den Menschen zumuten, sich bei der Beschaffung obendrein andere Dinge einzufangen, erhöhen wir in diesem Land die Kosten im Gesundheitswesen und sorgen noch dafür, dass sich das Ganze verschlimmert.Nun fahre ich mit meiner Rede fort. Ich wäre kein Gesundheitspolitiker, wenn ich nicht kurz auf die Nutzung von Cannabis in der medizinischen Versorgung eingehen würde. Wir haben in der letzten Legislaturperiode hier etwas erreicht. Wenn ich mich recht entsinne, geht keine der Vorlagen darauf ein. Aber auch an dieser Stelle müssen wir noch einmal ansetzen; denn nach dem, was zumindest ich sowohl von Ärzten als auch von denjenigen höre, die die entsprechenden Medikamente brauchen, gibt es sowohl bei der Abgabe als auch bei der Beschaffung erhebliche Probleme. Da ist das Gesetz – das sage ich selbstkritisch – vielleicht noch nicht ausreichend. Da müssen wir noch einmal herangehen.Zum guten Schluss. Ich finde es gut, dass wir darüber erneut und mit Vehemenz diskutieren. Es ist gut und richtig, dass wir uns noch einmal der Frage zuwenden, wie wir eine Drogenpolitik insbesondere im Hinblick auf den Umgang mit Cannabis betreiben können, die wirklich für die Menschen da ist, die die Menschen nicht kriminalisiert, sondern dafür sorgt, dass sie gesundheitlich geschützt und nicht weiter gefährdet werden. Das wäre der richtige Weg; denn damit würden wir einen vernünftigen Jugendschutz auf die Beine stellen. Jugendschutz ist zwingend und erforderlich. Wir auf jeden Fall möchten gerne darüber beraten und freuen uns auf die anstehenden Beratungen. Ich bin sehr gespannt, was wir am Ende gemeinsam daraus machen können.Vielen Dank.




	59. Stephan Harbarth (CDU) Herr Präsident! Verehrte Kolleginnen und Kollegen! Meine sehr geehrten Damen und Herren! Wenn man über § 219a des Strafgesetzbuches debattiert, dann muss man alle betroffenen Interessen berücksichtigen. Die vorgelegten Gesetzentwürfe nehmen die Interessen von Ärzten in den Blick – zu Recht.Die vorgelegten Gesetzentwürfe nehmen die Interessen von Schwangeren in den Blick – zu Recht. Zwei der vorgelegten Entwürfe lassen aber einen wichtigen Grundrechtsträger außer Acht: das ungeborene Kind. Dies halten wir für falsch.Die Interessen eines Arztes sind nicht mehr wert als die Interessen eines Kindes.Man kann nicht Kinderrechte im Grundgesetz einfordern, wenn man die vom Bundesverfassungsgericht anerkannten Grundrechte ungeborener Kinder mit keinem Wort und keiner Silbe erwähnt.Es war das Bundesverfassungsgericht, das klare Vorgaben zum Schutz des ungeborenen Lebens gemacht hat. Ich möchte aus der Entscheidung des Bundesverfassungsgerichts zitieren:Das Grundgesetz verpflichtet den Staat, menschliches Leben, auch das ungeborene, zu schützen …… Rechtlicher Schutz gebührt dem Ungeborenen auch gegenüber seiner Mutter …… Der Schwangerschaftsabbruch muß für die ganze Dauer der Schwangerschaft grundsätzlich als Unrecht angesehen und demgemäß rechtlich verboten sein …… Der Staat muß zur Erfüllung seiner Schutzpflicht ausreichende Maßnahmen normativer und tatsächlicher Art ergreifen, die dazu führen, daß ein – unter Berücksichtigung entgegenstehender Rechtsgüter – angemessener und als solcher wirksamer Schutz erreicht wird …… Das Untermaßverbot läßt es nicht zu, auf den Einsatz auch des Strafrechts und die davon ausgehende Schutzwirkung für das menschliche Leben frei zu verzichten.Das sagt das Bundesverfassungsgericht, und dem habe ich nichts hinzuzufügen.Da das heranwachsende Kind sich nicht selbst schützen kann, sondern hierzu auf den Staat angewiesen ist, regelt unsere Rechtsordnung, dass ein Schwangerschaftsabbruch grundsätzlich rechtswidrig ist,er aber unter näher bestimmten Voraussetzungen straffrei bleibt. Eine der Voraussetzungen ist die zwingende Beratung der Schwangeren. Hier heißt es im Gesetz: „Die Beratung dient dem Schutz des ungeborenen Lebens.“Diese verpflichtende Beratung durch eine unabhängige Stelle, nämlich gerade nicht durch denjenigen Arzt, der einen möglichen Schwangerschaftsabbruch vornehmen würde, kann nur funktionieren, wenn sie nicht durch Werbung und durch äußere Einflüsse konterkariert wird.Deshalb würde die Zulassung von Werbung das derzeitige Beratungsmodell infrage stellen. Genau das wollen wir nicht.§ 219a ist eine wichtige Schutznorm für das ungeborene Leben, eine Vorschrift, die verhindern soll, dass der Schwangerschaftsabbruch in der Öffentlichkeit als etwas Normales dargestellt wird.Es ist für uns klar: Schwangere Frauen und insbesondere solche, die sich in einer Not- und Konfliktsituation befinden, müssen jede Hilfe erhalten, die sie brauchen. Daran gibt es überhaupt keinen Zweifel.Eine Frau, die sich als Ergebnis der Beratung zur Durchführung eines Schwangerschaftsabbruchs entschließt, muss im Vorfeld die Möglichkeit haben, sich insbesondere bei staatlichen und neutralen Stellen zu informieren, bei welchem Arzt der Schwangerschaftsabbruch medizinisch kompetent durchgeführt werden kann. Daran besteht überhaupt kein Zweifel.Aber dies ist die Aufgabe der Beratungsstellen. Plakate, Zeitungsanzeigen, Internetwerbung sind dafür kein probates Mittel.Wir wollen im Abtreibungsbereich keine Werbung,sondern wir wollen Mut und Hilfe bei der Entscheidung gegen einen Schwangerschaftsabbruch geben. Deshalb werden wir diese Gesetzentwürfe ablehnen.Herzlichen Dank.




	60. Katharina Kloke (FDP) Sehr geehrter Herr Präsident! Liebe Kolleginnen und Kollegen! In der ganzen Stadt ist Berlinale. Es werden die neuesten Filme gezeigt. Nur hier im Deutschen Bundestag kommen Linke und Grüne schon wieder mit den alten Kamellen.Erfolglose Anträge zum Lobbyregister haben sie schon mehrfach eingebracht: in der 16., in der 17. und auch in der letzten Wahlperiode. Und pünktlich zum Beginn der neuen Spielzeit ist das Thema wieder da: das Lobbyregister, ein Evergreen.Lassen Sie uns ein paar Schlaglichter auf den konkreten Gesetzentwurf der Linken werfen. Zitat:Ohne eine gesetzliche Grundlage … ist es nicht möglich, den notwendigen Grad an Verbindlichkeit und Klarheit … zu erreichen.Liebe Kolleginnen und Kollegen, da wäre es natürlich toll, wenn sich der richtige und hehre Grundsatz der Klarheit auch in Form und Inhalt Ihres Entwurfes widerspiegeln würde.Sie aber stellen einen langen Katalog von registrierungspflichtigen Handlungen und Daten auf und hängen einen ebenso langen Ausnahmekatalog hintendran. Sie sagen „Transparenz“, aber Sie schaffen noch mehr Bürokratie.Ihr wichtigster Punkt: Die Einrichtung eines Bundesbeauftragten für politische Interessenvertretung. Die FDP lehnt es ab, ohne Not noch eine steuerfinanzierte Behörde einzurichten. Wenn es ein Sachproblem gibt, kann es mit Sicherheit durch bestehende Institutionen gelöst werden. Die Führung des öffentlichen Verbänderegisters hat die Bundestagsverwaltung jedenfalls bislang nicht in die Knie gezwungen.Zu den Kosten. In den letzten Wochen haben viele vielfach das intensive Ringen der GroKo-Parteien um die Postenverteilung kritisiert. Viel eleganter ist da natürlich der hier aufgezeigte Weg, einfach ganz neue Ämter zu schaffen.Das Gehalt des Beauftragten ist der einzige Punkt, zu dem Sie sich konkrete Gedanken gemacht haben, was es denn kosten soll. Zu den Kosten für das Register selbst: Schweigen im Walde! Das finde ich intransparent. Die Wählerinnen und Wähler haben es verdient, von Ihnen zumindest eine ungefähre Hausnummer zu hören.Ihr Entwurf sieht besondere Ausnahmen von der Registerpflicht vor, etwa für Anwältinnen und Anwälte bei der „Wahrnehmung oder Vertretung der rechtlichen Interessen … im Zusammenhang mit einem verwaltungsbehördlichen oder gerichtlichen Verfahren“. Das muss auch sein; denn sonst wäre ein Anwalt gezwungen, nicht alle Möglichkeiten seines Mandates auszuschöpfen, oder er müsste sehenden Auges gegen die Vertraulichkeit des Mandates verstoßen. Das Problem Ihres Vorschlags liegt in der Praxis: Wie soll man die politische Interessenvertretung trenngenau von der rechtlichen Interessenvertretung abgrenzen? Vielleicht ist das der Grund, warum in Österreich Anwälte gleich ganz von der Registerpflicht ausgenommen sind.Zur legislativen Fußspur: Das Bundesministerium der Justiz – das ist nicht, wie es in Ihrer Begründung steht, das „Bundesministerium für Recht“ – und für Verbraucherschutz veröffentlicht schon jetzt freiwillig die zu einem Gesetzesvorhaben eingegangenen Stellungnahmen auf seiner Homepage. Das scheint in der Sache also kein großes Problem zu sein. Ich frage mich, ob es Ihnen vielleicht nur um den Showeffekt geht, wenn Sie ein funktionierendes Prozedere gesetzlich regeln wollen, nur damit am Ende das Gleiche dabei herumkommt, was wir jetzt schon auf freiwilliger Basis haben.Klare Vorstellungen haben Sie zum Bestrafen von Verstößen. Sie finden, bis zu 10 Prozent des Gesamtumsatzes eines Geschäftsjahres sind für ein Unternehmen nicht zu viel. Transparenz ist ein wichtiger Grundsatz im Rechtsstaat, Verhältnismäßigkeit ist aber auch einer.Ihr Entwurf zeigt vor allem, wie viel Abneigung die Schimäre einer vermeintlichen Lobbyistenrepublik in Ihnen hervorruft. Kernaussage Ihres Entwurfs ist, dass es bei denen im Bundestag nicht mit rechten Dingen zugeht, dass nicht die Bundesregierung, sondern eine Lobbying­industrie das Land regiert, dass man den Abgeordneten stets auf die Finger schauen muss, damit sie nicht das Kungeln anfangen. Auch dieses plakative Anklagen vermeintlich schlimmer Zustände trägt zur Politikverachtung bei. Ich bitte Sie, das zu bedenken. Draußen mag Berlinale sein, aber hier sind wir nicht im Kino.Noch ein kurzer Hinweis an die lieben Kollegen von der AfD: Ich finde es bemerkenswert, dass Sie sich jetzt auf einmal, wo es Ihnen passt, des Grundgesetzes annehmen. Ansonsten scheinen Sie es völlig zu ignorieren.Ich danke für Ihre Aufmerksamkeit.




	61. Niels Annen (SPD) Vielen Dank, Herr Präsident. – Meine sehr verehrten Damen und Herren! Ich glaube, wir alle haben die Bilder von der Münchner Sicherheitskonferenz im Kopf. Besonders viel Sicherheit hat diese Konferenz nicht ausgestrahlt. Die Reden waren konfrontativ. Ich hatte manchmal den Eindruck, die dort vertretenen Außenminister und Ministerpräsidenten haben fast nur zu ihrer nationalen Öffentlichkeit geredet. Der Dialog ist zu kurz gekommen. Die Konferenz hat uns auch daran erinnert, wie fragil die Lage im Nahen Osten ist.Natürlich, der militärische Kampf gegen die Terrormiliz IS, gewissermaßen gegen das Kalifat, ist erfolgreich gewesen. Er ist noch nicht beendet – es gibt weiterhin einzelne Gebiete, die vom IS dominiert werden –, aber ich glaube, man kann schon sagen, dass es eine insgesamt erfolgreiche internationale Operation gewesen ist.Die Frage ist jedoch, meine sehr verehrten Damen und Herren: Was passiert eigentlich nach dem IS? Auf diese Frage haben wir bisher keine Antwort gefunden. Die Terrorgruppe ist auf dem Rückzug. Sie ist aber weiterhin in der Lage, Anschläge zu verüben und Terror zu verbreiten. Was wir im Moment erleben, ist, dass die Kräfte, die sich zum Teil taktisch verbündet haben, um gegen den IS vorzugehen, jetzt anfangen, sich quasi für die Post-IS-Phase in Stellung zu bringen. Das ist wahrscheinlich von der Formulierung her eine Untertreibung; denn wir erleben eine neue Welle der Gewalt in Syrien, aber auch im Irak. Der Kampf um diese Ordnung hat begonnen, und die internationale Gemeinschaft scheint auf diese Phase nicht ausreichend vorbereitet zu sein; auch wir scheinen auf diese neue Herausforderung keine richtige Antwort zu haben. Deswegen ist es wichtig, dass wir diesen Prozess miteinander gestalten.Ich kann das, was der Kollege Wadephul hier gesagt hat, nur wiederholen und unterstreichen: Die humanitäre Katastrophe, die wir im Moment in Syrien erleben, ist nicht nur markerschütternd, sondern auch vollkommen inakzeptabel. – Das Regime von Diktator Assad bombardiert erneut und wiederholt seine eigene Bevölkerung. Man hat offensichtlich geglaubt, dass man nach dem Aufschrei der Weltöffentlichkeit während des Kampfes um Ost-Aleppo – am Ende stand der militärische Sieg von Assad – wieder dieselbe Taktik anwenden kann, dass die Weltöffentlichkeit, wir alle, vielleicht ein bisschen müde geworden sind, ja, auch, dass wir uns an diese Bilder gewöhnt haben.Meine lieben Kolleginnen und Kollegen, ich glaube, wir alle müssen mit unserer Reaktion dafür sorgen, dass dieses Kalkül von Assad nicht aufgeht.Denn was wir jetzt erleben, ist eine tagelange Bombardierung. Es gibt Berichte von einem 13-stündigen, ununterbrochenen Bombardement der Zivilbevölkerung. Die Zahlen variieren, aber es gab wahrscheinlich über 300, bis zu 400 Tote allein in den letzten Tagen, darunter viele Frauen und Kinder. Eine Organisation, von der man sagen muss, dass sie wirklich nicht verdächtig ist, in irgendeinem Punkt eine politische Agenda zu vertreten, die SOS-Kinderdörfer, berichtet, dass ihre Einrichtungen gezielt von Assad bombardiert werden. Dieser Staatsterror muss enden.Ich sage auch: Alle, die diese menschenverachtende Politik von Assad unterstützen, tragen eine Mitverantwortung; man muss hier sowohl den Iran als auch Russland namentlich nennen. Ich glaube, das gehört bei unseren bilateralen Gesprächen auf den Tisch.Jetzt kommt ein weiterer Eskalationspunkt hinzu. Wir alle wissen, dass sich die türkische Regierung entschieden hat, in Afrin und in der Gegend um Afrin militärisch einzugreifen. Die türkische Regierung beruft sich auf das Selbstverteidigungsrecht nach Artikel 51 der Charta der Vereinten Nationen. Ich will hier mal in aller Freundlichkeit sagen: Die Belege, die die türkische Seite zur Legitimierung dieses Vorgehens vorlegt, haben mich nicht überzeugt.Die Türkei hat das Recht auf Selbstverteidigung, die Türkei ist und bleibt auch ein Partner Deutschlands, und, ja, die Türkei ist Opfer von Terrorismus geworden, auch von IS-Anschlägen; aber die Truppen, die dort jetzt bekämpft werden, haben gegen den IS gekämpft.Das muss man einmal ganz klar und deutlich aussprechen.Es zeigt sich in dieser Krise, wie fundamental sich der Blick der türkischen Regierung auf den Syrien-Konflikt von unserem Blick unterscheidet – übrigens nicht nur von unserem Blick, sondern auch vom Blick der amerikanischen Regierung. Diese Lage hätte ich mir nie vorstellen können: Ein NATO-Partner, die Türkei, marschiert auf einer zumindest fragwürdigen rechtlichen Grundlage – persönlich glaube ich nicht, dass das vom Völkerrecht gedeckt ist – in ein Nachbarland ein und bekämpft dort Truppen, die von einem anderen NATO-Partner, den Vereinigten Staaten von Amerika, politisch, logistisch und auch militärisch unterstützt werden.Ich möchte meiner Enttäuschung darüber Ausdruck verleihen, dass ich von der NATO dazu im Grunde genommen gar nichts gehört habe.Die Äußerungen des von mir sehr geschätzten Generalsekretärs waren bestenfalls windelweich. Hier brauchen wir aber eine politische Initiative des Bündnisses, das sich gerade in der Gefahr befindet, an der Frage, wie man sich in Syrien verhalten soll, auseinanderzubrechen. Das können wir doch nicht zulassen, liebe Kolleginnen und Kollegen.Aus diesem Hause sollte es eine klare Botschaft der Unterstützung und vielleicht auch der Ermutigung an Herrn Stoltenberg geben, das Thema auf die Tagesordnung zu setzen. Notfalls müssen wir es auf die Tagesordnung setzen, meine sehr verehrten Damen und Herren.Wir reden immer über Herausforderungen und Krisen. Zu einem Blick auf die Lage gehört aber auch, dass wir das, was wir in den letzten Jahren an vorsichtiger Annäherung und Stabilisierung erreicht haben, betonen und den Prozess weiter stützen. Das bedeutet, dass wir ein klares Wort führen müssen gegenüber jenen Kräften in den Vereinigten Staaten von Amerika, offensichtlich an der Spitze der Präsident, die jetzt versuchen, das Abkommen mit dem Iran zu unterminieren, das die eigene Vorgängerregierung mit ausgehandelt hat. Wir konnten uns bisher immer darauf verlassen, dass man sich, wenn eine Regierung wechselt – was ja ein normaler Vorgang ist –, an die Verträge hält, die vorher ausgehandelt worden sind.Herr Kollege Annen, erlauben Sie eine Zwischenfrage des Kollegen Dr. Neu?Ja, bitte. Selbstverständlich, Herr Neu. Lange nichts gehört.Herr Kollege Annen, wir beide waren auf der Münchner Sicherheitskonferenz. Sie haben eben gewissermaßen die Aussage des Ministerpräsidenten Yildirim wiederholt, der sich auf das Selbstverteidigungsrecht gemäß Artikel 51 der Charta der Vereinten Nationen berufen hat. Ich glaube, wir sind beide der Auffassung, dass diese Argumentation nicht zieht, weil sie nicht wirklich schlüssig ist.Als es die Attentate in Paris gegeben hat, wurde vonseiten Frankreichs mit Artikel 51 argumentiert. Deshalb meine Frage: Sehen Sie den Einsatz Frankreichs in Syrien und im Irak unter Berufung auf Artikel 51 auch als völkerrechtswidrig an?Herr Kollege Neu, warum sollte ich das als völkerrechtswidrig ansehen? Ich habe bisher niemanden getroffen – möglicherweise sind Sie die einzige Ausnahme –, der bestritten hat, dass Frankreich Opfer eines islamistischen Anschlages geworden ist.Die Argumentation, die ich vorhin vorgetragen habe, besagt, dass es eine Widersprüchlichkeit in der türkischen Argumentationslinie gibt; denn die Türkei behauptet, dass sie angegriffen worden ist. Sie hat aus meiner Sicht aber keine plausiblen Unterlagen oder Dokumente beibringen können. Insofern gibt es Fragen bezüglich der Kausalität in der türkischen Argumentation. Die Kausalität in der französischen Argumentation hingegen hat uns allen, auch der Mehrheit in diesem Hause, eingeleuchtet. Ich habe nicht bestritten, dass die Türkei und Frankreich ein Recht haben, sich auf Artikel 51 zu berufen. Insofern ist das eine etwas verwegene Argumentation, die Sie hier vortragen.Herr Kollege, erlauben Sie eine weitere Zwischenfrage des Kollegen Neu? – Frau Kollegin Roth, ich habe Ihr Zeichen nicht richtig verstanden.Dann muss Herr Annen freundlicherweise ins Mikrofon sprechen, dann verstehen wir ihn besser.Herr Präsident, ich spreche lauter und beantworte gerne noch eine weitere Zwischenfrage des geschätzten Kollegen Herrn Neu.Wenn ich Ihre Antwort gerade richtig verstanden habe, dann akzeptieren Sie das Selbstverteidigungsrecht Frankreichs angesichts eines Anschlages in Paris, aber das Selbstverteidigungsrecht der Türkei, die ähnlich argumentiert, nicht. Das hieße ja, Sie unterstützen Doppelstandards. Habe ich Sie da richtig verstanden?Nein, Sie haben mich komplett falsch verstanden. Ich habe darauf hingewiesen, dass beide Länder das Recht auf Selbstverteidigung haben. Auch unser Land hat das Recht auf Selbstverteidigung, aber nach den geltenden einschlägigen Völkerrechtsnormen muss man dieses Recht sehr plausibel darstellen. Man muss im Sicherheitsrat der Vereinten Nationen Belege vorlegen.Ich habe darauf hingewiesen, dass mir die Belege der türkischen Seite nicht ausreichen, um sich plausibel auf Artikel 51 berufen zu können. Frankreich hingegen hat sehr plausible Argumente, Belege und Dokumente vorgelegt, die zumindest die internationale Gemeinschaft und dieses Haus, offensichtlich mit der Ausnahme des Abgeordneten Neu, überzeugt haben.In der mir verbleibenden Redezeit möchte ich jetzt doch noch einmal darauf hinweisen, dass es aus meiner Sicht wichtig ist, dass wir alles tun, um das wichtige und komplexe Abkommen mit dem Iran aufrechtzuerhalten, weil es ein Beitrag zur Stabilisierung in dieser von Instabilität geprägten Region ist. Ich will aber auch darauf hinweisen, dass wir zwei Dinge voneinander trennen müssen: Das Abkommen mit dem Iran verfolgt das klare Ziel, die nukleare Bewaffnung dieses Landes zu verhindern. Gemeinsam haben wir nach Jahren des Verhandelns einen historischen Fortschritt erzielt. Das bedeutet aber nicht, dass wir gegenüber dem Iran schweigen werden, wenn die iranische Politik weiterhin und fortgesetzt aggressive Bewegungen in den Nachbarländern unterstützt.Deshalb will ich enden mit dem, was auch der Kollege Wadephul hier gesagt hat: Es geht um die Sicherheit des Staates Israel. Das aggressive Vorgehen von Milizen, die von der iranischen Regierung unterstützt werden, bis an die Grenze Israels ist inakzeptabel. Ich glaube, dass wir alle gut daran tun, dies in der Öffentlichkeit und in den diplomatischen Gesprächen, die wir führen, deutlich zu machen. Ich will auch darauf hinweisen, dass Deutschland weiterhin bereit ist, mit der UNIFIL-Mission an genau dieser fragilen Grenze einen Beitrag zur Stabilisierung, auch zur politischen Stabilisierung und zur Förderung des Dialogs zu leisten. Das sollte im Mittelpunkt unserer Beratungen stehen.Ich danke für die Aufmerksamkeit.




	62. Frank Schwabe (SPD) Frau Präsidentin! Sehr verehrte Damen und Herren! Selbstverständlich sind wir uns mit der Union einig darüber, dass wir, wenn der Koalitionsvertrag zum Tragen kommt, gemeinsam hinter der Entschlossenheit stehen, die die Bundesumweltministerin vertreten hat.– Große Zustimmung. – Die Mehrheit dieses Hauses will jedenfalls auf nationaler und globaler Ebene verantwortlich handeln. In der Tat gibt es eine dramatische Zuspitzung der Klimakrise. Wir müssen erleben, dass der Meeresspiegelanstieg deutlich höher ausfällt als bislang angenommen. Allein in Bangladesch müssen 35 Millionen Menschen noch schneller ihre Heimat verlassen. Das ist dramatisch und stellt uns vor ganz neue Herausforderungen. Ich sage das als jemand, der im Deutschen Bundestag langjährig Klimapolitik betreibt. Es ist bitter, deutlich machen zu müssen, dass es sehr unrealistisch ist, das 2020-Ziel zu erreichen. Aber ehrlich gesagt, wussten wir das schon vorher. Am Ende ist das Ausdruck des politischen Versagens derjenigen, die in den zehn Jahren, als wir noch Zeit hatten, das Ziel zu erreichen, Verantwortung getragen haben.Wir werden dennoch alles versuchen, um dieses Ziel erreichen.Wir sind aber international verpflichtet, ein anderes Ziel zu erreichen, nämlich das 2030-Ziel. Man kann sagen, dass das noch weit weg ist. Aber so weit weg ist das, ehrlich gesagt, gar nicht. Wenn wir dieses Ziel erreichen wollen, müssen wir alles tun, um die Kohleverstromung abzubauen bzw. letzten Endes aus ihr auszusteigen. Wenn der Koalitionsvertrag in der vorliegenden Form zum Tragen käme, wäre es das erste Mal, dass eine Bundesregierung das beschlossen hätte. Wir müssen zudem alles tun, um den Anteil der erneuerbaren Energien auf 65 Prozent zu erhöhen, genauso wie es im Koalitionsvertrag verankert ist. Das wird schwierig genug.Damit ist das Ziel, bis 2030 die Treibhausgase um 55 Prozent zu reduzieren, zu erreichen. In einer Studie von Aurora Energy Research wird prognostiziert, dass dieses Ziel zu erreichen ist, wenn es uns gelingt, die im Koalitionsvertrag verankerten Zahlen Realität werden zu lassen. Dafür sind ein Ausbau der erneuerbaren Energien und der Kohleausstieg notwendig. Es wird schwierig genug, das in einer Kommission so zu organisieren, dass weder Menschen noch Regionen ins Bergfreie fallen. Darauf legen jedenfalls Sozialdemokratinnen und Sozialdemokraten Wert.Wir werden ein Klimaschutzgesetz verabschieden. Die Grünen haben zusammen mit uns Sozialdemokraten und den Linken lange für ein solches Gesetz gekämpft. Wir hatten nie einen Mangel an wunderbaren Zielen. Die FDP war auch einmal dabei, als es, glaube ich, 2009 darum ging, das 40-Prozent-Ziel in einer Koalitionsvereinbarung zu verankern.Wie gesagt, wir hatten nie einen Mangel an Zielen; diese waren immer ganz okay. Aber es herrschte ein Mangel bei der Umsetzung; denn es gab keine gesetzliche Unterlegung der Ziele. Nun gibt es eine solche Unterlegung. Es ist unsere gemeinsame Verantwortung, daran zu arbeiten, dass dieses Klimaschutzgesetz besonders gut wird. Der Koalitionsvertrag liefert dafür eine gute Grundlage.Felix Matthes vom Öko-Institut – dieser Zeuge dürfte für die Grünen sicherlich eine wichtige Rolle spielen – hat das genauso bewertet. Er hat gesagt, dass im Koalitionsvertrag im Hinblick auf das 2020-Ziel zu wenig, aber im Hinblick auf das 2030-Ziel deutlich mehr erreicht worden sei als im Entwurf der Jamaika-Koalition.Es ist realistisch, zu sagen: Dass das 2020-Ziel nicht erreicht werden wird, ist bitter. Keine Regierung hätte dieses Ziel erreichen können. Aber die vereinbarten Maßnahmen, die dazu dienen, das 2030-Ziel zu erreichen, sind gut. Lassen Sie uns alle gemeinsam daran arbeiten, das Realität werden zu lassen.Vielen Dank.Vielen Dank, Frank Schwabe. – Damit schließe ich die Aussprache.Interfraktionell wird Überweisung der Vorlagen auf den Drucksachen 19/821 und 19/830 an die in der Tagesordnung aufgeführten Ausschüsse vorgeschlagen. Die Vorlage auf Drucksache 19/830 – Tagesordnungspunkt 7 b – soll federführend im Ausschuss für Wirtschaft und Energie beraten werden. Sind Sie damit einverstanden? – Das ist der Fall. Dann sind die Überweisungen so beschlossen.Tagesordnungspunkt 7 c. Beschlussempfehlung des Ausschusses für Umwelt, Naturschutz, Bau und Reaktorsicherheit auf Drucksache 19/856. Der Ausschuss empfiehlt unter Buchstabe a seiner Beschlussempfehlung die Ablehnung des Antrags der Fraktion Bündnis 90/Die Grünen auf Drucksache 19/83 mit dem Titel „Klimakonferenz in Bonn – Schneller Ausstieg aus der Kohle ist jetzt nötig“. Wer stimmt für diese Beschlussempfehlung? – Wer stimmt dagegen? – Wer enthält sich? – Die Beschlussempfehlung ist angenommen. Zugestimmt haben CDU/CSU, SPD, FDP und vereinzelte Kollegen der AfD. Dagegengestimmt haben Bündnis 90/Die Grünen und Die Linke und vereinzelte Abgeordnete der AfD. Enthalten haben sich einige Kollegen der AfD. Die Beschlussempfehlung ist damit angenommen.Tagesordnungspunkt 7 c. Unter Buchstabe b seiner Beschlussempfehlung empfiehlt der Ausschuss die Ablehnung des Antrags der Fraktion Bündnis 90/Die Grünen auf Drucksache 19/449 mit dem Titel „Klimaschutzzusagen einhalten – An Zielen für 2020 festhalten“. Wer stimmt für diese Beschlussempfehlung? – Wer stimmt dagegen? – Enthaltungen? – Die Beschlussempfehlung ist angenommen. Zugestimmt haben CDU/CSU, SPD, FDP und AfD. Dagegen waren Bündnis 90/Die Grünen und Die Linke. Noch einmal: Die Beschlussempfehlung ist damit angenommen.




	63. Canan Bayram (BÜNDNIS 90/DIE GRÜNEN) Vielen Dank. – Herr Präsident! Meine Damen und Herren! Herr Kollege Curio, wir haben vor einem Jahr im Berliner Abgeordnetenhaus gemeinsam eine ähnliche Debatte geführt. Mir fällt auf, dass Sie noch nicht einmal eine neue Rede geschrieben haben.Ich muss sagen: Sie sind mit Ihrem Antrag ein bisschen spät dran. Interessant ist außerdem, dass schon dem Berliner Abgeordnetenhaus kein Gesetzentwurf, sondern ein Antrag wie der jetzige vorgelegt wurde, in einem Satz hingerotzt: Die Landesregierung soll uns mal ein Gesetz machen, durch das wir die Vollverschleierung verbieten können. – In keinem der bundesweiten Landesparlamente konnte die AfD einen eigenen Gesetzentwurf vorlegen. Da stellt sich wirklich die Frage: Warum ist das so?Einige Anwaltskolleginnen haben Ihnen die Begründung schon genannt – dem will ich mich anschließen –: weil unsere Verfassung das nicht will. Und das ist gut so.Hier wurde mehrfach deutlich gemacht, dass Sie sich auf eine österreichische Regelung beziehen. Sie wollen, dass eine Vollverschleierung im öffentlichen Raum eine Ordnungswidrigkeit darstellt. Sie wollen, dass diese Frauen sich nicht im öffentlichen Raum bewegen können sollen. Ich frage Sie ernsthaft – ich finde, das hat die Kollegin von der FDP eindrücklich dargestellt –: Wollen Sie dasselbe machen wie die Taliban? Wollen Sie den Frauen, denen verboten wird, sich im öffentlichen Raum zu bewegen, wenn sie keinen Schleier tragen, verbieten, sich im öffentlichen Raum zu bewegen, weil sie einen Schleier tragen?Das kann doch nicht Ihr Ernst sein. Wir sind eindeutig dagegen.– Frau von Storch, ich höre Sie sogar. Sie sagen, wir müssen die Männer in die Verantwortung nehmen. Ich interpretiere und ergänze das so: wenn sie die Frauen unterdrücken. Aber das steht doch schon alles in unseren Gesetzen, und das gilt für alle Männer, die ihre Frauen unterdrücken, egal aus welchem Grund sie ihre Frauen unterdrücken.Das ist doch gar nicht die Frage, und das steht auch nicht in Ihrem Antrag. Frau von Storch, schreiben Sie doch einmal einen Antrag, in dem Sie das festhalten. Der Antrag, über den wir hier diskutieren, gibt das nicht her. Wir lehnen diesen Antrag eindeutig ab.Ich will auch Folgendes deutlich machen: Für uns von Bündnis 90/Die Grünen ist jede Form der Unterdrückung von Frauen ein Problem. Auch wir sehen in Nikab oder Burka eher ein Symbol der Unterdrückung. Deswegen müssen wir uns darüber unterhalten, wie wir die Frauen stärken, damit sie in der Lage sind, sich gegen die Männer zu wehren, die ihnen das aufzwingen wollen.Aber das heißt doch nicht, dass wir die Frauen bekämpfen. Das heißt doch genau das Gegenteil: Wir müssen die Frauen stärken, damit sie sich dagegen wehren, unterdrückt zu werden.An dieser Stelle will ich klarmachen: Wir müssen nicht nur die migrantischen Frauen stärken, sondern wir müssen alle Frauen stärken, die in unterschiedlichen Situationen Gefahr laufen, unterdrückt zu werden. Das ist der Anspruch, den wir haben müssen.Darüber hätte ich mich lieber mit Ihnen unterhalten als über den Antrag, den Herr Curio hier heute vorgestellt hat.Vielen Dank, Frau Bayram. Ich bin nach wie vor begeistert, dass die Redezeiten eingehalten werden.




	64. Thomas Seitz (AfD) Sehr geehrter Herr Präsident! Sehr geehrte Damen und Herren Kollegen! Liebe Landsleute! Wir verdanken Johann Wolfgang von Goethe eine kluge Einsicht:Sage mir, mit wem du umgehst, so sage ich dir, wer du bist; weiß ich, womit du dich beschäftigst, so weiß ich, was aus dir werden kann.Nicht mehr muss, aber auch nicht weniger darf ein Lobbyregister leisten. Goethes Einsicht sollte uns demokratische Pflicht sein. Die Bürger wollen wissen, mit wem wir Abgeordnete Umgang haben und womit wir uns beschäftigen. Daher unterstützt die AfD-Fraktion als die einzige wirklich freiheitliche und auch patriotische Kraft im Bundestagjede Forderung nach mehr Beteiligung des Volkes, nicht nur in Form von mehr direkter Demokratie, sondern auch, wie hier, auf lediglich informatorische Art und Weise.So freiheitlich der Abgeordnete agieren darf, so sehr bleibt er doch Vertreter. Er muss bereit sein, seinem Auftraggeber jederzeit Auskunft zu erteilen – über alle Akteure im In- und Ausland, die auf die parlamentarische Willens- und Meinungsbildung Einfluss nehmen wollen. Artikel 38 Grundgesetz zeigt, worum es geht: Der Abgeordnete ist „Vertreter des ganzen Volkes“. Hier für Sie alle zur Erinnerung: Gemeint ist „des ganzen deutschen Volkes“.Das Volk der Deutschen hat Ihnen Ihr Mandat gegeben und nicht irgendeine Bevölkerung.Dass die Sorge um Intransparenz und Manipulation keinesfalls theoretisch ist, zeigt die Praxis. Als den „Wettbewerb der Gauner“ brandmarkt beispielsweise Hans-Hermann Hoppe die real existierende Parteiendemokratie, und völlig unrecht hat er nicht, liebe Kollegen.Die Freiheit des Abgeordneten ist ein fundamentaler Baustein unserer parlamentarischen Demokratie. Sie steht, wie das Bundesverfassungsgericht sagt, in praktischer Konkordanz mit der Verpflichtung jedes Abgeordneten, für Transparenz und Aufklärung über seine Geschäfte zu sorgen.Immanuel Kant, der große Denker aus Königsberg, der Landeshauptstadt der einstigen Provinz Ostpreußen,hat in seiner Schrift „Zum ewigen Frieden“ nachgewiesen, dass der Rechtsstaat nur unter den Bedingungen der Publizität handeln kann.Aus Artikel 38 Grundgesetz folgen drei allgemeingültige Grundwerte für ein Lobbyregister: Die Angehörigen von Legislative und Exekutive sind, erstens, „Vertreter“ und nicht selbst der Souverän und Auftraggeber, zweitens die „Vertreter des … Volkes“ – das heißt, des deutschen Volkes und gerade nicht die Vertreter aller sich zufällig hier aufhaltenden Menschen – und drittens die „Vertreter des ganzen Volkes“.Sie sind nicht die Vertreter von ideologischen Gesinnungsgemeinschaften und Meinungskriegern, die sich in Wahrheit die Ausgrenzung und Diffamierung eines Teiles des Volkes, von Andersdenkenden und von Patrioten, auf die Fahnen geschrieben haben.Nun sind wir bei einem wichtigen Punkt. Es geht nicht nur um die Publizität von Kontakten mit Subventionsempfängern, mit der Großindustrie oder mit Finanzakteuren, woran vermutlich die Fraktionen von Linkspartei und Grünen vor allem denken. Zumindest die Grünen kennen sich inzwischen mit Lobbyismus und politischer Korruption gut aus. Der jüngste Fall heißt Simone Peter.Aber die berechtigten Erkenntnisinteressen unseres Volkes gehen darüber hinaus. So richtig es ist, die Kontakte des Parlaments mit Vertretern ökonomischer Interessen aufzudecken, so wichtig ist es auch, den Einfluss von Vertretern ideologischer Gesinnungen aufzuklären. Oder in abgewandelten Worten von Bertolt Brecht: Was ist ein Einbruch in eine Bank gegen die parlamentarische Mitgründung und staatliche Finanzierung der zumindest linksradikalen Amadeu-Antonio-Stiftung.Wir brauchen deshalb Aufklärung über den Einfluss ideologischer Einflüsterer auf Legislative und Exekutive, im Grunde bis hin zu den Kommunen.Ein Beispiel. In der Gemeinde Kandel hat sich bekanntlich ein Bürgermeister als Kuppler minderjähriger Mädchen mit erwachsenen muslimischen Männern hervorgetan. Diese perverse Idee endete mit der grausamen Ermordung eines blutjungen 15-jährigen Mädchens.Können Sie sich vorstellen, dass ein Bürgermeister seine Jugendlichen solcher Lebensgefahr aussetzt ohne Ermutigung, ja geradezu Anstiftung durch antideutsche linke Gruppierungen?Das wird Gott sei Dank im Augenblick vor Ort aufgeklärt. Seien Sie gewiss: Die AfD steht in der Causa Kandel für brutalstmögliche Aufklärung, und im Gegensatz zu Herrn Koch meinen wir es auch so.Noch eine kurze Anmerkung speziell zum Gesetzentwurf der Linken. Ihr verschwurbelter Gendersprech ist eine einzige Beleidigung unserer schönen deutschen Sprache. Inhaltlich träumen Sie von einem weiteren bürokratischen Monster. Dass dies auch einfacher geht, wird im Rahmen der Ausschussarbeit zu eruieren sein.Lassen Sie mich, frei nach Goethe, eine Voraussage treffen:Wenn die Deutschen wissen, mit wem ihre Minister, Staatssekretäre und Volksvertreter Umgang haben, dann werden sie als Wähler selbigen schon zeigen, was sie davon halten. Und wenn viele von Ihnen auch vom Gegenteil überzeugt sind: Unser Volk ist weder dumm noch naiv, es ist leider nur nicht ausreichend informiert. Die AfD vertraut deshalb der Klugheit des deutschen Volkes.Vielen Dank.




	65. Carina Konrad (FDP) Sehr geehrter Herr Präsident! Liebe Kolleginnen und Kollegen! Liebe Kollegen von den Grünen, meine Meinung: Die Überschrift Ihres Antrages ist falsch. Ich meine das aus zweierlei Gründen:Erstens. Sie schreiben „reduzieren“; am Ende Ihres Antrages zeigt sich aber, dass sie doch wieder nur das Verbot von Pflanzenschutzmitteln meinen.Sie merken schon – das ist der zweite Grund, den ich meine –: Ich rede nicht von Pestiziden, ich rede von Pflanzenschutzmitteln – und sie heißen auch so. Pflanzenschutzmittel dienen, wie der Name schon sagt, dem Schutz unserer Kulturpflanzen. Sie sollen ihre Gesundheit erhalten und ihre Vernichtung durch Krankheiten und Schädlinge verhindern.Das gilt im Übrigen sowohl für den konventionellen wie auch für den ökologischen Landbau.– Ja, Sie sprechen es gerade an. Vielen Dank, dass Sie mir den Hinweis „Mottenkiste“ geben. Auch das zeigt, dass Ihr Antrag in regelmäßigen Abständen im Plenum vorgeführt wird.Das wird dem Anspruch, das wird diesem Thema nicht gerecht. Ich möchte Sie da um ein bisschen mehr Fairness bitten.Das Interesse an nachhaltiger Produktion, Biodiversität, Gewässer- und Umweltschutz auf der einen Seite ist genauso berechtigt wie das Interesse der Landwirte auf der anderen Seite, die schließlich von ihrer Hände Arbeit leben müssen.Meine sehr geehrten Damen und Herren, im Jahr 2016 ist die Weltbevölkerung um 83 Millionen Menschen gewachsen.Frau Kollegin, gestatten Sie eine Zwischenfrage der Abgeordneten Künast?Sehr gern.Bitte sehr.Danke, Frau Kollegin. – Sie haben gerade über die berechtigten Interessen geredet. Deshalb muss ich Sie einmal fragen: Wie gehen Sie eigentlich mit dem Recht auf Gesundheit um? Wir alle sind da in der Pflicht. Die Politik hat da eine Verpflichtung. Wir haben international unterschrieben, zum Beispiel in der Kinderrechtskonvention, dass wir uns darum bemühen werden, eine Lebenssituation zu schaffen, die Kinder bestmöglich gesund hält. Das ist rechtlich gesehen kein Recht, das man abwägen kann. Wie passt das in Ihr System, bei dem Sie – jetzt allgemein formuliert – berechtigte Interessen der Konsumenten berechtigten Interessen der Landwirte gegenüberstellen?Zweiter Teil meiner Frage: Ist Ihnen aufgefallen, dass es auch das berechtigte Interesse von Landwirten auf Gesundheit gibt? Wir haben immer mehr Hinweise darauf, dass die Anwendung von Glyphosat Krebs auslöst.Zum Beispiel gibt es in Kalifornien eine große Gruppe von Landwirten, die das Non-Hodgkin-Syndrom haben.Sie sagen, dass es durch Glyphosat ausgelöst wurde, und klagen. Es ist ein großes Gerichtsverfahren, das da läuft.An der Stelle wird Monsanto – davon gehe ich einmal aus – wahrscheinlich verlieren.Also: Wo bleibt bei Ihrer These die Gesundheit der Landwirte?Frau Künast, zu Ihrer ersten These stelle ich einmal die Gegenfrage: Sind Sie der Meinung, dass wir in Deutschland keine hochwertigen Nahrungsmittel erzeugen? Ich behaupte: Wir erzeugen hier in Deutschland die hochwertigsten Nahrungsmittel weltweit.Zu Ihrer zweiten These sage ich Ihnen: Ich kenne Ihre Referenzstudie nicht. Wir können gern einmal über Ihre Referenzstudie reden. Aber wenn man den Verlauf der Studien sieht, die Sie und Ihre Kollegen immer für sich in Anspruch nehmen, dann muss man feststellen, dass Sie sich immer nur die Peaks der Kurven herausgreifen und nie die Kurven insgesamt begutachten und fachlich korrekt bewerten. – Ich hoffe, ich konnte damit Ihre Frage beantworten.Meine sehr geehrten Damen und Herren, Verantwortung zu übernehmen heißt, dass auch die Ressource Boden so effizient wie möglich genutzt werden muss. Wir können es uns schlicht nicht leisten, in Zeiten einer wachsenden Weltbevölkerung Ressourcen zu verschwenden, indem wir unsere Kulturpflanzen nicht vor dem schützen, was sie bedroht. Das gewährleistet flächendeckend eine Versorgung mit gesunden Nahrungsmitteln.Ich möchte sachlich und fachlich mit Ihnen debattieren. Die Reduktion des Einsatzes von Pflanzenschutzmitteln und der Einsatz von alternativen Pflanzenschutzmitteln liegen im ureigenen Interesse der Landwirte und werden bereits praktiziert. Herr Ebner, ich erinnere nur an das Beispiel des Einsatzes der Schlupfwespe gegen den Maiszünsler.Liebe Kolleginnen und Kollegen, seien Sie sich bewusst: Mit jedem Verbot, jedem Gesetz, jeder Verordnung, die Sie sich hier aus diesem warmen, trockenen Parlament heraus wünschen, schließen sich da draußen in Deutschland Hoftore für immer. Landwirte hören auf und fangen nie wieder an.Abgeordnete in diesem Parlament sind keine Landwirte.Die Aufgabe von Abgeordneten ist es nicht, Ackerbau zu betreiben, sondern unsere Aufgabe hier ist es, vernünftige Rahmenbedingungen zu schaffen.Dazu gehören für mich Förderung, Forschung und Entwicklung. Zum Beispiel können moderne Pflanzenschutzspritzen in Zukunft mit Sensortechnik punktgenau applizieren. Dadurch werden enorme Mitteleinsparungen möglich. Das setzt aber voraus, dass man hier tätig wird und flächendeckend den Breitbandausbau und den Netzausbau vorantreibt.Aber vor allem – da geht mein Appell an die Bundesregierung – muss der Knoten bei der Zulassung von Pflanzenschutzmitteln gelöst werden. Die EU-Kommission gibt eine Frist von 120 Tagen vor. Hier in Deutschland braucht es im Schnitt – jetzt halten Sie sich fest! – mehr als 732 Tage bis zur Zulassung. Die Zahl der Mittel, die fristgerecht zugelassen werden, liegt bei null. Das ist ein Skandal, und das gefährdet die Wettbewerbsfähigkeit unserer Betriebe in Deutschland.Deshalb mein Appell zum Schluss: Beschleunigen Sie die Zulassungsverfahren! Verzichten Sie auf nationale Zusatzkriterien, und erkennen Sie die Zulassungen der Mitgliedstaaten der sogenannten mittleren Zone ohne Vorbehalt an!Denn, Herr Ebner, mir liegt die Zukunft der Landwirtschaft am Herzen. Ich möchte, dass auch in Zukunft in Deutschland Landwirtschaft betrieben wird – und das nachhaltig.Vielen Dank.




	66. Franziska Brantner (BÜNDNIS 90/DIE GRÜNEN) Sehr geehrter Herr Präsident! Sehr geehrte Damen und Herren! Sie reden hier von der nationalen Pflicht zur Buntheit. Eines steht fest: Es gibt internationale Menschenrechte, und es gibt das europäische Versprechen: nie wieder Rassismus und Antisemitismus!In der letzten Tagungswoche des Europaparlaments hat der zuständige Ausschuss über transnationale Listen abgestimmt. Mich hat es wirklich sauer gemacht, dass die CDU mit allen Rechts- und Linkspopulisten, die es im Europaparlament gibt, gegen die transnationalen Listen gestimmt hat.Da gab es einmal einen Hoffnungsschimmer für die Stärkung der europäischen Demokratie: Die Briten treten aus; ein paar Plätze werden frei. Man hätte daraus transnationale Listen machen können, mit Menschen, die aus allen EU-Ländern kommen. „Nix da!“, sagt die CDU: Man könnte eher ein bisschen weniger europäische Demokratie gebrauchen. – Schade, liebe CDU, dass Sie diesen Hoffnungsschimmer so schnell gekillt haben.Frau Merkel – sie ist gerade nicht anwesend –, es reicht auch nicht, wenn Sie sagen: Die EVP hat einen Spitzenkandidaten. – Das ist zwar wunderbar, aber worum es bei Ihrem Treffen in den nächsten Tagen geht, ist, sicherzustellen, dass die nächste Kommissionspräsidentin vorher auch Spitzenkandidatin war. Das ist die Aufgabe, und das müssen die Regierungschefs jetzt festlegen, damit dann nicht wieder am Ende irgendjemand aus dem Hut gezaubert wird, den keiner kennt und der dann Kommissionspräsident wird.Darum geht es, und da reicht kein eigener EVP-Spitzenkandidat, sondern dafür brauchen Sie eine Festlegung auf europäischer Ebene.Erlauben Sie mir, etwas zum Haushalt zu sagen. Herr Lindner, es ist ein interessantes Konstrukt von Ihnen, dass die Große Koalition mit ihrem Versprechen, mehr Geld für Europa zur Verfügung zu stellen, Herrn ­Oettinger in den Rücken fallen würde.Herr Oettinger ist ja selber dafür, dass wir für einen größeren Haushalt sorgen.– Um den Kohäsionsfonds? Frau Merkel sagt, sie will auch für Deutschland weiter Geld. Das war auch keine Idee von Herrn Oettinger, sondern von Herrn Juncker. Also, Sie müssen schon aufpassen, wem Sie in den Rücken fallen wollen.Herr Oettinger kämpft für einen größeren EU-Haushalt, und offensichtlich die nächste Regierung auch. Das macht auch Sinn.Herr Bartsch, mich hat es überrascht, dass Sie auf einmal keinen größeren EU-Haushalt wollen. Das kann man zwar wollen, aber Sie haben vorher noch zigfach die Sozialunion angemahnt. Wie das zusammengehen soll – weniger Geld für Europa und gleichzeitig die Sozialunion –, müssen Sie uns einmal erklären.Der europäische Haushalt hat doch eine Aufgabe: die europäischen Aufgaben gemeinsam durchzuführen und dann auch zu finanzieren. Das ist eine gemeinsame Aufgabe beim Klimaschutz, bei Innovationen, Fairness und sozialer Gerechtigkeit, aber auch beim Handeln nach außen.Erlauben Sie, dass ich noch einmal auf Syrien zu sprechen komme. Frau Merkel hat die dramatische Situation in Ost-Ghuta schon beschrieben. 400 000 Menschen, die seit vier Jahren belagert und von allem abgeschnitten sind und eh schon keine Hoffnung mehr haben, werden noch dazu jetzt wieder bombardiert. Das ist unerträglich.Gleichzeitig gibt es den völkerrechtswidrigen Angriff der Türkei auf Nordsyrien. Herr Kauder, das ist mehr als ein Konflikt zwischen NATO-Partnern. Das ist ein völkerrechtswidriger Angriff auf Syrien. Als solchen müssen Sie ihn endlich benennen.Wenn sich nun die Regierungschefs zusammensetzen, dann muss es am Ende der Beratungen eine gemeinsame europäische Position geben. Es kann doch nicht sein, dass es in Syrien so weiterläuft wie bisher und dass dort Russland, die Türkei und die USA agieren, dass aber Europa noch nicht einmal eine gemeinsame Position dazu hat. Das ist nicht mehr erträglich. Wir brauchen ein gemeinsames Handeln. Die Bombardierungen – egal von welcher Seite – müssen aufhören. Der humanitäre Zugang zu allen Beteiligten muss ermöglicht werden, und sinnvolle Gespräche sind zu führen.Dann kann Europa wieder seinem Versprechen gerecht werden, für Frieden und Freiheit zu stehen. Ich fordere die Regierungschefs der EU-Mitgliedstaaten auf, ein deutliches Zeichen zu setzen.




	67. Volker Münz (AfD) Herr Präsident, vielen Dank für das Zulassen der Kurzintervention. Ich finde es eigentlich erbärmlich, dass in so einer Debatte um eine ganz wichtige Frage, in der es angeblich um das Gewissen geht, Zwischenfragen nicht zugelassen werden. Ich finde das undemokratisch.Frau Högl hat schon als zweite Rednerin den Versuch gemacht, hier zwischen Information und Werbung zu differenzieren.Ja, ist die Abgrenzung zwischen reiner Information und Werbung nicht genauso willkürlich wie die Abgrenzung zwischen schützenswertem Leben und nicht schützenswertem Leben?Diese Abgrenzung ist willkürlich, und es gibt keinen prinzipiellen Unterschied zwischen beidem. Das ist Augenwischerei. Das bitte ich zur Kenntnis zu nehmen, statt immer zu suggerieren: Information ist etwas ganz Tolles und Harmloses, und wir wollen ja gar nicht werben. – Denn diesen prinzipiellen Unterschied gibt es nicht.Was Sie hier versuchen, ist das Herausbrechen einer ganz wichtigen Regelung. Das Werbeverbot ist in diesem filigranen Konstrukt ganz wichtig. Wenn das herausgebrochen wird, dann bedeutet das letztendlich eine völlige Freigabe und Liberalisierung des Abtreibungsrechtes. Das ist genau das, was Sie wollen.Führen Sie die Kollegen und vor allem die Bürger nicht in die Irre!Frau Kollegin Högl, wollen Sie darauf antworten?Nein.




	68. Sabine Dittmar (SPD) Herr Präsident! Liebe Kolleginnen und Kollegen! Cannabis ist in Deutschland die am meisten konsumierte illegale Droge. Es ist unbestritten – Herr Kollege ­Pilsinger, da gebe ich Ihnen recht –, dass der Gebrauch von Cannabis gesundheitliche Risiken mit sich bringt und ein Suchtpotenzial birgt, genauso wie eben Alkohol und Nikotin.Umso besorgniserregender ist die Feststellung der Deutschen Hauptstelle für Suchtfragen, dass der illegale Beschaffungsmarkt von breiten Gesellschaftsgruppen genutzt wird und in seiner jetzigen Form auch Kindern und Jugendlichen uneingeschränkt zur Verfügung steht. Deshalb sollte es uns, Herr Kollege Pilsinger, mächtig zu denken geben,wenn ein bedeutender Teil von Sachverständigen aus der Praxis und der Forschung, also Verbände und Organisationen, Menschen, die tagein, tagaus mit Betroffenen zu tun haben, aber auch namenhafte Strafrechtler und zu guter Letzt der Bund Deutscher Kriminalbeamter zu der Feststellung kommen, dass unsere derzeitig praktizierte Verbotspolitik gescheitert ist.In meiner Partei wird seit langem kontrovers und leidenschaftlich über den richtigen Umgang mit der Drogenproblematik diskutiert. Ich will gar nicht verhehlen, dass es nach wie vor Stimmen gibt – es werden Gott sei Dank immer weniger –, die einer restriktiven Drogenpolitik das Wort reden. Ich bin dankbar, dass die progressiven Stimmen in meiner Partei einfordern,sich an den gesellschaftlichen Realitäten zu orientieren und neue Wege in der Drogenpolitik zu gehen.In der Arbeitsgruppe Gesundheit haben wir uns gemeinsam mit den Fachpolitikern aus der AG Recht und der AG Innen intensiv mit der Thematik beschäftigt und auch ein entsprechendes Positionspapier verfasst. Gemeinsam haben wir uns dafür ausgesprochen, es den Kommunen zu ermöglichen, Modellprojekte zur regulierten Abgabe von Cannabis an Erwachsene anzustoßen und die Ergebnisse zu evaluieren. Wir sind davon überzeugt, dass man damit nicht nur den Schwarzmarkt zurückdrängen, sondern auch ein neues Kapitel der Drogen- und Suchtprävention aufschlagen und vor allem Konsumenten wirkungsvoll entkriminalisieren kann.Gesundheitsrisiken würden reduziert, da man bei einer regulierten Abgabe dem gesundheitlichen Verbraucherschutz einen ganz anderen Stellenwert geben kann. Vor allem aber fänden wir einen sehr viel besseren Zugang zur Prävention und zu einem effektiven Jugendschutz. Wir könnten frühzeitig beraten, frühzeitig Hilfestellung anbieten und Schaden minimieren. Dass die Strafandrohung als Weg der Prävention ganz offensichtlich gescheitert ist, zeigen die Zahlen.Ich frage Sie: Welchen Sinn hat es, wenn sich Polizei und Justiz mit Endkonsumenten beschäftigen müssen, die nur geringe Mengen Cannabis bei sich haben, und einen unendlichen Aktenberg produzieren? Welches öffentliche Interesse sollten wir daran haben, die Konsumenten zu stigmatisieren und ihnen mit Akteneinträgen unter Umständen die persönliche Zukunft zu verbauen?Meine Damen und Herren, ich wünsche mir, dass wir die Debatte ohne ideologische Scheuklappen führen und uns in der Drogenpolitik von Fakten leiten lassen.Es ist kein Geheimnis, dass es in der zurückliegenden Legislaturperiode ein langer Weg war, Erleichterungen in der Substitutionstherapie zu vereinbaren oder auch den Zugang zu Cannabis als Medizin neu zu regeln. Sie, liebe Kolleginnen und Kollegen von der FDP und von den Grünen, wissen aus Ihren Sondierungsgesprächen sicherlich auch, dass es bei der Union starke Verharrungstendenzen gibt.Frau Kollegin, lassen Sie eine Zwischenfrage zu?




	69. Johann Wadephul (CDU) Moin, Herr Präsident! Meine sehr verehrten Damen und Herren! Die „Frankfurter Allgemeine Zeitung“ hat vor kurzem gefragt: „Kann jeder in Syrien bomben?“ und hat die Situation dort zu der Broken-Windows-Theorie in Beziehung gesetzt, die wir aus dem Bereich der inneren Sicherheit kennen. Sie besagt: Wenn erst einmal ein Fenster zerbrochen ist und man das hinnimmt, es keine Sanktionen gibt, es keine Reaktionen gibt, es keine Ordnung gibt, dann wird das allgemein akzeptiert und die Verwahrlosung schreitet weiter fort. Das erleben wir traurigerweise auch in Syrien. Ich kann im Grunde an das anknüpfen, was Eckhardt Rehberg gerade eben gesagt hat: Wir wollen eine regelbasierte Ordnung. Wir wollen, dass internationales Recht eingehalten wird. Dafür, meine sehr verehrten Damen und Herren, müssen wir in Syrien eintreten.Was wir jüngst erlebt haben, ist der Einsatz türkischer Streitkräfte im Norden Syriens um Afrin herum. Dazu möchte ich aus Sicht meiner Fraktion einige Anmerkungen machen.Wir haben immer ein nüchternes Verhältnis zur Türkei gehabt. Wir haben immer gesagt: Wir wollen enge Bindungen, aber einen europäischen Weg können wir uns nicht vorstellen. Die Türkei ist für uns NATO-Partner. Die Türkei ist in einer geostrategisch entscheidenden Lage und Situation. Die Türkei hilft uns sicherlich, Sicherheit für Südosteuropa zu gewährleisten. Die Türkei leistet Großes – das will ich auch heute noch einmal sagen – bei der Bewältigung der Flüchtlingskrise. Es wird dort eine große Zahl von Flüchtlingen aufgenommen. Das wissen wir, und das würdigen wir. Aber ich muss sagen: Ein derartiger Militäreinsatz – mit dieser Frage habe ich begonnen – darf nicht einfach so ermöglicht werden. Vielmehr kann er nur durch internationales Recht gerechtfertigt werden.Eine Belagerung Afrins, die der türkische Präsident Erdogan angekündigt hat, ist mehr als der legitime Einsatz von Gewalt gegen Terroristen, die gegen den türkischen Staat kämpfen. Deswegen fordern wir die Türkei an dieser Stelle auf: Halten Sie internationales Recht ein! Wahren auch Sie die Souveränität Syriens! Wenn es eine Belagerung geben sollte, ist das nicht gedeckt von internationalem Recht, und das könnte nicht unsere Unterstützung erfahren. Hier müssen wir klar sein, meine sehr verehrten Damen und Herren.Nichts ist besser geworden in den deutsch-türkischen Beziehungen in der Vergangenheit, was den Parlamentarismus angeht.Viele Kolleginnen und Kollegen waren bei der Münchner Sicherheitskonferenz. Wenn der Kollege Özdemir, mit dem man sich politisch über vieles streiten kann, dort unter Polizeischutz gestellt werden muss, weil er Bedrohungen ausgesetzt wird, dann verurteilen wir alle dies und sind solidarisch mit dem Kollegen aus der Fraktion der Grünen.Volker Kauder hat darauf hingewiesen: Viele brechen internationales Recht. Viele meinen, dass sie einfach bomben dürfen. Dazu gehört auch Russland. Wenn Russland – dass das passierte, habe ich für einen Fehler gehalten – von Amerika nicht als Regionalmacht bezeichnet werden möchte, wenn Russland als internationaler Spieler, als internationaler Partner ernst genommen werden möchte, dann muss auch Russland aufhören, einen Diktator zu unterstützen, der Giftgas einsetzt, in Ghouta wieder mordet und Zivilisten sterben lässt, dann muss auch Russland seiner internationalen Rolle endlich gerecht werden und seine Streitkräfte aus Syrien abziehen, meine lieben Kolleginnen und Kollegen. Das muss deutlich angesprochen werden.Das gilt auch für den Iran. Ich will deutlich sagen, dass wir Deutsche, wir Europäer ganz klar dafür eintreten, dass das Nuklearabkommen mit dem Iran eingehalten wird. Das ist – das muss man auch dem amerikanischen Präsidenten in vollem Selbstbewusstsein sagen – kein bilaterales Abkommen der Amerikaner mit den Iranern, sondern es ist – daran hat die deutsche Diplomatie einen großen Anteil; dem früheren Außenminister Frank-Walter Steinmeier sei gedankt – ein multilaterales Abkommen. Wir haben ein Interesse daran, mit den moderaten Kräften im Iran zusammenzuarbeiten. Wir haben ein Interesse daran, dafür zu sorgen, dass der Iran nicht nuklear bewaffnet wird. Deswegen halten wir, solange der Iran das Abkommen einhält – das macht er zurzeit –, an diesem Abkommen fest, liebe Kolleginnen und Kollegen. Das sollten wir auch im Bündnis klar formulieren.Dazu gehört genauso, dass auch der Iran seine Rolle in der Region wahrnehmen muss und sich nicht schlicht und ergreifend als eine Macht verstehen darf, die jetzt hegemonialen Einfluss durchsetzen will, die militärisch das durchführt, was ihr tatsächlich möglich ist, und die sich jetzt dauerhaft in Syrien, auch mit Militärbasen, verankern will. Auch das richtet sich gegen die Souveränität Syriens. Das ist insbesondere – das ist für uns Deutsche von großer Bedeutung – offenkundig eine Bedrohung Israels. Ich möchte in diesem Hause noch einmal betonen: Der Iran muss wissen, dass wir an dieser Stelle unsere internationale Politik, unsere Außenpolitik an folgendem Verhalten ausrichten: Jeder, der die Sicherheit des Staates Israel bedroht, muss wissen, er stellt sich damit außenpolitisch und sicherheitspolitisch auch in eine Gegnerschaft zu Deutschland. Deutschland ist an dieser Stelle solidarisch mit Israel. Das müssen alle, die in Teheran verantwortlich handeln, ganz eindeutig wissen, liebe Kolleginnen und Kollegen.Letztlich werden wir – deswegen kann ich an die Debatte, die wir vorhin geführt haben, anknüpfen – in dieser gesamten Region, auch mit unserem gewandelten Einsatz im Irak, den die Verteidigungsministerin schon skizziert hat, den wir aber auch noch miteinander parlamentarisch diskutieren müssen, nichts allein ausrichten. Deutschland allein wird in dieser Region – Kollege Post hat zu Recht an uns appelliert: was tun wir denn selber? – überhaupt nichts ausrichten, sondern wir werden dort nur gemeinschaftlich in Bündnissen mit anderen etwas ausrichten. Deswegen ist es gerade in dieser Zeit so wichtig, wo leider auch innerhalb des westlichen Bündnisses, seitens des amerikanischen Präsidenten, die NATO infrage gestellt worden ist, dass auch wir zur NATO stehen, dass wir unsere Bündnisverpflichtungen erfüllen und dass wir uns immer bewusst sind, dass wir das Geschehen in der Region nur dann werden beeinflussen können, wenn wir als Europäer gemeinsam auftreten. Deshalb die Lehre aus dem Desaster in Syrien: Macht Europa stark! Lasst uns im europäischen Verbund agieren! Dann können wir uns in dieser Region engagieren, und dann können wir dafür sorgen, dass das Leid geringer wird.Herzlichen Dank für die Aufmerksamkeit.




	70. Matthias Bartke (SPD) Herr Präsident! Liebe Kolleginnen und Kollegen! Ich glaube, Sie kennen es alle: Sie haben eine Besuchergruppe aus ihrem Wahlkreis zu Besuch. Dann kommt die Frage: Sagen Sie einmal, Herr Abgeordneter, haben Sie auch Kontakt zu Lobbyisten? – Hier schwingt immer ein leichter Grusel mit, der den Begriff „Lobbyismus“ begleitet. Richtig ist zweifellos, dass die Grenze zwischen legitimer und illegitimer Interessenvertretung bis hin zur Korruption überschritten werden kann. Richtig ist aber auch – das wurde auch gesagt –, dass die Politik natürlich einen Austausch mit Interessengruppen braucht. Im letzten Koalitionsvertrag gab es eine Formulierung im Bereich der Behindertenpolitik. Die lautete:Nichts über uns ohne uns.Das wurde bei der Entwicklung des Bundesteilhabegesetzes auch berücksichtigt. Die Menschen mit Behinderung und ihre Verbände wurden an der Ausgestaltung dieses Gesetzes beteiligt.Sie waren Experten in eigener Sache. Das war im Grunde ein Paradebeispiel für Lobbying. Aber, meine Damen und Herren, liebe Kolleginnen und Kollegen, ich kann nichts Schlechtes daran finden. Das ist doch sinnvoll.Seit 1972 gibt es eine Verbändeliste. Sie ist derzeit das zentrale Element. Aber sie kann ohne Frage nicht das leisten, was ein Lobbyregister leisten kann. In der Verbändeliste stehen – der Name sagt es – nur Verbände. Aber seit 1972 hat sich die Zeit geändert, und nicht nur Verbände machen Lobbying, sondern wir haben heute Unternehmensrepräsentanten, Rechtsanwaltskanzleien, Public-Affairs-Agenturen, Unternehmensberatungen usw. Die Zeit ist schlicht weitergegangen. Der Eintrag in die Verbändeliste soll freiwillig erfolgen. Auch da sagen wir: Das kann es heute nicht mehr sein. Deswegen sagt die SPD ganz klar: Die Verbändeliste reicht nicht mehr. Wir befürworten ein verpflichtendes Lobbyregister beim Deutschen Bundestag.Herr Schnieder, die Transparenz ist derzeit gerade nicht gegeben. Missbrauchen Sie nicht das freie Mandat, um damit irgendwelche Mauscheleien abzudecken. Das ist schlicht nicht in Ordnung.Sie haben zutreffend auf den § 108e StGB verwiesen, der in der letzten Legislaturperiode eingeführt wurde. Was war das für ein Kraftakt, um diese Norm einzuführen? Es war die CDU/CSU, die immer dagegengehalten hat. Das ist nicht in Ordnung. Die SPD hat in der letzten Legislaturperiode einen Gesetzentwurf vorgelegt. Der Gesetzentwurf sollte regeln, dass alle Personen und Verbände, die gegen Bezahlung Einfluss auf die Arbeit des Bundestages und der Bundesregierung nehmen wollen, sich in das Lobbyregister eintragen lassen müssen und dass sie Auftraggeber und Höhe der Einnahmen offenlegen müssen. Ein weiterer wichtiger Punkt unseres Gesetzentwurfes war der exekutive Fußabdruck. Es ist ja so: Die Mehrheit aller Gesetze wird in den Ministerien vorbereitet. Wenn der Lobbyist aktiv ist, dann fängt er nicht erst an, wenn das parlamentarische Verfahren beginnt, sondern er geht auf die Ministerien zu.Deswegen war in unserem Gesetzentwurf vorgesehen, dass allen Gesetzesvorhaben eine Liste beigefügt sein soll, in der die Interessenvertreter und die Sachverständigen benannt sind, die Einfluss genommen haben oder angehört worden sind.Ich sage Ihnen: Es war ein super Gesetzentwurf. Aber es gehört zur traurigen Wahrheit, dass wir ihn nicht eingebracht haben; denn es fehlte am Willen unseres Koalitionspartners. Das gehört zu den Schattenseiten der Großen Koalition, die ja dieser Tage intensiv diskutiert werden. Zu einer Großen Koalition gehört leider dazu, dass sich die SPD nicht in allen Fragen durchsetzen kann.Wir haben allerdings die Hoffnung, dass wir es in dieser Legislatur doch können.Wenn ich mir den Koalitionsvertrag mit unserem – hoffentlich – zukünftigen Koalitionspartner anschaue, dann sehe ich, dass da gute Sachen drinstehen – Kettenbefristungen sollen deutlich eingeschränkt werden, ein sozialer Arbeitsmarkt für 150 000 Langzeitarbeitslose soll geschaffen werden, die Mietpreisbremse soll verschärft werden –,dann merke ich, dass in der CDU/CSU doch kluge Leute sind, die sich von guten Argumenten überzeugen lassen und so etwas in den Koalitionsvertrag aufnehmen.Ich habe die Hoffnung, dass beim Lobbyregister das Gleiche passiert.Insofern möchte ich Ihnen, liebe Union, am Ende zurufen: Auch der Mehrheit Ihrer Wählerinnen und Wähler würden Sie eine Freude machen, wenn wir in dieser Legislaturperiode gemeinsam ein Lobbyregister einführen würden. Herr Schnieder, sehen Sie das ein!Danke.




	71. Daniela Raab (CSU) Frau Präsidentin! Liebe Kolleginnen und Kollegen! Der vielzitierte Sieben-Punkte-Brief ist Anlass für diese Aktuelle Stunde. Eigentlich geht es um nur einen Punkt, und schon daran wird deutlich: Mit dieser Aktuellen Stunde springt man viel zu kurz.Damit nicht der Eindruck zurückbleibt, dass wir hier – mit Verlaub – nur über einen Brief der Bundesregierung reden, möchte ich Ihr Augenmerk darauf lenken, dass wir das Thema Luftreinhaltung und die Verpflichtung Deutschlands zur Reduzierung der Stickoxide auch im europäischen Kontext sehr ernst nehmen und deswegen in der letzten Zeit Maßnahmen ergriffen haben, die – auch das ist heute viel zu kurz gekommen – bereits Wirkung zeigen.Ich will gar nicht darauf anspielen, dass wir 5,3 Millionen Autos nachrüsten, sondern – und das ist viel wichtiger –: Wir haben ein 1 Milliarden Euro schweres Sofortprogramm „Saubere Luft 2017-2020“ aufgelegt. Darum geht es doch. Was steht da drin? Wir wollen die Elektrifizierung des urbanen Wirtschaftsverkehrs. Dazu gehört die Elektrifizierung von – wichtig! – Taxis, Mietwagen, Carsharing-Fahrzeugen. Dazu gehören natürlich auch Elektrobusse. Die gibt es auf dem Markt, und die funktionieren. Wir müssen Dieselbusse nachrüsten, weil es im ÖPNV sonst nicht anders funktionieren wird. Das sind nur einige wenige Maßnahmen, die schon laufen und für die wir bereits Geld in die Hand nehmen. Darauf möchte ich Ihr Augenmerk lenken.Ich habe oft gehört, das Thema Fahrradverkehr würde bei dieser Bundesregierung – in der derzeitigen und auch in der letzten Legislaturperiode – zu kurz kommen. Das ist natürlich ein Märchen, wie wir alle wissen. Wir stocken die finanziellen Mittel zur Förderung des Radverkehrs auf insgesamt jährlich 200 Millionen Euro auf. Lieber Herr Klare, ich glaube, wir sind uns darüber einig: Da ist natürlich noch Luft nach oben.Förderung des Radverkehrs, Elektrifizierung auch im ÖPNV – das alles sind Maßnahmen, die dazu führen werden, dass die Luftbelastung in vielen Städten dieser Republik geringer wird.Wenn wir uns die Zahlen des Umweltbundesamtes anschauen – 2016 lagen 90 Kommunen deutlich über den Grenzwerten –, dann sehen wir: Diese Maßnahmen, die ich gerade genannt habe, greifen: 10 dieser 90 Kommunen liegen bereits nicht mehr über den Grenzwerten, 15 Kommunen sind wohl auf dem besten Weg, ebenfalls nicht mehr unter diese 90 Kommunen zu fallen. Das zeigt uns: Diese Maßnahmen sind eben kein Tropfen auf den heißen Stein, sondern sie sind zielgerichtet, sinnvoll und erreichen genau das, was wir erreichen wollen, nämlich eine bessere Luft in dieser Republik.Natürlich ist es richtig – das wird ja auch von fast niemandem in diesem Haus bestritten –, dass Dieselmotoren einen nicht unerheblichen Anteil an den Stickoxidemissionen haben. Selbstverständlich! Dennoch: Es gab auch Zeiten, da haben wir den Leuten gesagt: Kauft bitte den Diesel. – Warum? Wegen des geringeren Kraftstoffverbrauchs und der besseren CO 2 -Bilanz. Das alles soll jetzt plötzlich nicht mehr stimmen? Jetzt wollen wir den Haltern der 15 Millionen Dieselfahrzeuge in unserem Land mit Fahrverboten, Sonderplaketten oder was sonst noch alles in der Gegend herumschwirrt, die alleinige Verantwortung dafür aufbürden, dass die Luft besser wird? Nein! Ein ganz klares Nein vonseiten meiner Fraktion. Das ist nicht der Weg, um das zu erreichen, was wir anstreben, nämlich eine nachhaltige Verkehrspolitik und eine nachhaltige Mobilität, aber nicht auf dem Rücken der Dieselfahrer; denn sie können am wenigsten dafür.Auch die ganzen mittelständischen Handwerker können nichts dafür. Zwei Drittel ihrer Fuhrparke sind Dieselfuhrparke. Auch denen müssen wir mindestens eine angemessene Übergangsfrist zubilligen, um sich umzustellen. Das sind nämlich die Leistungsträger in unserer Gesellschaft, und die können wir nicht aus der Stadt aussperren.Im Gegenteil. Wir wollen, dass sie in die Städte fahren, weil wir wollen, dass die Arbeitsplätze, die diese Mittelständler mit ihren Fuhrparken auf dem Land geschaffen haben, auch weiterhin erhalten bleiben, meine lieben Freunde.Beim Thema nachhaltige Mobilität wird der ÖPNV bei der Reduzierung der Stickoxide in den stark belasteten Städten eine große Rolle spielen. Das ist selbstverständlich. Aber bitte mit Augenmaß und keine Schnellschüsse.Wir werden beim Thema Carsharing noch mehr tun müssen. Die letzte Regierung hat ein Carsharing-Gesetz auf den Weg gebracht. Warum? Weil auch das ein Mittel zum Zweck ist. Mittlerweile ersetzt ein Carsharing-Auto 20 private PKWs in deutschen Innenstädten. Auch das ist ein richtiger und wichtiger Weg, den wir beschreiten müssen.Ein Letztes: Wir müssen auch über die Digitalisierung unserer Verkehrslenkungen nachdenken. Wir können nicht weiterhin vertreten, dass Verkehr in der Stadt größtenteils steht. Wir müssen fließende Verkehre produzieren. Dabei kann uns die Digitalisierung helfen. Bei diesen Technologien können wir in Deutschland auch Vorreiter sein.Deswegen brauchen wir natürlich – erster Punkt – eine Senkung des Stickoxidausstoßes bei Fahrzeugen und – zweiter Punkt – einen viel besseren ÖPNV in diesem Land. Das heißt aber nicht automatisch, dass er deswegen kostenlos sein muss, sondern er muss erst einmal besser werden. Dritter Punkt: Wir brauchen eine vernünftige Verkehrslenkung, die wir mit den Erkenntnissen aus unserer Wirtschaft mit Sicherheit gut erreichen können. Dann ist mir auch um die Luftqualität in diesem Land nicht bange.Vielen herzlichen Dank.Die Aktuelle Stunde ist beendet.Sehr geehrte Kolleginnen und Kollegen, ich bitte jetzt all diejenigen, die an den weiteren Verhandlungen hier teilnehmen wollen, zügig ihre Plätze einzunehmen. – Das Präsidium des Bundestages hat sich davon überzeugt, dass für jedes gewählte Mitglied dieses Hauses ein Sitzplatz vorhanden ist. Ich bitte, von dieser Möglichkeit Gebrauch zu machen, und diese Bitte gilt fraktionsübergreifend.




	72. Daniela Kluckert (FDP) Verehrte Frau Präsidentin! Meine Damen und Herren! Ich möchte Ihnen von einer kleinen Reise berichten.Obwohl: Nahverkehrsodyssee trifft das,was ich auf meinem Weg in eine kleine brandenburgische Gemeinde erlebt habe, dann doch deutlich besser.Ironischerweise wollte ich zu einer Veranstaltung zum Thema Mobilität. Ich bin mit der S-Bahn gefahren und wollte für den letzten Kilometer oder, wie man auch gerne sagt, die letzte Meile vom Bahnhof zum Veranstaltungsort ein Taxi nehmen. Im Waggon eingeklemmt, habe ich dann 40 Minuten lang versucht, dieses Taxi zu bekommen – per Telefon, per App –, und habe am Ende, als ich inzwischen fast alleine im Zug war, entnervt aufgegeben.Am S-Bahnhof des Zielorts angekommen, war dann auch noch meine allerletzte Hoffnung, vielleicht ein Taxi vor Ort am Gebäude zu finden, pulverisiert. Mit dem Wissen, dass nun die Veranstaltung ohne mich beginnen würde, habe ich mit offenen Schuhen – geregnet hat es auch noch – den drei Kilometer langen Fußmarsch angetreten, den Wind immer im Gesicht. Aber das sind wir als FDP-Mitglieder sowieso gewöhnt.Zu meinem Glück hat mich dann ein freundliches Ehepaar aufgelesen und mich mit dem Auto zur Veranstaltung gefahren.Aber Glück, meine Damen und Herren, kann doch nicht Ersatz für fehlende Konzepte für die letzte Meile sein. Um nachhaltig die Verkehrsgewohnheiten von Menschen zu verändern, kann es aus meiner Sicht nur zwei Ansatzpunkte geben: erstens die Reibungslosigkeit auf der letzten Meile zu sichern und zweitens die Attraktivität des ÖPNV insgesamt zu erhöhen. Denn glauben Sie ernsthaft, dass sich der gutverdienende Autofahrer mit Tiefgarage am Unternehmenssitz hier in Berlin-Mitte durch die Ersparnis von 80 Euro – das ist nämlich der Durchschnittspreis eines ÖPNV-Monatstickets in Berlin – dazu bringen lässt, morgens um 7.30 Uhr in einen völlig überfüllten U-Bahn-Waggon oder S-Bahn-Waggon zu steigen, den er sich dann vielleicht auch noch mit Partygästen der letzten Nacht teilen muss? Das wird er sicher nicht.Damit der ÖPNV attraktiver wird und von mehr Menschen genutzt wird, die derzeit mit dem Auto fahren und dadurch die Emissionen in unseren Städten produzieren, meine ich erstens, dass die Kapazitäten des ÖPNV erhöht werden müssen. Nur mehr Züge können auch mehr Menschen befördern. In weiten Teilen des Netzes sind wir längst am Limit der möglichen Fahrgastzahlen angelangt. Zweitens meine ich, wir müssen bei der Gestaltung des ÖPNV neu und vor allem auch innovativ denken. Die Chancen der Digitalisierung müssen genutzt werden. Das Personenbeförderungsgesetz, das übrigens der Grund ist, warum ich in Brandenburg nur auf ein Taxi hoffen konnte und nicht auf eine Vielzahl von unterschiedlichen Mobilitätsangeboten, gehört dringend modernisiert.Innovative Mobilitätslösungen müssen zugelassen werden; denn wenn wir die Emissionen verringern wollen und gleichzeitig – hier wird häufig vergessen, was eigentlich mit dem ländlichen Raum ist – für die Menschen im ländlichen Raum Mobilität schaffen wollen, darf keine Idee ausgegrenzt werden. Es ist doch verrückt, dass wir uns hier über zu hohe Emissionswerte unterhalten, aber gleichzeitig leere Taxen vom Flughafen Schönefeld zurück nach Berlin-Mitte fahren, weil es den Berliner Taxen nicht gestattet ist, in Brandenburg Fahrgäste aufzunehmen. Das ist weder sozial noch ökologisch noch ökonomisch vertretbar.Der Vorschlag der geschäftsführenden Bundesregierung für einen kostenlosen ÖPNV ist darüber hinaus weder mit den Verkehrsbetrieben noch mit den Kommunen abgesprochen.Frau Lühmann, wenn Sie hier erst sagen, dass der Brief völlig unverbindlich ist, und im nächsten Moment betonen, dass Sie schon mit Modellregionen gesprochen haben, dann frage ich mich, was davon stimmen mag.Erst heute habe ich vom Verband Deutscher Verkehrsunternehmen – ein durchaus ernstzunehmender Verband in dieser Debatte – einen Brief bekommen, der vor diesem Vorschlag eindringlich warnt. Eine Zunahme der Fahrgastzahlen um 10 bis 15 Prozent führt derzeit unweigerlich zum Kollaps. Und dabei sind diese zusätzlichen Fahrgäste mit hoher Wahrscheinlichkeit nicht die von mir vorhin erwähnten Autofahrer.Kostenlos ist dieser Vorschlag ohnehin nicht. Er wird nur, wie so häufig, von jemand anderem bezahlt. In diesem Fall ist es wahrscheinlich die Umverteilung vom bayerischen Bauer hin zum Kreuzberger Beamten.Der vermeintlich kostenlose ÖPNV in Städten löst weder unsere Luft- noch unsere Verkehrsprobleme.Liebe Kolleginnen und Kollegen, lassen Sie uns gemeinsam an klugen Verkehrskonzepten arbeiten, mit besseren Angeboten des ÖPNV und zusätzlichen innovativen Mobilitätslösungen, die individuellen Verkehr kostengünstig für jedermann ermöglichen.Ich danke Ihnen, meine Damen und Herren.




	73. Rudolf Henke (CDU) Herr Präsident! Verehrte Damen! Meine Herren! Liebe Kolleginnen und Kollegen! Das ist nicht die erste Debatte, die wir über dieses Thema führen. Mit einem ähnlichen Gesetzentwurf von Bündnis 90/Die Grünen haben wir uns schon in der letzten Legislaturperiode befasst. Die Positionierung der Grünen und der FDP war Gegenstand der Sondierungsberatungen. Aber alle diesbezüglichen Punkte blieben gelb markiert, waren also offene Punkte. Der Abbruch der Verhandlungen durch die FDP hat dazu geführt, dass wir nie erfahren werden, wie die Sondierungsgespräche in diesem Punkt ausgegangen wären. „Selbst daran schuld“, muss man in Richtung FDP sagen.Das Spannungsverhältnis, das wir als Union zur SPD in diesem Punkt haben, bemerken wir auch nicht zum ersten Mal; das gab es schon in der letzten Legislaturperiode. Das führte dazu, dass wir im Jahre 2016 eine sehr ausführliche Anhörung zum Gesetzentwurf der Grünen durchgeführt haben. Ich will daran erinnern, was der Einzelsachverständige Professor Thomasius dazu vorgetragen hat.– Hören Sie erst einmal zu! – Er hat darauf aufmerksam gemacht, dass das Betäubungsmittelgesetz, über dessen Zukunft wir heute im Wesentlichen sprechen, ein Element einer Vier-Säulen-Politik ist, die in den letzten Jahren auch im Cannabisbereich ständig ausgebaut wurde. Es gibt Präventionsangebote und viele Hilfsmöglichkeiten für Cannabisabhängige. Es wurde schon darauf aufmerksam gemacht, dass sowohl im stationären als auch im ambulanten Bereich Cannabis, gemessen an der Häufigkeit der durchgeführten Therapien, an der Spitze der illegalen Stoffe steht. Cannabis ist also die Droge, die zu den meisten notwendigen Behandlungen führt.– Ich habe von den illegalen Drogen gesprochen. Ich komme gleich noch auf Alkohol und Tabak zu sprechen, weil wir in diesem Bereich natürlich eine integrierte Präventionspolitik brauchen.Die Gefährlichkeit von Drogen entscheidet sich in der Tat nicht allein an der Frage Legalität/Illegalität, sondern sie ergibt sich aus den Wirkmustern.Herr Kollege, Frau Strack-Zimmermann würde gern eine Zwischenfrage stellen.Bitte.Vielen Dank, Herr Kollege. Das ist sehr nett. – Es soll ein Projekt in der Landeshauptstadt Düsseldorf geben, aus der ich komme. Die Katholische Hochschule Nordrhein-Westfalen, deren Zentralverwaltung in Köln ist, hat sich bereit erklärt, es wissenschaftlich zu begleiten. Haben Sie das Gefühl, dass die Katholische Hochschule diesbezüglich nicht alle Tassen im Schrank hat? Glauben Sie nicht auch, dass sie vielmehr ein hohes Interesse daran hat, das sehr professionell, sehr sachlich und natürlich auch mit der nötigen Ernsthaftigkeit zu begleiten?Dazu kann ich mir kein Urteil erlauben, weil ich dazu in die wissenschaftliche Debatte über diese Studie eintreten müsste. Ich glaube aber, dass alle die, die derzeit Anträge stellen, übersehen, wie die Gesetzeslage ist. Das ist der Grund dafür, warum auch Sie selber sagen: Wir möchten Modellversuche ermöglichen. – Es wurden Beispiele dafür genannt, dass auch außerhalb von Düsseldorf Anträge gestellt wurden, beispielsweise in Münster.– Ja, diese Anträge sind alle auf der Basis der vorhandenen Rechtslage abgelehnt worden. Das ist, finde ich, nachvollziehbar.Einer der Teile, die wir zu diskutieren haben werden, betrifft die Frage, ob wir die Verbote für Modellversuche oder generell, wie es andere beantragen, aufheben wollen.Nur, ich finde, etwas Bestimmtes geht nicht, und dagegen wende ich mich. Wir haben dieses Vier-Säulen-Konzept: Wir haben die Prävention, wir haben die Hilfsmöglichkeiten – wir haben eine sehr ausdifferenzierte Therapielandschaft –, wir haben Konzepte der Schadensminimierung für riskant konsumierende Cannabiskonsumenten, und wir haben das Betäubungsmittelgesetz. Das Betäubungsmittelgesetz hat das Ziel, die Erzeugung und den Handel mit Betäubungsmitteln einzudämmen, um Kinder und Jugendliche vor der Verführung zu schützen.Ich finde, wir müssen schon mit Blick auf die Anhörung, die wir gehabt haben, widersprechen, wenn im FDP-Antrag steht: „Der Kampf gegen den Cannabiskonsum durch Repression ist gescheitert.“ Durch Umsetzung des FDP-Antrags würde die Wirksamkeit der Cannabiskontrollpolitik reduziert. „Die Verbotspolitik im Bereich Cannabis ist vollständig gescheitert“, heißt es in dem Antrag, den die Linke eingebracht hat. Oder im Gesetzentwurf der Grünen: „Die Prohibitionspolitik im Bereich von Cannabis ist vollständig gescheitert.“Ich finde, wir müssen schon zur Kenntnis nehmen, dass es vier oder fünf skandinavische Länder gibt, in denen noch weniger regelmäßig gekifft wird als in Deutschland. Aber gerade die Staaten, die Cannabis kontrolliert abgeben, die einen sehr nachlässigen Umgang mit Cannabis pflegen und wenig Prävention leisten und wenig Hilfsangebote machen, wie beispielsweise Tschechien, Portugal, Spanien oder Italien, haben deutlich höhere Konsumquoten.Zugleich zeigt die 2015 veröffentlichte Studie der Europäischen Beobachtungsstelle für Drogen und Drogensucht, dass nirgendwo sonst in Europa so viele regelmäßige Cannabiskonsumenten Hilfsangebote wie in Deutschland erhalten. Das heißt, wir haben mit dieser Vier-Säulen-Strategie, zu der das, was im Betäubungsmittelgesetz verankert ist, zählt, von der Europäischen Beobachtungsstelle für Drogen und Drogensucht Bestnoten bekommen. Deswegen, finde ich, sollte man ein bisschen mit der Notwendigkeit operieren, dass auch eine Evaluation der vierten Säule tatsächlich stattfindet.Herr Kollege, es gibt noch den Wunsch nach einer Zwischenfrage einer Kollegin der Fraktion Bündnis 90/Die Grünen.Einverstanden.Herr Kollege, vergleichen Sie da nicht Äpfel mit Birnen, wenn Sie sagen: „Wir in Deutschland haben die besten Präventionsangebote“, und dann Staaten mit einem steigenden Cannabiskonsum anprangern, von denen Sie selber sagen, sie hätten völlig unzureichende Präventionsangebote? – Das ist das eine.Suggerieren Sie nicht völlig falsche Zahlen, wenn Sie vernachlässigen, dass über 2 Millionen Menschen in Deutschland von legalen Schmerzmitteln und über 1 Million Menschen von Alkohol abhängig sind, also von legalen Drogen? Wir wissen doch, dass die Deklaration von „legal“ und „illegal“ willkürlich ist. Ihrem Vergleich liegen völlig unterschiedliche Voraussetzungen zugrunde. Widersprechen Sie sich da nicht selber?Nein, ich widerspreche mir da nicht. Wir sind uns komplett einig. Das ist aber nicht Gegenstand dieser Vorlagen.Wir haben ungefähr 100 000 Tabaktote in Deutschland im Jahr. Wir haben ungefähr 40 000 Alkoholtote in Deutschland im Jahr. Wer den Eindruck erweckt, als wäre das Thema Sucht mit den illegalen Suchtstoffen abgehandelt, der hat nicht verstanden, wie es tatsächlich ist.Für Tabak wie für Alkohol wie für die illegalen Suchtstoffe gilt „Prävention, Prävention, Prävention“ als der zentrale Ansatzpunkt. Wir haben aus diesem Grund in der letzten Legislaturperiode gemeinsam die Möglichkeiten geschaffen, die das Präventionsgesetz nun bietet, und wir haben uns für diese Legislaturperiode vorgenommen, die Möglichkeiten auszubauen. Wir haben in der Koalitionsvereinbarung gesagt, dass wir bei Tabak und Alkohol die heutigen Möglichkeiten gezielt ergänzen wollen. Dazu würden mir viele Dinge einfallen. Das werden wir dann, wenn es dazu kommt, zu bereden haben.Dass wir im Bereich von Tabak und Alkohol ein ungelöstes Problem haben, ändert doch nichts daran, dass es falsch ist, im Bereich Cannabis aus einem mäßig bis schlecht beherrschten Problem ein völlig ungelöstes Problem zu machen. Das ist doch keine Logik, die überzeugt.Mich stört an allen drei Vorlagen, die jetzt zur Debatte stehen, dass so getan wird, als sei die deutsche Cannabiskontroll- und -ablehnungspolitik gescheitert.Wenn wir uns die Prävalenzraten ansehen, also die Zahl von Jugendlichen und jungen Erwachsenen zwischen 15 und 34 Jahren, die in den letzten zwölf Monaten Cannabis konsumiert haben, stellen wir fest: Wir liegen in Deutschland mit 13,3 Prozent unter der Rate der Niederlande von 16,1 Prozent. Wir liegen unter der Rate von Spanien: 17,1 Prozent. Wir liegen deutlich unter der Rate von Frankreich: 22,1 Prozent. Das sind Zahlen der Europäischen Beobachtungsstelle für Drogen und Drogensucht.Was ich damit sagen will, ist, dass ich es fahrlässig finde, von vornherein das über Bord zu werfen, was wir als Erfolg haben. Es trifft nicht zu, dass die Cannabispolitik in Deutschland gescheitert ist. Wer sich anschaut, was die Beobachtungsstelle für Drogen und Drogensucht ausweist, der sieht das genaue Gegenteil. Wir bekommen mit einigen skandinavischen Ländern zusammen Bestnoten, was die Frage der Cannabiskontrolle angeht.Daran möchte ich nicht einfach im Hauruckverfahren, nur weil es populär ist und weil es in den Schulen gut ankommt, etwas ändern.Herr Kollege, denken Sie an Ihre Zeit.Ja, die ist nun schon erheblich überschritten.Allerdings. Ich habe die Uhr bei der Zwischenfrage auch angehalten.Es gab ja noch ein paar weitere Fragen. Aber die lassen Sie vielleicht nicht mehr zu. Ich wäre bereit, sie zu beantworten. Wenn es hier nicht geht, dann gern in den Ausschussberatungen, auf die ich mich freue.Ich glaube, dass wir die Ausschussberatungen brauchen. Ich finde, wir sollten sie nicht mit einer ideologischen Vorfestlegung,nämlich „Das Betäubungsmittelgesetz muss weg“, führen.Herr Kollege.Wir sollten seine Erfolge fair bewerten. Auf diese Diskussion freue ich mich.Herzlichen Dank für Ihre Aufmerksamkeit.Herr Kollege, ich glaube, es ist eine gute Idee, das in den Ausschüssen weiterzuberaten. – Ich schließe die Aussprache.




	74. Christian Lindner (FDP) Herr Präsident! Liebe Kolleginnen und Kollegen! Die Vorrednerin der AfD hat über vieles gesprochen. Wir haben nicht vergessen, dass vor der Bundestagswahl auf Einladung der AfD der frühere Chef der britischen Europahasser in Berlin zu Gast war.Deshalb ist es kein Zufall, dass die Vorrednerin zwar über vieles gesprochen hat, dass sie aber nicht erwähnt hat, dass die deutschen Beiträge zum Haushalt der Europäischen Union unserer Wirtschaft den Zugang zum größten Binnenmarkt der Welt eröffnen.Wer das unterschlägt oder bekämpft, der will unser Land in das gleiche Chaos stürzen wie das Vereinigte Königreich.Europa steht vor großen Herausforderungen: Flüchtlingskrise, Globalisierung und Digitalisierung, die Konflikte in unserer Nachbarschaft, die Schuldenkrise. All das kann man mit Tatkraft bewältigen. Die wirkliche Gefahr geht von den simplen und falschen Antworten von Nationalismus und Populismus aus.Meine Damen und Herren, Europa steht vor großen Herausforderungen. Das Ausscheiden Großbritanniens aus der EU sowie die veränderte Weltlage sind der Anlass und die Wahl von Emmanuel Macron in Frankreich ist die Chance, jetzt ein Jahrzehnt der Erneuerung des europäischen Einigungsprojektes zu begründen. Frau Bundeskanzlerin, Frankreich ist dabei zum Taktgeber avanciert. Der französische Präsident unterbreitet klare und konkrete Vorschläge. Sie dienen zum Teil der Interessenlage seines Landes;wer wollte ihm das verdenken. Deshalb kann ein schlichtes Echo auf die Pariser Ideen nicht die deutsche Haltung widerspiegeln. Es ist nicht bereits Europafreundlichkeit, Anrufe von Herrn Macron entgegenzunehmen.Wir haben heute, Frau Bundeskanzlerin, eine klare und konkrete deutsche Position erwartet. Wir haben diese klare und konkrete Position heute nicht gehört – zumindest nicht von Ihnen, sondern höchstens von Frau Nahles.Erstens werden Reformen der europäischen Institutionen diskutiert. Sie haben die Verkleinerung des Europäischen Parlaments zu Recht begrüßt. Wenn die EU mit 450 Millionen Einwohnern nach 2019 ein kleineres Parlament haben wird als die Bundesrepublik Deutschland mit 82 Millionen Einwohnern, dann nimmt das dieses Haus in die Pflicht für eine Parlaments- und Wahlrechtsreform.Wir haben, Frau Bundeskanzlerin, aber nichts von Ihnen gehört, wie Sie die Rolle des Europäischen Parlaments aufwerten wollen – beispielsweise dadurch, dass von der Kommission ausgehandelte Freihandelsabkommen zukünftig prioritär auf europäischer Ebene statt in den Mitgliedstaaten parlamentarisch gebilligt werden. Denn: Die wirtschaftliche Existenzfrage der Handelspolitik braucht Handlungsfähigkeit.Frau Bundeskanzlerin, zur Verkleinerung der Europäischen Kommission war ein Rotationsmodell vorgesehen. Die Bundesrepublik Deutschland ist in Vorleistung getreten und hat auf einen Kommissar verzichtet. Aber dieses Rotationsmodell ist noch nicht umgesetzt. Der französische Präsident schlägt nun 15 statt 27 Kommissare vor; das geht in die richtige Richtung. Hat die Bundesregierung dazu keine Position? Eine effizientere, auf die Kompetenzen der EU konzentrierte Kommission muss ein deutsches Anliegen sein.Es gibt Einvernehmen, die Gemeinsame Außen- und Sicherheitspolitik auszubauen. Dazu hat Deutschland zunächst einmal – wie wir in dieser Woche erneut vor Augen geführt bekommen haben – zu Hause Aufgaben zu erledigen. Die Bundeswehr braucht mehr als warme Worte angesichts der Tatsache, dass sie nach zwölf Jahren einer unionsgeführten Bundesregierung nur noch bedingt einsatzbereit ist.Die Gemeinsame Außen- und Sicherheitspolitik kann sich allerdings nicht nur auf die Verteidigungskomponente beziehen, sondern muss die diplomatische Seite miteinbeziehen. Wir wollen deshalb, dass die Rolle der Hohen Vertreterin gestärkt wird. Entscheidungen müssen auch mit qualifizierter Mehrheit möglich sein. Wir dürfen uns durch das Einstimmigkeitsprinzip auf europäischer Ebene nicht länger selbst lähmen.Und: Die Europawahl muss aufgewertet werden. Wir wissen, dass die EVP Spitzenkandidaten für die europäischen Parteienfamilien begrüßt. – Sie haben das hier gesagt; das ist auch richtig so. Damit kann die stärkste Fraktion mit vielleicht 30 Prozent aber nicht automatisch die Spitze der Kommission für sich beanspruchen; denn das wäre nicht Ausdruck von Europafreundlichkeit, sondern nur von Machtkalkül.Von wirklich praktischer Bedeutung indessen wären transnationale Listen. Dazu haben wir nichts gehört. Nur mit transnationalen Listen wird aus 27 nationalen Einzelwahlen eine wirklich europäische Wahl, und darauf sollte die Bundesregierung hinwirken. Es ist nicht zu verstehen, dass ausgerechnet die EVP sich gegen diesen Vorschlag gewandt hat.Wir werden deshalb heute dem von Bündnis 90/Die Grünen vorgelegten Entschließungsantrag zustimmen.Unverändert hat das Parlament mit Brüssel und Straßburg zwei Sitze. Dieser Wanderzirkus ist teuer, ineffizient und überholt.Deshalb sollten Sie mit dem französischen Präsidenten darüber sprechen, Brüssel zum alleinigen Sitz des Parlaments zu machen.Statt über diese Fragen zu reden, haben Sie, Frau Bundeskanzlerin, über europaweite Bürgerdialoge gesprochen. Dagegen ist ja gar nichts zu sagen. Aber aus den schon geführten Gesprächen mit den Bürgerinnen und Bürgern wissen wir, dass sie die bereits bekannten Strukturprobleme endlich gelöst sehen wollen.Frau Merkel, Sie haben zweitens zum Finanzrahmen der EU gesagt, die Debatte über den Haushalt dürfe nicht von der Debatte getrennt werden, welches Europa wir wollen. Wenn man Ihr Wort ernst nimmt, dann muss zuerst über Aufgaben und Ziele gesprochen werden und danach über die dafür benötigten Mittel. Wir Freien Demokraten wollen überall da ein starkes Europa, wo das Zusammenwirken Mehrwert schafft. Das ist in der Migrations-, der Verteidigungs- und in der Entwicklungspolitik zweifelsohne der Fall. Mehr Investitionen in disruptive Technologien und die Förderung von Projekten des privaten Sektors sowie die Forschung begrüßen wir ohnehin. Insbesondere Ihre Aussagen zur Personalverstärkung bei Frontex unterstützen wir. Diesen Worten müssen Taten folgen.Haushaltskommissar Günther Oettinger hat zur Finanzierung dieser neuen Prioritäten vorsichtige Kürzungen bei der Gemeinsamen Agrarpolitik und bei der Kohäsionspolitik vorgeschlagen. Sie gehen nicht weit genug, aber immerhin. Der Finanzwissenschaftler Friedrich Heinemann hat den Stand der Forschung in einem Beitrag für „Die Welt“ vom 14. Februar mit den Worten zusammengefasst:… die Kohäsionspolitik steht … im Verdacht, eher die Interessen von Landräten und Bürgermeistern zu befriedigen, als wirklich einen europäischen Mehrwert abzuliefern.Die Parteien der Großen Koalition haben hingegen bereits vorab erklärt, dass sie zu höheren Beiträgen zum Haushalt der EU bereit seien. Sie bekennen sich zu einer starken Kohäsionspolitik. Damit schwächen Sie die deutsche Verhandlungsposition. Damit stellen Sie das Verfahren auf den Kopf. Damit fallen Sie letztlich Günther Oettinger in den Rücken. Es ist nicht europafreundlich, pauschal mehr Geld ausgeben zu wollen. Es ist europafreundlich, das Geld der Bürgerinnen und Bürger zunächst besser einsetzen zu wollen.Erst wenn die Effizienzreserven im Haushalt der EU gehoben sind, dann kann über die Höhe der Beiträge befunden werden.Wir wollen höhere deutsche Zahlungen nicht ausschließen, aber pauschal anbieten sollte man sie auch nicht.Die Aussagen zum Finanzrahmen sind im Grunde exemplarisch dafür, dass es eine Zäsur in der deutschen Europapolitik gibt; Frau Nahles hat das ja auch in aller Klarheit ausgesprochen. Frau Nahles, wenn Deutschland seinen Partnern jetzt auch noch vorschreiben will, wie sie ihren Sozialstaat zu organisieren haben, dann läuten Sie die nächste Phase der Europaskepsis auf diesem Kontinent ein.Über diese Veränderungen in der Politik, insbesondere hinsichtlich der Wirtschafts- und Währungsunion, hat Frau Merkel nur kurz gesprochen, Frau Nahles länger. Frau Nahles, Sie haben offen ausgesprochen, dass es um den Ausgleich von Unterschieden geht. Dem ist zuzustimmen, wenn es um die Stärkung von Wettbewerbsfähigkeit und um private Investitionen geht. Die Jugendarbeitslosigkeit in Italien hat aber nichts mit der Austeritätspolitik von Merkel oder Schäuble zu tun. Das fällt in die Verantwortung der Berlusconis in Europa, die über Jahrzehnte notwendige Reformen verschleppt haben.Im Koalitionsvertrag steht sehr deutlich, um was es geht: um einen Investivhaushalt für die Euro-Zone, der unter anderem für Stabilisierung und Konvergenz genutzt werden soll. Das deutsche Wort dafür heißt Finanzausgleich. Mehr noch: Der französische Finanzminister Bruno Le Maire hat in einem Interview mit der „FAZ“ am 14. Februar erklärt, der einst als Rettungsschirm gedachte Stabilitätsmechanismus solle mehr oder weniger in ein Transferinstrument umgebaut werden. Es müsse über Liquiditätshilfen gesprochen werden, die in definierten Fällen bereitgestellt würden, also nicht nur im Falle von Krisen. Der ESM solle unter Unionsrecht gestellt werden, damit er schneller entscheiden kann. – Welche Rolle spielt dann der Bundestag? – Der ESM solle die Letztabsicherung für die Abwicklung nicht lebensfähiger Finanz­institute übernehmen, mit dem Geld der Steuerzahler.In der Sache hat der Präsident der Deutschen Bundesbank solche Vorschläge eingeordnet. Er sagt, das wirke auf ihn wie die „Lösung auf der Suche nach einem Problem, zumindest wenn man die bestehenden Regeln und Vorkehrungen in der Wirtschafts- und Währungsunion ernst nimmt“.Herr Weidmann hat recht. Mit diesen Vorschlägen wird die finanzpolitische Eigenverantwortung der Mitgliedstaaten ausgehöhlt. Das ist nicht nur rechtlich problematisch, es ist ökonomisch unklar. Le Maire hat in diesem Interview übrigens bekräftigt, er sei in diesen Fragen mit seinen deutschen Gesprächspartnern bereits einig.Wir hätten gerne heute von Ihrer Regierung gewusst, Frau Merkel: Mit wem hat er verhandelt? Welche Zusagen hat es gegeben? Wenn solch weitreichende Fragen nicht Gegenstand einer Regierungserklärung sind, welche dann?Frau Merkel, Sie haben heute eine Regierungserklärung abgegeben, aber aufschlussreicher als das, was Sie gesagt haben, war das, was Sie nicht gesagt haben; denn erklärt hat sich diese Regierung nicht.




	75. Silke Launert (CSU) Sehr geehrter Herr Präsident! Liebe Kolleginnen und Kollegen! Sehr geehrte Damen und Herren! Auch wenn es schon gesagt wurde, so ist es doch so wichtig, dass ich es noch einmal wiederhole:Die Würde des Menschen ist unantastbar. Sie zu achten und zu schützen ist Verpflichtung aller staatlichen Gewalt.Ich bin mir bewusst – viele haben es gehört, und sie werden es noch öfter hören –: Das sind nun einmal zwei Sätze, die alles ausmachen, wofür unsere Gesellschaft steht. Das ist der Kern unserer gesellschaftlichen Ordnung. Sie zitieren diesen Artikel ja auch oft genug. Es geht hier in der Tat um die Würde des Menschen. Ich möchte Sie bitten: Hören Sie zu, vielleicht erkennen Sie, weshalb wir hier einen Zusammenhang annehmen und weshalb es nicht nur um den § 219a StGB geht.Es geht hier um das Recht auf Leben. Es geht um die Menschenwürde. Das Bundesverfassungsgericht hat ganz klar und eindeutig festgestellt: Die Menschenwürde – Artikel 1 Grundgesetz – und das Recht auf Leben – Artikel 2 Grundgesetz – stehen auch dem ungeborenen menschlichen Leben zu.Dieses Recht auf Leben besteht gegenüber jedem, auch gegenüber der Mutter.In Artikel 1 Satz 2 unseres Grundgesetzes steht noch etwas anderes: Der Staat muss die Würde nicht nur achten, sondern er ist auch verpflichtet, sie zu schützen. Wir, der Staat, sind verpflichtet, das ungeborene Leben mit seinem Recht auf Leben zu schützen.Das ist unser Auftrag. Ganz ehrlich: Deshalb kommen mir in allen Gesetzentwürfen – ich habe sie alle gelesen – dieses Recht und unsere Schutzpflicht viel zu kurz. Es tut mir leid.Ich verkenne nicht, was es für eine Frau bedeutet. Ich kann mich sehr gut in die Lage einer Frau hineinversetzen, die in einer schwierigen Lebenssituation nicht so gerne schwanger werden will. Glauben Sie mir: Wir von der Union können das durchaus nachvollziehen. Auch das Bundesverfassungsgericht sieht die Gegenrechte. Natürlich gibt es Gegenrechte der Mutter:das allgemeine Persönlichkeitsrecht und das Recht auf Leben und Gesundheit.Wir verkennen auch nicht, dass es sich keine Frau in dieser Situation leicht macht. Aber das Bundesverfassungsgericht führt aus – offensichtlich muss man mehrfach das Bundesverfassungsgericht zitieren, weil es sonst keiner glaubt; es ist nun mal so –: „Die Vorkehrungen, die der Gesetzgeber“ – also wir – „trifft, müssen für einen angemessenen und wirksamen Schutz“ des ungeborenen Lebens „ausreichend sein“. Wo habe ich irgendetwas dazu gehört?Das Bundesverfassungsgericht führt weiter aus, Mindestanforderung sei es, dass der Schwangerschaftsabbruch – wortwörtlich – „grundsätzlich als Unrecht angesehen wird und demgemäß rechtlich verboten ist“. Das ist ein Zitat des Bundesverfassungsgerichts.Auch wenn Ausnahmeregelungen zugelassen werden, gerade weil wir die schwierige Lage von Frauen – –– Hören Sie zu! Vielleicht verstehen Sie dann den Zusammenhang.– Nein, offensichtlich nicht. Bemühen Sie sich mal, sich reinzuversetzen! – Also: Selbst dann, wenn wir Ausnahmeregelungen zulassen, was das Bundesverfassungsgericht akzeptiert hat, dann – so hat es das Gericht ausgeführt – gibt es einen Schutzauftrag des Staates. Unsere Pflicht ist es– hören Sie zu! –,„den rechtlichen Schutzanspruch des ungeborenen Lebens im allgemeinen …“ – –– Herr Präsident!Lassen Sie sich nicht stören, Frau Kollegin. Machen Sie einfach weiter.Ich wiederhole es noch einmal – Zitat des Bundesverfassungsgerichts –: Der Schutzauftrag verpflichtet den Staat, „den rechtlichen Schutzanspruch des ungeborenen Lebens im allgemeinen Bewußtsein zu erhalten und zu beleben“.Wir sind verpflichtet, sicherzustellen, dass die Menschen verstehen, dass das kein Standard ist, dass das kein Normalfall ist.Das ist unsere Pflicht; das sagt uns das Bundesverfassungsgericht ganz klar.Deshalb haben wir dieses ausdifferenzierte System. Genau deshalb gibt es in Ausnahmefällen eine Zulassung, weil man sagt, es sei für die Frau unzumutbar.Wir haben gleichzeitig eine Beratungspflicht. Die Beratung durch unabhängige Stellen muss von dem Bemühen geleitet sein, der Frau die Möglichkeit zu geben, sich vielleicht doch für das Kind zu entscheiden.Diese Beratung darf nicht vom Arzt durchgeführt werden, der dann die Abtreibung vornähme. Wenn das zulässig wäre, bestünde die Gefahr der Kommerzialisierung, und es wäre im Hinblick auf die anderen Regelungen, die wir für Frauen erreicht haben, grob fahrlässig und gefährlich, das auszublenden.Nur noch ganz kurz: Für Ärzte gibt es auch ansonsten Werbeverbote.Eigentlich ist die Zeit zu Ende, Frau Kollegin.Die Zurufe haben viel Zeit gekostet. –Wir haben im Moment eine Gesellschaft, in der man die Werbung für Tabak verbieten, aber die Werbung für den Abbruch der Schwangerschaft legalisieren will.Das verstehe, wer will.Ich schließe die Aussprache.Interfraktionell wird Überweisung der Gesetzentwürfe auf den Drucksachen 19/630, 19/820 und 19/93 an die in der Tagesordnung aufgeführten Ausschüsse vorgeschlagen. Gibt es dazu anderweitige Vorschläge? – Das ist nicht der Fall.Dann berufe ich die nächste Sitzung des Deutschen Bundestages auf morgen, Freitag, 23. Februar, 9 Uhr, ein.Ich wünsche einen allseits schönen und erfolgreichen Abend. Die Sitzung ist geschlossen.




	76. Esther Dilcher (SPD) Sehr geehrter Herr Präsident! Sehr geehrte Kolleginnen und Kollegen! Meine verehrten Damen und Herren! Menschen in Europa und Deutschland sind sich durchaus einig, einen Ganzkörperschleier als zivilisations- und frauenfeindlich zu betrachten. Die Mehrzahl der Muslima aber trägt gar keine Burka oder keinen Nikab.Betroffen ist nur eine verschwindend kleine Minderheit.Das Kopftuch hingegen ist viel weiter verbreitet und auch bei uns sichtbar. Auch das Tragen dieser religiösen Kopfbedeckung bewegt gelegentlich die Gemüter. Aber auch Frauen ohne Kopftuch können überzeugte und praktizierende Muslima sein. Der Vorsitzende der Türkischen Gemeinde in Deutschland, Herr Sofuoglu, äußerte hierzu, es sei besser, auf die Moscheegemeinden einzuwirken, statt Verbote zu erlassen. Warum äußert er das? Warum hat er damit auch recht? Der EuGH, der Europäische Gerichtshof, hat, wie schon zitiert, die Verbote der Vollverschleierung in Belgien und Frankreich bestätigt. Aber trotzdem ist gerade in diesen Ländern die Zahl der Burka- oder Nikabträgerinnen keineswegs zurückgegangen und auch nicht rückläufig. Die Terrorgefahr in diesen Ländern ist seit diesem Erlass der Gesetze sogar noch gestiegen. Es gibt daher den von Ihnen dargestellten kausalen Zusammenhang keineswegs.Aber unsere Gesellschaft wird weiter gespalten und Hass auf alles Fremde weiter geschürt.Der vorliegende Antrag wird mit „dem Schutz des Individual-Freiheitsrechts …, der inneren Sicherheit und dem staatlichen Ziel der Sicherstellung des gesellschaftlichen Zusammenlebens“ begründet. Dieses Ziel wird mit dem Verbot der Vollverschleierung jedoch nicht erreicht; schauen wir nur in unsere Nachbarländer. Liebe Kolleginnen und Kollegen, hier schleicht ein Wolf im Schafspelz durch unser Hohes Haus.Motiviert ist der Antrag nämlich nicht aus den aufgeführten Gründen, sondern, wie die Antragsteller in der Presse bereits lanciert haben – dort lag der Antrag nämlich schon vor, bevor wir ihn auf den Schreibtischen hatten – und wie es gerade bestätigt worden ist, als Maßnahme gegen die kulturelle Landnahme. Diese Begrifflichkeiten sind uns auch schon bekannt, und zwar aus einer Zeit, die mehr als 70 Jahre zurückliegt.Sehen Sie wirklich eine Gefahr für unser Land durch 300 Muslima? Es geht hier einmal mehr darum, den Islam in Deutschland zu stigmatisieren. Die Mütter und Väter unseres Grundgesetzes haben aufgrund der selbsterlebten Geschichte und der Erfahrungen mit nationalsozialistischem Gedankengut zu Recht in unserem Grundgesetz in Artikel 1 formuliert:Die Würde des Menschen ist unantastbar.Ganz bewusst wird nicht auf die Würde der Deutschen abgestellt. In Artikel 4 des Grundgesetzes wird die Freiheit des Glaubens als unverletzlich festgestellt und die ungestörte Religionsausübung gewährleistet. Jetzt fragen wir uns, wenn wir tatsächlich eine Gefahr für unsere innere Sicherheit sehen: Wie können wir diese gewährleisten? Wie können wir die Unterdrückung muslimischer Frauen tatsächlich verhindern? Und wie können wir unser Zusammenleben sichern? Mit Sicherheit nicht mit Verboten, die diese verschwindend geringe Minderheit von Frauen weiter ins Abseits stellen und ausgrenzen, sondern durch Angebote zur Integration, durch Kommunikation und durch Respekt vor dem Fremden,durch Solidarität in der Gemeinschaft und durch Toleranz.Meine Würde als Mensch gebietet es mir, solchen Anträgen, die auf Diffamierung und Hetze ausgerichtet sind, entgegenzutreten und diese nicht weiter zu unterstützen. Wie können wir sonst sicher sein, dass wir nicht selbst irgendwann zu den Minderheiten gehören, die nicht in das nationale Gedankengut der AfD passen und gegen die sich dann ihre Angriffe richten?Vielen Dank, meine Damen und Herren.




	77. Martin Rabanus (SPD) Vielen Dank, Frau Präsidentin. – Liebe Kolleginnen und Kollegen! Meine sehr verehrten Damen und Herren! In der Debatte ist schon deutlich geworden, dass es hier mitnichten – auch bei dem Antrag – um den Fall Deniz Yücel geht, sondern um eine Haltung. In düsterer Art und Weise ist hier deutlich geworden, was die Kolleginnen und Kollegen der AfD zu Pressefreiheit und derlei mehr denken. Es gibt aber auch den Anlass – das ist das Schöne an der Debatte –, festzustellen, dass eine übergroße Mehrheit dieses Parlamentes das anders sieht. Das ist sozusagen die positive Botschaft, die auch mit der Debatte verbunden ist.– Man hat so ein klein bisschen ein Gefühl dafür, was die Redner so sagen, und das ist auch gut so. – Wenn wir über Haltung reden und über die Frage, wie wir uns unser Deutschland vorstellen – denn darum geht es –, dann will ich sagen: Ich stelle mir ein Deutschland vor, wie es im Jahre 2018 ist, und nicht ein Deutschland wie in den 80er-Jahren des letzten Jahrhunderts, in den 50er-Jahren und schon gar nicht in den 30er- oder 40er-Jahren des letzten Jahrhunderts, sondern eben ein heutiges Deutschland.Ich möchte in einem Deutschland leben, in dem unsere Jungs in der Fußballnationalmannschaft – Cem Özdemir hat das angedeutet – ganz selbstverständlich Müller, Kimmich oder Draxler heißen, aber genauso selbstverständlich eben auch Özil, Khedira oder Boateng.Und wenn sie bei mir in der Nachbarschaft wohnen, dann gefällt mir das auch ausgesprochen gut.Ich möchte in einem Land leben, in dem es tolerant und weltoffen zugeht.– Ja, und das soll auch so bleiben.Und genau deswegen müssen wir hier auch ziemlich deutlich werden. Denn wir müssen unsere politische Arbeit doch darauf richten, Teilhabe von Menschen zu ermöglichen, Menschen am Arbeitsmarkt, an Bildung teilhaben zu lassen, Zugang zu sozialen Sicherungssystemen, Kunst und Kultur zu ermöglichen, um nur ein paar Stichworte zu nennen. Ich möchte in einem Deutschland leben, in dem unterschiedliche Lebenskonzepte, Familien- und Beziehungskonzepte möglich sind.– Doch, um all diese Fragen geht es. Wie sieht dieses Deutschland eigentlich aus?Ich möchte in einem Land leben, in dem Meinungsfreiheit herrscht,in dem die Freiheit von Kunst und Kultur geachtet und geschützt wird, in dem Pressefreiheit respektiert und geschützt wird.Ich möchte in einem Land leben, in dem Journalisten, Satiriker, Kritiker, Querdenker, Kabarettistenihre Arbeit frei machen können, wo sie zuspitzen können,wo sie provozieren können, und zwar egal, ob das jetzt Dieter Nuhr aus Wesel oder Deniz Yücel aus Flörsheim ist. Das ist nämlich ein hessischer Bub, sehr geehrte Damen und Herren von der AfD.Mir muss übrigens auch nicht alles gefallen, was die Kabarettisten sagen. Darum geht es überhaupt nicht. Ich muss mir das auch nicht zu eigen machen. Ich muss auch nicht bei allem jubeln. Auch ist klar in unserem Land: Wer in der Ausübung seiner eigenen Rechte, seiner Freiheitsrechte, seiner künstlerischen Freiheit die Rechte anderer verletzt, der muss sich dafür verantworten. Auch das ist klar. Aber das ist eine Frage, die im rechtsstaatlichen System rechtsstaatliche Stellen zu klären haben und die nicht politische Reglementierung hervorrufen darf.Ich jedenfalls will in einem Deutschland leben, das nie wieder auch nur in den Verdacht kommt, dass politische Zensur zulässig sei.Um es klarzumachen: Es geht Ihnen nicht um den Fall. Es geht Ihnen um Ihre Haltung, eine, die Meinungs-, Kunst- und Pressefreiheit mehr oder weniger subtil infrage stellt. Dazu kann ich Ihnen abschließend zwei Sachen sagen. Das eine: Es ist unanständig und erbärmlich. Das Zweite: Es wird auf unseren Widerstand stoßen.Vielen Dank.




	78. Eckhardt Rehberg (CDU) Herr Präsident! Liebe Kolleginnen und Kollegen! Ich nehme mir mal das Recht heraus, als Haushälter im Deutschen Bundestag durchaus kritisch zu hinterfragen, was mit dem Steuergeld passiert, das wir nach Brüssel überweisen.Erstens stelle ich mir die Frage: Welcher Mehrwert entsteht für uns, wenn wir zusätzliches Geld nach Europa geben? Beschränkt man sich auf das Wesentliche – Kollege Hahn ist darauf eingegangen: Grenzschutz, Verteidigung, Fragen der Migration, Digitalisierung des Binnenmarkts –, oder beschäftigt man sich mit Bürokratie, wovon etwa die Fischer bei mir zu Hause betroffen sind? Landwirte in Mecklenburg-Vorpommern müssen einen Antrag mit fast 100 Seiten ausfüllen, um einen kleinen regionalen Hofladen einrichten zu können. Das alles sind Vorschriften der Europäischen Union!Liebe Kolleginnen und Kollegen der SPD, zusätzliches Geld für Europa ist nicht in den 46 Milliarden Euro für prioritäre Maßnahmen enthalten. Es gibt auch keine finanzielle Vorsorge für massive Aufstockungen der Überweisungen aus dem deutschen Bundeshaushalt an den europäischen Haushalt.Das anzusprechen, gehört zur Wahrheit dazu. Wir reden über etwas, was bisher nicht vorgesehen ist. Nur wenn wir zusätzliche Spielräume nach 2020 für die nächste Förderperiode haben, können wir uns all die Dinge vornehmen, die mancher beschreibt.Zweitens. Ich bin schon dafür, auch mal zu fragen: Wie sieht es mit den Gegenleistungen der Nettoempfängerländer aus – nicht nur bei der Verteilung der Asylbewerber, sondern auch bei der Einhaltung von Rechtsstaatsprinzipien? Oder – Europa ist auch deswegen so erfolgreich, weil wir regelbasiert sind –: Werden Regeln eingehalten? Ich kann nicht verstehen, dass die EU-Kommission bei weit über 100 Fällen von Verstößen gegen den Fiskalpakt gesagt hat: Schwamm drüber. – Das gehört für mich zu einer ehrlichen Debatte dazu.Liebe Kolleginnen und Kollegen, ein Zitat des möglichen neuen Bundesfinanzministers Scholz:Wir wollen anderen europäischen Staaten nicht vorschreiben, wie sie sich zu entwickeln haben.Da sind in der Vergangenheit sicherlich Fehler gemacht worden.Ich weiß nicht, wo wir wem etwas vorschreiben wollen, aber wenn gemeinsam verabredete Regeln – wie im Fiskalpakt, wie in anderen Verträgen – nicht eingehalten werden, dann sind, glaube ich, ein Hinweis und auch ein kritisches Nachfragen nötig.Drittens. Wir brauchen Klarheit hinsichtlich Einsparungen im EU-Haushalt. Frau Nahles, die 6 Milliarden Euro zum Kampf gegen die Jugendarbeitslosigkeit sind abgeflossen. Wenn ich mir die Kritik des Europäischen Rechnungshofes oder die „Erfolge“ dieses 6-Milliarden-Programms anschaue, glaube ich, dass der eine oder andere Mitgliedstaat nach Deutschland hätte gucken können und sich von unserem dualen Ausbildungssystem eine Scheibe hätte abschneiden können. Bei der hohen Zahl der Jugendarbeitslosigkeit haben die 6 Milliarden Euro nach meinem Dafürhalten jedenfalls keinen Effekt gebracht.Noch eine weitere Frage muss gestellt werden dürfen. Gucken Sie sich den EU-Haushalt 2017 an: Kürzungen in Höhe von 10 Milliarden Euro bei wichtigen Fonds: beim Europäischen Sozialfonds, beim Europäischen Strukturfonds – EFRE –, beim Fonds für die ländlichen Räume – ELER. Grund: Die Kommission und die Mitgliedstaaten sind nicht in der Lage, das Geld umzusetzen. 10 Milliarden Euro, liebe Kolleginnen und Kollegen! Vorausschau für 2018: Noch einmal Kürzungen in Höhe von 5 Milliarden Euro. Das heißt: In zwei wichtigen Jahren ist die Europäische Kommission in Brüssel nicht in der Lage, Geld umzusetzen; denn mit diesen Geldern kann man Strukturreformen machen und Wettbewerbsfähigkeit herstellen. Deswegen: Ehe man nach neuem, nach mehr Geld ruft, sollte man das vorhandene vernünftig und effektiv einsetzen.Viertens. Wir brauchen Klarheit über die Strukturfondsmittel. Ich kann mich sehr gut an die Debatte 2012/2013 erinnern. Für Deutschland, die neuen Bundesländer, die Phasing-out-Regionen gab es damals Kürzungen um etwa 30 Prozent. Ehe wir darüber reden, dass wir mehr Geld zur Verfügung stellen, möchte ich wissen: Wie sehen die Fördergebietskulissen aus? Wie sehen die Beihilferegelungen aus? Dabei erwarte ich von Brüssel einen transparenten Prozess. Es kann nicht so laufen, wie der französische Finanzminister es macht: Mehr Geld fordern, ohne dass andere wichtige Dinge – übrigens auch für französische Landwirte, für die ländlichen Räume dort – geklärt sind. Deswegen, liebe Kolleginnen und Kollegen: Klarheit bei den Strukturfondsmitteln!Fünfte und letzte Bemerkung. Der Bundeshaushalt hat im letzten Jahr einen deutlichen Jahresüberschuss gehabt. 7,4 Milliarden Euro sind aus Brüssel zurückgekommen.Das war letztes Jahr. Dieses Jahr erwarten wir wieder über 4 Milliarden Euro. Das heißt, wir kriegen Geld zurück – ich bin darauf eingegangen –, weil Brüssel, die Mitgliedstaaten nicht in der Lage sind, das Geld umzusetzen. Ich will auch darauf hinweisen, dass wir eine rechtliche Verpflichtung haben: Wenn die Mitgliedstaaten die Projekte im Kohäsionsfonds und in anderen Fonds nicht umsetzen können, müssen wir dieses Geld aus dem Gesamthaushalt wieder zurückgeführt bekommen. Letztendlich haben wir in Brüssel eine Schuld in Höhe von knapp 12 Milliarden Euro für den bundesdeutschen Haushalt. Auch das bitte ich bei allen Debatten mal mit einzupreisen.Liebe Kolleginnen und Kollegen, statt einfach das Motto zu haben: „Mit mehr Geld wird in Europa alles besser“, bin ich der Auffassung, erst einmal zu gucken: Wie machen wir manches effizienter? Wie sorgen wir mit dem vorhandenen Geld dafür, wirklich Wettbewerbsfähigkeit herzustellen? Wie sorgen wir mit dem vorhandenen Geld dafür, die Jugendarbeitslosigkeit abzubauen? Liebe Kolleginnen und Kollegen, statt immer mehr Geld nach Europa zu geben, immer mehr in das tägliche Leben der Bürgerinnen und Bürger und der Wirtschaft einzugreifen, sollten wir uns auf das Wesentliche konzentrieren. Das ist, glaube ich, das Gebot der Stunde. Mein Schlusssatz ist: Vertrauen und Transparenz kann man nur dann herstellen, wenn wir klug deutsche und europäische Interessen miteinander verbinden.Herzlichen Dank.Herr Kollege Rehberg, herzlichen Dank. – Nach Ihren Worten schließe ich die Aussprache.Bevor wir zur Abstimmung kommen, wünsche ich allen Kolleginnen und Kollegen weiterhin einen gesunden und erfolgreichen Tag.




	79. Anja Weisgerber (CSU) Sehr geehrte Frau Präsidentin! Werte Kolleginnen und Kollegen! Klimaschutz ist eine große Herausforderung. Er ist wichtig, für manche Inseln im Pazifik ist er eine Überlebensfrage. Aber auch bei uns in Deutschland sind die Auswirkungen des Klimawandels bereits spürbar.Die Entwicklung von Innovationen im Bereich Klimaschutz ist eine Riesenchance für die Wirtschaft. Wir müssen am Ball bleiben. Wir müssen den Wandel gestalten. Wir dürfen ihn nicht verschlafen.Ich war gestern im Umweltministerium, wo der Deutsche Innovationspreis für Klima und Umwelt an mutige Unternehmer verliehen wurde. Auch mir machte diese Veranstaltung Mut, weil ich gesehen habe, mit welcher Kraft und mit welcher Überzeugung die Unternehmer bei der Sache sind. Es wurden tolle Projekte vorgestellt. Zum Beispiel kann Papier aus Gras hergestellt werden, um die Ressource Holz als wertvollen CO 2 -Speicher zu schonen.Ein weiteres Projekt war eine energiesparende Maschine zur Stoffherstellung, die gleichzeitig spinnt und strickt. Sie sehen: Es gibt eine Menge Innovationen durch mutige mittelständische Unternehmer, wodurch Arbeitsplätze geschaffen und auch erhalten werden können.Im ausverhandelten Koalitionsvertrag haben wir ganz klar festgehalten, dass wir uns zu unseren nationalen, europäischen und internationalen Klimazielen bekennen, und zwar in allen Sektoren. Dafür habe ich mich als Klimapolitikerin persönlich eingesetzt. Damit zeigen wir: Wir nehmen unsere klimapolitische Verantwortung wahr, in Deutschland und in der Welt.Für uns ist aber auch wichtig, wie wir dieses Ziel erreichen. Das wollen wir durch Anreize statt Zwang und durch Technologieoffenheit bewerkstelligen. Wir wollen auch sicherstellen, dass unsere Wirtschaft als Basis unseres Wohlstands international wettbewerbsfähig bleibt. Außerdem soll die Energieversorgungssicherheit weiterhin gewährleistet werden.Es liegt eine Studie der deutschen Industrie vor, die sich damit befasst hat, wie wir das langfristige Klimaziel bezogen auf 2050 erreichen wollen. Die Studie besagt: Das wird schwierig, aber es kann funktionieren. Und auch die Wirtschaft sagt: Die Wirtschaft kann Wandel. Sie hat es bei der Industrialisierung bewiesen. Jetzt nimmt sie die Herausforderungen der Digitalisierung an, Stichwort Industrie 4.0. Die Wirtschaft kann auch Klimaschutz, wenn die Politik verlässliche Rahmenbedingungen schafft und Anreize zum Beispiel für effiziente Umwelttechnologien setzt. Und genau das machen wir, meine Damen und Herren.Wenn Sie, die Antragsteller, deren Vorlagen wir heute beraten, den ausgehandelten Koalitionsvertrag aufmerksam lesen, dann werden Sie feststellen, dass viele Ihrer Punkte bereits enthalten sind, zum Beispiel die Forderung nach einer konsequenten Energieeffizienzpolitik. Es gibt bereits den Nationalen Aktionsplan Energieeffizienz, den wir weiterentwickeln werden. Außerdem wollen wir eine ambitionierte, sektorübergreifende Energieeffizienzstrategie unter dem Leitprinzip „Efficiency first“ installieren. Wir haben uns das ehrgeizige Ziel gesetzt, eine Halbierung des Energieverbrauchs bis 2050 zu schaffen. Im Koalitionsvertrag ist vorgesehen, bis 2030 einen Anteil an erneuerbaren Energien von 65 Prozent zu erreichen. Ebenso haben wir verankert – ich muss ganz ehrlich sagen: endlich –, die energetische Gebäudesanierung steuerlich zu fördern. In vielen Reden an diesem Rednerpult habe ich gefordert, dass wir dieses Klimaschutzinstrument nutzen. Meine Damen und Herren, ich setze darauf, dass nun auch die Grünen in den Bundesländern für dieses Instrument werben; denn in den Jamaika-Sondierungen war die Einführung dieses Instruments Konsens.Damit bin ich beim Thema: Wie sollen diese Instrumente ausgestaltet sein? Wie soll dieses Steuerungsinstrument ausgestaltet sein? Wir setzen auf Anreize statt auf Zwang. Unser Vorschlag sieht ein Wahlrecht des Antragstellers vor: Zuschussförderung oder Reduzierung des zu versteuernden Einkommens. Damit erreichen wir einen möglichst großen Personenkreis. Daneben wollen wir das CO 2 -Gebäudesanierungsprogramm fortführen und damit den Austausch alter, ineffizienter Heizungen fördern. Ferner wollen wir eine Kommission einsetzen, die auf Basis und in Vernetzung mit dem laufenden Prozess zur Umsetzung des Klimaschutzplans bis Ende des Jahres Maßnahmen vorschlagen soll, wie die Lücke zur Erreichung des Klimaziels 2020 so schnell wie möglich reduziert werden kann und im Energiesektor das 2030-Ziel erreicht werden kann. Dazu zählt auch die schrittweise Reduzierung und Beendigung der Kohleverstromung. Das Abschlussdatum für die Kohleverstromung soll ebenfalls durch diese Kommission festgelegt werden.Hört! Hört! Meine Damen und Herren, das geht sogar über das hinaus, was am Ende der Jamaika-Sondierungen ausverhandelt war. In den Jamaika-Verhandlungen lag der Fokus – Herr Dr. Köhler hat es bereits erwähnt – auf dem überhasteten, schnellen Stilllegen, auf der Reduzierung der Kapazität der Kohleverstromung um 7 Gigawatt vor 2020, und zwar ohne Rücksicht auf die Arbeitsplätze. Auch wir wollen die schrittweise Reduzierung der Kohleverstromung; aber wir wollen dabei die Energieversorgung weiterhin sicherstellen. Wir wollen, dass Energie für Verbraucher und Wirtschaft bezahlbar bleibt. Das ist ganz wichtig.Ich habe das Gefühl, das wird von den Grünen öfter ausgeblendet. Außerdem wollen wir nicht, dass es in den betroffenen Regionen zu Strukturbrüchen kommt.Sie sehen: Wir reden nicht nur, sondern wir handeln auch, aber durchdacht und mit Weitsicht.Zur Globalen Allianz für den Kohleausstieg, die in einem Antrag erwähnt wird: Gefordert wird, dass Deutschland dieser Allianz beitritt. Ich habe mir die Mühe gemacht, mir den Energiemix der teilnehmenden Staaten anzusehen. Man muss sich da ehrlich machen. Wie ist das in Frankreich? Dort stammen nur 3 bis 4 Prozent der Stromproduktion aus der Kohle, und Frankreich will von dem Ziel abrücken, den Strom aus Kernenergie bis 2025 auf 50 Prozent zu reduzieren. Kanada investiert mehr als 25 Milliarden in Atomreaktoren, damit sie für weitere 25 bis 30 Jahre laufen. Großbritannien will aus der Kohleverstromung aussteigen, gleichzeitig aber den Anteil des Stroms aus Kernenergie deutlich steigern.Wir wollen beides: Wir wollen die Kohleverstromung reduzieren und den Ausstieg aus der Kernenergienutzung weiter gestalten. Keiner behauptet, dass das leicht ist. Dennoch gehen wir diesen Weg voller Überzeugung für unsere Kinder, für unsere Enkel und für die Menschen, die darauf setzen, dass die Politik die Weichen richtig stellt und durch Innovationen im Umweltbereich Arbeitsplätze erhalten und geschaffen werden.Vielen Dank.




	80. Stephan Protschka (AfD) Habe die Ehre, Herr Präsident! Servus, liebe Kolleginnen und Kollegen! Grüß Gott, liebe Gäste hier im Hohen Haus! Nachdem wir bereits im Dezember hitzig über das Breitbandherbizid Glyphosat debattiert haben, bereichern uns die Kollegen der Grünenfraktion mit einem weiteren Ausfluss ihrer Selbstgerechtigkeit. Wie üblich ist die Angriffsfläche klar: Chemieunternehmen, Landwirte, Agrar­unternehmen. Alle sind dieselben bösen Truppen, die für 30 Silberlinge alles, vom Singvogel bis zum Wasserfloh, über den ökologischen Jordan gehen lassen.Was dagegen hilft, sind natürlich Verbote, Strafen, Sanktionen. Wir sind hier schließlich bei der Verbotspartei schlechthin, die es fertigbringt, im Antragstext die ersten drei Forderungen jeweils mit einer autoritären Wortwahl zu füllen,die jeden preußischen Hauptmann vor Neid erblassen lassen würde.Bei all dem Spott – Entschuldigung! – will ich den Kollegen von links außen gar nicht in Abrede stellen, die Krux des Ganzen wenigstens in Grundzügen verstanden zu haben. Das sollte man aber mittlerweile erwarten können; denn in der Regierungszeit von 1998 bis 2005 hat man diese Thematik mehrheitlich verschlafen.Zurück zum Thema. Wir erkaufen uns die internationale Wettbewerbsfähigkeit der deutschen Landwirtschaft wirklich allzu häufig mit dem Einsatz von Mitteln, deren Langzeitfolgen ungeklärt oder sogar als allgemein schädlich anerkannt sind. Aber ohne einen bedarfsgerechten Pestizideinsatz, den unsere Landwirte mit Sicherheit leisten, ist im Moment eine gewinnbringende Landwirtschaft, mit der der Landwirt seine Familie ernähren kann, leider nicht möglich. Glücklicherweise garantiert uns bereits jetzt der Einsatz moderner Maschinen, Stichwort „Digitalisierung“ – da sollte man vielleicht mehr investieren –, die minimale Verwendung von Pestiziden bei gleichzeitiger Ertragssteigung.Vor diesem Hintergrund erscheint in Ihrem Antrag die steile These, 60 Prozent der eingesetzten Pestizide seien herausgeworfenes Geld, allzu steil. Wenn das, was Sie behaupten, stimmen würde: Glauben Sie, dass irgendein Landwirt mit dem Einsatz auf dem Feld Geld freiwillig zum Fenster hinauswerfen würde, anstatt auf die Pestizide einfach zu verzichten? Ganz kann ich der von Ihnen angeführten Studie nicht glauben; denn Sie geben in Ihrem Antrag keine Quelle an.– Ich brauche nicht nach Dänemark zu schauen. Nennen Sie die Studie in Ihrem Antrag; dann kann man es nachvollziehen. Aber in einem Antrag einfach irgendwelche plumpen Behauptungen aufzustellen und keine Belege anzuführen nennen, führt dazu, dass man damit nichts anfangen kann.Ihrem Antrag fehlt es ganz einfach an einem ganzheitlichen Konzept, wie Agrarpolitik neu gedacht werden muss und soll. Wer auf der einen Seite die deutschen Bauern dem gigantischen internationalen Marktdruck aussetzt, kann ihnen nicht auf der anderen Seite Fesseln anlegen, die ihnen die Marktteilnahme definitiv unmöglich machen.Wir werden das Pferd jetzt von der richtigen Seite aufzäumen. Lasst uns die deutsche Landwirtschaft endlich vor dem globalen Wettbewerbsdruck durch transnationale Freihandelsabkommen schützen! Wenn deutsche Bauern nicht mehr mit ihren Kollegen aus dem Ausland konkurrieren müssen, die häufig mit weniger Aufwand viel größere Flächen bewirtschaften können und auch weniger Kontrollen und Bürokratie unterliegen, dann können wir auch hier wieder über weitgehende Einschränkungen des Pestizidgebrauchs diskutieren.Aber wie so oft: Am besten für Mensch und Umwelt ist eine kleinräumige, regionale und familiär geprägte Landwirtschaft. Den aktuell schon vorherrschenden Trend des Höfesterbens und der Flächenkonzentration durch derartige Vorhaben von den Grünen lehnen wir ab. Deutsche Agrarpolitik sollte sich an der Versorgungssicherheit des deutschen Volkes orientieren und nicht an einer planmäßigen Höfevernichtung durch internationalen Wettbewerbsdruck auf der einen und durch grün-linke Verbotspolitik auf der anderen Seite.Wir werden der Überweisung an den Ausschuss natürlich zustimmen, und wir werden im Ausschuss mitarbeiten; denn dann werden wir mit Sicherheit für die Landwirte und die Verbraucher eine vernünftige Lösung finden und keine, die von grün-linker Ideologie geprägt ist.Danke, meine Damen und Herren.




	81. Sebastian Münzenmaier (AfD) Sehr geehrter Herr Präsident! Meine Damen und Herren! Was steckt eigentlich hinter einer Befristung? Auf der einen Seite wünscht sich ein Arbeitgeber natürlich eine gewisse Flexibilität am Arbeitsmarkt. Er möchte Personal kennenlernen und erproben. Er weiß auch nicht immer vom ersten Tag an, ob eine bestimmte Stelle für sein Unternehmen auf Dauer wirtschaftlich tragbar ist. Aus Sicht eines Arbeitsgebers kann eine Befristung somit äußerst sinnvoll sein, und sie dient als Gegenstück zu einem sehr starken Kündigungsschutz in Deutschland, den wir als AfD ausdrücklich begrüßen.Natürlich wünscht sich im Gegensatz dazu ein Arbeitnehmer möglichst eine direkt unbefristete Anstellung und Planungssicherheit für sich und seine Familie. Wir alle wissen doch um die Probleme, die eine Befristung mit sich bringen kann: Die Familienplanung wird erschwert. Arbeitnehmer sind gezwungen, in teilweise absurd kurzen Abständen zu planen. Hauskredite sind nahezu unmöglich. Beim Kauf eines neuen Autos fangen die Probleme schon an. Aber genau hier gehen sowohl der Antrag der Linken als auch die Parolen der SPD „Sachgrundlose Befristungen abschaffen“ am eigentlichen Problem vorbei.Das Hauptproblem im Bereich der Befristungen sind überhaupt nicht die sachgrundlosen Befristungen, die maximal 24 Monate, in Zukunft vielleicht 18 Monate andauern können und die auch nur bei der Ersteinstellung erfolgen dürfen, sondern die unsäglichen Kettenbefristungen, die auf § 14 Absatz 1 des Teilzeit- und Befristungsgesetzes basieren und Befristungen mit Sachgrund sind. Denn nur Befristungen mit Sachgrund sind bisher ohne zeitliche Höchstgrenze erlaubt und werden auch immer wieder verlängert.Ja, wir kennen alle die Fälle der Schullehrer, deren befristete Verträge beispielsweise kurz vor den Sommerferien enden. Wir kennen den Fall einer Briefträgerin aus Mecklenburg-Vorpommern, die sage und schreibe 88‑mal einen Arbeitsvertrag mit Sachgrundbefristung erhalten hat. Die Abschaffung der sachgrundlosen Befristung würde aber überhaupt keine Probleme lösen. Im Gegenteil: In Zukunft würden Arbeitgeber sehr lange darüber nachdenken, ob sie überhaupt jemand Neues einstellen oder ob man die Aufgaben nicht durch Mehrarbeit der bisher Beschäftigten abfängt.Wir würden also Überstunden produzieren, und wir würden eine Flucht in die Leiharbeit verursachen. Das kann doch wirklich nicht im Sinne der Linken sein.Die Debatte, die Die Linke hier anstößt, ist eine reine Phantomdebatte. Meine Damen und Herren, Ihr Antrag ist überhaupt nicht das Papier wert, auf dem er geschrieben ist.Liebe Abgeordnete der Linken, geben Sie sich doch bitte in Zukunft zumindest einmal Mühe und kümmern sich um die echten Probleme der Beschäftigten in Deutschland!Wir als AfD möchten pragmatisch arbeiten und die Arbeitnehmer in Deutschland schützen. Deshalb machen wir uns Gedanken über eine grundlegende Reform des Teilzeit- und Befristungsgesetzes. Deshalb arbeiten wir an einer Lösung, um die Kettenbefristungen aus der Welt zu schaffen, und planen einen eigenen Gesetzentwurf, der Hand und Fuß hat. Wir müssen beispielsweise darüber nachdenken, ob wir bei Befristungen nicht generell eine zeitliche Höchstgrenze einführen.– Nicht nur für sachgrundlose Befristungen. Es steht also nicht im Koalitionsvertrag. – So bleibt eine gewisse Flexibilität für Arbeitgeber erhalten, und wir lösen das Problem der Kettenbefristungen trotzdem effektiv und dauerhaft.Wir können auch gerne darüber diskutieren, ob beispielsweise eine höhere Zahlung der Arbeitgeber in die Arbeitslosenversicherung bei Befristungen nicht ein Anreiz für Unternehmen sein könnte, Befristungen einzudämmen.Ehrlich gesagt: Obwohl ich nicht so oft wie Herr Schulz mit Macron telefoniere,weiß ich trotzdem, dass in Frankreich Arbeitnehmer mit sachgrundlos befristeten Verträgen, die nicht übernommen werden, eine Abfindungszahlung erhalten. Wir können gerne im Ausschuss über die verschiedenen Möglichkeiten diskutieren, und wir können gerne konstruktiv und gemeinsam an einer Lösung arbeiten.An die ganzen Sozialisten und Kommunisten hier im Saal: Hören Sie doch bitte endlich mal mit Ihren falschen Versprechungen sowohl im Wahlkampf als auch im Parlament auf, und hören Sie mit der Schelte der Privatwirtschaft auf!Die SPD fordert jetzt beispielsweise eine Obergrenze für sachgrundlose Befristungen von 2,5 Prozent. Frau Nahles ist heute leider wieder nicht da; vielleicht telefoniert sie jetzt mit Macron, nachdem Schulz abgesägt wurde.Aber ist es nicht so, dass im Sozialministerium von Frau Nahles 13,7 Prozent der Belegschaft sachgrundlos befristete Verträge hatten?– Doch, es ist wahr. – Ist es nicht korrekt, dass im öffentlichen Dienst und insbesondere in SPD-geführten Ländern die Anzahl der Befristungen weit über dem von Ihnen geforderten Wert liegt? Natürlich ist es so.Sie machen überhaupt keine Politik für Arbeitnehmer; Sie machen Klientelpolitik für die Gewerkschaften und machen Politik vor allem für die Kameras, meine Damen und Herren.Aber machen Sie sich keine Sorgen: Genau das merken doch die Arbeitnehmer da draußen, unsere Bürger. Der kleine Mann hat die SPD einst groß gemacht; aber der kleine Mann tritt ihr jetzt gewaltig in den Hintern und wählt die AfD, meine Damen und Herren.Obwohl wir Ihren Antrag für fachlich vollkommen falsch halten, stimmen wir einer Überweisung an den Ausschuss gerne zu und freuen uns dort auf eine sachliche Diskussion über die wahren Probleme der Arbeitnehmer in Deutschland.Vielen herzlichen Dank.




	82. Dirk Spaniel (AfD) Vielen Dank, Frau Präsidentin. – Sehr geehrte Damen und Herren! Wie heißt in Deutschland das goldene Kalb des 21. Jahrhunderts? – Klimaschutz. Wenn dieses Wort fällt, setzt jede rationale Diskussion in diesem Haus aus.Ausgehend von 1990 wollen wir den CO 2 -Ausstoß bis 2020 um 40 Prozent reduzieren. Deutschlands Anteil an den weltweiten CO 2 -Emissionen beträgt circa 2 Prozent. Mit unseren Einsparzielen können wir also circa 1 Prozent des weltweiten CO 2 -Ausstoßes beeinflussen. Selbst unter der Annahme, dass CO 2 -Emissionen tatsächlich einen Einfluss auf das Weltklima hätten,ist der deutsche Einfluss praktisch unbedeutend.Ich will aber trotzdem die deutschen Ziele einmal kurz in Maßnahmen ableiten.Vereinfachend betrachten wir einmal die energiebedingten Emissionen. Der Verkehrssektor hat über die letzten zehn Jahre einen weitgehend konstanten CO 2 -Beitrag. Warum das auch so bleibt, erkläre ich etwas später in meiner Rede.– Hören Sie mir zu, dann lernen Sie was.Nehmen wir einmal an, alle Bereiche außer Verkehr erfüllen die Zielvorgabe. Was bedeutet das für die Stromerzeugung? Die Summe aus Stromerzeugung und Verkehr betrug 2015 circa 500 Megatonnen CO 2 . Von unserem Einsparziel waren wir circa 90 Megatonnen CO 2 entfernt. In den letzten zwei Jahren hat man circa 9 Terawattstunden auf erneuerbare Energiequellen übertragen. Da fehlen aber noch 81 Terawattstunden. 2019 schalten wir, weil wir genug Reserven haben und es uns leisten können, noch ein CO 2 -neutrales Kernkraftwerk ab. Bis 2020 müsste ein Viertel der Kraftwerkskapazität der Öl-, Kohle- und Gaskraftwerke durch erneuerbare Energien ersetzt werden.Auf Windkraft bezogen erfordert das einen Zubau von 15 000 Megawatt pro Jahr, nicht 2 900 Megawatt.Vorwiegend in Süddeutschland müssten circa 15 000 neue Windräder der 3-Megawatt-Klasse gebaut werden; das sind die 200 Meter hohen Türme.Ich wiederhole das gerne für diejenigen, die vielleicht nicht mitkommen – es sind ja auch viele Zahlen; für Philologen usw. schwierig –:Um die CO 2 -Ziele zu erreichen, bräuchten wir circa 15 000 neue Windräder mit 200 Meter Höhe, vorwiegend in Hessen, Bayern und Baden-Württemberg.Mit circa 6 000 neuen Windrädern in Fernsehturmhöhe in Bayern ist das Thema „Wohlfühlen in der Heimat“ erledigt. Da hilft Ihnen auch kein Heimatministerium.Merken Sie eigentlich was?Nicht einmal CO 2 -Einsparungen von 40 Prozent können mit realistischen Annahmen erreicht werden.Wie weit sich dieses Parlament von der Realität entfernt hat, erkennt man daran, dass in den letzten Jahren niemand hier die CO 2 -Ziele in technisch realisierbare Maßnahmen abgeleitet hat.Das hat jetzt mit dem Einzug der AfD ein Ende.Die Energiewende ist ein schönes Märchen. Es gibt keinen Plan zur Realisierung. Für das Parlament eines modernen Industriestaats ist die permanente Vortäuschung der Machbarkeit des Energiewendemärchens ein absolutes Armutszeugnis.Jetzt komme ich zu meinem Lieblingsthema. Etwas ganz Besonderes ist die immer wieder proklamierte CO 2 -Einsparung durch Elektromobilität im Verkehrssektor. Da sollten Sie einmal zuhören;da kenne ich mich besser aus als wahrscheinlich jeder hier.In der idealen Welt soll der Strom für Elektrofahrzeuge aus erneuerbaren Energien kommen. In der realen Welt des Jahres 2020– ja, Sie sagen „aus der Steckdose“ – haben wir einen Strommix mit circa 51 Prozent fossilen Energieträgern. Darin sind übrigens die CO 2 -neutralen Kernkraftwerke noch enthalten. Mit diesem Strommix erzeugt das derzeit populärste Elektroauto – dieses schicke Produkt aus Kalifornien – mehr CO 2 als ein gleichwertiges Fahrzeug mit Dieselmotor.Noch einmal, ganz langsam: Ein Elektroauto erzeugt aufgrund des deutschen Strommixes im Jahre 2020 mehr CO 2 als ein moderner Diesel aus regionaler Erzeugung.Wer die Förderung von Elektroautos betreibt, kann also keinesfalls die CO 2 -Reduzierung als Ziel haben.Statt die Probleme in diesem Land sachlich zu lösen, vernebeln Sie die Fakten. Die Menschen in diesem Land haben ein Recht darauf, dass wir hier ihre Interessen vertreten. Durch Ihre Politik lösen Sie kein tatsächliches oder vermeintliches Problem im Bereich Umwelt. Sie schaffen aber ein Problem für viele Arbeitnehmer in der deutschen Maschinenbau- und Autoindustrie.Denken Sie bitte an Ihre Redezeit? Sie ist schon deutlich abgelaufen.Ja. – Das einzig denkbare Motiv ist die angestrebte Deindustrialisierung des Wirtschaftsstandortes Deutschland.Bei den Damen und Herren von Linken und Grünen überrascht mich das nicht. Die Protektion von Elektroautos durch CDU/CSU, SPD und FDP kann ich mir nur mit mangelndem Sachverstand erklären.Das macht nichts. Dieses Defizit lösen wir jetzt auf.Entschuldigen Sie, Sie sind weit über die Redezeit.Die AfD-Fraktion lehnt die scheinheilige Proklamation der angestrebten Klimaziele für 2020 ab, –Ihre Redezeit ist abgelaufen!– zumindest so lange, bis es ein schlüssiges Konzept gibt, wie man fossile Energieträger ersetzen will. – Ich bedanke mich; ich bin gleich fertig.Nein, Sie sind nicht gleich fertig, Sie sind jetzt fertig!Ich bin jetzt fertig.Gut.Fürs Erste folgen wir der Beschlussempfehlung des Wirtschaftsausschusses zur Annahme dieses Antrags.Danke.




	83. Elisabeth Winkelmeier-Becker (CDU) Herr Präsident! Liebe Kolleginnen und Kollegen! In unserer Klinik fühlen Sie sich wohl. Wir haben Verständnis für Ihre Situation. Unser Team hat jahrelange Erfahrung, tagtägliche Routine. Bitte bringen Sie bequeme Kleidung und Bargeld mit. – So könnte eine Klinik auf ihrer Internetseite werben. Die Bilder würden eine junge Frau mit melancholischem Blick zeigen, ein sympathisches Ärzteteam und eine moderne Klinik.In der Rubrik Bewertungen könnten Patientinnen schreiben: Ich war rundum zufrieden und gebe fünf Sterne.So oder so ähnlich – so wie Kliniken in Deutschland heute zum Beispiel für Schönheitsoperationen und Nasenkorrekturen werben –könnte auch für Abtreibungen geworben werden, wenn wir den Gesetzentwürfen der Fraktion der Linken und der Grünenoder auch dem Gesetzentwurf der FDP folgen würden, die das Verbot der Werbung komplett abschaffen bzw. einschränken wollen; denn unsachlich oder grob anstößig wäre solche Werbung nicht.Das zeigt, dass das, was in Ihren Gesetzentwürfen steht, weit darüber hinausgeht, als nur zu erlauben, dass ein Arzt auf seiner Homepage sagen kann: Schwangerschaftsabbruch gehört zu meinem Leistungsspektrum. – Es geht weit darüber hinaus und nimmt Formen an, die wir nicht akzeptieren werden. Solche Werbung wird ganz konkret von Firmen und Kliniken im Ausland schon betrieben. Das ist mit dem Recht des ungeborenen Kindes, mit seinem Recht auf Menschenwürde von Anfang an nicht vereinbar.Ich denke, so geht man an das Thema nur heran, wenn man den Gedanken an ein Kind mit eigenen Grundrechten ausblendet und nur von Schwangerschaftsgewebe spricht. In diesem Sinne ist es letztlich sogar folgerichtig, jede Einschränkung abzulehnen. Hier liegt der Unterschied zu unserer Bewertung. Nach unserer Überzeugung, die auch dem Ansatz des Bundesverfassungsgerichts entspricht, geht es um ungeborenes menschliches Leben, mit Menschenwürde und Lebensrecht von Anfang an – unabhängig davon, ob es ihm jemand zuspricht.Das ganze Individuum ist von Anfang an, schon in diesen ersten Stadien, angelegt. Es muss nichts mehr hinzukommen, was den Menschen erst zum Menschen macht. Er entwickelt sich, sagt das Bundesverfassungsgericht, als Mensch und nicht zum Menschen.Meine lieben Kolleginnen – das sage ich ausdrücklich, weil es hier um dieses heranwachsende Leben und sein Lebensrecht geht –, man kann beim Thema Abtreibung nach meiner Überzeugung nicht so diskutieren, als ginge es ausschließlich um Rechte von Frauen, um sexuelle Selbstbestimmung, um Freiheit, Emanzipation und Gleichberechtigung. Das wird dem Recht des Kindes nicht gerecht.Für das Kind steht alles auf dem Spiel. Gleichzeitig ist es in der denkbar vulnerabelsten Situation; denn es ist abhängig von der Mutter, die bei einer ungewollten Schwangerschaft – das ist uns natürlich sehr bewusst – selbst auch in ihrer eigenen Lebensplanung und in ihrer eigenen körperlichen Integrität massiv betroffen ist.Der Staat muss hier eine Schutzpflicht gegenüber dem Kind wahrnehmen. Das hat er in einem ziemlich klugen Konzept auch umgesetzt. Es gibt dem Lebensrecht des Kindes Raum, aber erkennt auch an, dass die ungewollte Schwangerschaft für die Mutter mit Zumutungen verbunden ist und dass nur sie allein letztendlich die Entscheidung treffen kann und treffen muss, wie es weitergeht.Der Schutz des Kindes kann nur mit der Mutter zusammen erfolgen. Das ist Grundlage des Schutzsystems. Deshalb stehen wir auch zu diesem System. Es ist sehr gut.Der Beratung kommt aber in diesem System eine ganz wesentliche Bedeutung zu. Sie ist es erst, die den Verzicht auf die Strafe auch verfassungsgemäß macht. Da gibt es ein paar weitere Regeln: Derjenige, der die Beratung durchführt, darf nicht selbst den Abbruch durchführen. Außerdem darf es kein materielles Eigeninteresse der Beratungsstelle geben und auch keine Verbindung zu denjenigen, die für einen Schwangerschaftsabbruch ein Honorar erhalten. Das hat seinen guten Grund. Es geht um den Konflikt zwischen Lebensrecht auf der einen Seite und Selbstbestimmungsrecht auf der anderen Seite, und da dürfen materielle Interessen Dritter keine Rolle spielen.Frau Kollegin, Sie denken an die Zeit.Noch etwas hat das Bundesverfassungsgericht deutlich gemacht: In der Rechtsordnung muss an anderer Stelle deutlich werden, dass der Abbruch missbilligt wird, auch wenn auf Strafe verzichtet wird. Das passt nicht mit Werbung zusammen. Werbung darf den Effekt der Beratung nicht konterkarieren. Werbung stünde im Widerspruch zur Rechtswidrigkeit des Schwangerschaftsabbruchs.Frau Kollegin.Nicht zuletzt würde die Trennung zwischen Beratung und eigenem wirtschaftlichen Interesse aufgehoben. Das gefährdet den Schutz des Kindes. Uns geht es um den Schutz des Kindes mit der Mutter zusammen. Deshalb halten wir an der geltenden Regelung fest.Vielen Dank.




	84. Katrin Dagmar Göring-Eckardt (BÜNDNIS 90/DIE GRÜNEN) Herr Präsident! Liebe Kolleginnen und Kollegen! Ein Jahr vor der Europawahl ist es höchste Zeit, dass wir hier im Parlament grundsätzlich über Europa reden und nicht sogar die Regierungserklärungen sein lassen, nach dem Motto „Wir haben jetzt andere Prioritäten“. – Andere Prioritäten kann es nicht geben.Ja, wir brauchen mehr Europa und definitiv keinen Nationalismus und auch weniger nationale Bezüge, wie wir sie heute in dieser Debatte immer wieder gehört haben.Ja, wir sind der privilegierteste, der sicherste und der freieste Kontinent auf unserem Planeten, und das verdanken wir der Europäischen Union. Trotzdem wird unser gemeinsames europäisches Projekt immer und immer wieder attackiert. Diese Attacken kommen aber nicht von außen, sondern sie kommen von innen. Und warum? Weil wir über so viele Jahre nicht ausreichend und grundsätzlich über Europa debattiert haben.Mit Europa ist es wie mit einer Beziehung: Sie macht Arbeit, sie muss immer neu begründet werden, und sie ist niemals fertig, außer sie geht kaputt. Frau Merkel, Sie haben den Satz wiederholt, dass es Deutschland nur gut geht, wenn es Europa gut geht. Ja, das stimmt, aber dazu gehört doch Leidenschaft, dazu gehört doch nicht nur Verwaltung, dazu gehört nicht nur Klein-Klein, dazu gehört Gestaltungswille, dazu gehört Zukunftsdrang. Genau das hat Ihnen so sehr gefehlt in den letzten Jahren der Großen Koalition und leider auch davor, meine Damen und Herren.Genau das, keine Ideen, keine Leidenschaft, gibt jenen Oberwasser, die nur Nörgler und Spalter sein wollen.Es ist schon bemerkenswert, dass sich die beiden künftigen Koalitionspartner darauf berufen, dass Europa im künftigen Koalitionsvertrag ganz vorne steht. Ich weiß nicht, warum ich das jetzt machen muss, aber ich finde, man könnte Martin Schulz an dieser Stelle einmal dafür danken, dass er europäische Leidenschaft in den Vertrag hineingebracht hat, meine Damen und Herren.Was für eine großartige Beziehung wir in und mit Europa haben, das sehen wir tragischerweise am Leid der Briten. Ja, den Brexit haben sie sich selbst eingebrockt, aber wir sehen: Europa von innen zu torpedieren, heißt immer vor allem, sich ins eigene Knie zu schießen. Natürlich hat Emmanuel Macron einen Weckruf gestartet, und natürlich braucht es endlich das entsprechende deutsche Handeln.Meine Damen und Herren, warum ist Europa so stark? Sie haben es ganz am Ende Ihrer Rede gesagt, Frau Merkel, als Sie allgemein auf Werte verwiesen haben. Aber worum geht es bei diesen Werten? Es geht um Solidarität, es geht um Frieden, es geht um Demokratie, es geht um Nachhaltigkeit, es geht um Freiheit. Und dann stehen Sie hier und reden über Migration, reden über die Sicherung der Außengrenzen und reden über Solidarität innerhalb Europas. Aber wenn man über Werte redet, dann muss man doch auch über die Solidarität mit den Menschen reden, die auf der Flucht sind, meine Damen und Herren.– Sie können hier gerne stören. Ich weiß, dass Sie von der AfD es mit der Humanität und Menschlichkeit nicht so haben. Das haben wir am Aschermittwoch gemerkt, als Sie nicht nur jeden, der Ihnen irgendwie in die Quere kam, beleidigt haben, sondern sogar jede Art von Menschlichkeit, jede Art von Humanität infrage gestellt haben. Wir haben gesehen, wes Geistes Kind Sie sind. Das hat überhaupt nichts mehr damit zu tun, dass Sie Teil der Demokratie sind. Sie stellen die Demokratie infrage, die Humanität und die Menschlichkeit gleichermaßen, und Sie beleidigen all diejenigen, die für Freiheit und Demokratie eintreten. Das ist das, was Sie tun.Ich will, dass wir an dieser Stelle tatsächlich über Europa reden. Der mehrjährige Finanzrahmen bietet die Möglichkeit, zu zeigen, worum es eigentlich geht. Er ist nämlich nichts anderes als das materialisierte Versprechen zur Stärkung der gemeinsamen Zukunft. Herr Lindner, da bin ich ganz anderer Meinung als Sie. Man kann sich hierhinstellen und sagen: An der Arbeitslosigkeit der italienischen Jugendlichen ist jemand anders schuld. Über die Finanzkrise haben Sie übrigens kein Wort verloren, als ob es sie nicht gegeben hätte. Das ist ja auch ein unbequemes Thema. Ich finde, dass jeder Jugendliche in diesem Europa mit Europa Hoffnung verbinden sollte. Mir ist es egal, ob es um einen Jugendlichen aus meinem Nachbardorf geht oder um einen Jugendlichen in irgendeinem italienischen Dorf. Sie gehören alle zu Europa, und wenn dieses Europa eine Chance haben soll, dann müssen alle wissen: Das ist unser Europa, und wir werden nicht im Stich gelassen, meine Damen und Herren.Ich glaube, wenn wir es irgendwie schaffen wollen, dann geht es mit diesem business as usual nicht weiter. Wir brauchen Solidarität mit den ärmeren Regionen in Europa. Wir brauchen eine nachhaltige, der Zukunft zugewandte Wirtschaftsinvestitionspolitik. Mit „nachhaltig“ meine ich natürlich auch die ökologischen Ziele. Dabei hilft uns keine nationale Nettozahlerdebatte nach dem Motto: Du kriegst einen Flughafen, du die Autobahn und der Nächste ein paar Goodies für die Bauernlobby. Bei der Gestaltung des nächsten Haushaltes könnte doch endlich einmal klar werden, dass wir eine breite, eine europäische Debatte führen, und zwar – ja bitte – in unseren Parlamenten. Dazu gehören Klarheit und Werte.Frau Merkel, Sie haben hier über Syrien geredet – sehr eindringlich und notwendigerweise –; aber wenn man über Syrien spricht, dann muss man auch Afrin und die Rolle der Türkei erwähnen, meine Damen und Herren.Lieber Dietmar Bartsch, auch in Ihre Richtung sage ich: Dann muss man auch die Rolle Russlands erwähnen. Auch das darf man nicht weglassen, wenn man diese humanitäre Katastrophe wirklich beleuchten will.Ja, es geht bei den europäischen Werten um Rechtsstaatlichkeit, es geht um Meinungsfreiheit, es geht um Gewaltenteilung. Aber ein bisschen Freiheit kann es genauso wenig geben wie ein bisschen Europa. Deswegen lautet meine herzliche Bitte, Frau Merkel: Wenn Sie mit Herrn Orban reden, dann lassen Sie ihn nicht raus aus seinen Verpflichtungen. Nachdem er vieles antieuropäisch gelöst hat, richtet er sich jetzt gegen die Nichtregierungsorganisationen in seinem Land. Das widerspricht dem europäischen Recht, das widerspricht der Europäischen Menschenrechtscharta. Es kann nicht sein, dass die Nichtregierungsorganisationen keine Chance mehr haben. Nein, wer sich solidarisch um Menschen kümmert, stellt nicht die Einheit Ungarns infrage und ist schon gar nicht eine Bedrohung für Europa. Reden Sie mit Herrn Orban, Ihrem Parteifreund. Vielleicht kann die CSU ein bisschen mithelfen. Es muss ganz klar sein: So etwas geht in Europa nicht! Das ist ein Angriff auf die Demokratie und die Freiheit.Deswegen sage ich ganz klar: Wir brauchen dieses gemeinsame Europa, wir brauchen die Verständigung auf echte Werte, wir brauchen eine gemeinsame europäische Außenpolitik – ja selbstverständlich –, wir brauchen die transnationalen Listen, weil wir nur dann ein echtes Bekenntnis zu Europa abgeben können. Die Europäische Union ist das größte politische Projekt der letzten Jahrhunderte. Das Versprechen der Europäischen Union ist größer als das, was wir alle zusammen hier versprechen können. Jetzt ist das Momentum: Zeigen Sie, dass Ihnen wirklich daran liegt, mit Leidenschaft und Anstrengung. Zeigen Sie, dass Sie Europa wirklich großartig finden, dass es nicht nur um kleinteilige, bürokratische Verhandlungen geht.Herzlichen Dank.




	85. Gottfried Curio (AfD) Sehr geehrter Herr Präsident! Sehr geehrte Abgeordnete! Die AfD fordert das Verbot der Vollverschleierung im öffentlichen Raum. Es geht um den Schutz der Individualfreiheitsrechte der muslimischen Frau gegenüber Gruppendruck und Einschüchterung in Parallelgesellschaften. Es geht um die Bewahrung der Werte unserer Gesellschaft, die die optische Unkenntlichmachung der Person nicht dulden kann.Diese Auslöschung des Gesichts ist das Symbol der Unterdrückung weiblicher Selbstbestimmung, ist geschlechtsspezifische Diskriminierung schlechthin. Sah man – außer im kriminellen Geschehen – je einen vollvermummten Mann? Diese Negation der Menschenwürde zu dulden, wäre selbst grundgesetzwidrig. Die Menschenwürde nämlich verbietet, den Menschen zum bloßen Objekt zu machen, seine Subjektqualität, sein Personsein infrage zu stellen. Kann man jemandem nicht ins Gesicht schauen, wird ein integraler Teil des sozialen Menschseins entfernt, ja vernichtet.So die Menschenwürde zu verletzen, ist ausnahmsweise sogar der Selbstbestimmung entzogen. Deshalb können wir dieses schwarze Stück Tuch, das Frauen das Gesicht rauben soll, nicht dulden. Man denke daran: Textstellen, die man nicht mehr erkennen soll, werden geschwärzt. Sollen jetzt Menschen geschwärzt werden?Wie sehr die Totalverhüllung die Frau zum Objekt degradiert, wird an der sexistischen Herkunft dieser Unsitte deutlich. Sie verhüllt das Objekt der Begierde, Motto „Aus den Augen, aus dem Sinn“. Sie war und ist auch nicht etwa Religionsausübung. Sie ist hier und heute Zeichen bewusster Abgrenzung gegen westliche Kultur und die Werte der Aufklärung.Nie in Jahrtausenden hat es in Europa solche Stigmatisierung eines Geschlechts gegeben. Und jetzt bitte nicht mit „multikulti“ kommen: Die Burka steht gerade nicht für Liberalität oder Weltoffenheit.Der sich hier ausdrückende Herrschaftsanspruch über die Frau ist nur eines: gruppenbezogene Menschenfeindlichkeit, Geschlechterrassismus pur.Wollen wir, ja dürfen wir die öffentliche Frauenverhüllung als politisches Statement zulassen und hoffähig machen? Haben wir nicht die Pflicht, die Freiheitsrechte der Frau zu schützen, auch gegen den sozialen Gruppendruck, der bei Duldung der Vollverschleierung entsteht –heute sind es drei, morgen viele –, und sei es als freiwilliger Kulturkampf? Das erzeugt inflationäre Selbstghettoisierung, macht Integration zunichte, bevor sie beginnen könnte.Meine Damen und Herren, wollen wir Geschlechterapartheid auf unseren Straßen? Soll die Hälfte der Menschheit vermummt herumlaufen? Nein, die Burka ist der Offenbarungseid der islamischen Kultur.Die Duldung der Frauenvermummung wäre ein fatales Zeichen, dass unser Rechtsstaat zurückweicht vor der kulturellen Landnahme durch radikalen Islamismus. Nach unseren Grundwerten begegnen sich Menschen frei und gleichrangig. Die Vollverschleierung wäre das Signal, an unserer offen kommunizierenden Gesellschaft nicht teilhaben zu wollen, sich von der Lebensart der Ungläubigen abgrenzen zu wollen. Der Nikab ist die Fahne der Salafisten, die Burka atmet den Geist der Scharia. So wird ein falscher Standard etabliert.Sure 33 empfiehlt den Frauen, ihre Kleider herunterzuziehen, damit sie nicht belästigt werden. Diese Assoziation von Verhüllung und Ehrbarkeit heißt doch im Umkehrschluss: Wer sich nicht so kleidet, ist offenbar die berühmte „ungläubige Schlampe“ – von Männern in einschlägiger Weise zu betrachten mit allen Kölner Konsequenzen.Wenn jetzt wegen verfehlter Zuwanderungspolitik unsere Frauen bald einer Mehrheit von jungen Männern aus archaischen, frauenfeindlichen Gesellschaften gegenüberstehen: Sollen Frauen dann erst auf kurze Röcke verzichten, dann besser ihre Haare mit einem Kopftuch verhüllen, um am Ende in einer Burka eingesperrt herumlaufen zu müssen? Sollen angstfreie Räume Mangelware werden und abendliches Joggen eine Mutprobe? Dahin darf es nicht kommen.Selbst innerislamischer Konsens ist: Die Freiheit der Religionsausübung wird durch ein Burkaverbot nicht berührt. Laut Europäischem Gerichtshof für Menschenrechte ist das französische Burkaverbot im öffentlichen Raum – wie in Belgien und Österreich – eine legitime Maßnahme, die Voraussetzungen des Zusammenlebens zu wahren, welche von der Burka als einer Barriere zwischen Trägerin und Umwelt untergraben würden.Höre ich da etwa: „Niemand hat die Absicht, eine Barriere zu errichten“?Wir sagen Ihnen: Mr. de Maizière, tear down this barrier!Meine Damen und Herren, laut forsa sind 60 Prozent der Befragten für ein Verbot der Vollverschleierung. Verehrte Abgeordnete der künftigen GroKo: Zeigen Sie uns, dass Sie nicht gegen den erklärten Willen von 60 Prozent der Bevölkerung regieren wollen.Vielen Dank.




	86. Lukas Köhler (FDP) Sehr verehrte Frau Präsidentin! Meine sehr verehrten Damen und Herren! Die Lage ist ernst. Das wissen wir nicht erst, seit in dieser Woche neben dem Potsdam-Institut für Klimafolgenforschung auch eine Reihe von DAX-Unternehmen schnelle Maßnahmen gegen den Klimawandel gefordert haben. Ich unterstelle Ihnen, liebe Kolleginnen und Kollegen von der Union, von der SPD, von den Linken und von den Grünen, dass Sie das Problem nicht nur erkannt haben, sondern auch lösen wollen,dass auch Sie unseren Kindern und Kindeskindern eine Welt hinterlassen wollen,die ihnen Chancen eröffnet, die wir selbst hatten und haben. Es geht hier und heute also nicht darum, ob wir den menschengemachten Klimawandel bekämpfen wollen, sondern darum, wie wir das tun wollen. Wir als Freie Demokraten stehen ohne Wenn und Aber zu den Klimazielen 2030 und 2050.Was wir dazu nicht brauchen, ist blinder Aktionismus. Jetzt auf die Schnelle 20 Kohlekraftwerke abzuschalten, um das nationale, rein symbolische Ziel für 2020 zu erreichen, bringt überhaupt nichts.Denn das gefährdet nicht nur Arbeitsplätze, sondern es vernichtet sie, insbesondere in strukturschwachen ­Regionen wie der Lausitz. Bis 2021 wandern die dort eingesparten Emissionen über den Emissionshandel sowieso direkt zu unseren europäischen Nachbarn.Damit ist dem Klima nicht geholfen. Die Zeche zahlen aber die Menschen, die ihre Jobs verlieren, und die Geringverdiener, die unter den steigenden Preisen ganz besonders leiden.Es ist gut, dass der neue Koalitionsvertrag beim 2020-Ziel mehr Ehrlichkeit und Realismus beinhaltet. Aber wie wir an den Forderungen des PIK und der deutschen Wirtschaft gesehen haben, ist keine Zeit mehr, abzuwarten und das Problem in Kommissionen auszulagern.Meine Damen und Herren, wenn wir das Ziel verantwortungsbewusst erreichen wollen, müssen wir schnell, aber überlegt handeln. Als Politiker müssen wir dafür den Rahmen vorgeben, müssen Ziele setzen und herausarbeiten, wie viel CO 2 emittiert werden darf. Und dann überlassen wir es doch den Ingenieuren und Tüftlern, wie sie diese Ziele erreichen wollen.In der Vergangenheit wurde Klimapolitik leider viel zu oft mit Energiepolitik gleichgesetzt. Dabei zeigen die Zahlen und Daten des Umweltbundesamtes, dass wir gerade in den Sektoren ohne Emissionshandel noch die größten Einsparpotenziale haben. Es ist daher auch hier nicht die Frage, ob wir, sondern wie wir CO 2 in Sektoren wie Verkehr, Gebäude oder Landwirtschaft einsparen wollen.Insbesondere im Verkehr ist der Handlungsbedarf dabei offensichtlich. Die Zahlen zeigen, dass wir keine Einsparungen erreicht haben. Da wir wissen, dass eine Flottenerneuerung im Schnitt etwa zehn Jahre dauert, müssen wir jetzt handeln und nicht erst in fünf Jahren – dann, wenn es zu spät ist. Sonst können wir die Klimaziele für 2030 vergessen.Wir müssen den Verkehr so schnell wie möglich in den europäischen Emissionshandel aufnehmen, damit CO 2 auch hier einen Preis bekommt. Mir ist klar, dass das nicht leicht wird. Genau deshalb müssen wir in Deutschland mit gutem Beispiel vorangehen. Die EU-Emissionshandelsrichtlinie erlaubt es ausdrücklich, dass wir den Emissionshandel auch nur auf nationaler Ebene auf weitere Sektoren ausweiten können. Das ist natürlich nicht unser endgültiges Ziel – das ist ziemlich klar –, aber es ist ein wichtiger Zwischenschritt auf dem Weg zu einer europäischen, am besten sogar zu einer weltweiten Lösung.Das ist zwar nur ein Zwischenschritt, aber immerhin ist es ein Schritt in die richtige Richtung.Meine Damen und Herren, nationale und europäische Schritte sind wichtig, aber unser Ziel muss ein weltweiter CO 2 -Preis sein. Es ist der Auftrag der Bundesregierung, dieses Thema bei der COP 24 in Kattowitz auf international höchster Ebene zu platzieren. Am besten ist es, dieses Thema auf einem High Level Panel voranzubringen. Wir müssen dafür sorgen, dass die Staaten, die das Pariser Klimaabkommen unterzeichnet haben, eine internationale CO 2 -Bepreisung einführen.Liebe Kolleginnen und Kollegen, jetzt ist die Zeit zum Handeln. Ich möchte Sie einladen, den Klimawandel mit uns gemeinsam und interfraktionell zu bekämpfen. Deshalb bitte ich um Ihre Zustimmung zu unserem Antrag, um die Zukunft unserer Kinder zu sichern und zu schützen.Vielen Dank.




	87. Frauke Petry (Plos) Sehr geehrte Frau Präsidentin! Meine sehr geehrten Damen und Herren! Während Frau Baerbock ihre Stimme erholt, können wir vielleicht zu einer sachlichen Debatte zur Klimapolitik zurückkommen.In der Tat stellt sich die Frage, warum wir viel zu oft über angebliche Klimaziele reden und dabei vergessen, dass der CO 2 -Emissionshandel in der Tat nicht viel mehr ist als ein moderner Ablasshandel.Dieser Emissionshandel belastet deutsche und europäische Unternehmen gegenüber dem Rest der Welt, und die hohen Kosten müssen gerade im Mittelstand auf die Produktkosten umgelegt werden. Das ist das Gegenteil von verantwortungsvoller Politik für das Land. Darum geht es Ihnen aber auch nicht so häufig; vielmehr geht es Ihnen darum, dass die Ideologie stimmen muss, koste es, was es wolle.Die Freien Demokraten sind zumindest ehrlich genug, zuzugeben, dass die Klimaziele nicht zu erreichen sind. Ehrlich sind sie allerdings nicht, liebe FDP, wenn es darum geht, dass die ganze Klima- und Energiepolitik falsch gelagert ist. Bei jedem FDP-Antrag hoffe ich auf ein tatsächlich freiheitliches Denken, aber dazu fehlt Ihnen nach wie vor der Mut.Wir sollten nicht nur über die Klimaziele, sondern vielmehr auch darüber reden, wie ineffizient die derzeitige Energieerzeugung ist und dass der CO 2 -Ausstoß steigt anstatt zu sinken – völlig losgelöst davon, ob wir die CO 2 -Debatte nicht überhaupt endlich wissenschaftlich führen sollten, das heißt, überprüfen sollten, ob die Hypothese der Grünen, dass CO 2 ein Schadstoff ist, überhaupt zutrifft.Bevor dieser Bundestag nicht endlich wieder auf dem Boden der Wissenschaft angekommen ist, sind solche Debatten wie diese hier wahre Zeitverschwendung. Wir brauchen eine Energie- und Umweltpolitik, die tatsächlich hilft, die Vielfalt der Flora und Fauna zu erhalten und vor allem Energie für Unternehmen und Bürger bereitzustellen – und das nicht zum Höchstpreis, der, wie Sie wissen, in Europa und in Deutschland massiv gestiegen ist, sondern zu einem vernünftigen Preis, der in einem Industrieland notwendig ist.Herzlichen Dank.




	88. Gabriela Heinrich (SPD) Danke schön. – Sehr geehrter Herr Präsident! Meine Damen und Herren! Kolleginnen und Kollegen! Wenn wir heute über die aktuelle Entwicklung im Nahen und Mittleren Osten reden, dann müssen wir natürlich auch über die humanitäre Lage und auch über die humanitäre Hilfe reden, die übrigens – das ist der humanitären Hilfe immanent – eben nicht von Bedingungen abhängig ist.Die Not in dieser Region ist für die meisten von uns unvorstellbar: in Syrien, im Jemen, im Irak, in den Flüchtlingslagern in Jordanien, im Libanon, in der Türkei. Die Weltgemeinschaft muss mittlerweile Unsummen aufbringen, um die Menschen zu ernähren und medizinisch zu versorgen.Die Zerstörung ist Folge von Kampfhandlungen, direkt durch Bomben verursacht oder als Teil einer Strategie, die die Zivilbevölkerung in Geiselhaft nimmt, als vermeintliche Schutzschilder zwischen den Kämpfern oder herabgewürdigt als Erpressungspotenzial bei Belagerungen, ohne jede Rücksicht auf Leben, auf Gesundheit, auf die psychische Zerstörung, und ohne auch nur einen Gedanken an den Hass zu verschwenden, der immer und immer über die Generationen weitergegeben wird: so viel Hass und Leid, dass Frieden kaum noch möglich scheint.Wenn Erdogan jetzt ankündigt, Afrin belagern zu wollen, nimmt er Hunger, Krankheit und Tod bei 320 000 Menschen nicht nur in Kauf. Er setzt damit die Not von 320 000 Menschen als Kriegsmittel für seine Zwecke ein.Ähnliches passiert aber auch anderswo: Diese Woche wurden mehrere Krankenhäuser in der syrischen Rebellenhochburg Ost-Ghuta bei Damaskus bombardiert. UNICEF hat eine fassungslose, letztlich leere Pressemitteilung herausgegeben. Ich darf zitieren:Wir haben nicht länger die Worte, um das Leiden der Kinder und unsere Empörung zu beschreiben. Haben diejenigen, die das Leid verursachen, noch Worte, um ihre barbarischen Handlungen zu rechtfertigen?UNICEF ist sprachlos. Gestatten Sie mir, kurz daran zu erinnern, dass es hier in diesem Haus – auch heute wieder – ernsthaft Forderungen gab, aktuell Flüchtlinge nach Syrien zurückzuschicken.Wenn beklagt wird, dass dies ein Missverständnis sei, dann kann ich nur sagen: Zurückschicken und freiwillig ist in sich ein Widerspruch.Die humanitären Helfer verzweifeln daran, dass viele Hilfsgüter nicht durchkommen: Straßen in Syrien sind zerstört. Im Jemen sind vor einiger Zeit ganz gezielt Hafenkräne bombardiert worden. Jetzt können dort keine großen Container mehr entladen werden. Dadurch fehlen Nahrungsmittel und Impfstoffe. Im Jemen ist jetzt nach der Cholera die Diphterie ausgebrochen.Was bedeutet das für die Kinder im Jemen? Seit der Eskalation der Gewalt 2015 sind im Jemen über 3 Millionen Kinder geboren worden. Über 5 000 Kinder sind verletzt oder getötet, 1,8 Millionen Kinder leiden an Mangel­ernährung. 400 000 dieser Kinder kämpfen deshalb um ihr Leben. Diese Kinder wachsen im Dreck auf, gehen nicht zur Schule, sind traumatisiert. Dies ist nicht nur im Jemen so; im Jemen jedoch sind praktisch alle Kinder auf humanitäre Hilfe angewiesen.Meine sehr verehrten Kolleginnen und Kollegen, es ist im Vorfeld bereits viel darüber gesprochen worden, welche politischen Initiativen es geben muss, um der schrecklichen Notlage der Menschen ein Ende bereiten zu können. Die Bombardierung der Zivilgesellschaft muss sofort aufhören. Die Belagerung von Städten, Gemeinden und Dörfern muss aufhören, damit die Menschen überleben können.Die humanitäre Hilfe muss durchkommen. Das sind die allerersten Bedingungen, für die wir uns einsetzen müssen, um hier weiterzukommen. Aber wir werden – genauso wie in den letzten Jahren im Bundestag – die Haushaltsmittel für die humanitäre Hilfe weiter erhöhen. 2017 waren es bereits 1,2 Milliarden Euro, und wir werden weiter erhöhen müssen, nicht nur, aber auch für die Menschen im Nahen und Mittleren Osten. Das ist ein Kernanliegen der SPD, und das ist auch bei den Verhandlungen deutlich geworden. Wir werden hier nicht lockerlassen.Vielen Dank.




	89. Axel Gehrke (AfD) Herr Präsident! Verehrte Kolleginnen und Kollegen! Wir stehen heute vor einer sehr sensiblen Entscheidung. Die Legalisierung von Cannabis als bisher verbotene Droge wird allein schon dadurch erschwert, dass wissenschaftlich belastbare Fakten bestenfalls zur medizinischen Wirkung, nicht aber zur Drogenszene bestehen.– Lassen Sie mich doch ausreden. – Daraus die Schlussfolgerung zu ziehen, wir müssten Cannabis legalisieren, um zu verlässlichen Daten zu kommen, wäre schon der erste Trugschluss.Nein, meine Damen und Herren, die Drogenszene bliebe bestehen. Und da Dealer im Gegensatz zu dem von Ihnen eingebrachten Gesetzentwurf keine Mindestaltersgrenze kennen, würde sie sich vermutlich sogar noch ausweiten. Zumindest ergibt sich für minderjährige Jugendliche kein Unterschied, ob mit oder ohne Legalisierung.Aus medizinischer Sicht ist die Sache ziemlich klar. Es handelt sich bei Cannabis und daraus abgeleiteten Produkten um psychoaktive Substanzen, die insbesondere Jugendliche in ihrem Reifeprozess erheblich gefährden, und zwar umso mehr, je jünger diese sind.– Wir reden nur von Jugendlichen; ganz besonders darum geht es.– Na gut. Aber ich stehe ja jetzt hier am Pult. Deswegen lassen Sie mich das doch sagen.Nun treten die Befürworter derzeit in breiter Front an. Ein typisches Argument ist: „Wir haben in der Jugend auch gekifft, und es hat uns nicht geschadet“, wobei allein das schon eine subjektive Bewertung ist.Herr Schinnenburg, das gilt auch für Ihren Nachbarn.Es soll auch nur nach strengen Regeln verkauft werden: nur über 18 Jahre, nur in Apotheken oder in gesonderten Fachgeschäften. Diese dürfen aber laut Ihrem Gesetzentwurf nicht in der Nähe von Schulen und Jugendtreffs angesiedelt sein –als ob das eine Rolle spielen würde! Allein das zeigt die ganze Hilflosigkeit Ihres Vorhabens gegenüber den echten Fakten einer Sucht.Nein, meine Damen und Herren, das Problem ist nicht, wie weit ein Jugendlicher laufen muss, um an den Stoff zu kommen.Vielmehr steht zu befürchten, dass Cannabinoide als Einstiegsdroge für härtere Gangarten zu werten sind. Interes­santerweise kommen diese Befürchtungen genau von denjenigen, die sich am meisten mit der Drogenszene befassen, von Strafverteidigern, Psychiatern, Drogenbeauftragten, Elternverbänden und Bewährungshelfern. Meine sehr verehrten Damen und Herren, ich möchte ausdrücklich warnen: Öffnen Sie nicht die Büchse der Pandora!Wir haben in der Drogenbekämpfung doch schon so viel erreicht. Bei unserer Jugend ist es nicht mehr cool, Drogen zu nehmen. Der Zigarettenverbrauch ist dramatisch gesunken. Saufpartys à la Ballermann sind geächtet.Eine erst kürzlich erhobene Umfrage belegt: 85 Prozent der Deutschen haben Angst vor Drogen; bei der Befragung von Eltern mit Kindern liegen diese Zahlen noch höher.Was wurde bisher erreicht? Laut einer Studie der Bundesregierung sind etwa 10,8 Prozent der Bevölkerung irreversibel nikotinabhängig. 3,48 Prozent sind alkoholabhängig, aber nur 0,6 Prozent irreversibel drogenabhängig. Das zeigt ganz eindeutig:Drogenprävention lohnt sich, und dabei sollten wir bleiben, meine Damen und Herren.Lassen Sie sich nicht durch beweisende Studien in die Irre führen. Es steckt fast immer eine hocheffiziente Lobby dahinter. Denn wer verdient an einer Freigabe? Für den Staat ist es ein Milliardengeschäft. An der Zigarettensteuer verdient der Staat jährlich 14 Milliarden Euro. Die damit verbundenen Folgekosten schätzen Fachleute auf 17 Milliarden Euro. Das Erste nimmt der Staat ein, das Zweite zahlt der Bürger. Wollen Sie das wiederholen?Herr Kollege, lassen Sie eine Zwischenfrage aus der FDP-Fraktion zu?Ja.Herr Kollege, vielen Dank, dass Sie die Zwischenfrage zulassen. – Sprechen Sie sich dann konsequenterweise auch für ein Verbot von Alkohol und Nikotin aus?Eigentlich müssten entsprechend auch Alkohol und Nikotin verboten werden; das ist ganz klar. Wir haben aber keinen Grund, eine dritte Droge zuzulassen, und darum geht es hier.Hier wie dort sind es weitere Steuern auf Kosten der Gesundheit der Bürger, und so etwas lehnen wir grundsätzlich ab.Aber auch für die Produktions- und Pharmafirmen, die den Stoff vertreiben, ist das ein Milliardengeschäft. In den USA gehen die Marihuana-Aktien durch die Decke; Anstiege in kurzer Zeit um über 2 000 Prozent sind dort die Regel. Aber auch bei uns ist der sogenannte Cannabis-Index, der die Börsenaktivitäten dokumentiert, allein vom Oktober 2017 bis heute in froher Erwartung um über 213 Prozent gestiegen. Nein, meine Damen und Herren, diese Spekulation auf Kosten unserer Jugendlichen darf es nicht geben.Wir haben bereits eine sehr liberale Drogengesetzgebung. Wir haben die Freigabe von Cannabis für medizinische Zwecke. Das ist schon umstritten genug und, wenn überhaupt, nur für wenige Indikationen gesichert. Aber immerhin: Es ist möglich.Herr Kollege, lassen Sie eine Zwischenfrage von Frau Strack-Zimmermann von der FDP-Fraktion zu?Bitte, ja.Vielen Dank, Herr Kollege, dass Sie eine Zwischenfrage zulassen. – Meine Frage geht dahin: Sie haben mitbekommen, dass der Bund Deutscher Kriminalbeamter darauf aufmerksam gemacht hat, dass die Jagd auf Konsumenten sehr viel Energie und vor allen Dingen Personal bindet. Er hat dringend empfohlen, diese Praxis beim Eigenkonsum zu beenden, weil die Polizei in diesem Land Wichtigeres zu tun hat. Sind Sie einer Meinung mit den Kriminalbeamten, dass die Polizei in diesem Land Wichtigeres zu tun hat, als Konsumenten, die ein Pfeifchen rauchen, zu jagen?Nein, Frau Kollegin, der Meinung bin ich nicht. Ich schließe mich der Meinung des Präsidenten der Bayerischen Polizei an, der genau dies als völligen Unsinn bezeichnet hat. Und da hat er, denke ich, vollkommen recht.Die Tatsache, dass der Besitz von Cannabis strafrechtlich verfolgt wird, hat bisher eindeutig gute, abschreckende Wirkung gezeigt.Es ist mehrfach widerlegter Irrglaube, dass durch eine Legalisierung der Drogenmarkt ausgetrocknet wird. Das Gegenteil ist der Fall: Die Dealer weichen auf billigeres Material und härtere Drogen aus. Die Niederlande lassen grüßen.Im Namen der Mehrzahl der Eltern dieses Landes fordere ich Sie auf: Bleiben Sie so cool wie unsere Jugendlichen, und lassen Sie den Geist nicht aus der Flasche! Die AfD jedenfalls lehnt zum Schutze unserer Jugend die vollständige Legalisierung von Cannabis kategorisch ab.Vielen Dank.




	90. Roderich Kiesewetter (CDU) Herr Präsident! Meine sehr geehrten Damen und Herren! Liebe Kolleginnen und Kollegen! – Jetzt geht das Pult immer weiter herunter, wahrscheinlich damit sich das Blickfeld erweitert, und genau das brauchen wir für Europa.Meine lieben Kolleginnen und Kollegen, wir sind am Ende zweier sehr spannender und zugleich sehr ernsthafter Debatten zu diesem ernüchternden Thema: Europa/europäische Eigenverantwortung und unsere unmittelbare Nachbarschaft, der Nahe und der Mittlere Osten. Wie wollen wir Millionen von Syrern überzeugen, in ihr Heimatland zurückzukehren, wie wollen wir Hunderttausende Syrer, die mit ihren Familien in Deutschland sind, überzeugen, in ihre Heimat zurückzukehren, wenn wir Europäer es zulassen, dass Assad stabilisiert wird, und wenn wir mit europäischen Steuergeldern das, was er zerstört hat, zu seiner Festigung wieder aufbauen? Darauf haben wir keine Antwort.Liebe Kolleginnen und Kollegen, ich glaube, aus der heutigen Debatte wird klar: Wenn wir Europa glaubwürdig, selbstbewusst und handlungsfähig machen wollen, müssen wir aus der Zuschauerrolle heraus und genau das ansprechen, was viele Kolleginnen und Kollegen hier heute Vormittag sehr deutlich aufgezeigt haben: Wie bekommen wir europäische Handlungsfähigkeit?Heute findet eine Sitzung des Weltsicherheitsrats statt, und Russland verhindert dort einen Waffenstillstand, weil Russland sagt, die Rebellen seien für das weitere Sterben verantwortlich. Daneben aber arbeitet Schweden gemeinsam mit Kuwait daran, auf der nächsten Sitzung des Weltsicherheitsrats eine 30-tägige Waffenruhe durchzusetzen.Seien wir selbstbewusst! Was ist denn Europa? Die Europäische Union ist eine der wenigen Regionen in dieser Welt, die noch für eine regelbasierte internationale Ordnung stehen, und dafür müssen wir Strahlkraft entwickeln. Das bedeutet: Wir müssen – so wie Kuwait – Partner überzeugen, dass sie unser Ansinnen für eine friedliche Lösung, für eine Unterstützung de Misturas intensiv begleiten. Regelbasierte Ordnung ist aufwendig, verträgt keinen Populismus, verträgt auch nicht, dass wir uns aus der Region zurückziehen; vielmehr brauchen wir – ich glaube, das ist in Teilen deutlich geworden – eine abgestimmte europäische Strategie, wie wir sie bereits im Libanon und in Jordanien haben. Diese Länder, die jeweils so viele Flüchtlinge aufgenommen haben, wie einem Viertel ihrer Bevölkerung entspricht – über 1 Million jeweils –, stabilisieren wir: Grenzsicherung, Wiederaufbau, gesunde Ernährung, Bildung in den Flüchtlingslagern. Im Übrigen: Das wäre so, als wenn die Bundesrepublik Deutschland 20 Millionen Flüchtlinge aus dem Umfeld der EU aufgenommen hätte. Ich sage das, um einfach einmal zu verdeutlichen, welch eine Dimension das hat, welch eine Leistung dort vor Ort erbracht wird. Deshalb gilt es, den Libanon und Jordanien zu stützen.Ein ganz konkretes Beispiel für europäische Unterstützung ist der Irak. Wir müssen verhindern, dass um Syrien herum alles weiter zusammenbricht. Der aus meiner Sicht völkerrechtswidrige Angriff der Türkei ist heute hier genug angesprochen worden. Mir geht es konstruktiv darum, dass wir auf der einen Seite den Irak zusammenhalten, den Irak als Gesamtstaat erhalten, auf der anderen Seite die Kurden in Nordkurdistan ermutigen, alles zu tun, um ihre Autonomie zu erhalten. Das muss unsere Aufgabe sein.Hier haben wir zwei Ansätze: erstens die Eindämmung des IS, die zunehmend gelingt – wir werden das auch mit einem entsprechenden Mandat fortsetzen –, und zweitens die Befähigung des Gesamtstaates; „wieder Vertrauen in staatliche Strukturen schaffen“, „Aussöhnung“, „Wiederaufbau“, „Bildung“ heißen die Stichworte.Wenn wir das alles zusammenbinden, bedeutet das für uns als Parlament aber auch, dass wir mit Blick auf die Haushaltsausstattung für unsere Entwicklungshilfe und unsere Sicherheitspolitik alles tun müssen, dass wir den Anforderungen der Parlamentsarmee und einer verantwortlichen Sicherheitspolitik gerecht werden. Das geht nur, wenn wir bereit sind, mehr Verantwortung nicht nur zu predigen, sondern unsere Entwicklungszusammenarbeit finanziell auch besser auszustatten und unsere Bundeswehr zu befähigen, dauerhaft in dieser Region tätig zu sein. Das erreichen wir mit PESCO-Projekten, indem wir zum Beispiel auf europäischer Ebene für Hubschrauber, aber auch für Entwicklungszusammenarbeit, Ausbildung und Bildung sorgen und gemeinsam europäisch finanzieren.Liebe Kolleginnen und Kollegen, in diesem Sinne können wir glaubwürdige europäische Nachbarschaftspolitik, Entwicklungszusammenarbeit und europäische Interessenvertretung leisten. Dafür werbe ich. Dafür steht die Union.Herzlichen Dank.




	91. Ulle Schauws (BÜNDNIS 90/DIE GRÜNEN) Sehr geehrter Herr Präsident! Liebe Kolleginnen und Kollegen! § 219a StGB verhindert, dass Frauen, die ungeplant schwanger und in einer Notlage sind, sich umfassend, schnell und sachlich informieren können.Gerade dort, wo sie sich als Erstes verlässliche Hilfe und Informationen erhoffen, nämlich bei einer Frauenärztin oder einem Frauenarzt, genau dort finden sie diese Informationen auf schnellem Wege nicht. Denn es ist Ärztinnen und Ärzten nach § 219a verboten, sachliche Informationen über Schwangerschaftsabbrüche öffentlich zu machen und online zu stellen. Es ist an der Zeit, dass wir endlich parlamentarisch über diesen veralteten Paragrafen des Strafgesetzbuches debattieren. Wir Grüne haben dazu eine klare Haltung: Wir wollen die Aufhebung von § 219a.„Werbung für den Abbruch der Schwangerschaft“, so lautet die Überschrift. Dieser Titel ist gewissermaßen irreführend. Denn unter Strafe gestellt wird hier deutlich mehr als nur Werbung. § 219a stellt auch die öffentliche und sachliche Information unter Strafe, und das ist absurd.Ich sage sehr deutlich: Wir Grünen wollen nicht, dass Werbung für Schwangerschaftsabbrüche erlaubt ist; das ist wohl klar.Was wir aber wollen, ist Rechtsklarheit für Ärztinnen und Ärzte. Denn sie müssen Schwangere sachlich und medizinisch informieren dürfen, ohne dafür vor Gericht zu landen.Erlauben Sie eine Zwischenfrage, Frau Kollegin, von der AfD?Nein, ich erlaube keine Zwischenfrage. – Das ist der Alltag in Deutschland. Dies ist der Ärztin Kristina Hänel passiert; das haben Sie der Presse entnommen.Abtreibungsgegnerinnen und ‑gegner gehen gezielt gegen Ärztinnen und Ärzte vor, die online über Schwangerschaftsabbrüche informieren, und zeigen diese an. Für eine Anzeige reicht es oft schon aus, wenn das Wort Schwangerschaftsabbruch auf einer Homepage steht. Und: Die Zahl der Anzeigen – das ist der Punkt – steigt seit Jahren stetig. Es ist jetzt an uns, liebe Kolleginnen und Kollegen, etwas dagegen zu tun, und zwar zügig.Noch einmal zur Klarstellung: Ärztinnen und Ärzte dürfen schon jetzt nicht für Schwangerschaftsabbrüche Werbung machen; denn das Berufsordnungsrecht der Ärzte untersagt anpreisende Werbung. Das wissen Sie alle. Sollte es darüber hinaus Sanktionsbedarf für Fehlverhalten geben, dann kann das auch im Ordnungswidrigkeitsrecht geregelt werden. Was wir dafür aber nicht brauchen, liebe Kolleginnen und Kollegen, ist das Strafgesetzbuch.Darum sage ich: Lassen Sie uns jetzt an dieser Stelle für Klarheit sorgen.Eine Aufhebung des § 219a – das sage ich in Richtung Union – berührt den Gesamtkompromiss des § 218 nicht.Das hat auch die Strafrechtsprofessorin Elisa Hoven am Montag in der FDP-Anhörung deutlich ausgeführt.Insofern können Sie hier beruhigt sein, liebe Unionskolleginnen und -kollegen. Sowohl das Schutzkonzept als auch die Beratungsregelung bleiben bestehen. Es geht ausschließlich um die Streichung des § 219a aus dem Strafgesetzbuch.Ein weiterer wichtiger Aspekt, liebe Kolleginnen und Kollegen: Aktuell wird die Suche nach einem Arzt oder einer Ärztin oft unnötig verkompliziert. § 219a verhindert nämlich die Transparenz darüber, in welchen Arztpraxen und nach welcher Methode dort Abbrüche vorgenommen werden. Es ist sogar für die Beratungsstellen schwer – das wissen wir –, an diese Informationen zu kommen. Sie können Frauen nach einer Konfliktberatung oft keine aktuelle Liste mit Adressen von Arztpraxen geben. Das ist ein echtes Problem; denn Beratungsstellen wollen und sollen die Frauen verlässlich informieren. Von pro familia wird deswegen unter anderem auch die Aufhebung von § 219a gefordert. Genau darum sollten wir dies im Sinne der Selbstbestimmung von Frauen jetzt tun.Liebe Kolleginnen und Kollegen, es ist unsere Aufgabe, dafür zu sorgen, dass Frauen diese höchstpersönliche Entscheidung so informiert und so gut wie möglich treffen können. Informationen müssen online zugänglich sein; wir leben im 21. Jahrhundert.Lassen Sie uns gemeinsam nach guten Lösungen suchen: im Sinne der Informationsfreiheit für Frauen und im Sinne der Rechtssicherheit für Ärztinnen und Ärzte.Vielen Dank. Ich freue mich auf die Beratungen.




	92. Karsten Möring (CDU) Frau Präsidentin! Liebe Kolleginnen und Kollegen! Als ich in der Tagesordnung das Thema der Aktuellen Stunde gelesen habe, habe ich mich ein bisschen gewundert. Ich hätte ja verstanden, wenn man an dem Tag, an dem wir – anlässlich der Revision von Urteilen der Verwaltungsgerichte Düsseldorf und Stuttgart – das Urteil des Bundesverwaltungsgerichts zu Fahrverboten erwarteten, dies zum Thema gemacht hätte, vielleicht auch das möglicherweise drohende EU-Vertragsverletzungsverfahren wegen der Überschreitung der Grenzwerte für Stickoxid. Aber das, was wir von der FDP zum Thema Mobilität vorgelegt bekommen haben, hat mich dann doch ein bisschen irritiert. Das gilt vor allen Dingen für Herrn Luksic, der seine Rede damit eingeleitet hat, dass wir nicht denken, sondern handeln sollen. Denn ich habe ein Flugblatt der FDP vor Augen, auf dem steht: „Mobilität neu denken.“ Gleichzeitig habe ich das Gefühl, dass für die Bundesregierung Denkverbote verlangt werden.Was steht eigentlich in dem Brief? Frau Lühmann hat es vorhin schon einmal zitiert. Darin steht: Wir denken darüber nach. –Wir denken über sieben Punkte nach, aber auch über die Dutzende von Punkten, die wir als Ergebnis der Dieselgipfel I und II und der weiteren Gespräche auf die Tagesordnung gesetzt haben.Von den Linken habe ich nichts anderes erwartet als einen Vorschlag, die Kosten des ÖPNV, der ja nicht kostenlos ist, zu sozialisieren und ihn auf diese Weise zu bezahlen. Das ist aber auch nicht das richtige Modell.Dann habe ich mich ein bisschen über Herrn Gelbhaar gewundert. Herr Gelbhaar, der ÖPNV ist bezahlbar, und da, wo er nicht bezahlbar ist, weil die Preissteigerungen groß sind – die Preise werden ganz wesentlich von der Lohnentwicklung bestimmt –, haben die Städte und zum Teil die Länder zum Mittel der Sozialtickets gegriffen, zum Beispiel bei mir in Köln. Dass das Modell richtig funktioniert, zeigt sich daran, dass wir jedes Jahr eine massive Zunahme der Fahrgastzahlen haben. Marktwirtschaftlich ist es also offensichtlich richtig, und sozial abgefedert ist es auch.Die Frage, was ein ticketfreier ÖPNV kostet und wie diese Kosten gedeckt werden sollen, steht auf einem ganz anderen Blatt. Der VDV hat ausgerechnet, dass die deutschen Nahverkehrsunternehmen insgesamt rund 12 Milliarden Euro durch Fahrgastbeförderung eingekommen haben. Angesichts dieser Summe, muss man nicht lange darüber nachdenken, um zu dem Ergebnis zu kommen, dass das eine ziemlich irreale Summe ist. Aber es wird ja auch nicht ernsthaft darüber diskutiert, eine solche Maßnahme einzuführen.In der ganzen Diskussion ist folgender Punkt noch gar nicht zur Sprache gekommen. Es geht um die Erfahrung, die wir in den letzten Jahren gemacht haben, nämlich dass die durch Emissionen entstandenen Belastungswerte deutlich zurückgegangen sind. Alle Maßnahmen, die bisher ergriffen worden sind, haben also Wirkung gezeigt.Die Frage, die heute bei der Erörterung vor dem Bundesverwaltungsgericht eine Rolle gespielt hat, nämlich die Frage der Verhältnismäßigkeit von Maßnahmen in Bezug auf die gesundheitliche Beeinträchtigung der Bevölkerung durch Emissionen, wird auch noch einmal eine Rolle spielen, wenn wir über praktische Maßnahmen in den Kommunen diskutieren. Wir arbeiten mit einem ganzen Bündel von Maßnahmen, die ich jetzt aber nicht wiederholen will. Meine Kollegen Jung und Klare haben das bereits thematisiert bzw. konkretisiert.Lassen Sie mich auf einen Punkt der sieben Punkten, die in dem Brief der drei Bundesminister aufgelistet worden sind, eingehen, der da lautet: den Rechtsrahmen verändern. Dahinter verbirgt sich die Überlegung, dass wir die Kommunen in die Lage versetzen sollen, Regelungen für punktuelle, lokal wirksame Maßnahmen zu treffen, zum Beispiel eine Beeinflussung der Ausgestaltung, wie Taxen, Busse oder Ähnliches mehr fahren sollen. Das zeigt: Bei vielen Punkten ist eine Ergänzung vorgenommen worden, über die wir diskutieren können. Das ist schlicht und einfach sinnvoll.Es wird gefordert: Die Autoindustrie soll für Nachrüstungen zahlen. Ja, es sieht so aus, als seien technische Nachrüstungen möglich und sinnvoll; sicherlich nicht in Bezug auf alle Fahrzeuge, weil die Nachrüstung bei einigen Fahrzeugen möglicherweise deutlich zu teuer und damit wirtschaftlich nicht vertretbar ist. Leider haben wir eine ganz klare Rechtslage, die besagt: Die Autoindus­trie, mit Ausnahme der Nutzung von Betrugssoftware, hat sich an die EU-Rechtsvorgaben gehalten. Die Vorgaben enthielten Lücken, die dazu geführt haben, dass wir heute in diesem Bereich Probleme haben. Allerdings will ich auch sagen: Die Autoindustrie wäre gut beraten, wenn sie, statt Hunderte von Millionen Euro in Imagekampagnen zu stecken, diese Mittel benutzen würde, um sich an den Nachrüstungskosten zu beteiligen. Das wäre meiner Ansicht nach eine lohnendere Investition.Ich will zum Schluss einen Blick auf die Situation in Köln werfen. Köln ist die belastetste Stadt in Nordrhein-Westfalen, ein Hotspot für Überschreitungswerte. Es gibt aber auch Messpunkte, an denen wir die Grenzwerte unterschreiten. Das zeigt doch eines, nämlich dass wir keine flächendeckenden Maßnahmen brauchen, sondern punktuell wirksame Maßnahmen. Die Punkte, die in der Diskussion sind, bieten hierfür genügend Möglichkeiten.Kollege Möring, ich bitte darum, den Punkt zu setzen.Schießen wir nicht mit Schrot auf Punktziele.Ich danke Ihnen.




	93. Hermann Färber (CDU) Sehr geehrter Herr Präsident! Liebe Kolleginnen und Kollegen! Wenn man den Antrag der Grünen zur Reduktion von Pestiziden liest, bekommt man den Eindruck, dass man einfach nur Pflanzenschutzmittel weglassen müsste oder am bestem auf ökologische Landwirtschaft umstellen sollte, und dann wäre alles gut.In Wirklichkeit aber stehen traditionelle und ökologische Landwirtschaft im Bereich des Pflanzenschutzes vor gewissermaßen gleich großen Herausforderungen; denn auch die ökologische Landwirtschaft kommt eben nicht ohne Gift aus. Zwar werden im Biolandbau keine synthetischen Stoffe eingesetzt, aber die Zulassung der Pestizidanwendungen im ökologischen Landbau in Deutschland umfasst immerhin 99 DIN‑A4-Seiten. Diese Liste enthält Stoffe wie Mineralöle, Bakterienstämme, Viren, Mikroorganismen, Pflanzenextrakte, aber natürlich auch Chemikalien wie Kupfer. Daran sieht man schon, dass die Begriffe „natürlich“ oder „natürliche Gifte“ nicht automatisch auch „harmlos“ und „gesund“ bedeuten.Um beispielsweise den Pilzbefall auf den Reben zu vermeiden, müssen die Winzer, die Weinbauern im Ökolandbau Chemikalien, prophylaktische Lösungen mit Kupfersalzen, spritzen. Kupfer aber ist ein Schwermetall, das sich im Boden sehr schnell anreichert und bereits in geringen Dosen höchst toxisch auf Mikroorganismen, auf Weichtiere und auch auf Kleinstlebewesen auswirkt. Die EU-Kommission würde deswegen Kupfer am liebsten verbieten. Das hätte jedoch verheerende Folgen; denn für den Ökolandbau gibt es keinen möglichen Ersatz.Herr Kollege, gestatten Sie eine Zwischenfrage der Grünen?Ja, gerne.Bitte sehr.Danke, Herr Präsident. – Danke, Herr Kollege Färber, dass Sie die Frage zulassen – Sie haben jetzt sehr auf den Ökolandbau abgezielt. Ich möchte Sie deshalb fragen, ob Sie mir zustimmen, dass wir eigentlich jetzt an einem Zeitpunkt angekommen sind, an dem es nicht mehr um gegenseitige Schuldzuweisungen geht und um die Frage, welche Landwirtschaft besser oder schlechter arbeitet, sondern darum, dass wir Lösungen brauchen. Deshalb haben wir sie aufgeschrieben.Das haben wir aus genau dem Grund gemacht, den Sie gerade benannt haben, weil wir im Weinbau durchaus ein Problem haben, weil wir mehr Investitionsmittel in die Forschung stecken müssen und dass es unsere gesamtgesellschaftliche Aufgabe ist, das zu tun, dass wir nämlich genau hier die Alternativen anbieten möchten. Deshalb frage ich Sie nochmals: Stimmen Sie mir zu, dass es jetzt darum geht, Lösungen zu erarbeiten, und dass wir uns darum gemeinsam bemühen müssen?Vielen Dank, lieber Herr Kollege Ebner. Wenn Sie mir noch ein bisschen zugehört hätten, dann wäre ich auf diesen Bereich noch zu sprechen gekommen; denn es geht in der Tat um eine Lösung für die Landwirtschaft insgesamt. Aber vielleicht lassen Sie mich meine Rede zu Ende bringen.Ich gehe gerade auf Ihren Antrag ein. Ich weiß ja, dass Sie persönlich viele Dinge besser wissen, als es manchmal aus den Anträgen ersichtlich wird.Mir liegt eben nur der Antrag vor, auf den ich jetzt eingehe und den Sie hier im Hause eingebracht haben.Aber ich möchte, ohne auf der Ökolandwirtschaft herumtrampeln zu wollen, einfach ein paar Dinge sagen. Ein Beispiel ist die Verwendung von Pyrethrinen. Diese werden aus Chrysanthemenarten gewonnen, und sie werden auch im Bioanbau als Insektizide zur Schädlingsbekämpfung eingesetzt. Doch sie wirken auf alle Insektenarten neurotoxisch, auch auf Nützlinge.Was mich an diesen Dingen besonders stört, ist, dass diese Chrysanthemen, die in Kenia und in Südamerika angebaut werden, eben nicht ökologisch angebaut werden, sondern dass Pestizide und Organophosphate für ihren Schutz eingesetzt werden. Das ist ein klassisches Beispiel dafür, wie man ein Problem nur ins Ausland verlagert. Wir können doch nicht hier sagen, wir arbeiten ökologisch, und dann die Produktionsmittel für diesen ökologischen Landbau in anderen Ländern erzeugen. Das ist alles andere als ökologisch.Ein weiteres Insektizid ist Spinosad, das in die höchste Kategorie für Bienengifte einsortiert ist. Es wirkt toxisch auf Schmetterlinge, ist giftig für Algen und Fische und sehr giftig für Wasserorganismen, wird aber in der ökologischen Landwirtschaft eingesetzt.Diese Beispiele zeigen, lieber Herr Ebner, dass wir eben keinen Keil zwischen die ökologische und die konventionelle Landwirtschaft treiben dürfen. Beide stehen vor den gleichen Herausforderungen. Ich bitte Sie alle hier im Hause – insbesondere natürlich auch Sie von den Grünen –, dass wir wieder zu einer lösungsorientierten, aber insbesondere zu einer sachlichen Debatte zurückkehren.Es ist wichtig und richtig, dass wir Pflanzenschutzmittel so sparsam wie möglich anwenden und dass wir bei der Ausbringung darauf achten, dass sie für Mensch und Natur möglichst unbedenklich sind.Was wir brauchen, ist mehr Forschung: Forschung für den integrierten Pflanzenschutz, Resistenzforschung, Forschung für den vorbeugenden, nichtchemischen Pflanzenschutz und insbesondere eine Ursachenforschung zum Rückgang des Insektenbestandes. Denn, meine sehr geehrten Damen und Herren, die vielzitierte Krefelder Studie – die einzige Studie, die es zum Rückgang des Insektenbestands gibt –, sagt nämlich über die Ursache dieses angeblichen Rückgangs überhaupt nichts aus.Deshalb muss die Ursache, die definitiv nicht nur monokausal ist, noch erforscht werden.Diese Maßnahmen sind im Koalitionsvertrag vereinbart. Darin werden unter anderem folgende Punkte genannt: die Umsetzung einer Ackerbaustrategie für umwelt- und naturverträgliche Anwendungen, die Ergänzung dieser Ackerbaustrategie durch ein Innovationsprogramm für digital-mechanische Methoden zur Reduzierung des Einsatzes von Pflanzenschutzmitteln, aber auch die Intensivierung der Forschung, um die Bandbreite innovativer, aber auch verfügbarer Pflanzenschutzmittel zu sichern. Denn die Resistenzen, die Sie vorhin angesprochen haben, Herr Ebner, werden ja nicht draußen auf den Äckern produziert. Da entstehen sie. Produziert werden sie in der Politik, weil immer weniger Wirkstoffe für die entsprechenden Problemfelder zur Verfügung stehen. Ein „Aktionsprogramm Insektenschutz“ und der Aufbau eines Monitoringzentrums unter Einbeziehung des Bundesumwelt- und des Bundeslandwirtschaftsministeriums werden die Biodiversität stärken.Der Antrag der Grünen ist daher nicht nur unsachlich und unvollständig, sondern er ist obendrein auch völlig überflüssig.Herzlichen Dank.




	94. Michael Donth (CDU) Herr Präsident! Liebe Kolleginnen und Kollegen! Herr Kollege Luksic, als ich Sie gerade gehört habe, habe ich mich gefragt, warum Sie eine Aktuelle Stunde zum kostenlosen öffentlichen Personennahverkehr beantragt haben, wenn Sie doch eigentlich über das Vertragsverletzungsverfahren sprechen wollen. Warum haben Sie dann keine Aktuelle Stunde zu diesem Thema beantragt?Das ist der erste Punkt, der mir auffiel. Ich will mich auf den öffentlichen Nahverkehr konzentrieren, den Sie in Bausch und Bogen weggewischt haben. Das sehe ich im Grunde genommen, wenn man es pauschal betrachtet, genauso. Dann wundert es mich aber – das ist mein zweiter Punkt –, dass gerade die FDP in Berlin die Einzigen sind, die sich hier in der Stadt dafür einsetzen. Auch in Osnabrück ist es, glaube ich, die FDP, die so etwas fordert. Aber das ist Ihr Problem.Lassen Sie uns zum Thema kommen, so wie es beantragt wurde. Ein Freifahrtschein für alle und immer, das klingt durchaus nach Schlaraffenland. Deshalb sollte es auch keinen verwundern, dass es nichts anderes ist als eine Utopie. In der Realität kann ein uneingeschränkt kostenloser ÖPNV nicht finanziert werden und ist auch nicht beabsichtigt. Der Verband kommunaler Unternehmen rechnet hoch: 13 Milliarden Euro Minimum müssten wir jedes Jahr dafür einsetzen.Ich glaube, wichtig ist, dass wir, nachdem sich die Wogen geglättet haben und sich der Nebel gelegt hat, die Diskussion über verschiedene Möglichkeiten, wie wir die Schadstoffbelastung in unseren Städten reduzieren können, nun besonnen führen können.In ihrem Brief an die EU-Kommission hat die Bundesregierung am 11. Februar dieses Jahres in den Katalog der Möglichkeiten, die zusammen mit den Ländern und Kommunen diskutiert oder auch schon umgesetzt werden, zusätzliche Werkzeuge für den Instrumentenkasten eingebracht. Dazu gehört auch die Frage, wie man möglichst viele Autofahrer dazu animieren kann, auf den ÖPNV umzusteigen, um die Abgasbelastung aus dem Individualverkehr zu reduzieren.Dass ein generell kostenloser ÖPNV nicht finanzierbar ist und womöglich sogar zu unerwünschten Effekten führt, haben Versuche gezeigt, die schon in einigen Städten durchgeführt wurden. Im brandenburgischen Templin zum Beispiel konnte man das Projekt nach fünf Jahren nicht mehr finanzieren und musste es beenden.In der estnischen Hauptstadt Tallinn hat der kostenlose ÖPNV zum Beispiel dazu geführt, dass plötzlich vermehrt Fußgänger und Radfahrer in die Busse eingestiegen sind und das System an den Rand des Zusammenbruchs brachten. Auch das ist widersinnig; denn wir wollen ja nicht nur das Bus- und Bahnfahren, sondern auch das Radfahren fördern.Ein kostenloser ÖPNV an bestimmten einzelnen Tagen mit hoher Schadstoffbelastung könnte aber eine Variante sein, die vielleicht die eine oder andere Stadt umsetzen möchte. Ich glaube, das ist ein wichtiger Baustein, der zur Schadstoffreduktion führt. Das können wir auch in der Hauptstadt meines Bundeslandes, in Stuttgart, sehen, wo in den letzten Jahren verbilligte Tickets angeboten wurden und die Werte kontinuierlich zurückgehen. Wenn die Maßnahmen aus dem Sofortprogramm Saubere Luft noch hinzukommen, werden die Werte sich sicherlich noch weiter verbessern.Ich finde es richtig, dass wir den Kommunen mehrere unterschiedliche Maßnahmen zum Testen an die Hand geben, um die Luft in den Innenstädten zu verbessern, ohne dass Fahrverbote ausgesprochen werden müssen. Wir setzen lieber auf eine Politik, die gute Ideen in den Kommunen fördert und an die Vernunft der Bevölkerung appelliert, als die Menschen mit Verboten zu drangsalieren.Weil aber jede Stadt und jede Gemeinde anders ist, ist es richtig, dass die Bundesregierung vorschlägt, zunächst in fünf Modellstädten mit unterschiedlichen Ausgangslagen hinsichtlich Einwohnerzahl, Schadstoffbelastung, Verkehrssituation etc. auszuloten, welche Maßnahme jeweils konkret erfolgversprechend ist. Das mag in Essen anders sein als bei mir in Reutlingen in meinem Wahlkreis in der Lederstraße.Deshalb begrüße ich es, dass die Bundesregierung in ihrem Schreiben ein ganzes Bündel an Maßnahmen vorschlägt. Jetzt muss der Umweltkommissar in Brüssel das prüfen und rückmelden, ob der Weg gangbar ist.Ich bin mir sicher, dass die Modellstädte, wie beispielsweise Reutlingen, zusammen mit Bund und Land ein innovatives Konzept erarbeiten werden. Dazu kann ein kostenloser Nahverkehr an Tagen mit hoher Belastung gehören; muss es aber nicht. Man muss das auf jeden Fall genau prüfen, um den Pkw-Verkehr zu reduzieren, ohne gleichzeitig dem ÖPNV zu schaden – im Gegenteil.Vielen Dank.




	95. Frauke Petry (Plos) Sehr geehrter Herr Präsident! Meine sehr geehrten Damen und Herren! Frau Möhring, Sie waren emotional, aber Sie haben denselben Fehler wie andere Redner vor Ihnen gemacht: Sie haben wieder einmal die schwere Situation der Frau gesehen – jawohl –, aber die Situation des Kindes völlig ausgeblendet.Seien wir doch einmal ehrlich: Ein Mangel an Informationen dazu, wie Schwangerschaftsabbrüche ablaufen, gibt es objektiv nicht. Dass Sie das behaupten, Frau Schauws oder Frau Högl, entspricht einfach nicht der Realität. Es geht, meine Damen und Herren, auch nicht um die technischen Informationen.Es geht nicht darum, zu wissen, wie die OP abläuft. Nein, meine Damen und Herren, es geht bei der Frage, ob eine Schwangerschaft beendet wird oder nicht, um die psychisch unglaublich belastende Situation, in der sich die Frau, die Familie befindet. Diese psychische Situation lässt sich nicht durch das Aufrufen einer Internetseite auflösen. Dazu braucht es Gespräche, dazu braucht es gerade das Beratungsgespräch.Deswegen ist § 219a, auch wenn er trocken klingen mag, nach wie vor notwendig; denn er macht sich zum Anwalt der ungeborenen Kinder. Er macht vielleicht nicht ausreichend, aber dennoch deutlich, dass er die verletzlichste und schwächste Form des Lebens schützt.Meine Damen und Herren, ich frage mich, wie viele von Ihnen, die hier für eine Abschaffung des Werbungsverbots plädieren, schon einmal mit Frauen gesprochen haben, die einen solchen Eingriff hinter sich habenund die sich zum Teil ihr Leben lang genau diesen Eingriff nicht mehr verzeihen.Meine Damen und Herren, der Erhalt von § 219a – egal, wie laut Sie werden – ist mehr als nur ein Symbol. Er schützt gleichermaßen die Frau und das Kind. Wenn uns diese Kinder wirklich am Herzen liegen – es sind in der Tat nach wie vor viel zu viele; vermutlich mehr als 100 000 pro Jahr in Deutschland –, dann sollten wir bei § 219a nicht zurückweichen und den Schutz, den er nach wie vor bietet, ernst nehmen.Wir werden auf die Probe gestellt. Ich bin der Meinung, wir sollten es an dieser Stelle genauso belassen, wie es ist, und hier dem ungeborenen Leben Tribut zollen.Danke.




	96. Johannes Fechner (SPD) Herr Präsident! Liebe Kolleginnen und Kollegen! Liebe Zuhörerinnen und Zuhörer auf der Tribüne! Anlass unserer Debatte ist das Urteil des Amtsgerichtes Gießen. Eine Frauenärztin wurde zu einer hohen Geldstrafe verurteilt, nur weil sie für ein legales Tun einen sachlichen Hinweis gab. Das zeigt ganz klar: Wenn eine Ärztin bestraft wird, obwohl sie nur sachlich auf eine legale Handlung hingewiesen hat, dann müssen wir den § 219a ändern.Denn keine Frau macht sich die Entscheidung leicht, einen Schwangerschaftsabbruch vorzunehmen. Aber wenn sich eine Frau für eine legale Abtreibung entschieden hat, dann sollten wir ihr als Gesetzgeber in dieser schwierigen Situation keine Steine in den Weg legen, sondern ihr den einfachen Zugang zu sachlichen Informationen ermöglichen.Denn wenn wir als Gesetzgeber Schwangerschaftsabbrüche mit guten Gründen in bestimmten Fällen zulassen, dann können wir diese Grundentscheidung doch nicht dadurch wieder infrage stellen oder sogar umkehren, dass wir diejenigen strafrechtlich verfolgen, die die von uns zugelassenen Schwangerschaftsabbrüche dann auch tatsächlich durchführen. Diesen Widerspruch können wir nicht hinnehmen.Nicht nur für die betroffenen Frauen müssen wir Rechtssicherheit schaffen, sondern vor allem auch für die Ärztinnen und Ärzte. Es kann doch nicht sein, dass wir Schwangerschaftsabbrüche in bestimmten Fällen zulassen, aber diejenigen dann strafrechtlich verfolgen, die hierauf sachliche Hinweise geben wollen.Zu meiner Vorrednerin: Bei einer Änderung oder Abschaffung dieses Paragrafen ist nicht mit einer Werbeflut zu rechnen; denn nach ärztlichem Standesrecht ist diese reißerische Werbung, die niemand will – auch wir nicht –, verboten und kann mit bis zu 200 000 Euro Geldbuße bestraft werden. Es ist deshalb aus meiner Sicht völlig falsch, zu sagen, dass hier eine Werbeflut droht. Das zeigen auch die sehr wenigen Fälle an Verurteilungen, die es nach dieser Vorschrift gab.Kurzum: Auch wir wollen keine reißerische Werbung für Schwangerschaftsabbrüche. Wir wollen aber ermöglichen, dass sachliche Informationen gegeben werden. Deswegen brauchen wir Rechtssicherheit für Frauen und Ärzte durch die Abschaffung oder mindestens eine deutliche Änderung dieses Paragrafen, liebe Kolleginnen und Kollegen.Wir verhindern doch keine Schwangerschaftsabbrüche durch das Strafrecht, durch Einschüchterungen oder durch Drohungen,sondern durch die bewährte und kompetente Beratung wie etwa bei pro familia oder durch gute Betreuungseinrichtungen oder indem wir gerade junge Familien finanziell noch besser fördern. Das schafft Perspektiven für Eltern und Kinder. Damit verhindern wir Schwangerschaftsabbrüche, aber nicht mit Strafrecht und Repressionen, liebe Kolleginnen und Kollegen.Die konsequenteste Lösung wäre sicher die Abschaffung, die auch wir in der SPD beschlossen haben. Wenn sich dafür keine Mehrheit findet, dann sollten wir über andere Lösungen sprechen, die zukünftig zumindest sachliche Informationen ermöglichen.Erlauben Sie eine Zwischenfrage?Nein, das Thema ist mir zu wichtig. Gerne hinterher bei einer Kurzintervention.Beim FDP-Gesetzentwurf stört mich, ganz offen gesagt, dass Sie in Absatz 2 eine zusätzliche Strafbarkeit, eine zusätzliche Strafnorm einführen wollen und dass Sie das Tatbestandsmerkmal „Werben“ benutzen. Ich finde, wir sollten noch einmal prüfen, ob es nicht präzisere Regelungen gibt. Heute steht das Wort „Anpreisen“ im Tatbestand. Hierüber werden wir sicher Diskussionen führen können.Ich freue mich, dass wir schon in der nächsten Sitzungswoche zu einem Treffen kommen und darüber beraten, ob wir einen Gruppenantrag erstellen können. Ich bin optimistisch, dass wir uns auf einen Antragstext einigen und das schaffen, was dringend notwendig ist: Rechtssicherheit für die betroffenen Frauen und vor allem für die betroffenen Ärztinnen und Ärzte. Die SPD-Fraktion wird dafür werben und steht dafür bereit.Vielen Dank.




	97. Karl-Heinz Brunner (SPD) Sehr verehrter Herr Präsident! Meine Kolleginnen und Kollegen! Meine Damen und Herren auf den Zuschauerrängen! Dass die AfD heute Genderpolitik mit uns betreiben will, hat mich schon etwas verwundert. Aber es hat sich nun gezeigt, dass diese Genderpolitik sich in einem Antrag gegen Vollverschleierung widerspiegelt, der eher der Verschleierung ihrer eigenen Ziele als der Entfernung eines Grauschleiers über der Politik dient.Ich, meine sehr verehrten Damen und Herren, liebe Kolleginnen und Kollegen, komme, wie Sie wissen, aus Bayern, wo, wie man sagt, die Welt noch in Ordnung ist:Im Winter ist es kalt, im Sommer warm, zumindest meistens. – Was wir in Bayern immer haben, sind Menschen, die gerne zu uns kommen, die einen zur ärztlichen Behandlung, die anderen zum Einkaufen – in München sehen wir in der Maximilian- oder der Theatinerstraße ab und an welche –, wieder andere, um die Schönheiten Neuschwansteins zu sehen und dann entweder mit Burka oder – als Japaner – mit Gesichtsmaske den Berg hinauf­zuhecheln. Die meisten kommen aber wegen unserer Gastfreundschaft, unserer Weltoffenheit.Das ist, meine Damen und Herren, liebe Kolleginnen und Kollegen, nicht selbstverständlich. Deshalb haben wir unsere Verfassung, unsere Grundwerte, unser Grundgesetz. Das ist es, was uns wirklich wichtig ist; denn wir wollen nicht wie die Österreicher wegen „Verwaltungsübertretung“ verzwergte Rohrkrepierer oder wie die Letten Gesetze mit homöopathischer Wirkung für fünf oder sechs Personen beschließen. Vielmehr verstehen wir in Deutschland, dass Menschen auch Angst haben, Angst davor, dass sich unter einer Burka, dem Nikab oder was auch immer etwas Verbotenes verbergen kann.Wir alle wissen auch, dass Terrorismus nicht vor Kleiderordnungen haltmacht und dass er nur durch beherztes Vorgehen der Sicherheitskräfte verhindert werden kann. Das haben wir in der Großen Koalition gemacht, und das machen wir auch weiterhin.Uns ist aber auch wichtig, meine sehr verehrten Kolleginnen und Kollegen, dass Frauen die gleichen Rechte wie Männer haben, ohne Einschränkung, ohne Zwang und ohne gesellschaftlichen Druck. Deshalb will ich, dass Frauen so frei leben können, wie sie wollen,dass sie selbst entscheiden, mit wem, wann, wo und wie sie ihr Leben gestalten, dass sie selbst entscheiden, was sie anziehen, was ihnen gefällt und dass sie, wenn sie es wollen, auf ihrem Dekolleté gerne ein Kreuz, den ­Davidstern oder den Halbmond tragen können. Und Frauen, die sich vor Blicken schützen wollen, sollen sich auch vor Blicken schützen können, und zwar so, wie sie es wollen, ohne dabei unsere Rechtsordnung zu beeinträchtigen. Dazu brauchen wir Regeln: Regeln gegen Ungleichbehandlung, Regeln für Chancengleichheit, Regeln für gleichberechtigte Teilnahme am gesellschaftlichen Leben auch bei uns in Deutschland. Vielleicht brauchen wir auch Quoten, und ja, Equal Pay brauchen wir dazu auch.Was wir nicht brauchen, meine Kolleginnen und Kollegen, ist ein Verbot der Vollverschleierung; das ist irrelevant, es hat mit der Wirklichkeit nichts zu tun.Denn die Wahrscheinlichkeit, in Deutschland einer vollverschleierten Frau zu begegnen, ist – außer am Flughafen München bei der Einreise aus den Golfstaaten – nicht größer, als auf dem Ku’damm einem Strauß zu begegnen.Mir ist es auf jeden Fall noch nicht passiert.Meine sehr verehrten Damen, meine sehr verehrten Herren, ich gehe davon aus, dass wir uns in dieser Legislatur noch mit vielen Anträgen dieser Art auseinandersetzen werden müssen: mit Anträgen, die Ängste schüren, mit Anträgen, mit denen Wahrheiten nicht aufgedeckt, sondern Halbwahrheiten zurechtgebogen werden sollen. Sie werden niemandem helfen, nicht den Menschen, nicht unserer Gesellschaft. Sie fügen ihnen Schaden zu. Sie werden sich gegen religiöse Gruppen, gegen Homosexuelle, gegen ausländische Mitbürger, gegen Flüchtlinge, gegen Minderheiten, gegen jeden wenden. Aber es handelt sich immer nur um platte Symbolpolitik, die unsere Zukunft nicht besser macht, sondern einfach nur Schuld zuweist und keine Verantwortung übernimmt.Worum es aber geht, ist: Wir müssen in die Köpfe und Herzen der Menschen – wir müssen aber nicht in die Kleiderschränke.Vielen Dank.Einen schönen guten Tag wünsche ich Ihnen, liebe Kolleginnen und Kollegen. – Danke, Karl-Heinz Brunner.Letzter Redner in dieser Debatte ist Philipp Amthor. Sie waren letztens auch schon letzter Redner; das habe ich natürlich nicht vergessen.




	98. Stephan Mayer (CSU) Herr Präsident! Sehr verehrte Kolleginnen! Sehr geehrte Kollegen! Um gleich eines zu Beginn der Rede klar und ausdrücklich zu sagen: Burka und Nikab gehören nicht nach Deutschland.Burka und Nikab sind nicht vereinbar mit unserem Menschenbild und mit unseren Wertevorstellungen. Burka und Nikab sind auch nicht Symbole der Gleichbehandlung und der Gleichberechtigung, sondern sie sind Symbole der Ausgrenzung, der Abgrenzung, der Diskriminierung und vor allem der Ungleichbehandlung der Frau gegenüber dem Mann und der Unterdrückung der Frau.Die Vollverschleierung der Frau – um dies ausdrücklich zu sagen – ist ein enormes und erhebliches Integrationshindernis. Mir fehlt, mit Verlaub, die Fantasie, mir vorstellen zu können, wie eine burkaverschleierte Frau mit ihren Kindern auf dem Kinderspielplatz unvoreingenommen und vorbehaltlos Kontakt zu einheimischen Frauen aufnehmen soll und wie umgekehrt einheimische Frauen einen sozialen Kontakt aufbauen sollen zu Frauen, die mit Burka oder mit Nikab verschleiert sind.Wir brauchen aber – um auch dies ausdrücklich zu sagen – keinen Nachhilfeunterricht von der AfD,weil wir als CDU/CSU bereits in der letzten Legislaturperiode tätig geworden sind. Wir haben in der letzten Legislaturperiode eine sehr umfassende bereichsspezifische Regelung geschaffen, die im Juni letzten Jahres in Kraft getreten ist. Damit wird eines klipp und klar geregelt: Beamtinnen und Soldatinnen dürfen nicht vollverschleiert ihren Dienst ausüben.Sie dürfen vollverschleiert keine Tätigkeit ausüben, die einen Bezug zum Dienst hat.Was auch sehr wichtig ist – das gilt sowohl für Einheimische als auch für Menschen, die in der jüngsten Zeit zu uns gekommen sind –: Lichtbilder, die für Legitimationspapiere, für Pässe, aber beispielsweise auch für Identitätspapiere von Flüchtlingen angefertigt werden, dürfen die betreffende Person nicht vollverschleiert zeigen.Meine sehr verehrten Kolleginnen und Kollegen, wir haben umfassend von den Regelungsmöglichkeiten Gebrauch gemacht, die es für uns als Deutscher Bundestag gibt. Vor diesem Hintergrund möchte ich, weil die Regelungen in Frankreich, in Österreich oder in Belgien genannt wurden, eines klar sagen: Wir haben ein Grundgesetz bzw. eine Verfassung. Daran sind wir gebunden. Ich muss schon sagen: Ich habe, mit Verlaub, Schwierigkeiten damit und finde es auch sehr merkwürdig, dass insbesondere eine Partei, die immer wieder peinlichst Wert darauf legt, dass wir uns an Recht und Gesetz halten, und fordert, dass sich die Bundesregierung an die Rechtsordnung, an die Verfassung zu halten hat, einen Antrag stellt, der zum Gegenstand hat, dass wir eine himmelsschreiend verfassungswidrige Regelung in Kraft setzen sollen.Es ist nun einmal so, dass die Religionsfreiheit nach Artikel 4 Grundgesetz ein Kernpfeiler unserer Verfassung und unserer Freiheitsrechte ist, die auch entsprechend zu achten sind. Wir werden deshalb – um dies klar zu sagen – als CDU/CSU all das tun, was rechtlich möglich und rechtlich vertretbar ist, um auch in weiteren Kontexten die Vollverschleierung zu unterbinden. Aber wir werden nicht sehenden Auges eine verfassungswidrige Regelung in Kraft setzen.Deshalb möchte ich klar davor warnen, hier Scheindebatten zu führen. Wir tun sehr viel, um Menschen in unserem Land zu integrieren, die neu zu uns gekommen sind. Aber um dies auch zu sagen: Wir haben auch klare Erwartungen.Die klare Erwartung ist nicht nur, dass man sich an Recht und Gesetz hält, sondern auch, dass man unsere Bräuche, unsere Traditionen und unsere Gepflogenheiten zu achten weiß. Um dies ausdrücklich zu sagen: Die Kommunikation in einer pluralen und offenen Gesellschaft wie der unsrigen ist aus meiner Sicht nur möglich, wenn man im wahrsten Sinne des Wortes Gesicht zeigt.Insofern bin ich unserem Bundesinnenminister Dr. Thomas de Maizière sehr dankbar, dass er klargemacht hat, dass die Burka nicht zu Deutschland gehört. Vor diesem Hintergrund möchte ich auch ausdrücklich darauf hinweisen, dass wir gesellschaftspolitisch hier einen klaren Kurs fahren, dass wir aber nichts unternehmen werden, was nur im Entferntesten zu einer verfassungswidrigen Regelung führt.Meine sehr verehrten Kolleginnen und Kollegen, ich warne davor, hier Scheindebatten zu führen. Wir sollten uns mit den tatsächlichen Problemen unseres Landes beschäftigen. Wir sind bereits in der letzten Legislaturperiode tätig geworden. Das ist der AfD vielleicht nicht richtig aufgefallen.Es hat – das darf ich zum Abschluss sagen – sehr lange gedauert, bis Sie mit diesem Antrag ums Eck gekommen sind. Dieser Antrag, der schon lange angekündigt worden ist, ist erst vor einem Tag tatsächlich auf unseren Schreibtischen gelandet. Das zeigt aus meiner Sicht auch, dass Sie sich selbst sehr schwergetan haben mit der Formulierung Ihres Antrages. Die CDU/CSU wird diesem Antrag – um es klar zu sagen – die Ablehnung erteilen.




	99. Wilfried Oellers (CDU) Sehr geehrter Herr Präsident! Meine sehr geehrten Kolleginnen und Kollegen! Meine sehr geehrten Damen und Herren! Frau Kollegin Ferschl, Gratulation zur ersten Rede! – Sie hört gerade nicht zu. – Allerdings muss ich sagen: Bei dem, was sie gesagt hat, weiß ich gar nicht, wo ich anfangen soll, um das alles richtigzustellen.Deswegen versuche ich, mich auf meine vorgesehenen Ausführungen zu beschränken und darf als Korrektur gleich anmerken:Das Schreckgespenst, das Sie hier von Befristungen gemalt haben, entspricht mitnichten den Tatsachen. Bevor ich jetzt mit Zahlen komme, schicke ich vorweg, dass wir von Unionsseite auch lieber unbefristete Beschäftigungsverhältnisse sehen wollen und dass wir sicherstellen wollen, dass mit den Befristungen kein Missbrauch getrieben wird. Das steht komplett außer Frage.Schauen wir uns die Zahlen einmal genau an. Wenn wir von allen Beschäftigungsverhältnissen ausgehen, haben wir eine Befristungsquote, die lediglich zwischen 7 und 8 Prozent liegt. Das heißt, wir reden hier nicht davon, dass die überwiegende Zahl von Arbeitsverhältnissen befristet ist; vielmehr liegt die Zahl weit unter der Hälfte. Sie haben es eben anders dargestellt.Von diesen – ich sage jetzt mal – ungefähr 8 Prozent betreffen über 50 Prozent den öffentlichen Dienst und somit etwas unter 50 Prozent die Privatwirtschaft. Das Schreckgespenst Richtung Privatwirtschaft aufzubauen, passt also auch schon nicht.Zur Kenntnis muss man auch nehmen, dass zum öffentlichen Dienst auch der Bereich der Wissenschaft gehört, in dem ebenfalls ein Großteil befristeter Arbeitsverhältnisse zu finden ist. Wir sind in der letzten Legislaturperiode mit unserem Wissenschaftszeitvertragsgesetz schon tätig geworden, um in diesem Bereich bessere Regeln einzuführen.Wir können also feststellen, dass wir, wenn wir von den Zahlen ausgehen, eigentlich eine konstante – wenn nicht sogar eher fallende – Tendenz bei den befristeten Beschäftigungsverhältnissen haben.In diesem Zusammenhang darf man auch nicht verschweigen, dass die Übernahmequote mit über 40 Prozent recht beachtlich ist. Hier kommt den Befristungen eine Funktion zu, die man nicht außer Acht lassen darf: die Brückenfunktion von Arbeitslosigkeit hin zum ersten Arbeitsmarkt.Das bestätigt selbst das IAB. Das sollten Sie zur Kenntnis nehmen; denn das ist eine neutrale Stelle.Darüber hinaus ist festzuhalten: Wir brauchen in unserem Wirtschaftsleben das Flexibilisierungsinstrument der Befristungen, auch wenn Sie das in Abrede stellen.Es geht dabei nicht nur um Auftragsspitzen, die wir abdecken, sondern auch um andere Themenfelder. Wenn man sich einmal anschaut, dass wir den Arbeitnehmern einerseits Flexibilisierung durch Elternzeit, durch das Recht auf Teilzeit, das wir einführen wollen, ermöglichen,stellt man fest, dass dadurch andererseits neue Notwendigkeiten entstehen, befristete Arbeitsverhältnisse zu begründen. Das muss man auf jeden Fall zur Kenntnis nehmen.Wenn die Befristung auf der einen Seite Flexibilisierungsinstrumente schafft, kann man sie auf der anderen Seite nicht komplett abschaffen. Es ist wichtig, dass man mit der sachgrundlosen Befristung ein Flexibilisierungsinstrument hat, das einfach unbürokratisch ist.Ich betone: unbürokratisch. Als Arbeitgeber kann man für einen bestimmten Zeitraum in dem Wissen, einmal nichts dokumentieren zu müssen, Ruhe haben.In diesem Zusammenhang ist es richtig und wichtig, dass wir dieses Instrument erhalten – sicherlich mit gewissen Korrekturen, etwa beim Thema der Kettenbefristungen, so wie wir es vereinbart haben, wenn wir es denn umsetzen können. Ich denke, dass alle diese Argumente das Schreckgespenst der Befristungen etwas abmildern können.Herr Oellers, gestatten Sie eine Zwischenfrage oder Zwischenbemerkung einer Kollegin von der Linken?Gerne.Bitte sehr.Herr Oellers, wir beide haben ja schon öfter über sachgrundlose Befristungen gestritten. Wenn ich Sie jetzt höre, muss ich natürlich gleich wieder etwas dazu sagen.Meine Kollegin hat von jüngeren Menschen in einem befristeten Arbeitsverhältnis gesprochen. Bei jüngeren Menschen ist es ganz einfach so: Die haben den höchsten Anteil an Befristungen und die meisten sachgrundlosen Befristungen. Es ist richtig, dass es bei Arbeitnehmern unter 30 Jahren etwa 24 Prozent sind. Sie reden von Durchschnittszahlen. Es gibt eine Kaskade: Je älter man ist, desto größere Chancen hat man auf ein unbefristetes Arbeitsverhältnis. Aber junge Leute haben diese Chancen nicht; das entspricht der Wahrheit.Ein zweiter Punkt. Wie können Sie die Möglichkeit der Flexibilisierung, Elternzeit und Rückkehrrecht von Teilzeit in Vollzeit als Begründung dafür nehmen, dass man Befristungen doch braucht? Ich weiß nicht, ob Sie sich erinnern, dass es in Betrieben früher einmal so etwas wie Springer gegeben hat. Diese Kräfte hat man vorgehalten, um genau so etwas zu ermöglichen. Zu sagen, dass das nur mit befristeten Beschäftigungsverhältnissen geht, ist einfach nicht richtig. Man muss anfangen, darüber nachzudenken, was man machen kann. Da findet man Lösungen, die aber nicht in Befristungen liegen.Frau Kollegin Krellmann, was Ihr letztes Argument angeht, dass man Elternzeit oder Teilzeit nicht als Begründung für befristete Arbeitsverhältnisse heranziehen darf, muss ich Ihnen ganz deutlich widersprechen. Was erwarten Sie denn bitte schön von den Arbeitgebern? Wenn jemand kommt und sagt: „Ich möchte mein Recht auf Teilzeit, ich möchte Elternzeit in Anspruch nehmen“, und ich als Arbeitgeber sehe, wie lange dieser Arbeitnehmer ausfällt, dann ist doch klar, dass ich als Arbeitgeber jemanden einstellen möchte, der während genau dieser Zeit arbeitet.Es sind nicht alle Unternehmen so groß, dass sie das mit Springern auffangen oder entsprechendes Personal vorhalten können; das muss man einmal ganz deutlich sagen.Sie geben mir mit Ihren Ausführungen aber die Gelegenheit, darauf hinzuweisen, dass es in der Tat auch andere Beispiele gibt. Sie kennen vielleicht schon das Beispiel aus meinem Wahlkreis. Ich bringe es gerne noch einmal zur Kenntnis. Der Landrat des Kreises Heinsberg hatte vier Beschäftigte in seiner Kreisverwaltung, die alle befristet beschäftigt waren. Allerdings waren das alles Befristungen mit Sachgrund, weil es sich um Schwangerschaftsvertretungen handelte. Irgendwann hat der Landrat dann gesagt: Wir haben diese Verträge so oft verlängert. Jetzt müssen wir den Leuten doch auch die Möglichkeit geben, eine Festanstellung zu bekommen. Ich sehe, dass ich hier im Haus offensichtlich den Bedarf habe. – Das hat er gemacht. Die erste Kritik aus dem Kreistag kam von der Opposition. Ich muss jetzt dazusagen, dass die CDU nicht in der Opposition ist. Es war auch Ihre Fraktion, die behauptet hat, dass dadurch der Personalhaushalt aufgebläht würde. Man kann es Ihnen offensichtlich an der Stelle nicht recht machen. Das muss man auch einmal ganz deutlich sagen.Sie haben die jüngeren Beschäftigten angesprochen. Dazu muss ich sagen: Wenn man sich die Alterskurve anschaut, dann gebe ich Ihnen recht, dass der Anteil der jüngeren Arbeitnehmer unter den befristet Beschäftigten sehr hoch ist. Im Moment liegt die Quote bei ungefähr 23 Prozent. Vor einigen Jahren lag sie noch bei 28 Prozent. Die Kurve flacht aber in den anderen Altersstufen, die in diesen Diagrammen angegeben sind, drastisch ab, sodass wir bei Altersstufen, die weit in Richtung des Rentenalters gehen, zu einem Wert kommen, der bei 2 bis 3 Prozent liegt. Bei einer solchen Zahl sehe ich in der Tat keinen Missbrauch. Ich denke, dass damit Argumente gebracht sind, die Ihre Vorwürfe entkräften.Wir brauchen diese Flexibilisierungsinstrumente – wenn man den Arbeitnehmern diese berechtigten Instrumente an die Hand geben will –, um personalbedingte Engpässe bei Elternzeit, bei Pflegefällen abfedern zu können. Das ist überhaupt keine Frage. Aber dann muss man auch mit gleichem Maß messen und Flexibilisierungsinstrumente ermöglichen.Herzlichen Dank.




	100. Beate Müller-Gemmeke (BÜNDNIS 90/DIE GRÜNEN) Sehr geehrter Herr Präsident! Sehr geehrte Kolleginnen und Kollegen! Liebe Gäste auf den Tribünen! Zu keinem anderen Thema habe ich so oft geredet, insgesamt achtmal. Das zeigt: Das Thema ist wichtig.Wir haben schon alle Argumente ausgetauscht. Die einen sind für und die anderen sind gegen die sachgrundlose Befristung. Meine Kurzform lautet heute: „Grundlos“ meint im Wortsinn nichts anderes, als „einfach so“, „willkürlich“ zu befristen. So etwas darf es in der Arbeitswelt nicht geben; denn der Preis für die Beschäftigten und für ihre Familien ist zu hoch.Deshalb wollen auch wir Grünen die sachgrundlose Befristung abschaffen.Nur für Existenzgründerinnen und Existenzgründer wollen wir für eine gewisse Zeit die Möglichkeit erhalten; aber das ist nur ein Detail. Deshalb werden wir dem Antrag der Linken zustimmen.Eine Mehrheit werden wir zwar nicht bekommen, und doch tut sich vielleicht etwas. Die SPD konnte sich nicht komplett durchsetzen – die sachgrundlose Befristung wird bleiben –, aber falls die Große Koalition kommt, wird die sachgrundlose Befristung zumindest eingeschränkt.Das ist ein Anfang und gut für die Beschäftigten, die davon profitieren.Die geplanten Einschränkungen sind bei allen offenen Fragen und aller Kritik im Detail ein erster Schritt. Das begrüßen wir.Jetzt zur Unionsfraktion. Sie bleiben stur und halten an der sachgrundlosen Befristung fest, und das, obwohl diese Möglichkeit immer wieder krass und hemmungslos ausgenutzt wird. Ein Beispiel: Als Anfang 2017 der Mutterschutz reformiert wurde, hat die BWRmedia dazu ein Themenspezial für Arbeitgeber veröffentlicht. Ich habe das dabei und kann es nachher gerne Herrn Oellers geben. Die Überschrift lautet: Das ist der ideale Arbeitsvertrag; die Zwischenüberschrift lautet: So hebeln Sie den Kündigungsschutz von Schwangeren aus. Der Trick: Wenn sich junge Frauen bewerben, schließen Sie mit Ihrer Mitarbeiterin einen befristeten Arbeitsvertrag. Die Formel lautete – Zitat –, zweimal sechs und einmal zwölf – gemeint sind Monate – sachgrundlos zu befristen. Damit gibt es keinen Kündigungsschutz, wenn Frauen schwanger werden. – Das muss man sich einmal vorstellen. Solch eine Beratung ist unsäglich. Wenn die soziale Verantwortung in dieser Weise verloren geht, dann müssen Sie diese Fehlentwicklung endlich korrigieren.Sachgrundlose Befristungen sind nicht nur ungerecht und willkürlich, sondern auch unnötig. Es gibt eine ausreichend lange Probezeit. Kleine Betriebe sind ganz vom Kündigungsschutz befreit. Ich sage es immer und immer wieder: Wer gute Gründe hat, kann natürlich weiterhin befristen, vorübergehend bei Auftragsspitzen, bei Projektarbeit auf Zeit, bei Elternzeitvertretungen, bei Vertretungen wegen längerer Krankheit, sogar bei Erprobung. Für Befristungen gibt es also genügend gute Gründe, für die sachgrundlose Befristung aber nicht. Gehen Sie von der Union noch einmal in Klausur, und lassen Sie sich endlich von Ihrem Arbeitnehmerflügel, der CDA, überzeugen; denn die haben es verstanden.Durch Befristungen darf weder das unternehmerische Risiko auf die Beschäftigten übertragen werden noch der Kündigungsschutz umgangen werden. Nur das ist richtig und fair. Flexible Arbeitsverhältnisse dürfen keine Einbahnstraße sein; denn die Beschäftigten brauchen soziale Sicherheit. Das stärkt auch den dringend notwendigen gesellschaftlichen Zusammenhalt. In diesem Sinne werden wir, falls es zur Großen Koalition kommt, die Gesetzgebung kritisch, aber konstruktiv begleiten.Vielen Dank.




	101. Harald Ebner (BÜNDNIS 90/DIE GRÜNEN) Sehr geehrter Herr Präsident! Danke für die Worte zu den Geschwistern Scholl. Beide wurden bei mir im Wahlkreis geboren; ich habe sehr großen Respekt vor deren Handeln.Werte Kolleginnen und Kollegen, heute hat der Frühling kurz vorbeigeguckt. Dann denkt man gleich, dass es draußen vor der Tür wieder brummt und summt. Doch nach den alarmierenden Zahlen aus den letzten Jahren hoffe ich sehr, dass er nicht noch stummer werden wird, als er es ohnehin schon wurde. 75 Prozent weniger Insekten-Biomasse, 50 Prozent weniger Vögel, die Hälfte der Wildbienenarten ist in Deutschland bestandsgefährdet. Liebe Kolleginnen und Kollegen, so kann es wirklich nicht weitergehen.Es geht darum, ob wir unsere Lebensgrundlagen nach und nach zerstören oder ob wir sie für unsere Enkelkinder erhalten.Wir haben schon jetzt Rückstände von Pestiziden im Wasser, die teuer herausgefiltert werden müssen und als Rückstände ins Grundwasser gehen. Wir haben Unkräuter, die resistent geworden sind und gar nicht mehr auf Herbizide reagieren. Dass wir von den hohen Pestizidmengen herunterkommen müssen, erkennen immer mehr Menschen, auch in der konventionellen Agrarwirtschaft. Ich erinnere an die Deutsche Landwirtschafts-Gesellschaft und ihre zehn Thesen.Einige von Ihnen werden jetzt sagen: Ja, aber ohne Pflanzenschutz geht es doch gar nicht. – Natürlich, ohne Pflanzenschutz geht es nicht; da haben Sie recht. Pflanzenschutz ist aber ein System aus vielem: aus der Fruchtfolge, der Sortenwahl, der Züchtung, der Bodenbearbeitung und der Bodenkultur. Wer Pflanzenschutz auf chemisch-synthetische Helferlein reduziert, der hat ihn einfach nicht verstanden.Die Bäuerinnen und Bauern wissen selber am besten, dass ein intaktes Ökosystem die Grundlage für ihr Wirtschaften ist, dass sie Bestäuber und Nützlinge brauchen. Deshalb wollen wir sie beim wichtigen Prozess „Weg von immer mehr Pestiziden, hin zu einem ganzheitlichen Pflanzenschutz“ unterstützen. Da hat die Große Koalition in den letzten vier Jahren aber leider nichts hinbekommen. Das muss man so feststellen.Pestizide können niemals risikolos sein. Seit Jahrzehnten sehen wir immer wieder, was ich einen Pestizidzyklus nenne: Die Zulassungsverfahren versagen und müssen reformiert werden. Herstellerstudien erfassen die Risiken nicht umfassend, und die Entscheidungen der Zulassungsbehörden basieren auf diesen Herstellerstudien. Nach Jahren des Großversuches am Ökosystem stellen wir fest, dass doch etwas passiert ist. Dann wird ein Wirkstoff vom Markt genommen, aber der nächste steht schon bereit. – Diesen Pestizidzyklus müssen wir endlich beenden.Liebe Kolleginnen und Kollegen von Union und SPD, in Ihrem Koalitionsvertragsentwurf sprechen Sie davon, den Einsatz chemisch-synthetischer Pestizide wirksam reduzieren zu wollen. Sie wollen angeblich sogar den Glyphosat-Einsatz schnellstmöglich und grundsätzlich beenden, obwohl der Herr Noch-Minister Schmidt den greifbaren europäischen Glyphosat-Ausstieg ja gerade erst wirksam versenkt hat. Erst zündeln und sich dann fürs Löschen loben lassen – das lassen wir Ihnen nicht durchgehen, Herr Schmidt.Es ist trotzdem gut, dass Sie das aufgeschrieben haben. Machen Sie das! Lassen Sie es nicht bei warmen Worten bewenden! Machen Sie Nägel mit Köpfen. In unserem Antrag haben wir für Sie die wesentlichen Punkte aufgeschrieben, auch dass Glyphosat und Neonicotinoide – die Bienenkiller – sofort vom Markt müssen und dass wir den Pflanzenschutz mit wirksamen Maßnahmen voranbringen müssen. Wer hier wirksam werden will, der muss Anreize schaffen, der muss Wissen schaffen, der muss auch die Umsetzung des Wissens fördern.Wir nehmen Sie mit Ihrem Koalitionsvertrag beim Wort. Sie haben geschrieben: Uns liegt „der Schutz der Bienen besonders am Herzen“.Unterstützen Sie unseren Antrag zum Wohl der Bienen.Danke schön.




	102. Alice Weidel (AfD) Sehr geehrter Herr Präsident! Liebe Kolleginnen und Kollegen! In Ihrer Regierungserklärung, Frau Bundeskanzlerin, haben Sie von Verantwortung gesprochen – von der Verantwortung Deutschlands für Europa. Doch was verstehen Sie darunter, und was bedeutet das letztlich für die Bürger? Mehr Verantwortung für Europa heißt bei Ihnen – darauf sind Sie in Ihrer Rede herzlich wenig eingegangen – mehr Geld – deutlich mehr Geld – und Souveränitätsabgabe an die Europäische Union, und das bedeutet mehr Geld des deutschen Steuerzahlers für ein Projekt, das längst nicht mehr dessen Interessen vertritt, ganz im Gegenteil.Wie wenig die wohl dritte Auflage der Großen Koalition die Interessen der deutschen Steuerzahler vertritt, hat diese bereits im Koalitionspapier sehr deutlich gemacht. Ich zitiere:Wir sind zu höheren Beiträgen Deutschlands zum EU-Haushalt bereit.So heißt es dort wortwörtlich.– „Stimmt!“, höre ich da von SPD-Seite. Richtig: Sie können nämlich nur das Geld anderer Leute ausgeben; etwas anderes können Sie nicht.Purer Sozialismus: Immer geht es um das Umverteilen des Geldes, das Ihnen nicht gehört.Das ist insgesamt eine Bankrotterklärung. Sie, sehr geehrte Damen und Herren von Union und SPD, opfern freiwillig das Königsrecht eines jeden Parlamentes, und das ist die Budgethoheit.Doch was ist eigentlich der aktuelle Anlass für diese Freizügigkeit der Koalitionsparteien im Umgang mit dem Geld des deutschen Steuerzahlers? Der aufgeblähte EU-Apparat steht vor einem Problem: Durch den bevorstehenden Brexit – Sie haben ja leider nur kurz erwähnt, was er eigentlich bedeutet und warum die Briten sich entschieden haben, auszutreten – entsteht eine riesige Haushaltslücke in der Europäischen Union. Vernünftigerweise könnte man jetzt meinen, dass eine durch den Brexit kleinere EU ihren Haushalt entsprechend anpasst, sprich: Kürzungen vornimmt; denn eine kleinere EU bedeutet auch einen kleineren Haushalt, vor allem wenn ein Leistungsträger, etwa ein Nettozahler von der Größe Großbritanniens, wegfällt. Das ist keine Exklusivmeinung der AfD; das sehen Regierungen anderer europäischer Staaten genauso: Das sieht man in Den Haag so, das sieht man auch in Wien so. Doch offensichtlich sieht man das hier, in Berlin, nicht so.Diese wirtschaftlich logische Maßnahme, nämlich die Kürzung des EU-Haushalts, ist für Sie, sehr geehrte Bundeskanzlerin, offensichtlich ein Horrorszenario; das kann man nicht anders sagen. Warum? Ein geringeres Budget für Brüssel könnte einen Einflussverlust der ohnehin überbezahlten EU-Bürokraten, von denen es im Übrigen viel zu viele gibt, mit sich bringen.Das ist auch der Grund – Sie haben damit ja Erfahrungen, Herr Schulz –, warum die EU an Großbritannien ein Exempel statuieren möchte – eine Strafmaßnahme im Übrigen jenseits jeder ökonomischen und politischen Vernunft. So geht man nicht mit europäischen Partnern um, sehr geehrte Damen und Herren!Im Übrigen: Die vor dem Referendum vorhergesagte Rezession ist ausgeblieben. Ganz im Gegenteil: Die britische Wirtschaft ist sogar kräftig gewachsen. Nun hat man in Brüssel, Paris und Berlin Angst, dass das Beispiel Schule machen könnte, dass sich weitere Staaten in Europa ihre Souveränität zurückholen. Das ist nämlich auch der Grund, warum die EU-Kommission plant, den Zugang der Briten zum Binnenmarkt bereits in der Übergangsphase bei Bedarf willkürlich einzuschränken; das muss man sich mal vorstellen. Indem Sie diese Pläne der Ausgrenzung des wichtigsten Außenhandelspartners Deutschlands in der EU unterstützen – diese Ausgrenzungspläne unterstützen Sie alle –, machen Sie den freien Handel und den Wettbewerb innerhalb von Europa zur Geisel einer gescheiterten EU-Ideologie –ein törichter Fehler, ein folgenreicher Fehler für den europäischen Zusammenhalt; denn die historisch gewachsenen guten Wirtschaftsbeziehungen zwischen Großbritannien und dem übrigen Kontinent müssen bewahrt bleiben; sonst wird Europa weltwirtschaftlich ins Hintertreffen geraten.Deshalb ist es auch an der Zeit, den freien Warenaustausch mit dem Vereinigten Königreich auf eine neue verlässliche Grundlage zu stellen und diese Unsicherheiten endlich zu beenden. Das passende Vertragsformat – eine Lösung gibt es also bereits – ist das Abkommen über den Europäischen Wirtschaftsraum, EWR, von 1992. Einige von Ihnen kennen es, einige vielleicht aber auch nicht, offensichtlich nicht.Dieses Vertragswerk garantiert die unverzichtbaren Binnenmarktprinzipien: den freien Waren-, Zahlungs-, Dienstleistungs- und Personenverkehr. Das wollen wir auch ganz klar sagen: Diese Freizügigkeit sollte aber nicht den Zuzug europäischer und außereuropäischer Sozialmigranten in unsere Sozialsysteme beinhalten.Ganz einfach: Im Interesse Europas muss der EWR-Vertrag für Großbritannien geöffnet werden. Hören Sie endlich mit Ihren Drohgebärden gegenüber Großbritannien auf!Europa muss sich wieder auf die Grundfreiheiten eines demokratischen Kontinents der Subsidiaritäts- und der demokratischen Rechte souveräner Staaten besinnen.Wir, die Abgeordneten des Bundestages, haben es letztlich in der Hand: Entweder vertreten wir gemäß unserem Wählerauftrag den demokratischen Staat in einem demokratisch gewählten Parlament, oder wir entscheiden uns, die Verantwortung an eine europäische Zentralregierung abzugeben – eine Entscheidungsgewalt, die über keinerlei demokratische Legitimation verfügt und diese auch nicht anstrebt.Für die AfD-Fraktion im Übrigen ist die Entscheidung glasklar: Wir handeln im Sinne des Souveräns, im Sinne des deutschen Wählers. Wir stehen ein für ein Europa der Vaterländer, das nach innen die Prinzipien der Freiheit, der Selbstbestimmung, des Wettbewerbs und der Demokratie vertritt und sich nach außen einig und selbstbewusst auf der Weltbühne zeigt.Ich danke Ihnen für Ihre Aufmerksamkeit.




	103. Arno Klare (SPD) Frau Präsidentin! Meine sehr verehrten Damen und Herren! Liebe Kolleginnen und Kollegen! Ich komme aus dem Ruhrgebiet. Falls man das hören sollte: Es ist durchaus gewollt. Das Ruhrgebiet ist ein riesiger Ballungsraum, einer der größten in Europa. Für uns ist der Brief, der da geschrieben worden ist, wirklich ein sehr positives Signal. Ich werde versuchen, Ihnen das zu begründen.Die Stadt Essen wird als eine Modellstadt genannt. In die Stadt Essen pendeln pro Arbeitstag 150 000 Arbeitnehmer, die also nicht in Essen wohnen. Das sind übrigens rund 46 Prozent aller Menschen, die in Essen tagtäglich ihrer Arbeit nachgehen. Getoppt wird das in Nordrhein-Westfalen nur von Bonn und Düsseldorf, die noch höhere Einpendlerquoten haben. Das Problem ist: Nur 18 Prozent dieser Menschen pendeln mit dem ÖPNV; 62 Prozent nehmen das Auto. – Das ist für eine Metropolregion wie diese, aus der ich komme, ein wirklich schlechter Wert. Insofern ist das Signal, das von diesem Brief ausgeht: „Der ÖPNV muss attraktiver werden“, goldrichtig.Wir werden die Klimaziele nur erreichen können, wenn der ÖPNV wirklich deutlich attraktiver wird. Hier ist gerade die Rede davon gewesen, man habe keinen Kompass, es sei keine Landkarte zu sehen. Ich werde einmal ein paar Gegenbeispiele nennen.In der letzten Legislaturperiode haben wir die Regionalisierungsmittel um fast 1 Milliarde Euro erhöht. Für die, die es nicht wissen: Daraus wird alles, was im Schienenpersonennahverkehr fährt, finanziert. Wir haben ins Baugesetzbuch ein neues städtisches Gebiet, nämlich das urbane Gebiet, hineingeschrieben, um endlich wieder einmal Arbeit, Freizeit und Wohnen zusammenbringen zu können, so wie es in den Städten ehemals einmal war. Das setzt die Leipzig Charta von 2007 um.In meiner Region werden viele Mittel – in der letzten Ausbaustufe sind es dann fast 4 Milliarden Euro – in ein Backbone-Schienensystem – „RRX“, „Rhein-Ruhr-Express“ genannt – investiert; im Bundesverkehrswegeplan prioritär hinterlegt. Wir reden auch über Radwege. Der RS1, der genau an meinem Wahlkreisbüro vorbeigeht, ist aus dem Ministerium von Barbara Hendricks gefördert worden; weitere Maßnahmen werden aus dem Bundesverkehrswegeplan finanziert.Diese Linie setzt sich fort. Wir werden die Mittel für das Gemeindeverkehrsfinanzierungsgesetz-Bundesprogramm in dieser Legislaturperiode von 333 Millionen Euro auf 1 Milliarde Euro erhöhen.Als ich das bei einer Veranstaltung erwähnt habe, hat mir irgendein Journalist gesagt, das sei der typische Überbietungswettbewerb von Politikern. Herr Gastel ist mein Zeuge; er war auch bei der Veranstaltung. Jetzt steht das im Koalitionsvertrag.Wir werden auch die Mittel aus dem „Sofortprogramm Saubere Luft“ – 1 Milliarde Euro waren da angesagt – verstetigen und jedes Jahr im Haushalt haben.Jetzt zu sagen, wir hätten nichts getan, ist schlichter Blödsinn.Die Menschen legen pro Tag 3,4 Wege außerhalb ihrer Wohnung zurück. Das ist eine Zahl, die seit 100 Jahren gleich geblieben ist; die hat sich nicht verändert. Nur die Länge der Wege hat sich vervielfacht; die Summe liegt jetzt bei 42 Kilometern. 20 Prozent dieser Wege sind übrigens dem Freizeitbereich zuzuordnen.Im Modal Split bundesweit sind wir bei 11 Prozent ÖV und 55 Prozent MIV, also motorisierter Individualverkehr. Das müssen wir drehen. Wir müssen deutlich bessere Zahlen bekommen. Insofern ist dieser Brief wirklich ein Anstoß, ein Nachdenkanstoß, wie Kirsten Lühmann es gesagt hat. Ich bin froh und dankbar, dass wir heute begonnen haben, auf parlamentarischer Ebene ergebnisoffen darüber nachzudenken, wie wir den ÖPNV attraktiver machen können.– Ich weiß nicht, Herr Krischer, ob Sie mir nie zuhören. Diesmal haben Sie wieder nicht zugehört. Verdammt noch mal! Das kann doch nicht sein! Irgendwann müssen Sie doch mal zuhören! Ich habe doch gerade eine ganze lange Liste aufgezählt. Wieso hören Sie mir nicht zu?Sie von den Grünen haben, weil Sie den Haushalt abgelehnt haben, auch alle die Mittel abgelehnt, die ich gerade aufgezählt habe.Dieser Brief ist der richtige Impuls zur richtigen Zeit. Er steht in einer erfolgreichen Tradition und setzt diese Tradition fort. Wir werden das in dieser Legislaturperiode zu einem sehr guten Ende bringen.




	104. Frauke Petry (Plos) Sehr geehrter Herr Präsident! Meine sehr geehrten Damen und Herren! Eine Reduktion von Pflanzenschutzmitteln oder – im Grünensprech – Pestiziden um 40 Prozent klingt toll. Das ist ungefähr so, als würden Sie den Bauern erklären, dass wir die Behandlung beispielsweise mit Antibiotika um ein paar Tage reduzieren, um die gleiche Wirkung zu erreichen. Das ist unwissenschaftlich par excellence, liebe Grüne, und eines Antrags in diesem Hohen Haus unwürdig.Anstelle einer ernsthaften und niveauvollen Auseinandersetzung mit Pflanzenschutzmitteln, die wie jedes Medikament auch Nebenwirkungen haben, setzen Sie darauf, dass die Bürger auf Angst- und Panikmache hereinfallen und genauso wenig von chemischen Formeln verstehen wie Sie. Das ist schade. In der Landwirtschaft kommt insbesondere Ihre Klientel, liebe Linke, die wenig Geld hat und billige Produkte einkaufen muss, ohne Pflanzenschutzmittel einfach nicht aus. Es wäre schön, es wäre anders, aber wir leben nun einmal in der Realität. Deswegen sollten Sie zur Kenntnis nehmen, dass große Teile der Bürger kein Problem damit haben, konventionell erzeugte Produkte zu kaufen. Leider haben die allermeisten Bürger nicht die Wahl, ob sie billig oder teuer kaufen.Insofern sollten Sie Ihren Antrag eher umformulieren. Es geht per se nicht um die Menge. Weniger wäre immer schön; da bin ich ganz ehrlich. Die Bauern sind an dieser Stelle eher die Experten als wir. Es geht aber mehr um die Art der Anwendung. So müsste Ihr Antrag viel besser folgendermaßen lauten: Überprüfung der Zulassungsverfahren und Entwicklung der Anwendungsbestimmungen für Pflanzenschutzmittel. Sie wollen aber gar nicht überprüfen, Sie wollen einfach pauschal reduzieren.Meine Damen und Herren von den Grünen, vielleicht sind Sie im Ausschuss bereit, Ihren Antrag professionell nachzurüsten, den Bauern und der Umwelt zuliebe.




	105. Ulli Nissen (SPD) Sehr geehrte Frau Präsidentin! Liebe Kollegen und Kolleginnen! Eine Nachricht hat uns letzte Woche überrascht. Es hieß: Der Bund will kostenlosen Nahverkehr. Viele gingen davon aus, dass dies flächendeckend für ganz Deutschland gelten soll. Wer sich dann aber genauer informiert hat, konnte nachlesen: Es geht um einen Modellversuch in fünf Kommunen. Das ist ein wichtiger Vorschlag zur Erprobung von Maßnahmen zur Verbesserung der Luftqualität.Modellversuche in einigen Regionen halte ich für sinnvoll, dies kann aber nicht überall umgesetzt werden. In vielen Städten sind die Kapazitätsgrenzen auf vielen Linien des ÖPNV fast erreicht. In meinem Frankfurter Wahlkreis haben wir auf manchen Strecken seit 2010 einen Zuwachs von fast 60 Prozent erreicht – da geht kaum noch was. Wir versuchen in Frankfurt aber, durch neue Angebote die Menschen dazu zu bringen, auf Busse und Bahnen umzusteigen. Auf Initiative unseres Frankfurter SPD-Oberbürgermeisters Peter Feldmann hin haben wir im Dezember 2017 einen neuen Nachtverkehr für U-Bahnen am Wochenende eingeführt. Dieser Nachtverkehr wird von der Bevölkerung ausgesprochen gut angenommen. Deshalb denken wir darüber nach, diesen Nachtverkehr auf weitere U-Bahnen und Straßenbahnen auszuweiten.Grundsätzlich müssen die Bedingungen für den ÖPNV verbessert werden. Deshalb ist es gut, dass wir im Koalitionsvertrag vereinbart haben – das werden wir umsetzen, wenn er von unseren Mitgliedern angenommen wird –, dass wir den ÖPNV deutlich besser fördern wollen. Die Mittel für das Gemeindeverkehrsfinanzierungsgesetz, GVFG, werden bis 2021 für Aus- und Neubaumaßnahmen in zwei Schritten von derzeit 333 Millionen Euro auf 1 Milliarde Euro erhöht und – das ist wichtig – dynamisiert. Mit dem GVFG stellt der Bund Mittel für große Infrastrukturmaßnahmen zum Ausbau des Schienenpersonennahverkehrs vor Ort zur Verfügung. Damit können die Angebote im Nahverkehr ausgeweitet werden, sodass die Pendlerinnen und Pendler zum Beispiel nicht mehr in überfüllten Zügen zusammenstehen müssen. Auch dies trägt sicherlich dazu bei, dass die Bürgerinnen und Bürger bereit sind, vom Auto auf die Bahn umzusteigen.Natürlich ist auch ein geringerer Preis ein guter Grund, umzusteigen. Auch diesbezüglich bin ich unserem Frankfurter Oberbürgermeister dankbar. Er war und ist ein großer Befürworter des 1‑Euro-Tickets. In einem ersten Schritt können jetzt erstmals Schülerinnen und Schüler für nur 1 Euro am Tag – das ist das Wiener Modell – rund ums Jahr Bus und Bahn fahren, und dies in ganz Hessen. Das finde ich großartig.Herr Gelbhaar, in Frankfurt sind auch die Gebühren für das Einzelticket in diesem Jahr gesenkt worden. Es geht also auch anders, wenn man das will, und Oberbürgermeister Peter Feldmann will das.Wichtig ist mir auch, dass wir im Koalitionsvertrag vereinbart haben, dass Kommunen bei der Ausschreibung von ÖPNV-Leistungen auch privaten Unternehmen soziale und ökologische Standards vorschreiben können. Noch lieber wäre es mir natürlich, wenn der ÖPNV von kommunalen Unternehmen betrieben wird.Mit einer Senkung der Trassenpreise wollen wir mehr Güter vom Lkw auf die klimafreundliche Schiene bringen.Die Verbesserungen beim ÖPNV sind gut und notwendig; aber nicht jeder Weg kann mit dem ÖPNV erledigt werden. Deshalb müssen wir auch bei den Kraftfahrzeugen ansetzen. Beim Nationalen Forum Diesel im Sommer 2017 sind einige Beschlüsse gefasst worden. Die Automobilindustrie war nur bereit, die Kosten für ein Softwareupdate für gut 5 Millionen Dieselfahrzeuge zu übernehmen, um die Stickoxidemissionen zu reduzieren. Die Automobilindustrie hat angegeben, dass es keine sinnvollen Nachrüstungen für die Hardware gibt. Der damalige Verkehrsminister Dobrindt hat dies geglaubt. Ich habe das schon damals arg angezweifelt. Seit dieser Woche wissen wir: Die Automobilindustrie hat wieder einmal getrickst. Der ADAC hat aktuell nachgewiesen, dass Hardwarenachrüstungen an Dieselfahrzeugen nicht nur möglich, sondern auch hochwirksam sind. Bis zu 70 Prozent innerorts und bis zu 90 Prozent außerorts weniger Schadstoffe lassen sich laut den Messungen durch Nachrüstungen an solchen Fahrzeugen erreichen. Besonders wichtig ist: In besonders belasteten Gebieten könnte dies eine Verbesserung der Luftqualität um bis zu 25 Prozent mit sich bringen.Auch aufgrund des zu erwartenden Gerichtsurteils und des absolut notwendigen Gesundheitsschutzes ist es dringend erforderlich, dass neben den Softwareupdates an den betroffenen Dieselfahrzeugen auch ein Hardwareupdate erfolgt. Autos mit diesen Updates könnten von drohenden Fahrverboten in deutschen Innenstädten ausgenommen werden.Aus meiner Sicht ist vollkommen klar: Die Kosten für die Nachrüstung dürfen nicht an den Verbrauchern hängen bleiben.Sie müssen von den Verursachern getragen werden, von der Automobilindustrie. Das hat Barbara Hendricks schon immer gefordert. Die Automobilunternehmen zahlen in den USA zig Milliarden an Strafen. Und hier wollen sie die Kundschaft mit der Übernahme der lächerlich niedrigen Kosten für die Softwareupdates abspeisen? Viele befürchten jetzt einen hohen Wertverlust ihres Fahrzeugs und haben Angst, dass sie die betroffenen Innenstädte nicht mehr befahren dürfen. Viele Handwerker sehen die Gefahr, dass sie ihren Fuhrpark stilllegen müssen. Liebe Kolleginnen und Kollegen, lassen Sie uns den Menschen helfen und die Automobilindustrie dazu verpflichten, die Kosten für die notwendige Nachrüstung zu übernehmen.Ich danke Ihnen ganz herzlich für die Aufmerksamkeit.




	106. Heike Hänsel (DIE LINKE.) Herr Präsident! Liebe Kolleginnen und Kollegen! Die Münchner Sicherheitskonferenz, die man eher „Münchner Unsicherheitskonferenz“ nennen müsste, wurde mehrfach erwähnt. Sie hat letzte Woche stattgefunden, und die Eskalation im Nahen und Mittleren Osten war dort mit Händen zu greifen. Eine Kriegsdrohung jagte die nächste. Das Gewaltverbot in den internationalen Beziehungen – dazu gehört übrigens auch das Verbot, Gewalt anzudrohen – scheint in diesen Tagen nicht mehr viel wert zu sein. UN-Generalsekretär António Guterres bezeichnete die Situation als schlimm; der Krieg in Syrien drohe zu einem neuen regionalen Krieg zu werden.Die Situation in dieser Region ist aber nicht einfach so entstanden. Die bestehenden Konflikte wurden durch die Regime-Change-Politik der NATO-Staaten im Irak, in Libyen und Syrien weiter verschärft.Die Linke – das möchte ich betonen – war von Beginn an gegen all diese militärischen Interventionen. Wir haben von Anfang an gewarnt: Krieg ist niemals eine Lösung für all diese Konflikte.Wir sehen jetzt leider all die zerstörten und destabilisierten Staaten und auch die Erosion des Völkerrechts. Überall in diesen Ländern wird mit Waffen aus deutscher Produktion gekämpft, teilweise sogar auf beiden Seiten. Deshalb kann eine zentrale Schlussfolgerung nur sein, dass wir hier endlich ein Rüstungsexportverbot durchsetzen.Frau Merkel hat sich heute Morgen in ihrer Regierungserklärung über die Situation in Syrien und die Angriffe auf Ost-Ghuta bestürzt gezeigt. Auch die Fraktion Die Linke verurteilt die Angriffe auf Ost-Ghuta ganz klar. Sie müssen sofort gestoppt werden,genauso wie die Angriffe aus Ost-Ghuta auf Wohnviertel von Damaskus. Wir wollen einen sofortigen Waffenstillstand überall, wo in Syrien gebombt wird.Was aber nicht geht, ist, dass die Kanzlerin mit keinem Satz die türkische Offensive, den Angriffskrieg auf Afrin, auf die kurdische Region im Norden Syriens erwähnt. Das, muss ich sagen, war wirklich schändlich, und es zeigt die Doppelstandards dieser Bundesregierung.Noch dazu wird der Angriff mit Panzern aus deutscher Produktion geführt. Alle Fraktionen hier im Hause haben sich ganz klar positioniert und diesen Angriffskrieg verurteilt. Ich frage die Bundesregierung, warum sie sich bis heute weigert, ihn auch ganz klar so zu bezeichnen. Das ist ein Bruch des Völkerrechts.Im Auswärtigen Ausschuss gab es da gestern ein großes Rumgeeiere. Die Bundesregierung hat sich selbst auf harte Nachfragen nicht positioniert und diesen Angriff nicht als das bezeichnet, was er ist: ein Verbrechen an den Kurdinnen und Kurden in Syrien. Hinzu kommt, dass die Bundesregierung jetzt – ausgerechnet jetzt – zu einer Normalisierung der Beziehungen mit der Türkei kommen möchte. Das kann nicht sein. Wir können mit der Türkei nicht zur Tagesordnung übergehen.Von Reichskanzler Bethmann-Hollweg ist das vor hundert Jahren gefallene Wort überliefert, die Türkei müsse an der Seite Deutschlands gehalten werden, auch wenn die Armenier dabei zugrunde gingen. Heute drängt sich einem der Eindruck auf, dass die Bundesregierung auf die NATO-Partnerschaft mit der Türkei setzt, auch wenn dabei die Kurden zugrunde gehen. Diese Kumpanei mit der islamistischen Diktatur Erdogans muss sofort aufhören!Und stoppen Sie Ihre Rüstungsexporte in die Türkei! Dazu gehört auch die geplante Aufrüstung der Leopard-2-Panzer, die in Syrien die Kurden niederwalzen.Ich frage die Bundesregierung auch, warum sie angesichts dieses Angriffskrieges die Bundeswehr in den AWACS-Verbänden in der Türkei belässt. Mit diesen Aufklärungsflügen werden Daten erhoben, die für die Luftangriffe der Türkei nutzbar sind.Und Sie wollen uns jetzt wirklich glauben machen, dass das NATO-Mitglied Türkei diese Daten nicht nutzt? Mit Sicherheit können Sie es nicht ausschließen, Herr Gabriel. Deshalb gibt es in diesem Fall nur eines: Die Bundeswehr muss raus aus der Türkei!Aber es geht nicht nur um die Waffenlieferungen. Sie tragen auch sonst wenig zu einer friedlichen Lösung in dieser Region bei. Die Linke fordert seit Jahren, dass sich die Bundesregierung für einen sofortigen und umfassenden Waffenstillstand in ganz Syrien und für eine politische Lösung einsetzt. Wie lange haben Sie sich dem verweigert, weil es Ihnen allein um einen Regime-­Change ging und Sie fest davon überzeugt waren, dass dieser Krieg der islamistischen Rebellen in Syrien gewonnen wird? Bis heute nehmen Sie nicht Abstand von dieser unsäglichen Regime-Change-Politik. Aber es gäbe gute Möglichkeiten. So fordern Sie zum Beispiel umfassende humanitäre Hilfe für die gesamte Region. Warum verbinden Sie das nicht mit einem Angebot und fordern endlich ein Ende der verheerenden Sanktionen gegenüber der syrischen Bevölkerung? Das wäre ein Schritt dahin gehend, dass man auch über einen Waffenstillstand verhandeln könnte.Auch im Koalitionsvertrag steht bezüglich Waffenexporten sehr wenig Neues. Man dachte zu Beginn, Sie hätten vielleicht aus dem Gebaren der islamistischen Diktatur in Saudi-Arabien gelernt, die einen blutigen Krieg gegen die Bevölkerung im Jemen führt, und Sie würden dies jetzt wenigstens zum Anlass nehmen, einen generellen Rüstungsexportstopp im Hinblick auf die Golfdiktaturen zu fordern. Aber nein: Es ist weiterhin möglich, dass deutsche Rüstungsschmieden an diese Länder liefern, wenn diese Staaten zusichern, dass die Waffen im Land verbleiben. Für uns ist das, was Sie hier machen, inakzeptabel. Rüstungsexporte sind Beihilfe zum Mord. Wir fordern deswegen eine friedliche Außenpolitik, die fundamental dazu beiträgt, dass endlich friedliche Lösungen in der Region entstehen können.Danke.




	107. Dietmar Bartsch (DIE LINKE.) Herr Präsident! Meine Damen und Herren! Liebe Kolleginnen und Kollegen! Frau Merkel, Sie haben Ihre letzte Regierungserklärung im Juni 2017 abgegeben. Es ist also fast ein Dreivierteljahr her, dass Sie sich vor dem Parlament, vor dem Gremium, wohin es gehört, oder vor der Bevölkerung das letzte Mal erklärt haben. Ich weiß, es gab stressige Wochen – Wahlkämpfe, zweimal Sondierungen, Koalitionsverhandlungen und Weihnachten –, aber trotzdem hat es genug Anlässe gegeben, sich hier zu erklären. Alle Fraktionen hatten Sie auch dazu aufgefordert. Sie haben das nicht gemacht. Ich sage ganz klar: Das ist ein Ausdruck Ihrer Wertschätzung des Parlaments, ein Ausdruck des Verständnisses des Parlaments und auch ein Ausdruck der Haltung gegenüber der Bevölkerung.Frau Bundeskanzlerin, gerade haben Sie eine Erklärung zu Europa abgegeben. Wir haben gehört, dass es wieder vorangeht, dass wir europäische Antworten brauchen – Donnerwetter! –, dass wir uns auf den Erfolgen nicht ausruhen dürfen usw. usw. Das waren sehr viele Allgemeinplätze.Stellen wir einmal eine Frage: Wie hat sich Europa seit dem Jahre 2005 entwickelt, als Sie Kanzlerin geworden sind? Man kann einmal Bilanz ziehen. Im Jahre 2005 war die Osterweiterung gerade ein Jahr alt. Es gab ganz viele Hoffnungen auf ein solidarisches, ein soziales Europa. Es wurden Chancen gesehen, auch im Hinblick darauf, dass Europa nicht von Deutschland dominiert wird. Dann kam die Finanzkrise, und es wurde klar: Die Banken sind wichtiger als die Menschen. Wir müssen heute nach 13 Jahren Kanzlerschaft feststellen, dass Europa in einem schlechteren Zustand ist als im Jahre 2005. Das ist die Wahrheit. Der Brexit ist nur ein Ausdruck davon, meine Damen und Herren.Sie haben völlig zu Recht darauf hingewiesen: Europa war und ist zuallererst ein Friedensprojekt. Aber das ist in Gefahr, weil viele Menschen keine Perspektive auf eine gute Zukunft in Europa sehen, meine Damen und Herren.Frau Bundeskanzlerin, Sie haben auf die Wirtschaftsdaten verwiesen. Sagen Sie bitte aber auch die Wahrheit, dass in Europa 120 Millionen Menschen in Armut leben. Das ist fast jeder Vierte. Um ein paar Beispiele zu nennen: Die Überbelastung durch Wohnkosten in Europa nimmt immer mehr zu. 9 Prozent der Europäer haben nicht genügend Geld, um zu heizen, meine Damen und Herren. Über die Jugendarbeitslosigkeit ist geredet worden. Andrea Nahles hat die Zahlen genannt. In Griechenland sind es über 40 Prozent. In jeder Rede zu Europa höre ich von Ihnen etwas über Jugendarbeitslosigkeit. Aber es ist offensichtlich viel zu wenig getan worden. Hier muss entschlossen gehandelt werden.Man kann doch nicht einfach zusehen. Dass die jungen Leute in der Regel mies bezahlt werden, ist doch auch ein Problem.Auf der anderen Seite gibt es in allen europäischen Ländern auch obszönen Reichtum. Die Spaltung wird immer größer. Die Situation in Europa ist dramatisch. Deswegen nimmt auch die Akzeptanz für Europa mit jedem Jahr, in dem sich Banken an öffentlichem Eigentum bereichern und Staaten geplündert werden, weiter ab. Deshalb gibt es das Erstarken nationalistischer und rechtspopulistischer Parteien überall in Europa, und das ist eben eine dramatische Entwicklung.Da muss man eine Frage stellen: Haben Sie damit irgendetwas zu tun, oder haben nur andere dafür die Verantwortung? Ich sage ganz klar: Sie tragen für die europäische Entwicklung maßgebliche Verantwortung, –dafür, dass der Zustand Europas so desolat ist.Ich habe jetzt nur nebenbei gehört, dass die wegfallenden Sitze der Briten zum Teil aufgeteilt werden sollen. Wer ist denn auf die Schnapsidee gekommen?Wenn die Sitze wegfallen, dann müssen sie weg sein. Sie können doch nicht unter den anderen aufgeteilt werden. Wie wollen Sie das denn den Leuten erklären? Ich verstehe das überhaupt nicht.Der Kern dieser Entwicklung – der Grund dafür, dass Europa in diesem Zustand ist – ist Ihre Finanzpolitik, die Austeritätspolitik oder wie auch immer. Haushalte vor Menschen – das ist Ihre Herangehensweise, und es ist eine falsche Herangehensweise, meine Damen und Herren.Nicht nur Griechenland und andere Länder haben unter diesem Spardiktat gelitten. Es hat dramatische Folgen. Schauen Sie sich einige Daten an, zum Beispiel die Selbstmordrate in Griechenland! Das sind Folgen einer solchen Politik.Jetzt lese ich, der noch zu beschließende Koalitionsvertrag sehe hier eine Kursänderung vor. Es steht an sehr prominenter Stelle. Das ist ja alles schön; es ist auch sehr viel Lyrik. Papier ist ja geduldig, und wir werden sehen, was daraus wird. Aber eine Frage sei mir schon gestattet: Wie korrespondiert die Tatsache, dass Sie diesen Kurswechsel wollen, eigentlich damit, dass Sie sich so nachhaltig für Jens Weidmann als EZB-Chef einsetzen? Der steht nun wirklich für einen rigiden Sparkurs. Eins geht nur, finde ich – entweder, oder.Da muss sich die Bundesregierung dann schon mal entscheiden.Sie haben mit Ihrem Kurs das gesellschaftliche Klima in vielen Ländern Europas vergiftet. Das ist beim Brexit letztlich am deutlichsten zu sehen gewesen. Wir haben eine Generation, die Europa als ganz positiv betrachtet, für die Nationalismus überhaupt keine Option ist. Reisefreiheit, Kulturaustausch – das wird geschätzt. Diese Generation will nicht zurück in die dunklen Zeiten des Nationalismus. Diesen Fortschritt – und das ist ein Fortschritt – bringen Sie mit einer Politik deutscher Hegemonie in Gefahr, Frau Merkel. Das ist das Problem.Dort, wo es Hoffnungslosigkeit und Perspektivlosigkeit in Europa gibt, wo die Leute Angst vor einem Bürokratiemonster haben und wo Konzerne machen können, was sie wollen, da wächst Europaskepsis. Europa ist nicht zuerst Garant für die Freiheit der Konzerne und für Kapitalfreiheit – es ist mehr. Es muss um eine Sozial­union gehen – Ja zur Mindestlohnregulierung. Es reicht nicht, eine gemeinsame Währung zu haben, sondern es muss endlich etwas anderes geschehen.Es geht mir auch um die Tatsache, dass es die Panama Papers gibt, die Paradise Papers gibt und dass daraus de facto nichts folgt. Sie müssen sich mal angucken, was Apple eine unabhängige Kanzlei auf der Isle of Man gefragt hat: ob Gesetze eventuell zuungunsten von Apple geändert werden könnten, ob es eventuell eine Oppositionspartei geben könnte, die die Regierungsverantwortung übernimmt usw. Das alles ist an Dreistigkeit überhaupt nicht mehr zu überbieten. Und das Schlimme ist: Das ist alles legal.Da müssen die europäischen Staats- und Regierungschefs handeln und endlich einen Riegel vorschieben.Ich will ein Wort dazu sagen, dass Sie hier – völlig zu Recht – Assad und sein Agieren in Syrien kritisiert haben. Ich teile das. Aber es ist einfach ein Unding, dass Sie in dieser Situation kein Wort zu der Aggression der Türkei in Nordsyrien verlieren.Da wird ein völkerrechtswidriger Krieg geführt, und das erwähnen Sie hier nicht. Sie können doch nicht nur einseitig Stellung beziehen. Das gehört genauso dazu wie die Aggression, der Wahnsinn von Assad.Die 13 Jahre Ihrer Kanzlerschaft sind auch ein Nährboden für das, was europaweit abläuft, nämlich ein Kulturkampf, der von rechts geführt wird.– Lachen von ganz rechts. Ist es nicht schön? Danke für die Bestätigung.Meine Damen und Herren, ich will Folgendes noch kurz erwähnen: Im Koalitionsvertrag steht der richtige und gute Satz: „Wir verurteilen Rassismus und Diskriminierung in jeder Form.“ Ich finde das richtig. Aber bitte fangen Sie auch bei Ihren Partnern an. Herr Orban ist Mitglied Ihrer Parteienfamilie, er ist in Ihrer Fraktion im Europäischen Parlament, und was er so alles von sich gibt, das ist doch nicht zu akzeptieren. Er ist jemand, der Europa spaltet. Dazu muss man eine klare Haltung haben und Position beziehen.Mir sei zum Abschluss noch eine kurze Bemerkung an meine sozialdemokratischen Freunde gestattet. Liebe Andrea Nahles, die Rede eben hat demonstriert, dass Sie mental in der neuen Koalition schon angekommen sind. Ich weiß, es ist noch ein bisschen früh; man weiß nicht, wie die Abstimmung ausgeht.Ich will Ihre Haltung abgekürzt anhand zweier Tweets von Martin Schulz aus der Vergangenheit erläutern. Martin Schulz hat zu Emmanuel Macron getwittert: „Ich freue mich über das gute Ergebnis für @Emmanuel Macron. Um Europa zu reformieren, brauchen wir im September auch in Deutschland den Wechsel!“ Und zu Jeremy Corbyn hat er getwittert: „Was für eine Aufholjagd! Gratulation an @jeremycorbyn und @UKLabour!“ – Fällt Ihnen etwas auf? Fasst das das Dilemma nicht gut zusammen?Entweder liberaler Umbau oder soziale Wende, Sie müssen sich entscheiden. Aber bitte treffen Sie nicht die falsche Entscheidung. Wir brauchen eine neue Europapolitik.Das Aufgeschriebene kann ja ganz interessant sein, aber ich möchte endlich ein Handeln sehen.Wissen Sie, was ein schönes Motto ist? Sie werden es kennen. Es heißt „Ein neuer Aufbruch für Europa, eine neue Dynamik für Deutschland, ein neuer Zusammenhalt für unser Land“. Das ist die Überschrift Ihres Koalitionsvertrages. Ich hoffe, dass davon auch irgendetwas zustande kommt.Nach der Regierungserklärung und Ihrer Antwort fehlt mir ein wenig der Glaube.Herzlichen Dank.




	108. Volker Kauder (CDU) Herr Präsident! Liebe Kolleginnen, liebe Kollegen! Das informelle Treffen, das heute beginnt und morgen fortgesetzt wird, soll keine Entscheidungen bringen, aber es wird Weichen stellen für das, was in der nächsten Zeit in Europa wichtig wird. Es geht einmal um die finanziellen Rahmenbedingungen, und es geht zum anderen auch um die Vorbereitung der Wahlen für das nächste Europäische Parlament. In beiden Bereichen werden Entscheidungen vorbereitet, die für dieses Europa wichtig sind. Jetzt mag ja manche Diskussion über die Frage der Spitzenkandidaten und über die Anzahl an Kommissaren sehr kleinteilig daherkommen, aber damit werden Beispiele genannt, die die Menschen verstehen.Bevor wir über diese Dinge sprechen, halte ich es aufgrund von Äußerungen, die in diesem Parlament, die in dieser Debatte gefallen sind, für notwendig, zumindest eine Bedeutung Europas hervorzuheben. Es ist nicht so, dass sich die Menschen in Deutschland und in Europa grundsätzlich von Europa abgewandt hätten.Das hätten manche gerne, die es so formulieren. Die Menschen in unserem Land wissen ganz genau, dass Europa die größte Friedensversicherung ist, die es jemals auf diesem Kontinent gegeben hat, meine sehr verehrten Damen und Herren.Und ich freue mich sehr darüber, dass dies nicht nur eine ältere Generation sieht, der Kriegserfahrungen noch nah sind, sondern auch eine jüngere Generation. Wenn wir heute in den Nahen und Mittleren Osten schauen, dann wissen wir doch: Es ist eine unglaubliche Erfolgsgeschichte dieses Europas, dass wir seit mehr als 70 Jahren keinen Krieg in unserem Land gehabt haben, meine sehr verehrten Damen und Herren.Das ist etwas, das über dem steht, was die normalen Diskussionen über Europa beinhalten.Europa ist nach dem Zweiten Weltkrieg stark geworden durch Grundsätze. Es galt immer der Grundsatz der Solidarität und der Solidität. Das betrifft, sehr geehrte Frau Nahles, den zweiten Schritt, den Sie angesprochen haben, nämlich dass die Jugendarbeitslosigkeit bekämpft werden muss. Wir haben doch die Erfahrung gemacht, dass es überhaupt nichts bringt, immer noch mehr Geld in ein System einzuzahlen, das die Aufgabe gar nicht bewältigen kann. Deswegen ist es richtig, wenn auch bei diesem Treffen in Europa darüber gesprochen wird, dass nicht in erster Linie Geld ausgeben zum Erfolg führt, sondern die notwendigen Reformen, die die Wettbewerbsfähigkeit voranbringen.Daran muss auch festgehalten werden. Geld nützt relativ wenig, wenn die Wirtschaft in einem europäischen Mitgliedsland nicht wettbewerbsfähig ist; denn dann kann es auch keine Arbeitsplätze und keine Ausbildungsplätze geben. In einer solchen Situation ist die Mindestlohndebatte wenig zielführend. Wir müssen eine Debatte über die Reformen führen.Ein weiterer Punkt: Die Bundeskanzlerin hat angesprochen, dass der Weg der Bankenunion weitergegangen werden soll. Damit sind wir einverstanden.Der Deutsche Bundestag hat im Jahr 2015 in einer Entschließung, in der entsprechende Grundsätze formuliert sind, beschlossen, wie dies gelingen kann. Zunächst einmal müssen die Risiken verringert werden. Auf dem Weg sind wir, aber wir sind noch nicht am Ende angelangt. Es darf nicht nach dem Motto gehen: Die Risiken haben wir zwar noch nicht sehr verringert, aber die Bankenunion muss in diesem Sommer vollendet sein. – Da werden wir von der CDU/CSU-Bundestagsfraktion nicht mitmachen.Wir haben in der von der Großen Koalition gemeinsam beschlossenen Resolution auch formuliert, dass eine Haftungsgemeinschaft bei der Einlagensicherung nicht der richtige Weg ist. Wir verlangen, dass zunächst einmal die anderen Länder das tun, was wir in Deutschland bereits gemacht haben, nämlich ein Einlagensicherungsgesetz auf den Weg zu bringen; das ist in vielen Ländern noch nicht der Fall. Da nützen alle Diskussionen nichts; zunächst müssen diese Reformmaßnahmen durchgeführt werden. Erst wenn alle europäischen Länder die Einlagensicherungsrichtlinie umgesetzt haben, können wir uns über den Weg zu europäischen Lösungen unterhalten.Meine sehr verehrten Damen und Herren, wenn wir über diese schwierigen Sachverhalte in Europa reden, ist es notwendig, dass wir bei der Wahrheit bleiben. Es ist gesagt worden, wir hätten im Koalitionsvertrag das Königsrecht des Parlaments, nämlich die Haushaltsgestaltung, auf dem Altar in Brüssel geopfert. So werden aus einer bestimmten Richtung immer Halbwahrheiten, in dem Fall sogar Unwahrheiten in die Welt gesetzt.Deswegen will ich einmal sagen, was im Koalitionsvertrag steht. Dort steht ausdrücklich, dass die Rechte des Parlaments und des Haushaltsausschusses, was den Haushalt betrifft, gewahrt bleiben. Das ist etwas anderes als das, was Sie, Frau Weidel, in diesem Plenum gesagt haben.Die Wahrheit sieht anders aus, als Sie es hier berichtet haben. Das werden wir Ihnen nicht durchgehen lassen, dass Sie immer wieder mit falschen Argumenten versuchen, etwas zu begründen, was so nicht zu begründen ist.Meine sehr verehrten Damen und Herren, wir haben – die Bundeskanzlerin hat es gesagt – auch in finanzieller Hinsicht in Europa darauf zu achten, welche Aufgaben zu leisten sind. Der Schutz der Außengrenzen ist ein zentrales Thema. Dafür müssen wir auch die notwendigen Mittel zur Verfügung stellen. Wer will, dass Europa offen bleibt und dass wir nicht zu den alten Grenzen in Europa zurückkommen, die Europa beschädigen würden, der muss einen Beitrag dazu leisten, dass die Außengrenzen gesichert werden.Das gehört zusammen, und dafür müssen wir Geld zur Verfügung stellen.Für die großen Aufgaben ist Europa zuständig. Dazu gehört der Schutz der Außengrenzen. Dazu gehört aber auch, dass wir in Europa in der Reaktion auf große Herausforderungen unsere Sprachlosigkeit überwinden. Die Bundeskanzlerin hat die Situation im Nahen Osten, in Syrien, angesprochen.Wir haben immer gehofft, dass es kein zweites Aleppo in Syrien gibt. Was sich im Augenblick in Ghuta abspielt, kommt einem zweiten Aleppo sehr nahe.Denen, die meinen, wir müssen uns um eine bessere Beziehung zu Russland kümmern, kann ich nur sagen: Damit bin ich einverstanden. Wenn ich aber sehe, welchen Beitrag Russland dazu leistet, dass aus Ghuta ein zweites Aleppo wird, muss ich sagen: Auch das gehört in die Gespräche. Es kann nicht nach dem Motto gehen: Um gute Beziehungen zu Russland zu entwickeln, sagen wir nicht mehr die Wahrheit über die Völkerrechtsverletzungen, die dieses Land begeht.Das geht auf keinen Fall. Dazu muss Europa eine gemeinsame Position finden.Ich bin sehr besorgt darüber, dass auch in Syrien zwei NATO-Mitglieder, nämlich die USA und die Türkei, aneinandergeraten. Auch darüber muss gesprochen werden, wenn man sich jetzt in Europa trifft. Es kann nicht so weitergehen, dass ein NATO-Mitglied ein anderes provoziert. Auch das, was die Türkei in Syrien macht, findet meine Zustimmung nicht.Auf beide Positionen muss geantwortet werden. Das wünsche ich mir, und das erwarten die Bürgerinnen und Bürger; sie erwarten nicht, dass Europa ständig attackiert und schlechtgemacht wird,– sie erwarten, dass Europa die großen Aufgaben löst, die ein Nationalstaat nicht lösen kann. Daher muss Europa auch bei diesem Treffen zeigen, dass es diese Größe besitzt.




	109. Artur Auernhammer (CSU) Herr Präsident! Meine sehr verehrten Damen und Herren! Viele Reden hier im deutschen Haus sind nicht umsonst.Ob es kostspielig wird für unsere Nation, das zeigt sich immer hinterher.Meine sehr verehrten Damen und Herren, Diskussionen um die Landwirtschaft hier im Bundestag sind immer von starken Emotionen begleitet. Das war in der letzten Legislaturperiode so, und das ist auch in dieser Legislaturperiode so, merke ich. Im Deutschen Bundestag wurde oft über die Rolle des Einsatzes von Pflanzenschutzmitteln diskutiert. Mit Ihrer Erlaubnis, Herr Präsident, zitiere ich den Kollegen Ebner aus dem Protokoll der Plenarsitzung vom 23. März 2017:Solange es keine Alternativen gibt, fordern wir die Bundesregierung auf, sich bei der EU konsequent für die Prüfung einer zeitlich und mengenmäßig begrenzten Zulassung von Kaliumphosphonat im Ökoweinbau einzusetzen.Jetzt muss man natürlich wissen: Dieses Präparat ist ein Pflanzenschutzmittel, in Ihrer Sprechweise eigentlich ein Pestizid. Meine sehr verehrten Damen und Herren, lassen Sie uns etwas mehr abrüsten und hier im Deutschen Bundestag näher an der Sache orientiert diskutieren.So muss es auch in unser alle Interesse sein, dass wir die Leistungen der deutschen Landwirtschaft sehen. Die deutsche Landwirtschaft diskutiert längst nicht mehr die Frage: Soll ich konventionell oder ökologisch produzieren? Diese Entscheidung trifft jeder Betriebsleiter für sich, für seine persönlichen Rahmenbedingungen, das geht nicht über politische Diskussionen.Lassen Sie uns darüber diskutieren, wie wir die Rahmenbedingungen der gesamten Landwirtschaft verbessern. Dazu gehört auch der Einsatz von Pflanzenschutzmitteln. Die deutsche Landwirtschaft und die Weltlandwirtschaft haben eine große Herausforderung vor sich. Die Weltbevölkerung hat sich seit 1950 verdreifacht, und sie wird sich weiterhin rasant entwickeln. Die landwirtschaftliche Nutzfläche auf der Welt wird ständig kleiner. Es ist unsere Aufgabe, die Landwirtschaft mit modernen Methoden, mit modernen Technologien in die Lage zu versetzen, die Weltbevölkerung zu ernähren, und das auch im Einklang mit der Umwelt.Gerade wenn es um die Zukunftsperspektiven der Landwirtschaft – wir reden hier in erster Linie von der deutschen Landwirtschaft – geht, müssen wir immer auch im Fokus haben: Wie stehen wir im internationalen Vergleich da? Welche Auflagen machen wir unserer deutschen Landwirtschaft im internationalen Vergleich, und welche Rahmenbedingungen setzen wir ihr? Da hilft es wenig, wenn wir hier im Deutschen Bundestag einseitig gegen eine konventionelle Landwirtschaft wettern. Vielmehr hilft es, wenn wir gemeinsam um die beste Lösung ringen und hier diskutieren.Meine sehr verehrten Damen und Herren, eine große Herausforderung für unsere Landwirtschaft ist das gesamte Thema der Digitalisierung. Die Digitalisierung ist eine Herausforderung und eine Chance. Sie wird uns auch in der Pflanzenproduktion, in der Tierproduktion ermöglichen, effektiver, effizienter, moderner, ziel­orientierter und vor allem auch umweltfreundlicher zu produzieren. Wir haben hier schon viel über das Thema Glyphosat diskutiert. Warum ist Glyphosat so in Verruf gekommen? Weil es flächenweise eingesetzt worden ist. Heute gibt es moderne Technologien, bei denen mir der Satellit sagt, an welcher Stelle auf dem Feld zum Beispiel die sogenannte Gemeine Quecke steht. So kann ich mit dem Pflanzenschutzgerät hinfahren und muss die Pflanze nur so bekämpfen, wie es eigentlich notwendig ist. Das ist die große Chance, die wir durch die Digitalisierung in der Landwirtschaft haben. Dieses gemeinsam auf den Weg zu bringen, dieses gemeinsam hier im Deutschen Bundestag zu begleiten, ist, glaube ich, unser aller Aufgabe.Ich bedanke mich für die Aufmerksamkeit.




	110. Patrick Schnieder (CDU) Herr Präsident! Meine Damen! Meine Herren! Liebe Kolleginnen! Liebe Kollegen! Der Gesetzentwurf, den wir heute hier präsentiert bekommen, und auch der Antrag, den die Fraktion Bündnis 90/Die Grünen vorgelegt hat, sind nicht neu. Beide hat es fast unverändert in dieser Form schon gegeben. Beide sind nahezu unverändert kopiert. Der Gesetzentwurf ist quasi nach der Vorlage einer Lobbyorganisation hier eingebracht worden. Den Antrag von Bündnis 90/Die Grünen gab es schon in der 18. Wahlperiode. Das ist ja nun nicht schlimm, aber das sollte man der Transparenz halber deutlich sagen.Das Problem ist, dass die Punkte, die an dem Gesetzentwurf und dem Antrag in der Anhörung, die in der letzten Wahlperiode stattgefunden hat, als problematisch festgestellt worden sind, in keiner Weise ausgeräumt worden sind.Die Auswüchse, die hier an die Wand gemalt werden, wird es im Einzelfall geben; das will ich gar nicht in Abrede stellen. Es gibt auch schwarze Schafe in diesem Geschäft. Auch das will ich nicht in Abrede stellen. Aber ich glaube, dass wir mit einer Diskreditierung zulässiger Interessenausübung das Problem überhaupt nicht in den Griff bekommen. Ganz im Gegenteil: Wir beschränken unsere Möglichkeiten als Abgeordnete, die ein freies Mandat ausüben und die selbst entscheiden können, welche Interessen sie hören, wie sie sie abwägen und wie sie sie im Einzelnen bewerten. Deshalb kann ich mich dem Zerrbild, das zum Teil gezeichnet wird, überhaupt nicht anschließen. Dass wir in diesem Bereich der Interessenvertretung nur mit Klüngelei, mit Mauschelei, mit Korruption zu tun hätten, das ist mir eine Nummer zu hoch. Ich glaube, dass wir hier in einer neuen Art von Sachlichkeit über dieses Thema diskutieren sollten.Wir sollten die Frage des freien Mandates, das wir ausüben, in den Mittelpunkt rücken. Wir sollten Vertrauen in die parlamentarischen Abläufe schaffen. Ich stelle einfach fest: Wir haben kein Transparenzdefizit im Deutschen Bundestag. Alles ist verfolgbar, alles ist nachlesbar, alles ist nachvollziehbar. Selbst Transparency International hat festgestellt: Die Transparenz des Deutschen Bundestages kann als sehr hoch eingestuft werden. Insofern gibt weder der Gesetzentwurf noch der Antrag eine wirkliche Antwort auf die zugrundeliegende Problematik.Eine Antwort, die gegeben wird, ist, dass wir eine neue Behörde schaffen sollen, dass wir mehr Bürokratie einführen sollen. Nun habe ich nicht den Eindruck, dass wir in Deutschland schon zu wenig Bürokratie hätten. Ich schätze die Arbeit der Beamten, der Angestellten im öffentlichen Dienst sehr, aber es ist mir schon befremdlich, dass wir hier Registrierungspflichten aufstellen sollen, um überhaupt mit einem Abgeordneten Gespräche führen zu dürfen.Damit sind wir beim ersten Kernproblem dieser Vorgehensweise, nämlich der Frage: Beschneiden wir damit nicht das freie Mandat, die freie Mandatsausübung von uns Abgeordneten? Wir sind gewählt, um für die Bürgerinnen und Bürger Interessen abzuwägen, und wir entscheiden selbst, mit wem wir reden. Vor diesem Hintergrund stelle ich mir folgende Fragen: Wie soll denn so etwas praktikabel sein? Sollen wir nachprüfen, bevor wir mit jemandem reden, ob er auf einer solchen Liste eingetragen ist? Wie ist es, wenn wir selbst einen Gesprächswunsch haben? Darf ich nur mit denen reden, die auf einer solchen Liste vermerkt sind? Wie wollen wir das abgrenzen? Wer ist ein solcher Interessenvertreter oder Lobbyist, wer ist keiner? Ich kann nur sagen: Die größten Lobbyisten sind die, die bei mir im Wahlkreis sitzen, die Bürgerinnen und Bürger, die ihre Anliegen vortragen, ihre Interessen vortragen. Wie will ich das im Einzelfall abgrenzen? Ich halte das für hochproblematisch.Auf eine weitere Kernfrage geben Sie keine Antwort – mein Vorredner hat ihn kurz gestreift –, nämlich ob Grundrechte von solchen Maßnahmen betroffen sind. Das Problem ist doch – das wurde in der Anhörung in der letzten Wahlperiode eindeutig festgestellt – die verfassungsrechtskonforme Ausgestaltung eines solchen Lobbyregisters. Interessenvertretung kann sich immer auf Grundrechte stützen. Ich erlaube mir die Anmerkung – bei aller Wertschätzung, bei allem Gewicht, die Transparenz und Öffentlichkeit für uns und unsere Arbeit haben –: Transparenz steht im Grundgesetz jedenfalls nicht direkt hinter der Menschenwürde, sondern da stehen die Grundrechte. Das können Interessenvertreter immer geltend machen, je nach Ausprägung verschiedener Grundrechte.Immer betroffen ist zum Beispiel Artikel 12, Berufsausübung, freie Berufswahl, aber auch das Recht auf informationelle Selbstbestimmung, der Schutz von Betriebs- und Geschäftsgeheimnissen und nicht zuletzt auch Artikel 9 Absatz 3 des Grundgesetzes, die Koalitionsfreiheit.Danach steht es Unternehmen, aber auch Arbeitnehmervertretern und Gewerkschaften grundrechtlich garantiert frei, ihre Interessen frei und ungehindert vorzutragen. Das auszugestalten, sind Sie schuldig geblieben. Ich sehe auch nicht, wie man das hier adäquat lösen kann.Deshalb will ich zusammenfassen: Das Ziel, das Sie mit diesem Gesetzentwurf verfolgen, können Sie mit der Ausgestaltung des Lobbyregisters gar nicht erreichen; denn es kann die Kontaktaufnahme zu Abgeordneten letztlich nicht unterbinden. Ein Register kann auch nicht Auskunft darüber geben, ob es unzulässige oder rechtswidrige Beeinflussung gegeben hat. Das ist übrigens strafrechtlich geregelt. § 108e Strafgesetzbuch haben wir in der letzten Wahlperiode eingeführt. Auch hier haben wir gesetzliche Vorschriften, die das regeln. Das können wir nicht über ein Lobbyregister angehen und brauchen es auch nicht.Angesichts dieser durchgreifenden verfassungsrechtlichen Bedenken, vor allem aber auch wegen der bürokratischen Aufblähung sollten wir von dieser Art und Weise eines Lobbyregisters unbedingt Abstand nehmen. Ich werbe dafür, dass wir uns selbst vertrauen und uns selbst zutrauen, dass wir die Interessen, die an uns heran­getragen werden, sorgsam, sorgfältig und sachgerecht abwägen können. Es geht darum, das freie Mandat zu stärken und nicht schwach zu machen.Vielen Dank.




	111. Marc Biadacz (CDU) Sehr geehrter Herr Präsident! Liebe Kolleginnen und Kollegen! Meine Damen und Herren auf der Tribüne! Die soziale Marktwirtschaft ist seit 70 Jahren ein Garant für wirtschaftliche und soziale Stabilität in Deutschland. Warum? Sie stellt Menschen in den Mittelpunkt des wirtschaftlichen und gesellschaftlichen Handelns. An diesem bewährten Kompass hat sich die Arbeitsmarktpolitik zu orientieren. Wie gut dieser Kompass funktioniert, sehen Sie an den Erfolgen der CDU-geführten Bundesregierung.Wir haben die niedrigste Arbeitslosenquote seit der deutschen Wiedervereinigung. Bei mir im Wahlkreis Böblingen haben wir Vollbeschäftigung.Auch in der heutigen Debatte über sachgrundlos befristete Arbeitsverträge stehen Menschen im Mittelpunkt. Wir haben im Koalitionsvertrag verabredet, Missbräuche bei den Befristungen abzuschaffen, beispielsweise Kettenbefristungen. Wir dürfen aber umgekehrt nicht so tun, als ob sachgrundlos befristete Arbeitsverträge ein Massenphänomen wären.Wir haben derzeit 44 Millionen Erwerbstätige in Deutschland, davon 1,3 Millionen mit sachgrundlos befristeten Arbeitsverträgen.Das sind gerade einmal 2,7 Prozent. Von daher ist Ihr Antrag, liebe Kolleginnen und Kollegen der Linken, scheinheilig.Sie haben es in der letzten Wahlperiode geschafft, über dieses Thema sieben Mal hier im Plenum zu diskutieren. Damit geben Sie dem Thema einen Stellenwert, den es nicht hat. Sie tun so, als ob es unter den Arbeitgebern ausschließlich schwarze Schafe gäbe. In Ihrem Antrag sprechen Sie sogar davon, dass Unternehmen eine Machtposition hätten. Wer durch eine solch ideologische Brille die Welt betrachtet, wer sie in Gut und Böse, wer sie in Schwarz und Weiß unterteilt, der macht sich die Welt ziemlich einfach.Die Welt ist komplexer. Auch unsere moderne Arbeitswelt ist komplexer. Das wird hier jeder bestätigen können, der, so wie ich, selbst einmal in der freien Wirtschaft tätig war. Auch ich hatte einen sachgrundlos befristeten Arbeitsvertrag. Ich kann daher aus eigener Erfahrung sagen: Befristete Arbeitsverträge sind kein Teufelswerk, sondern ein wichtiges arbeitsmarktpolitisches Instrument, ein Instrument, das den Menschen Chancen eröffnet, in ein unbefristetes Arbeitsverhältnis zu kommen. Lassen Sie uns den Menschen weiterhin diese Chancen geben!Hier sind wir wieder bei unserem Kompass der sozialen Marktwirtschaft, also dem Anliegen, möglichst vielen Menschen berufliche und soziale Teilhabe zu ermöglichen. Denn letztlich ist es besser, befristet in Arbeit zu kommen, als unbefristet arbeitslos zu bleiben.Befristete Arbeitsverträge sichern Unternehmen eine gewisse Flexibilität, um wettbewerbsfähig zu bleiben, keine grenzenlose Flexibilität, sondern eine mit klaren Leitlinien. Wir haben dafür wichtige Elemente in den Koalitionsvertrag eingearbeitet. Was wären die Folgen, wenn wir, wie Sie es fordern, sachgrundlos befristete Arbeitsverträge abschaffen würden? Unternehmen würden Stellen abbauen, es gäbe mehr Zeitarbeit, und Arbeitsplätze würden ins Ausland abwandern.Am Ende wundert sich Die Linke, dass die Arbeitslosenquote steigt.Lassen Sie uns die wirklich drängenden Probleme auf dem Arbeitsmarkt lösen, anstatt zum x‑ten Mal hier im Parlament über Ihren Antrag zu diskutieren. Schauen wir auf Langzeitarbeitslose, auf Bildungsabbrecher, auf den demografischen Wandel, auf den digitalen Wandel. Davon sind weitaus mehr Menschen in Deutschland betroffen als von den sachgrundlosen Befristungen, über die wir heute sprechen.Die CDU/CSU-Bundestagsfraktion wird diese Wahlperiode dafür nutzen, den Blick auf diese zentralen Themen zu richten und den Menschen in den Mittelpunkt unserer Arbeit zu stellen, ganz im Sinne der sozialen Marktwirtschaft.Lassen Sie uns gemeinsam ohne Ideologie an die Arbeit gehen.Vielen Dank.




	112. Andrea Nahles (SPD) Herr Präsident! Liebe Kolleginnen und Kollegen! Es ist erfreulich, dass nach Jahren der Krise das Wachstum in der Europäischen Union wieder Fahrt aufnimmt, die Beschäftigung steigt und auch der Zuspruch der Menschen zur EU in den Mitgliedstaaten wieder wächst. Allerdings dürfen wir uns nichts vormachen: Das Grundvertrauen in die Europäische Union als Garant für Wohlstand, für Sicherheit und auch für Wachstum ist noch lange nicht wiederhergestellt, und es wird auch nicht von alleine wiederhergestellt. Wir müssen mehr tun, um den Zusammenhalt und das Vertrauen in Europa zu stärken, und das auf verschiedenen Ebenen.Die Bundeskanzlerin hat gerade etwas zu der Sicherung der Außengrenzen gesagt. Ich finde, das war sehr richtig, insbesondere was die personelle Verstärkung angeht. Dieser Aspekt der äußeren Sicherheit muss aber von einem wichtigen zweiten Aspekt begleitet werden. Wir müssen uns auch um die Ungleichheit der Lebensverhältnisse in Europa kümmern. Wir haben damals, als Trump in den USA zum Präsidenten gewählt wurde, viele Analysen über die Gründe gehört. Darin wurde insbesondere als einer der Gründe genannt, dass die Ungleichheit der Lebensverhältnisse im Rust Belt zu Unzufriedenheit und Gewalt geführt hat.Die Situation in Europa ist ungleich schlimmer. Denn die Ungleichheit der Lebensverhältnisse in Europa ist weitaus größer als in den USA. Ich möchte das an einer Zahl deutlich machen. Das Bruttoinlandsprodukt pro Kopf beträgt in Bulgarien 5 500 Euro im Jahr; in Luxemburg beträgt es 81 000 Euro. Oder nehmen wir die Arbeitslosigkeit: Die Arbeitslosenquote reicht von 2,3 Prozent in der tschechischen Hauptstadt bis 20,7 Prozent in Griechenland. Ich glaube, dass es diese Ungleichheiten sind, die den Zusammenhalt in Europa immer wieder neu gefährden. Deswegen muss es eine wichtige Priorität für uns sein, dass wir wieder näher zusammenkommen und diese Ungleichheit überwinden.Das ist auch im ureigensten Interesse Deutschlands und übrigens auch im ureigensten Interesse der deutschen Arbeitnehmerinnen und Arbeitnehmer. Denn eines möchte ich ganz klar sagen: Wenn Armutswanderung in vielen Städten in Deutschland ein großes Thema ist und wir es mit Armutsmigration und Lohndumping zu tun haben, dann ist es auch in unserem eigenen Interesse, etwas dafür zu tun, dass es anständige Löhne und anständige soziale Bedingungen in allen anderen europäischen Ländern gibt.Ich denke, dass dabei ein besonderes Augenmerk auf die Jugendarbeitslosigkeit gelegt werden muss, und sie wird auch auf der Tagesordnung des bevorstehenden Treffens stehen. Wir können nämlich noch immer kein Signal der Entspannung heute aus diesem Bereich senden. Wir haben in Griechenland eine Jugendarbeitslosigkeit von 40,8 Prozent, in Spanien von 36,8 Prozent und in Italien von 32,2 Prozent. Und wir müssen feststellen: Wir haben zwar in den letzten Jahren hier investiert, aber es hat Jahre gebraucht, bis das europäische Geld auch tatsächlich bei den jungen Leuten in den jeweiligen Ländern angekommen ist. Ich fordere nicht nur eine Verstetigung der Mittel. Vielmehr brauchen wir in der Europäischen Union bessere Verfahren, um schneller auf Krisen reagieren zu können. Auch das muss Thema sein, wenn nun die Staats- und Regierungschefs sich treffen und über den zukünftigen Finanzrahmen reden. Da geht es auch darum, wie wir in der Europäischen Union auf Krisen besser reagieren können. Hier gibt es einiges zu tun.Um den sozialen Zusammenhalt zu stärken, brauchen wir aber auch faire Regeln. Wir wollen kein Lohndumping. Wir brauchen einen Rahmen für Mindestlöhne. 235 Euro im Monat beträgt zurzeit der Mindestlohn in Bulgarien, während er in Luxemburg bei 2 000 Euro im Monat liegt. Ich bin weit davon entfernt, entspannt zu sein, wenn ich mir die sozialen Sicherungssysteme in der Europäischen Union anschaue. Es gibt viele Länder, die keine Sozialhilfe wie wir bzw. keine Existenzsicherung haben. Die Situation ist in den letzten Jahren nicht besser, sondern schlechter geworden. Wir wollen nicht, dass die Europäische Union Sozialhilfe zahlt; das müssen die Länder schon selbst machen. Aber wir müssen im europäischen Recht verankern, dass jeder Mitgliedstaat ein funktionierendes Sozialrecht auf Existenzsicherung schaffen muss und dass dieses Recht regelmäßig angepasst wird; denn nur so können wir Armutsmigration und Arbeitsmarktzuwanderung verhindern.Es ist außerdem sehr sinnvoll, den Aufbau von Mindestlohnsystemen zu organisieren. Es geht nicht darum, dass wir einen einheitlichen Mindestlohn vorgeben, sondern um einen Rechtsrahmen, innerhalb dessen sich die Mindestlöhne bewegen.Wir haben zudem für faire Regeln zu sorgen, wenn es um das Thema Steuerdumping geht. Wir müssen dem Steuerdumping endlich die Grundlage entziehen. In Deutschland und im restlichen Europa hat kein Mensch Verständnis dafür, dass Brüssel alles Mögliche regelt, dass aber in dieser Frage seit Jahren nichts erreicht wird. Deutschland muss hier wesentlich mehr Druck machen, als das in der Vergangenheit der Fall gewesen ist.Im Übrigen setzen wir uns für eine Sitzverlagerungsrichtlinie ein; denn es ist genauso wenig akzeptabel, dass die Verlagerung des Unternehmenssitzes dazu führt, dass Mitbestimmungsregeln unterlaufen werden, wie es gerade in Deutschland bei der Fusion von Thyssen und Tata passiert, um ein konkretes Beispiel zu nennen. Dem muss ein Riegel vorgeschoben werden.Bei der Steuerpolitik in Europa geht es aber auch um ein gemeinschaftliches Wettbewerbsmodell, das auf der Basis der sozialen Marktwirtschaft fußt. Unser Rechtsstaat kann aus meiner Sicht bestimmte Praktiken von Staatskonzernen oder Monopolisten – ob sie nun aus China oder aus Kalifornien kommen – nicht akzeptieren. Das wichtigste Bollwerk gegen solche Praktiken sind die Europäische Union und eine gemeinsame, abgestimmte Steuerpolitik auf der europäischen Ebene. Wir brauchen eine Stärkung des Zusammenhalts, um unser Modell der sozialen Marktwirtschaft in Europa gegen permanente Angriffe zu verteidigen und unseren Sozialstaat vor der Unterminierung seiner Grundlagen zu schützen.Wenn wir das alles zusammenfassen, kommen wir zu dem Schluss: Wir brauchen eine klare Neuausrichtung der Europapolitik. Das hat uns geleitet, als wir den Koalitionsvertrag zwischen CDU, CSU und SPD geschlossen haben. Wenn die Mitglieder der SPD und der Parteitag der Union zustimmen, dann wird das ein starkes Signal an unsere europäischen Partner und insbesondere an unsere französischen Freundinnen und Freunde sein. Denn im Rahmen dieses Koalitionsvertrages ist nichts weniger als eine neue Europapolitik verabredet worden: ein Investitionshaushalt und kein Sparhaushalt, ein entschlossenes Vorgehen gegen Steuerdumping, ein entschlossenes Ja zu sozialen Mindeststandards und Grundsicherungssystemen sowie ein klares Bekenntnis zur Sicherung der Mitbestimmung. Das alles zusammen leitet eine neue Europapolitik ein. Das ist wichtig und ein zentrales Signal, das wir dem nun bevorstehenden Rat senden.Ich möchte noch auf einen anderen Punkt eingehen, der ebenfalls Gegenstand der morgigen Beratungen in Brüssel sein wird. Wir brauchen eine Stärkung der demokratischen Legitimation und der Institution. Dabei gibt es aus meiner Sicht wertvolle Impulse aus dem Europäischen Parlament. Wichtig ist das zur Diskussion stehende Spitzenkandidatenprinzip, wonach sich nur dann jemand um das Amt des EU-Präsidenten bewerben kann, wenn er zuvor Spitzenkandidat in der europäischen Parteienfamilie war. Das war bei der letzten Europawahl noch umstritten. Ich freue mich, dass das mittlerweile von vielen anderen, auch vom konservativen Teil der europäischen Parteienfamilie, so gesehen wird.Nach dem Brexit geht es jetzt auch um die Frage einer neuen Bewertung der freiwerdenden Mandate. Ich persönlich hätte mir gewünscht, dass es zu transnationalen Listen kommt. Bedauerlicherweise hat die EVP-Fraktion dies für 2019 unterbunden, indem sie ihre Zustimmung dazu verweigert hat. Ich glaube aber, dass die Zukunft Europas auch darin liegt, dass es gemeinsame europäische Listen mit Kandidaten aus verschiedenen Mitgliedstaaten gibt und dass das zu einer Stärkung der demokratischen Legitimation des EU-Parlaments und der europäischen Parteienfamilie gehört.Ich bin sicher, dass wir das in Zukunft erreichen müssen und auch erreichen werden.Es ist klar: Es geht wirklich darum, dass wir uns für die nächsten Jahre auf der europäischen Ebene etwas vornehmen. Das Vertrauen muss wieder wachsen. Das bedeutet auch, dass man auf Herz und Nieren prüft, wo der Zusammenhalt gestärkt werden kann, wo sich Institutionen weiterentwickeln müssen und wo wir schlicht und ergreifend eine neue Politik brauchen. Ich bin zuversichtlich, dass es dafür mit dem neuen Koalitionsvertrag eine gute Grundlage gibt. Ich freue mich auf die Umsetzung dieser politischen Vereinbarung.Vielen Dank.




	113. Niema Movassat (DIE LINKE.) Herr Präsident! Meine Damen und Herren! Letzte Woche war ja politischer Aschermittwoch. Überall konnte man Bilder sehen, auf denen insbesondere CSU-Politiker genüsslich Bier tranken. Sie von der CSU zelebrieren also öffentlich das Alkoholtrinken, obwohl in Deutschland 74 000 Menschen im Jahr an den Folgen des Alkoholkonsums sterben, erzählen uns hier aber, Cannabis gehöre verboten. Dabei ist in Deutschland kein einziger Fall belegt, bei dem Menschen durch Cannabis gestorben sind. Es ist wirklich absurd, mit welchen ideologischen Scheuklappen Sie Drogenpolitik betreiben.Wir Linke haben heute einen Antrag vorgelegt, der erste Schritte beinhaltet. Wir wollen den Konsum von Cannabis entkriminalisieren und eine kontrollierte Abgabe ermöglichen. Wir sagen: Es muss endlich Schluss damit sein, dass Menschen, nur weil sie ein bisschen Cannabis besitzen, bestraft werden.Ich möchte gerne drei Argumente für unseren Antrag nennen:Erstens. Erwachsene Menschen dürfen sich selbst schaden. Das ist laut Grundgesetz ihre persönliche Freiheit.Deshalb darf man Alkohol trinken, rauchen und auch am Straßenverkehr teilnehmen, obwohl all das gefährlich ist. Wer mal einen Joint raucht, schadet niemandem Dritten; allenfalls schadet er sich selbst. 122 Strafrechtsprofessorinnen und -professoren haben schon vor fünf Jahren ähnlich argumentiert. Sie sprachen von einer Einschränkung der Bürgerrechte durch die jetzige Drogenpolitik. Was wir heute im Umgang mit Cannabiskonsumenten erleben, ist eine Entmündigung erwachsener Menschen, selbstbestimmt über ihr Leben zu entscheiden.Herr Kollege, lassen Sie eine Zwischenfrage von dem Kollegen von der AfD-Fraktion zu?Nein. – Zweites Argument. Die bisherige Cannabispolitik ist eine unfassbare Verschwendung der Ressourcen von Polizei und Justiz. Millionenbeträge werden versenkt, um ein paar Kiffer zu verfolgen. Laut Kriminalstatistik haben von den insgesamt 300 000 erfassten Rauschgiftdelikten 183 000 einen Cannabisbezug.Herr Kollege, es gibt den nächsten Wunsch nach einer Zwischenfrage, diesmal vom Kollegen der CDU/CSU-Fraktion.Ja, gerne.Herr Kollege, zum Thema Kosten. Wir wissen, dass, bedingt durch Cannabiskonsum, schon jetzt fast 1 Milliarde Euro für die Gesundheits- und Sozialsysteme aufgewandt werden muss. Ist Ihnen bewusst, dass diese Kosten von allen Bürgern zu tragen sind und nicht nur von denen, die Cannabis konsumieren?Herr Kollege, es ist natürlich richtig, dass Menschen, die nach bestimmten Drogen süchtig sind, Kosten im Gesundheitssystem verursachen.Das ist aber auch bei Alkohol und Tabak so. Dieses Problem löst man in diesem Fall nicht durch Verbote, sondern durch Gesundheitsangebote. Der Schwarzmarkt schafft da noch viel mehr Probleme, als wenn man das entkriminalisieren würde.Man muss wirklich sagen: Es ist ja eine „tolle“ Leistung, wenn man ein paar Menschen ihr Gras wegnimmt und ein paar Kiffer kriminalisiert. Aber das nützt doch niemandem in diesem Land. Das hat selbst der Bund Deutscher Kriminalbeamter gesagt, der jüngst forderte, endlich ein Ende der Kriminalisierung herbeizuführen. Es ist richtig: Polizei und Justiz haben Wichtigeres zu tun, als ein paar Cannabiskonsumenten zu verfolgen.Ich will Ihnen ein drittes Argument nennen. Die bisherige Verbotspolitik ist gesundheitsschädlich; denn durch die Kriminalisierung gibt der Staat die Kontrolle und die Regulierung des Cannabismarktes komplett auf. Erst die Kriminalisierung schafft einen dubiosen Schwarzmarkt, und dort werden dann verunreinigte oder gar tödliche Substanzen in Umlauf gebracht. In den Niederlanden, wo Cannabis geduldet ist, haben wir solche Probleme nicht. Deshalb ist ein legaler Zugang zu Cannabis ein echter Beitrag zum Gesundheitsschutz.Leider ignorieren insbesondere Sie von der Union alle fachlichen Argumente – das erinnert in seiner fachlichen Falschheit schon fast an die Leugnung des Klimawandels von Donald Trump –, die für eine Entkriminalisierung von Cannabis sprechen.An der Spitze dabei ist die Drogenbeauftragte Marlene Mortler, die ja immer wieder neue peinliche Äußerungen von sich gibt. Es wäre wirklich an der Zeit, sie endlich zu entlassen.Im Wahlkampf hat Martin Schulz gesagt, bei Cannabis gehe es um eine Gewissensfrage. Liebe Kolleginnen und Kollegen der SPD, beugen Sie sich nicht irgendwelchen ideologischen Koalitionsvereinbarungen, die noch gar nicht gelten. Noch sind Sie frei in Ihrer Entscheidung. Ihr Fraktionschef im Land Berlin, Raed Saleh, hat gestern richtigerweise sogar die Legalisierung von Cannabis gefordert. Lassen Sie uns hier im Bundestag mit einem kleinen Schritt starten, mit der Entkriminalisierung und der Schaffung von legalen Zugangsmöglichkeiten. Das wäre doch ein Anfang.Danke schön.




	114. Lars Castellucci (SPD) Frau Präsidentin! Meine sehr verehrten Damen und Herren! Liebe Kolleginnen und Kollegen! Der Antrag der AfD enthält tatsächlich einen vernünftigen Satz. Das ist mehr, als ich erwartet habe.Es ist gleich der erste:Der Deutsche Bundestag begrüßt die Freilassung von Deniz Yücel aus politischer Willkürhaft.In der Tat: Das begrüßen wir hier alle.Das ist nicht vom Himmel gefallen, und deswegen auch von meiner Seite und von der Seite meiner Fraktion herzlichen Dank allen, die sich dafür eingesetzt und die Öffentlichkeit erzeugt haben, zuvorderst unserem geschäftsführenden Bundesaußenminister Sigmar Gabriel und der Bundeskanzlerin. Allen, die mitgeholfen haben, einen herzlichen Dank für dieses Engagement! Schön, dass es von Erfolg gekrönt war.Wie man an dieser Stelle sieht, schützt der deutsche Staat seine Bürgerinnen und Bürger im Inneren und sogar dann, wenn sie irgendwo auf der Welt in Gefängnissen landen. Ich finde, das ist eigentlich ein Zeitpunkt, an dem wir stolz auf unser Land sein und diesem Stolz Ausdruck verleihen könnten, was Sie ja so gerne tun. Das wäre auch für die Kolleginnen und Kollegen von der AfD einmal eine Gelegenheit gewesen.Den ersten Satz, den Sie hier geschrieben haben, wollten Sie eigentlich weiterschreiben. Hier sollte stehen: „Der Deutsche Bundestag begrüßt die Freilassung von Deniz Yücel aus politischer Willkürhaft, in die er von unserem Bruder im Geiste, Tayyip Erdogan, hineingeworfen wurde“;denn der Erdogan hat es nicht so mit der Meinungsfreiheit und der Pressefreiheit, und Sie haben es auch nicht so mit der Pressefreiheit und der Meinungsfreiheit.Er sitzt in der Türkei; das ist schade für die dort. Sie sitzen aber hier im deutschen Parlament, und ich fordere Sie auf, dass Sie Politik auf der Grundlage unseres Grundgesetzes machenund diesen Antrag zurückziehen. Sie können uns hier nicht auffordern, irgendwelche Äußerungen zu missbilligen, die journalistisch fallen.Nun zur Vorzugsbehandlung: Dort sei einer bevorzugt worden, während andere, vergleichbare Leute das gleiche Schicksal erlitten haben und weiterhin in den Kerkern sitzen, weiterhin unschuldig eingesperrt sind. – Da habe ich gedacht: Oha, jetzt wird uns die AfD einmal aufschreiben, was man für diese ganzen anderen Leute machen könnte, die unschuldig irgendwo eingekerkert sind und in den Gefängnissen sitzen.Und dann, Pustekuchen, kommt natürlich überhaupt nichts, sondern Sie verwenden diese vergleichbaren Schicksale nur, um auf den einen anderen zu zeigenund uns dann vorzuwerfen: Ausgerechnet dem helft ihr!Wissen Sie, an was mich das erinnert? Sie sind doch das christliche Abendland. Jetzt gebe ich Ihnen einmal christliches Abendland.Kennen Sie die Geschichte, in der das teure Salböl vergossen wurde? Dann kam Judas, der später zum Verräter wurde, und sagte: Nein, dieses teure Salböl wollen wir nicht ausgießen. Das wollen wir verkaufen, damit wir das Geld daraus den Armen geben können. – Dann steht als nächster Satz geschrieben: Aber Judas ging es gar nicht um die Armen; denn er war ein Dieb.Das finde ich mit dieser Situation sehr vergleichbar.Sie reden auch manchmal von dem Rentner, der ein Leben lang gearbeitet hat und der nun mit einer kleinen Rente dasteht.Aber ich habe von Ihnen nicht einen einzigen Vorschlag zur Verbesserung von Renten in diesem Parlament gehört, sondern immer wird nur in der ausländerfeindlichen Suppe gerührt. Das ist verwerflich.Herr Castellucci, gestatten Sie eine Frage oder Bemerkung von Herrn Nolte?Ich werde mit Ihnen streiten und diskutieren, wenn wir einmal über Sachfragen reden, aber nicht so lange, wie Sie uns solch peinliche Anträge vorlegen, über die wir hier unsere Zeit verlieren. Vergeudete Lebenszeit!Herr Curio – wo ist er denn? –, Sie sind doch ein studierter Mann.Sie können mir doch nicht erzählen, dass Sie nicht in der Lage sind, zwischen einer Satire, aus der Sie hier ständig zitieren,und einem normalen Text zu unterscheiden. Das können Sie mir doch nicht erzählen.Da machen Sie uns doch etwas vor. Wissen Sie was? Ich helfe Ihnen dabei. Wir lernen hier ein bisschen Satire. Wissen Sie, was Deniz Yücel zu Ihrem Antrag sagen würde? Er würde sagen – Achtung Satire! –: AfD? Bester Antrag, wo gibt, von ganze Welt!




	115. Amira Mohamed Ali (BSW) Sehr geehrter Herr Präsident! Kolleginnen und Kollegen! Liebe Gäste! Wir sprechen heute über die Forderung, den Gebrauch von Pestiziden zu reduzieren. Wir sprechen über eine tickende Zeitbombe, über die Vernichtung der Natur, über Gesundheitsgefährdung. Und wir sprechen darüber, dass die Regierungsparteien in den letzten Jahren immer wieder vor der Agrarindustrie eingeknickt sind und nichts gegen die Vergiftung unserer Böden unternommen haben.Wir als Linke sagen: Damit muss endlich Schluss sein. Wir brauchen die Agrarwende – und zwar nicht irgendwann. Wir brauchen sie jetzt!Die Linke hält es für dringend geboten, den Einsatz von Pestiziden deutlich zu reduzieren, insbesondere den Gebrauch des Pflanzengiftes Glyphosat und der Neonicotinoide, der sogenannten Bienenkiller; denn Pestizide unterscheiden nicht zwischen Nützlingen und Schädlingen.Seit 1989 sind circa 80 Prozent unserer Insekten gestorben. An dieser Stelle schreien die Lobbyisten der Agrarindustrie in der Regel entsetzt auf und sagen: Das ist ja gar nicht nachgewiesen. Es gibt ganz viele Ursachen. Das muss man erst einmal in Ruhe untersuchen, bevor man handelt. – Ich möchte es einmal in aller Deutlichkeit sagen: Das sind Nebelkerzen, mit denen die Öffentlichkeit für dumm verkauft werden soll. Es gibt keine seriöse Studie, die den Zusammenhang zwischen dem Pestizidgebrauch und dem Insektensterben abstreitet.Natürlich sind es immer mehrere Faktoren, die ein System irgendwann zum Kippen bringen. Aber das ist doch kein Grund, die Ursachen, auf die wir unmittelbar Einfluss nehmen können, nicht abzuschalten.Kolleginnen und Kollegen, es geht übrigens nicht nur um den unmessbaren Schaden am Ökosystem – also für die Welt, in der wir leben – durch den Verlust unserer Bienen, unserer Hummeln, der Schmetterlinge, der Nachtfalter und all der anderen Insekten; mit dem Verlust der Bienen und aller bestäubenden Insekten verlieren wir auch unsere Unabhängigkeit.Ich bin Rechtsanwältin, und es gibt da einen Spruch, der lautet: Ein Anwalt arbeitet so manches Mal vergebens, aber niemals umsonst. – Bei den Bienen ist es genau andersrum; denn die Bienen arbeiten immer nützlich, und sie machen es vor allen Dingen umsonst.Es ist kostenlos für alle. Wenn die Bienen weiter sterben, dann müssen wir diese Dienstleistung teuer einkaufen.In China werden die Felder bereits manuell bestäubt. Heerschaaren von Arbeitern schwärmen aus und bestäuben die Pflanzen per Hand. In den USA gibt es sogenannte Wanderimkereien. Das klingt so niedlich, aber ist ein Millionengeschäft. Riesige Trucks voller Bienenstöcke fahren Tausende von Kilometern durch das Land, von Feld zu Feld, gegen Cash – weil es nicht anders geht.Meine Damen und Herren von Union und SPD, von der FDP und auch von der AfD: Wollen Sie das? Ich finde, das ist ein furchtbares Szenario.Die Linke will nicht, dass es in Deutschland so kommt. Wir fordern Sie auf: Handeln Sie!Ein weiterer Faktor für das Insektensterben ist übrigens die steigende Zahl der Monokulturen. Monokulturen zerstören Lebensräume für die Insekten und damit auch die natürlichen Fressfeinde. Es treten vermehrt Schädlinge auf. Die vielen Pestizide werden also auch wegen der Monokulturen gebraucht. Es ist ein kranker Kreislauf. Mit diesem kranken Kreislauf fahren Pestizidgroßhändler wie Monsanto ungestört ihre Milliardengewinne ein.Es gibt aktuell sechs Konzerne, die den Markt kontrollieren – den von Pestiziden und vom Saatgut. Die Unternehmen verkaufen beides: erst die Pestizide und dann das passende Saatgut, das diese Pestizide überlebt. Man muss Jahr für Jahr neues Saatgut kaufen. So werden die Bauern in die totale Abhängigkeit getrieben. Es ist ein Milliardengeschäft auf Kosten der Menschen, auf Kosten der Natur, auf Kosten unseres gesamten Ökosystems. Wir müssen uns aus diesem Würgegriff der Giftmischer endlich befreien.Die Linke fordert die Regierung auf, nicht länger vor den Konzerninteressen in die Knie zu gehen. Wir brauchen endlich eine Politik für die Bauern, für den Erhalt unserer natürlichen Lebensgrundlagen und für die Gesundheit der Menschen. Wir brauchen die Agrarwende. Die Pestizide einzudämmen, wäre ein Schritt in die richtige Richtung.Vielen Dank.




	116. Jens Beeck (FDP) Sehr geehrter Herr Präsident! Sehr geehrte Kolleginnen und Kollegen! Meine sehr verehrten Damen und Herren! Die Meinungen über die sachgrundlose Befristung – das ist offenbar geworden – gehen weit auseinander.Tatsächlich bietet eine Reihe von Befunden Anlass dazu, das Instrument der befristeten Arbeitsverhältnisse zu hinterfragen.Im Grundsatz gibt es befristete und unbefristete Arbeitsverhältnisse, und daran ist zunächst auch gar nichts Schlechtes zu erkennen. Denn für diejenigen, die erstmalig in den Arbeitsmarkt eintreten, oder diejenigen, die nach Umorientierung am Arbeitsmarkt erstmalig wieder Fuß fassen wollen, ist zunächst gar nicht entscheidend, ob eine Befristung vorliegt oder nicht, sondern das Arbeitsverhältnis als solches.Dessen ungeachtet gibt es – das zeigt die gesellschaftliche Diskussion – an vielen Stellen Fehlentwicklungen, und diesen Fehlentwicklungen müssen wir natürlich entgegentreten.Dazu gehört aber, zunächst einmal die Sachverhalte exakt zu benennen: Meinen wir einmalige Befristung, oder meinen wir Kettenverträge? Beleuchten wir den privaten oder den öffentlichen Sektor?Im öffentlichen Bereich greift eine besonders hohe Anzahl an befristeten Beschäftigungen Raum. Das haben wir gerade durch aktuelle Anfragen wieder erfahren. In Wissenschaft, Forschung und Lehre erhalten immer mehr auch hochqualifizierte Mitarbeiter nur noch befristete Verträge. Von 109 Arbeitnehmern in der öffentlichen Verwaltung erhalten 57 nur einen Zeitvertrag, im Erziehungs- und Sozialwesen sind es zwei Drittel und im Bereich Erziehung und Unterricht sogar 72 Prozent. Das sind beklagenswerte Zustände. Das sehen auch die Freien Demokraten so.Aber mit alldem, meine sehr geehrten Kolleginnen und Kollegen von der Linken, hat Ihr heutiger Antrag – auch wenn Sie ihn zum siebten Mal einbringen – überhaupt nichts zu tun.Denn die gerade beschriebenen Phänomene sind in der großen Zahl Beschäftigungsverhältnisse, die eben nicht sachgrundlos im Sinne des Teilzeit- und Befristungsgesetzes sind, sondern die in aller Regel den acht Ziffern des § 14 Absatz 1 des Teilzeit- und Befristungsgesetzes unterfallen und insofern von Ihren beantragten Streichungen nicht betroffen sind.Etwa im Bereich der Universitäten oder der Schulen würde Ihr Antrag zur Streichung der Absätze 2, 2a und 3 des § 14 Teilzeit- und Befristungsgesetz also überhaupt niemandem helfen.All das wäre zu verschmerzen, aber es ist doch so: Gerade für die schwächeren Teilnehmer am Arbeitsmarkt verursacht Ihr Antrag gravierende Verschlechterungen.Unsoziale Wege, die wir nicht mitgehen können, ergeben sich schon daraus, dass Sie sachgrundlose Befristungen, die durchaus ein Vorteil sein können, vollständig abschaffen wollen. Bei Start-ups, die sich im Wettbewerb noch erproben, machen sie Sinn. Für ältere Arbeitnehmer über 52, die aufgrund einer sachgrundlosen Befristung überhaupt erst wieder die Chance haben, auf dem Arbeitsmarkt Fuß zu fassen, wollen Sie die Regelung ebenfalls streichen. Das Gleiche gilt für langzeitarbeitslose Menschen, für die die sachgrundlose Befristung ebenfalls eine wichtige Brücke darstellt, um in den Arbeitsmarkt zu kommen.Das zeigen im Übrigen die aktuellen Zahlen des IAB. Aus den Zahlen geht auch hervor, dass aus vielen befristeten Beschäftigungen, die zunächst eingegangen werden, dauerhafte Beschäftigungen werden. Das In­strument ist also nicht nur vorhanden, es ist auch wirksam. Ihr Antrag ist in den Bereichen, in denen es unbestritten Fehlentwicklungen gibt, also wirkungslos. Zudem würden Perspektiven gerade für die Schwächeren im Arbeitsmarkt, für die Sie in Anspruch nehmen zu sprechen, vernichtet werden.Liebe Kolleginnen und Kollegen, es mangelt nicht an gesetzlichen Regelungen. Vielmehr mangelt es in der öffentlichen Verwaltung am Wollen. Die meisten Befristungen könnten in den Bundesministerien, in den Ländern und in den Kommunen problemlos aufgehoben werden.Wir Freien Demokraten stehen an der Seite der Menschen, für die die flexiblen Instrumente am Arbeitsmarkt Grundvoraussetzung sind für Teilhabe und für die Verwirklichung der eigenen Ziele, und an der Seite der Betriebe, die in Zeiten von Globalisierung und Digitalisierung flexibel reagieren können müssen, ausdrücklich auch durch befristete Projekte, die nun einmal am Anfang vieler Gründungen stehen.Wir stehen nach wie vor zur Möglichkeit der sachgrundlosen Befristung. Der vorliegende Antrag hilft niemandem, aber er schadet einigen. Er ist wachstums- und beschäftigungsfeindlich. Er ist unsozial, weil er Chancen vernichtet,ausgerechnet von jungen Gründern und älteren Erwerbslosen, deren Fähigkeiten und Erfahrung wir in Zeiten des Fachkräftemangels dringend brauchen. Der vorliegende Antrag ist insofern korrekt, als dass er ein Problem anspricht, das tatsächlich besteht; aber er bietet keine Lösung.Vielmehr ist der Antrag – erlauben Sie mir diese Bemerkung – erfrischend frei von Sachkenntnis oder handwerklich schlecht gemacht. Wir sollten ihn deswegen ablehnen.Vielen Dank, meine Damen und Herren.




	117. Roland Hartwig (AfD) Guten Morgen, Herr Präsident. – Meine Damen und Herren! Wir reden aktuell über einen der größten, wenn nicht den größten Krisenherd auf der Welt. Es ist völlig klar, dass diese Region damit ganz besondere Anforderungen an die deutsche Außenpolitik stellt. Wir müssen aber feststellen, gerade mit Blick auf die aktuellen Entwicklungen, dass die deutsche Außenpolitik diesen Anforderungen ganz offensichtlich nicht gewachsen ist.Die entscheidende Frage, die wir als AfD immer wieder stellen werden, lautet: Welchen deutschen Interessen dient die aktuelle Politik der Bundesregierung in dieser Region? Nehmen wir als erstes Beispiel Syrien. Assad ist seit 2000 Präsident von Syrien. Die Aufstände gegen das Assad-Regime begannen im Jahr 2011 im Zuge des Arabischen Frühlings. Die Bundesregierung stellte sich sehr früh, bereits 2012, gegen Assad. Die Frage ist doch: Warum? Gab es doch wirtschaftliche Gründe? War zum Beispiel der von Assad abgelehnte Bau einer Gaspipeline von Katar durch Syrien nach Europa der Grund? Oder gab es politische Gründe, etwa die indirekte Unterstützung Israels bei der Ausschaltung eines politischen Gegners? Offiziell – das ist der Standpunkt der Bundesregierung – geht es um die Unterstützung von Demokratie und Menschenrechten in Syrien. Aber gab und gibt es nicht zwei abschreckende Beispiele, die zeigen, dass das in dieser Region so nicht geht, dass man missliebige Regime nicht einfach beseitigen und erwarten kann, dass anschließend eine Demokratie entsteht und die Menschenrechte gewahrt werden?Nehmen wir den Irak als Beispiel. Das aus unserer Sicht völlig völkerrechtswidrige Angreifen des Iraks im Jahr 2003 durch die USA und Großbritannien basierte auf der gigantischen Lüge, Hussein sei im Besitz von Massenvernichtungswaffen.In der Folge mussten mehrere Hunderttausende Menschen ihr Leben lassen. Das Land versank in Gewalt und Anarchie.Nehmen wir Libyen: Gaddafi wurde 2011 gestürzt. Seitdem ist das Land weitgehend ohne staatliche Autorität. Derzeit durchqueren Zigtausende von Migranten Libyen auf dem Weg nach Europa. Sie sind lokalen Milizen und kriminellen Banden ausgesetzt.Die Frage ist: Warum hatte die Bundesregierung aus diesen Beispielen nichts gelernt? Warum hat sie sich mit der Forderung „Assad muss weg“ auf eine militärische Lösung in Syrien versteift? Das war ein Fehler mit ganz fatalen, mit dramatischen Auswirkungen: Nach sieben Jahren Bürgerkrieg sind mehr als 400 000 Tote zu beklagen, Millionen von Menschen sind auf der Flucht, das Land liegt in Trümmern. War das im deutschen Interesse? Ganz sicher nicht.Im Gegenteil: Das Ergebnis dieser verfehlten Politik, unterstützt von der Bundesregierung, hat auch unser Land dramatisch verändert. Hunderttausende Syrer sind illegal nach Deutschland gelassen worden, Menschen, die vorrangig in die von uns aufgebauten und finanzierten Sozialsysteme einwandern, Moslems, die, wenn sie hierbleiben, unsere Gesellschaft dramatisch verändern werden,weil sie unsere Werte nicht nur ablehnen, sondern zum Teil aktiv bekämpfen.Doch zurück zu Syrien. Nach dem Eingreifen der Russen wird Assad wohl bleiben. Der Bürgerkrieg, so dramatisch er bisweilen noch verläuft, scheint dem Ende zuzugehen, und es ist an der Zeit, über den Wiederaufbau Syriens nachzudenken. Es macht doch keinen Sinn, Assad aus dem Friedensprozess auszuklammern. Erfolgreiche Außenpolitik muss doch immer auch Realpolitik sein. Deshalb fordern wir den Bundestag auf, die Bundesregierung dazu zu veranlassen, in Verhandlungen mit Syrien über die Rückführung von Syrern nach Syrien einzutreten.Weil das hier bewusst missverstanden worden ist, wiederhole ich: Es geht um die freiwillige Rückkehr in sichere Räume.Dann kann es auch die in den letzten Wochen intensiv diskutierte Familienzusammenführung geben, die wir als AfD natürlich befürworten, aber bitte im Heimatland und nicht in Deutschland.Das wäre im deutschen Interesse.Blicken wir auf die Türkei. Das Vorgehen der Türkei gegen die kurdischen Milizen auf syrischem Hoheitsgebiet ist völkerrechtswidrig. Die angebliche Selbstverteidigung wird doch obsolet, wenn die türkische Regierung ankündigt, Afrin belagern zu wollen. Die deutsche Reaktion hierauf? Fehlanzeige, obwohl das doch zu erwarten gewesen wäre.Erinnern wir uns an die Krim. Deutschland ist Wortführer bei den Sanktionen gegen Russland wegen völkerrechtswidrigen Vorgehens, obwohl das gar nicht im deutschen Interesse liegt. Der Ost-Ausschuss der Deutschen Wirtschaft schätzt, dass ungefähr 150 000 Arbeitsplätze durch die Russland-Sanktionen verloren gegangen sind.Wir müssen ein Ende der Entspannungspolitik in Europa feststellen; das hat die Bundesregierung in Kauf genommen. Warum schweigt sie denn zu den Vorkommnissen in der Türkei? Ist das Flüchtlingsabkommen, das man mit dem Präsidenten Erdogan geschlossen hat, und die damit verbundene Angst, dass die Türkei wieder Hunderttausende Flüchtlinge über die Balkanroute auf den Weg nach Europa schicken würde, die wir an unseren Grenzen nicht aufhalten können, möglicherweise der Grund? Wir haben damals schon gesagt: Dieses Abkommen ist ein fundamentaler Fehler; denn es begründet eine Abhängigkeit von der Türkei. Auch hier haben wir offensichtlich recht behalten.Daher unser Appell an die Bundesregierung: Lassen Sie sich nicht unter Druck setzen! Wenn der türkische Präsident tatsächlich wieder die Balkanroute für Flüchtlinge aktivieren sollte, dann tun Sie das, was längst überfällig ist: Stellen Sie die Kontrollen über die deutschen Grenzen wieder her!Diese beiden Beispiele zeigen, dass das Fazit der bisherigen deutschen Außenpolitik im Nahen und Mittleren Osten mehr als ernüchternd ist. Jahre nach dem einst gefeierten Arabischen Frühling ist die Region instabil wie nie zuvor. Die Unterstützung von Revolutionsbewegungen, die nie demokratisch gesinnt waren, hat unzählige Tote gefordert. Die Sicherheitslage in Europa hat sich dadurch eklatant verschlechtert. Die Bundesregierung ist offensichtlich eher bereit, Konflikte nach Deutschland zu importieren, als diese Themen in der Region anzugehen und zu lösen. So kann das nicht weitergehen. Wir brauchen auch hier einen Neuanfang. Wir brauchen eine Außenpolitik, die wieder deutsche Interessen in den Mittelpunkt stellt und diese Interessen glaubwürdig und nachhaltig verfolgt.Jedes andere Land der Welt verfolgt seine eigenen Ziele. Wenn wir das nicht mehr tun, werden zunächst deutsche Interessen bedeutungslos und eines Tages auch wir selbst. Das werden wir als AfD nicht zulassen. Dem werden wir uns vehement entgegenstellen.Vielen Dank.




	118. Oliver Luksic (FDP) Vielen Dank – Sehr geehrter Herr Präsident! Liebe Kolleginnen und Kollegen! Wir haben eine Aktuelle Stunde beantragt, zum einen, weil es ein Vertragsverletzungsverfahren der EU-Kommission gegen die Bundesrepublik Deutschland gibt, und zum anderen, weil sich das Bundesverwaltungsgericht heute mit dem Thema Luftreinheit befasst hat und wir der Meinung sind, dass das, was die Bundesregierung tut, absolut nicht ausreicht. Es gibt massive Fehler und Versäumnisse der Industrie. Wir meinen, die Bundesregierung hat viel zu lange zugeschaut. Jetzt drohen Fahrverbote. Das ist heute das Thema beim Bundesverwaltungsgericht gewesen.Bei den Dieselfahrzeugen ist bereits eine Wertminderung zu verzeichnen. Wir brauchen dringend die Nachrüstung. Wir sind der festen Überzeugung: Die Bundesregierung darf hier nicht länger tatenlos zuschauen; sie muss dringend handeln.Dass zu wenig passiert, haben wir auch an dem Brief der zuständigen Fachminister an die EU-Kommission gesehen. Der Hauptvorschlag war der kostenlose ÖPNV. Er ist erstens nicht kostenlos, und vor allem bietet er keine sofortige Lösung für das Problem. Deswegen frage ich die Bundesregierung: Wieso wird eine so weitgehende Maßnahme in Brüssel vorgeschlagen, ohne dass sie mit dem Städtetag, den zuständigen Landesministerien und den betroffenen Kommunen abgestimmt wird? Die wussten nicht einmal, was los ist. Ein solcher Vorschlag ist unrealistisch. Alleine Hamburg würde das jedes Jahr so viel kosten wie der Bau der Elbphilharmonie. Das zeigt, dass die Regierung keinen Kompass hat. Gerade von unionsgeführten Ministerien hätte ich mehr erwartet als eine solche Verzweiflungstat.Wenn man schon aus dem Parteiprogramm der Piraten abschreiben muss, dann hat man eindeutig den falschen Weg eingeschlagen.Jetzt drohen Fahrverbote, auch wenn die Entscheidung vor dem Bundesverwaltungsgericht darüber heute noch einmal vertagt worden ist. Ein Fahrverbot ist unserer Meinung nach ein schwerer Eingriff in die Eigentumsrechte und bedeutet eine Wertminderung der Fahrzeuge. Wir sind der Überzeugung: Mobilität ist ein Grundrecht, das wir schützen und bewahren müssen.Liebe Kolleginnen und Kollegen, die Menschen in Deutschland haben in gutem Treu und Glauben ein Fahrzeug gekauft. Jetzt müssen wir dafür sorgen, dass das Vertrauen in die Dieseltechnologie wieder gestärkt wird. Wir brauchen den Diesel, weil er hinsichtlich Verbrauch und CO 2 -Ausstoß effizienter ist. Die Maßnahmen, die die Bundesregierung auf dem Dieselgipfel beschlossen hat, reichen aber hinten und vorne nicht aus. Ich möchte es an einem Beispiel verdeutlichen. Die Bundesregierung will die Förderung von Elektrobussen. Dummerweise gibt es aber gar keine Elektrobusse auf dem Markt; deswegen kann das nicht funktionieren.– Die sind für die lokalen Unternehmen nicht verfügbar.Wer ein Fahrzeug bestellt – das ist so; fragen Sie einmal nach in der Praxis –, der bekommt keins. Deswegen hilft diese Maßnahme genauso wenig wie das Software­update. Das Softwareupdate reicht nicht aus. Deswegen muss die Bundesregierung die Rahmenbedingungen beim Thema Hardwarenachrüstung, was Zulassung oder Haftung angeht, verbessern. Beispielsweise muss mit AdBlue-Technik nachgerüstet werden. Der ADAC-Test hat es gezeigt: 50 Prozent NO x -Reduktion können erzielt werden. Da müssen wir ran. Subventionen sind nicht der richtige Ansatz, Verbote schon gar nicht. Wir brauchen Innovationen und Nachrüstung. Das darf nicht der Autofahrer bezahlen, das darf auch nicht der Steuerzahler bezahlen, sondern der Verursacher des Problems muss dafür sorgen, dass das Vertrauen in den Diesel wiederhergestellt wird.Die einzige Alternative, die es dazu gibt, wäre die blaue Plakette. Diese lehnen wir Freie Demokraten ab, weil sie den Lebensnerv der Städte treffen würde. 60 Städte wären eventuell davon betroffen. Wir sind der Überzeugung, das würde zu chaotischen Zuständen führen. Wir hätten einen bundesweiten Fleckenteppich. Es wäre auch nicht kontrollierbar; das sieht man bei der grünen Plakette. Wir bräuchten zahlreiche Ausnahmen; insofern bringt es am Ende nichts. Denken Sie an den Handwerker. Wie soll er die Waschschüssel zum Kunden bringen? Denken Sie an den Rentner, der im Monat vielleicht nur wenige Kilometer fährt, aber er muss trotzdem noch seine Einkäufe erledigen können. Gerade diese Personengruppe würden wir mit einer solchen Maßnahme schwer treffen. Wir sind der Überzeugung, dass nicht ausgerechnet diejenigen, die sich keinen neuen Diesel leisten können, davon betroffen sein dürfen. Auch deswegen wollen wir eine solche Maßnahme unbedingt verhindern.Die Bundesregierung muss endlich handeln. Alles, was bisher auf den Weg gebracht wurde, ist zu wenig. Die Industrie ist am Zug und muss handeln. Sie muss das Vertrauen wiederherstellen. Wir brauchen auch mehr intelligente Verkehrssteuerung. Auch in diesem Bereich kann man kurzfristig sehr viel erzielen. Wir brauchen neue Mobilitätsformen – Stichwort: selbstfahrende Großraumtaxen oder Minibusse. Wir haben eine Kleine Anfrage zum PBefG gestellt und keine lektorierte Antwort von der Regierung bekommen. Gleichzeitig werden aber Briefe nach Brüssel geschrieben. Das ist ein Missverhältnis. Deswegen fordern wir die Bundesregierung auf: Hören Sie auf mit Verboten und mit Subventionen, wir brauchen Innovationen und Technologieoffenheit. Der Autofahrer darf nicht bestraft werden, weil die Industrie und die Bundesregierung nicht handeln. Tun Sie endlich etwas. Das ist dringend notwendig, sonst drohen Fahrverbote hier in Deutschland.




	119. Stephan Thomae (FDP) Herr Präsident! Verehrte Kolleginnen! Verehrte Kollegen! Meine Damen und Herren! Die Entscheidung des Amtsgerichtes Gießen vom 24. November letzten Jahres in der Angelegenheit der Ärztin Kristina Hänel hat einen Paragrafen ins Licht der Öffentlichkeit gerückt, der in der Vergangenheit ein eher unscheinbares Dasein gefristet hat: § 219a Strafgesetzbuch.Auch in der Vergangenheit gab es Fälle, in denen Ärzte gegen Strafbefehle keinen Einspruch eingelegt haben, gab es Fälle, in denen Ärzte eine Verfahrensbeendigung durch Einstellung unter Auflagen akzeptiert haben. Aber hier, in diesem Fall, kam es zu einer Anklageerhebung, zur Eröffnung eines Hauptverfahrens und zu einer Verurteilung. Dieses Urteil hat eine breite Diskussion über die Zeitgemäßheit des § 219a Strafgesetzbuch ausgelöst.Auf der einen Seite wird vorgetragen, dass es zeitgemäß ist, dass sich Menschen einfach breiter informieren wollen, dass sich junge Frauen, die ungewollt schwanger geworden sind, über einen Schwangerschaftsabbruch Gedanken machen, sich dazu vielleicht schon entschlossen haben. Das Erste, was sie tun, ist heutzutage etwas ganz Normales: Sie informieren sich im Internet, suchen die Seiten von Ärzten und Einrichtungen auf und lesen dort nach. Warum soll eine solche sachliche Information nicht zulässig sein?Auf der anderen Seite soll, wie wir gehört haben, § 219a Strafgesetzbuch erhalten bleiben, weil ein Schwangerschaftsabbruch keine ärztliche Leistung wie eine andere ist. Es ist keine Heilbehandlung, sondern er endet mit dem krassesten Eingriff, der denkbar ist: mit dem Ende werdenden Lebens. Deswegen muss man sich vielleicht auch diesen Gedanken vor Augen führen: Zu dem Schutzkonzept gehört, dass die Beratung über einen Schwangerschaftsabbruch reglementiert sein soll, dass es eben eine Sache anerkannter Beratungsstellen bleiben soll, darüber zu informieren, welche Hilfsangebote und Behandlungen es gibt.Weil es so eine große Spannbreite von ernstzunehmenden Ansichten gibt, hat die FDP-Fraktion am Montag dieser Woche einen Fachkongress mit vier Experten durchgeführt, die die ganze Bandbreite abgedeckt haben. An diesem Kongress haben 90 Abgeordnete aus allen Fraktionen des Deutschen Bundestages teilgenommen. Die Diskussion dort war sehr sachlich, sehr konzentriert, sehr differenziert. Ich hatte das Gefühl, dass dieses Haus dieses Thema sehr ernst nimmt.Insofern ist es angemessen, dass wir uns mit diesem schwierigen, ernsten sittlichen Thema auch ernsthaft befassen, dass wir uns noch einmal vor Augen führen, dass das Schutzkonzept für werdendes Leben eine ganz komplizierte Struktur innehat, die eine komplizierte Statik mit einem Indikationsmodell umfasst, verschränkt mit einem Fristenmodell. Dazu gehört eine Beratungslösung, die vorsieht, dass die Beratung im Gesetz genau beschrieben ist und eben nicht von jedermann durchgeführt werden kann, vor allem nicht von denen, die selber Eingriffe vornehmen.Nun kann man sagen: Die heutigen Zeiten sind anders. Es ist ganz normal, dass das Informationsbedürfnis der Menschen gewachsen ist und dass Ärzte um des Vermögensvorteils willen dafür Gebühren abrechnen, dass sie Abbrüche vornehmen. Auch das ist zunächst einmal kein Unrecht.Aber man muss eben auch Respekt vor der Ansicht derer haben, die sagen: Wir wollen bei dem Schutzkonzept für werdendes Leben nicht noch mehr Kompromisse eingehen. Es ist immerhin ein, wie ich schon ausführte, sehr krasser Eingriff, und deswegen ist auch diese Ansicht ernst zu nehmen.Die FDP-Fraktion hat sich in Anbetracht dieses ernsthaften Themas darauf verständigt, dass das Ganze aus ihrer Sicht nicht, wie es Kollegin Schauws sagte, nur eine Sache der Landesärztekammern und eine Sache des Standesrechtes der Ärzte sein kann; vielmehr müssen wir uns selbst als Gesetzgeber dazu verhalten, und wir müssen überlegen, wie wir zu dieser Werteentscheidung stehen. Deswegen müssen wir uns der Mühe unterziehen, uns selbst mit dieser Thematik zu beschäftigen, und wir können dies nicht einfach auf die Ärztekammern abwälzen.Es gibt einen vermittelnden Vorschlag der FDP-Fraktion, § 219a Strafgesetzbuch zu modernisieren, das Werbeverbot im Strafrecht weiterhin verankert zu lassen und deutlich zu machen, dass es Ärzten nicht erlaubt sein soll, in grob anstößiger Weise für Schwangerschaftsabbrüche oder für strafbare Abbrüche zu werben. Wir sind der Auffassung, dass in der Breite der Diskussion auch dieser Vorschlag in den weiteren Beratungen im Ausschuss vertreten sein muss. Deswegen werben wir dafür, unseren Gesetzentwurf an den Ausschuss zu überweisen.Ich danke Ihnen.




	120. Kirsten Lühmann (SPD) Frau Präsidentin! Liebe Kollegen! Liebe Kolleginnen! Sehr verehrte Zuhörende! Herr Donth hat es schon angesprochen: Der Kollege Luksic hat zum eigentlichen Thema nichts gesagt. Darum werde ich seine Meinung hier einmal kundtun; die habe ich nämlich im Internet gefunden.Da haben Sie gesagt:Das Versprechen eines kostenlosen ÖPNV hört sich schön an, lässt sich aber in der Realität kaum umsetzen.Nachdem ich das gelesen habe, habe ich verstanden, warum diese Aktuelle Stunde auf der Tagesordnung steht. Dann habe ich aber mal geguckt, wer dieses Versprechen gegeben hat. Ich habe niemanden gefunden. Das Einzige, was ich aus der letzten Zeit zu diesem Thema gefunden habe, war eine Passage aus einem Brief, den drei Minister unserer Regierung an die EU-Kommission geschrieben haben. Da steht: Zusammen mit den Ländern und Kommunen denken wir über kostenlosen ÖPNV als Mittel zur Senkung der Anzahl der Privat-Pkw nach.Wir denken darüber nach! Also wenn man der Bundesregierung jetzt verbietet, nachzudenken, und behauptet, das wäre ein Versprechen, dann wird es etwas schwierig.Ich finde, die Bundesregierung hat genau das Richtige gemacht, nämlich das, was wir von ihr erwarten: Sie hat Vorschläge gemacht, die jetzt diskutiert werden, und das sollten wir auch einmal zulassen, liebe Kolleginnen und Kollegen.Zudem ist der Punkt, den Sie hier zum Thema der Aktuellen Stunde gemacht haben, nur einer von sieben Punkten, die in dem Brief aufgelistet waren.Da sind noch ganz andere dabei, und darum steht in dem Brief auch, man möchte gern fünf Modellregionen schaffen, in denen man die Effektivität der Maßnahmen – Mehrzahl! – prüft.Es soll also nicht nur der Frage nachgegangen werden: „Ist ein kostenloser ÖPNV sinnvoll oder nicht?“, sondern es soll auch geprüft werden: Kann dies im Zusammenspiel mit anderen Maßnahmen sinnvoll sein?Kann es da verschiedene Spielarten geben? – Mein Kollege Arno Klare wird in seiner Rede gleich darauf noch eingehen.Es ist also absolut sinnvoll, dass wir sagen: Wir bilden Modellregionen und schauen, was aus dem Instrumentenkasten der sieben verschiedenen Maßnahmen – und vielleicht noch weiteren – das Effektivste ist.Ich finde, das ist schwer in Ordnung, und das sollten wir auch so machen. Darum wird es in der nächsten Woche das erste Treffen mit den entsprechenden Bürgermeistern geben, liebe Kolleginnen und Kollegen.Leider – das müssen wir auch sagen – kann man ja das Problem der Luftverschmutzung in unseren Städten nicht mit einer Maßnahme erschlagen.Es wäre ja schön, wenn man sagen könnte: Wir machen irgendein Gesetz und haben damit ab morgen weniger Luftverschmutzung. – Das funktioniert so nicht. Es gibt verschiedene Lösungsansätze, die auch die Bundesregierung favorisiert und auch schon auf den Weg gebracht hat, zum Beispiel ein Förderprogramm für Elektrobusse. Im Gegensatz zu Ihnen bin ich heute Morgen hier in Berlin mit einem Elektrobus gefahren,ganz normal im Linienverkehr der BVG. Das ging hervorragend, das war problemlos. Es war ein neues Modell, das getestet wird. Ich empfehle Ihnen also: Nutzen Sie einmal den ÖPNV, dann werden Sie sehen, dass es da hervorragende Lösungen zur Elektromobilität gibt.Ich bin allerdings mit Ihnen der Meinung, dass der Elektrobus nicht die alleinige Lösung ist. Warum nicht? Wir müssen uns wirklich angucken, was die Probleme der einzelnen Städte sind. Ich nenne Ihnen zwei Beispiele: In München, an der Landshuter Allee, verursachen Busse nur gut 5 Prozent der Stickoxidemissionen. An dieser Messstelle sind die Stickoxidemissionen aber um 50 Prozent erhöht. Also selbst wenn ich sämtliche Busse elektrisch fahren lasse, wird das mein Problem an der Landshuter Allee in München nicht lösen. Wenn ich allerdings nach Münster schaue – ich meine die Stadt Münster, die wir alle kennen, die schon seit Jahren vernünftige Konzepte macht, wo inzwischen 30 Prozent der Menschen, die pendeln, das Fahrrad nutzen, und 10 Prozent den Bus –, sehe ich, dass es, obwohl Münster seit Jahren diese guten Konzepte hat, immer noch eine Messstelle gibt, an der die Werte im Jahresdurchschnitt überschritten werden. An dieser Messstelle sind Busse die Ursache, da sie in Münster sehr alt sind und nur alte Euro-Normen erfüllen. Wenn also Münster seine Busse durch solche mit der modernen Euro-6-Norm, durch Erdgasbusse oder Elektrobusse ersetzen würde, wäre diese Maßnahme für Münster voll ausreichend – für München aber nicht.Wir können darum nicht einfach sagen: Es gibt eine Maßnahme, und die wird alles erschlagen. – Wir müssen verschiedene Dinge ausprobieren.Das macht diese Bundesregierung, und das ist auch gut so.Das, was wir aber auf alle Fälle in diesem Instrumentenkasten brauchen – das steht auch in dem Brief –, sind Nachrüstungen: Hardwarenachrüstungen von Pkw, Bussen und auch Nutzfahrzeugen. Wir haben gestern dazu ein Fachgespräch geführt. Es gibt inzwischen vier Firmen, die das anbieten. Wir haben nur ein klitzekleines Problem: Wir brauchen dazu die Automobilindustrie; denn die notwendige Abstimmung mit der Motorsteuerung kann man nicht ohne sie hinbekommen.Die Hinweise von Chefs deutscher Automobilfirmen, die sagen, das sei doch alles Quatsch, die Leute sollten doch Neufahrzeuge kaufen, finde ich allerdings ein bisschen zynisch.Denn wir haben 15 Millionen Menschen, die Dieselfahrzeuge fahren, und von denen hat die Mehrheit einfach kein Geld, ihr Fahrzeug – einige sind erst vier oder fünf Jahre alt – durch ein neues zu ersetzen. Es geht hier um die Glaubwürdigkeit unserer Automobilindustrie. Wenn sie da etwas machen will, sollte sie endlich mit der Politik und der mittelständischen Industrie zusammenarbeiten, die diese Lösungen entwickelt hat, damit wir zu einer vernünftigen Lösung kommen.Das Fazit ist also: Auch wenn die Luftverschmutzung in Deutschland sinkt, die Krankheiten der Atemwege nehmen zu. Das heißt, wir müssen etwas tun. Wir können nicht mehr warten. Es gibt keinen Königsweg. Lassen Sie uns die Modelle testen! Lassen Sie uns mit der Industrie eine vernünftige Nachrüstung vornehmen! Dann wird die Luft in unseren Städten deutlich besser werden.Herzlichen Dank.




	121. Stephan Pilsinger (CSU) Herr Präsident! Meine Damen und Herren! Zum Thema Marihuana gibt es passenderweise viele rauschhafte Fantasien. Ich möchte heute zu den wissenschaftlichen Fakten sprechen. Ich werde darauf eingehen, dass die Behauptungen von FDP, Linken und Bündnis 90/Die Grünen teilweise nicht der Realität entsprechen.Es wird behauptet, dass durch die Abkehr von Repressionen Polizei, Staatsanwaltschaft und Justiz entlastet würden. Fakt ist aber, dass mit dieser Argumentation alle Drogen weitestgehend freigegeben werden müssten.Behauptet wird oft auch, eine kontrollierte Abgabe von Cannabis als Genussmittel könnte einen Beitrag zum Gesundheitsschutz leisten, weil dadurch die Qualität von Cannabisprodukten kontrollierbar wäre.Fakt ist aber, dass mit dieser Argumentation – unter Gesundheitsgesichtspunkten – alle Drogen kontrolliert abgegeben werden müssten.Auch wird behauptet, dass durch eine mögliche Besteuerung von Cannabisprodukten erhebliche Einnahmen erzielt würden, die der Suchtprävention und der Aufklärung zugeführt werden könnten. Fakt ist aber, dass erhebliche Steuereinnahmen aus dem Vertrieb von ­Cannabisprodukten nur dann zu erzielen sind, wenn möglichst viel Cannabis auch vertrieben wird. Das finde ich grotesk: Tabakwerbung verbieten wollen, aber Cannabiskonsum zu Steuerzwecken nutzen!Darüber hinaus wird sehr gerne behauptet, dass die kontrollierte Abgabe von Cannabis völlig risikolos sei. Fakt ist aber, dass dies schlichtweg falsch ist, da Cannabiskonsum mit erheblichen Gefahren verbunden ist. Wie die jüngst veröffentlichte CaPRis-Studie der LMU München belegt, kann Cannabiskonsum zu psychischen Störungen wie Angststörungen, Depressionen, Suizidalität, bipolaren Störungen sowie Psychosen führen.Vielfach wird auch behauptet, dass Cannabiskonsum die Arbeitsfähigkeit der Menschen nicht beeinträchtigt. Fakt ist aber, dass Cannabiskonsum zu vielseitigen kognitiven Beeinträchtigungen führt.Laut einer prospektiven Beobachtungsstudie aus dem Jahr 2016, deren Ergebnisse in der Zeitschrift „JAMA Internal Medicine“ veröffentlicht wurden, konnte sich jeder zweite Teilnehmer in einem Test aus einer Liste von 15 Worten ein Wort weniger merken, wenn er seit fünf Jahren regelmäßig Marihuana geraucht hatte.Herr Kollege, würden Sie eine Zwischenfrage zulassen?Nein, danke. Jetzt bin ich gerade im Fluss.Zudem belegt die US-amerikanische CARDIA-Studie, dass ein hoher Cannabiskonsum langfristig Spuren im verbalen Gedächtnis hinterlässt. Außerdem schnitten bei der CARDIA-Studie Kiffer in Tests zu Verarbeitungsgeschwindigkeit, Exekutivfunktionen und zum verbalen Gedächtnis umso schlechter ab, je mehr Cannabis sie konsumiert hatten. Besonders erschreckend bei dieser Studie ist, dass die Cannabiskonsumenten bei den Tests auch dann noch schlechter abschnitten, wenn sie mittlerweile auf Cannabis verzichteten. Fast alle seriösen Studien machen mehr als deutlich: Intensiver Cannabiskonsum macht dumm!Herr Kollege, lassen Sie eine Zwischenfrage zu?Nein. – Auch wird behauptet, dass Cannabis nicht gesundheitsgefährdend sei. Fakt ist, dass Cannabis das Risiko für respiratorische Symptome erhöht. Akut bewirkt Cannabiskonsum erweiterte Blutgefäße, Bluthochdruck und beschleunigten Puls. Dies kann ebenfalls der CaPRis-Studie entnommen werden.Behauptet wird zudem, dass Cannabiskonsum keine negativen gesellschaftlichen Folgen mit sich bringt.Fakt ist aber, dass die sozialen Folgen von Cannabiskonsum nicht zu vernachlässigen sind, wie ebenfalls die CaPRis-Studie belegt. Früher Beginn und häufiger Cannabiskonsum in der Jugend sind mit geringen Bildungschancen assoziiert.Auch wird behauptet, dass Cannabis nicht zu Abhängigkeiten führen würde oder diese zumindest zu vernachlässigen wären. Fakt ist aber, dass die bereits viel zitierte CaPRis-Studie aufführt, dass in Europa die Zahl der Personen, die erstmals eine Suchtbehandlung wegen cannabisassoziierten Problemen beginnen, von 43 000 im Jahr 2006 auf 76 000 im Jahr 2015 gestiegen ist. Diese Entwicklung zeigt sich auch in Deutschland. Epidemiologischen Studien zufolge wird geschätzt, dass circa 9 Prozent aller Personen, die jemals Cannabis konsumiert haben, eine cannabisbezogene Störung entwickeln.Die vorgelegten Anträge sind für mich reiner Populismus. Mir geht es nicht darum, mich falsch anzubiedern. Mir geht es um die Gesundheit der Menschen. Deshalb will ich eine Welt mit weniger als mehr Drogen.Vielen Dank.




	122. Wieland Schinnenburg (FDP) Herr Präsident! Meine Damen und Herren! Es ist Zufall, dass ich meine erste Rede nach der denkwürdigen Debatte von eben halte. Eine Fraktion macht einen Anschlag auf die Pressefreiheit, und alle anderen fünf Fraktionen, die sonst viel trennt, verteidigen sie gemeinsam. Ich bin stolz, in diesem Parlament arbeiten zu können.Meine Damen und Herren, die auf Repression basierende Cannabispolitik in diesem Lande ist gescheitert, und zwar aus mehreren Gründen: Sie ist zum einen gescheitert, weil nach Jahrzehnten der Strafverfolgung immer noch mehrere Millionen Menschen Cannabis konsumieren. Sie ist zum anderen gescheitert, weil diese Menschen zusätzlich dadurch gesundheitlich gefährdet werden, dass sie sich Cannabis auf dem Schwarzmarkt besorgen müssen und dort oft Cannabis mit Verunreinigungen bekommen. Sie ist drittens gescheitert, weil auf diese Weise Justiz und Polizei unnötig mit Arbeit belastet werden. Diese Ressourcen könnte man anderweitig besser verwenden.Meine Damen und Herren, ich habe deutlich mehr Angst vor Einbruchdiebstählen als davor, dass mein Nachbar kifft.Ich habe drei Töchter. Eine meiner größten Sorgen war immer: Was passiert, wenn eine meiner Töchter drogensüchtig wird? Ich habe mich gefragt: Was tut dann der Staat für mich oder für meine Tochter? Die Antwort war: entweder Gefängnis oder Schwarzmarkt. Kurz gesagt: Das war unbefriedigend.Die Freien Demokraten haben schon vor einigen Jahren beschlossen: Wir wollen eine andere Drogenpolitik. Wir wollen eine kontrollierte Abgabe von Cannabis. Achtung: kontrollierte Abgabe, nicht etwa Freigabe. Wir wollen nicht, dass Cannabis irgendwo im Supermarkt im Regal liegt; nein, wir wollen, dass Cannabis in Apotheken und anderen speziell lizenzierten Geschäften kontrolliert und in überschaubarer Menge an Erwachsene abgegeben wird.Was spricht dafür, das so zu machen? Der erste Grund ist, dass man auf diese Weise verhindert, dass Menschen unreines Cannabis bekommen. Man kann es kontrollieren. Der zweite Grund ist, dass man so einen viel besseren Zugang zu den Drogenabhängigen hat und ihnen viel schneller Therapien anbieten kann. Der dritte Grund ist, dass der Staat dann viel Geld einnimmt – hoffentlich –, das man für Therapien und Prävention ausgeben kann.Meine Damen und Herren, das ist das Konzept der Freien Demokraten. Das fordern wir seit drei Jahren.Wir wissen natürlich, dass es in diesem Hause einige Fraktionen gibt, die dem noch ein bisschen skeptisch gegenüberstehen. Darum machen wir heute nur den ersten Schritt und fordern zunächst einmal ein Modellprojekt. Liebe Kollegen von den Fraktionen, die das eher konservativ sehen, Sie können doch nicht ernsthaft dagegen sein, dass wir klüger werden. Lassen Sie uns doch erst einmal einem Modellprojekt zustimmen, und nach dem Modellprojekt wissen wir, wie es geht.Unsere Bitte ist also: Stimmen Sie wenigstens einem Modellprojekt zu, auch wenn Sie insgesamt noch ein bisschen skeptisch sind.Meine Damen und Herren, ich bringe es auf folgenden Punkt: Wir brauchen eine andere Drogenpolitik. Wir brauchen eine Drogenpolitik, die die Interessen der Menschen in den Mittelpunkt stellt und nicht, wie aktuell, Dogmen. Das ist die Idee der Freien Demokraten. Ich kann es auch anders ausdrücken: Wir brauchen eine neue Generation von Drogenpolitik, und dafür wollen wir Ihre Unterstützung.Vielen Dank für Ihre Aufmerksamkeit.




	123. Mario Mieruch (LKR) Sehr geehrter Herr Präsident! Sehr geehrte Damen und Herren! Liebe Gäste! Lobbyregister, – was für ein Gesetzentwurf! Ich mag es gerne praktisch, daher habe ich mir ein tagesaktuelles Beispiel herausgezogen. Sie brauchen nicht mitzuschreiben: EuroNatur, Global Nature Fund, Institut für Ökonomik und Ökosystemmanagement, Bodensee-Stiftung, LiLu, Tropenwaldstiftung Oro Verde, Stiftung Initiative Mehrweg, Deutscher Naturschutzring, Heinrich-Böll-Stiftung, The Nature Conservacy, Internationale Klimaschutzinitiative der Bundesregierung, Leibniz-­Zentrum für Agrarforschung, Stiftung „Lebendige Stadt“, Agora Energiewende, European Climate Foundation, Stiftung Mercator, ClimateWorks Foundation, Potsdam-Institut für Klimaforschung, Bioenergie Grosselfingen, solarkomplex, Bund für Umwelt und Naturschutz, Europäisches Umweltbüro, EU-LIVE, ClimateWorks Australia, Toyota Filterhersteller, Hessisches Umweltministerium, UN-Umweltprogramm, Bundesnetzagentur, Deutsche Telekom. Warum zähle ich Ihnen das auf?Wenn man sich hier die Köpfe und Verantwortlichkeiten, die Mitwirkung, die Spenden und die Geldflüsse anschaut, dann taucht ein Verein immer wieder auf: die Deutsche Umwelthilfe. Deutsche Umwelthilfe, was für ein Begriff! Wie klingt das? Das klingt nach einem riesigen Verein – 274 Mitglieder. Wahnsinn. Global Nature Fund – oben genannt – stellt man sich noch größer vor, da globales Weltretten natürlich besonders wichtig ist – sieben Mitglieder. Sie sitzen auch alle im Präsidium. Zwei von ihnen sind zufälligerweise Mitglied bei der Deutschen Umwelthilfe.Lange Rede kurzer Sinn. Was findet man, wenn man da ein bisschen nachschaut? Man findet einen schlanken Aufbau, und man findet nachgeordnete Stiftungsstrukturen, die ein absolutes Alleinstellungsmerkmal in unserem Land bilden und die es auf diese Art und Weise geschafft haben – das hat meine Anfrage im Januar ergeben –, ungefähr 5 Millionen Euro an Fördergeldern für eine ganze Reihe von Projekten abzugreifen. Was macht man unter anderem damit? Die Deutsche Umwelthilfe hat eine ganz interessante Abmahnindustrie entwickelt, indem sie jede Menge Unternehmer zur Kasse bittet oder indem sie die Kommunen verklagt. In Leipzig wurde heute zum Glück kein Urteil gefällt; sie haben sich vertagt.Ich könnte das jetzt hier noch ewig ausführen; das Ganze ist superspannend. Die Quintessenz der ganzen Geschichte: Es muss gar kein Geld fließen. Wir können auch Gesetze erlassen. Es reicht, wenn man seinem Lobbyisten heutzutage das Verbandsklagerecht einräumt; dann läuft der Laden von ganz alleine, und dann braucht man das Gesetz überhaupt nicht.Wenn sich auch die Fraktion der Grünen für Transparenz in der Lobbyarbeit starkmacht, dann können wir vor der eigenen Tür wunderbar anfangen zu kehren. Ich werde euch auf jeden Fall dabei helfen – versprochen.




	124. Klaus-Peter Schulze (CDU) Herr Präsident! Liebe Kolleginnen und Kollegen! Meine sehr verehrten Damen und Herren! Ich bin ja dafür bekannt, dass ich versuche, sachlich in die Diskussion einzusteigen. Deshalb, lieber Kollege Ebner, muss ich das, was Sie beim Diskussionseinstieg behauptet haben, als Sie sagten, dass wir in Deutschland 50 Prozent weniger Vögel haben, korrigieren. Richtig ist, dass die Gruppe der Agrarvögel deutlich zurückgegangen ist; aber andere Gruppen haben zugenommen. Das ist nicht nur auf den Pflanzenschutzmitteleinsatz zurückzuführen. Da müssen wir zu unseren südeuropäischen EU-Partnern schauen, die jedes Jahr Hunderttausende, ja Millionen Vögel abschießen, obwohl sie unter Schutz stehen. Über die Netzanlagen in Nordafrika und unsere veränderte Landnutzung möchte ich auch nicht reden. Das alles sind Faktoren, die eine Rolle spielen.Wie hat sich denn der Herbizideinsatz beispielsweise in den letzten Jahrzehnten entwickelt? Im Jahr 1950 sind pro Hektar landwirtschaftlicher Nutzfläche 1,5 Kilogramm Herbizide eingesetzt worden, im vergangenen Jahr waren es 10 Gramm je Hektar. Das ist eine positive Entwicklung, reicht aber noch nicht aus; darüber sind wir uns alle im Klaren.Zum Thema Wald: In den letzten zwei Jahren sind etwa 0,15 Prozent der Gesamtwaldfläche in Deutschland mit Pflanzenschutzmitteln behandelt worden. – Das sind Zahlen, die man in der Diskussion auf jeden Fall berücksichtigen sollte.Die Kollegin Konrad von der FDP hat das Thema der Zulassung von Pflanzenschutzmitteln angesprochen. Das ist für mich ein ganz entscheidender Punkt. Die Zahlen, die mir zur Verfügung stehen, sind andere. Ich habe gehört, die durchschnittliche Bearbeitungszeit betrage 639 Tage; der Spitzenreiter sei eine Bearbeitungszeit von 1 667 Tagen – bei einer EU-Vorgabe von 120 Tagen –, und es liege nach wie vor keine Zulassung vor.Herr Schulze, gestatten Sie eine Zwischenfrage des Kollegen Ebner?Gern.Bitte sehr.Danke, Herr Kollege, dass Sie die Frage zulassen. – Sie haben gerade von x Gramm Wirkstoff pro Hektar und davon gesprochen, dass die Anwendungsmenge zurückgegangen ist. Da möchte ich Sie zwei Dinge fragen.Erstens. Wie erklären Sie sich dann, dass die absoluten Absatzzahlen, die verkauften Wirkstoffmengen in Tonnen, sich seit 1970 verdoppelt haben und seit 1995 um 50 Prozent angestiegen sind? Wie erklären Sie, dass wir damals 20 000 Tonnen hatten, 1995  35 000 Tonnen und 2015  48 000 Tonnen? Die Ackerfläche, Herr Kollege, ist in dieser Zeit ganz bestimmt nicht gewachsen, und es kauft auch keiner, um es nur im Keller im Regal zu lagern. – Das müssen Sie erst einmal auflösen; darum möchte ich Sie bitten.Zweitens möchte ich Sie fragen, ob Sie anerkennen, dass Wirkstoffe wie beispielsweise Neonicotinoide eine mehrhundertfache Toxizität im Vergleich zum damals eingesetzten DDT oder zu Lindan haben. Wenn die neueren Wirkstoffe in der gleichen Menge ausgebracht würden wie die damals eingesetzten, wäre dies eine richtige Katastrophe. Wir können nicht damit zufrieden sein, dass die Wirkstoffmenge gleich bleibt, wenn die Toxizität mehrhundertfach höher ist.Wenn Sie jetzt die Neos mit dem DDT vergleichen, dann muss ich sagen: Die Entscheidung, DDT nicht mehr als Pflanzenschutzmittel zu verwenden, war richtig. Dass wir auch bei den Neos Handlungsbedarf haben, streitet gar keiner ab.Die Zahlen zu den Wirkstoffen, die ich genannt habe – mir sind Zahlen bekannt, dass wir in den letzten fünf Jahren etwa 32 000 Tonnen Wirkstoffe in Deutschland eingesetzt haben –, sind die Zahlen, die mir zur Verfügung stehen. Wir können uns gerne hinterher darüber austauschen, woher Sie Ihre Zahlen haben, und dann vergleiche ich das. Danke.Ich setze wieder an der Stelle ein, an der ich stehen geblieben bin, nämlich beim Thema Zulassung. Hier bekommen wir mit dem Brexit das nächste Problem. In Großbritannien wird eine ganze Reihe von Pflanzenschutzmitteln zugelassen. Wenn Großbritannien nicht mehr Mitglied der EU ist, werden die Zulassungsanträge auf andere Länder und wahrscheinlich vornehmlich auf Deutschland übertragen. Deshalb müssen wir in dieser Hinsicht unbedingt etwas tun.Ich bin für folgende konkrete Maßnahmen: erstens eine externe Bewertung der Verfahrensabläufe und deren Optimierung mit den beteiligten Behörden, zweitens eine entsprechende Aufstockung des Personals – darüber haben wir uns auch im Koalitionsvertrag geeinigt – und drittens eine höhere Harmonisierung der Antragsbearbeitung zwischen den einzelnen europäischen Ländern.Die Grünen fordern in einem Antrag, dass das Thema „neue Sorten, bessere Sorten“ vorangebracht werden soll. Damit bin ich sehr einverstanden. Wir werden uns nach der Entscheidung des Europäischen Gerichtshofes zum Thema Crispr/Cas sicherlich unterhalten müssen. Wenn wir diese neue Züchtungsform einsetzen können, dann – da bin ich mir ganz sicher – werden wir schnell effiziente Sorten züchten, um später den Einsatz von Pflanzenschutzmitteln zu reduzieren.Ich bedanke mich für die Aufmerksamkeit.Vielen Dank, Herr Kollege. – Ich schließe die Aussprache.Interfraktionell wird Überweisung der Vorlage auf Drucksache 19/835 an die in der Tagesordnung aufgeführten Ausschüsse vorgeschlagen. Sind Sie damit einverstanden? – Ich sehe keinen Widerspruch. Dann ist das so beschlossen.




	125. Sonja Steffen (SPD) Sehr geehrter Herr Präsident! Liebe Kolleginnen und Kollegen! Mit dem Lobbyregister reden wir heute über ein Thema, das der SPD-Fraktion sehr am Herzen liegt. Und direkt vorweg: Die beiden Anträge der Linken und der Grünen sind gut.Die SPD-Fraktion hat in der Vergangenheit bereits ähnliche Anträge eingebracht, und ich bin mir sicher, dass meine Fraktionskolleginnen und ‑kollegen es sehr begrüßen, wenn wir in dieser Sache eine eindeutige Regelung finden.Denn eines ist ganz klar: Transparenz stärkt das Vertrauen der Bürgerinnen und Bürger in die Politik. Viele haben nämlich das Gefühl, dass die Lobbyisten inzwischen mehr zu sagen haben als manche Abgeordnete. Und Ihnen, Frau Kloke von der FDP, sage ich: Das Bürokratiemonster-Argument, das in diesem Hohen Haus so oft bemüht wird, ist als Gegenargument, finde ich, äußerst schwach, wenn es um Transparenz und Demokratie geht.Wir alle müssen hier aktiv gegensteuern; denn schon allein die Vermutung, dass es eine Einflussnahme durch Lobbyverbände geben könnte, die in erster Linie nur an sich und ihre Bankkonten denken, schadet dem Hohen Haus, schadet dem Ruf der Politik nachhaltig. Da haben wir hier gemeinsam eine Verantwortung.Hier können wir, liebe Kolleginnen und Kollegen, einen wichtigen Schritt mit einem verpflichtenden Lobbyregister machen.Und ja, Frau Kollegin Haßelmann, wir von der SPD haben versucht, das auch im Koalitionsvertrag zu verankern. Es war bis zuletzt ein ziemlicher Kampf, aber die CSU hat deutlich gemacht, dass das mit ihr nicht zu machen ist. Dazu kann sich jeder seine Gedanken machen. Den rechten Rand einmal ausgenommen, haben alle Parteien, die hier sitzen, bereits an Koalitionsverhandlungen teilgenommen und wissen, dass Kompromisse oft schmerzlich sind, aber dazugehören.Auch ohne ein verbindliches Register haben wir in der Vergangenheit – in den letzten vier Jahren – einiges erreicht. Vieles ist schon genannt worden: Die Zahl und die Namen der Lobbyisten hat die Bundestagsverwaltung bereits vor einiger Zeit veröffentlicht. Im Frühjahr 2016 hat der Ältestenrat des Deutschen Bundestages beschlossen, dass Vertreter von Unternehmen keine Hausausweise mehr bekommen. Das bedeutet, dass die Ausgabe von Dauerausweisen inzwischen stark eingeschränkt worden ist. Und auch die Karenzzeit für Regierungsmitglieder nach ihrem Ausscheiden ist eine wichtige Neuerung.Diese Schritte reichen jedoch nicht aus. Ich sage es hier noch einmal ganz klar: Wir brauchen ein für alle verpflichtendes Lobbyregister.Dabei ist übrigens uns allen bewusst, dass Lobbyarbeit durchaus kein Teufelswerk ist. Sie ist gut, und sie ist wichtig, weil sie uns Abgeordneten eine Informationsbreite verschafft. Wir tun beispielsweise Ärzte ohne Grenzen oder dem Deutschen Tierschutzbund unrecht, wenn wir sie mit den schwarzen Schafen in eine Schublade stecken.Wir können mit einem verbindlichen Lobbyregister viel dazu beitragen, hier Klarheit zu schaffen und die Spreu vom Weizen zu trennen. Ich kenne viele Kolleginnen und Kollegen in diesem Haus – das ist ja in der Debatte zur Sprache gekommen und trifft, wie ich denke, übrigens auch für die Union zu –, die dazu bereit wären, ein verbindliches Lobbyregister einzurichten. Die Lobbyisten selbst sind es übrigens auch; auch denen ist daran gelegen.Meine Damen und Herren, wenn ich wetten müsste, welche Vorhaben trotz Nichtbehandlung im Koalitionsvertrag Chancen haben, noch in dieser Legislatur umgesetzt zu werden – vorausgesetzt, es kommt zu einer Großen Koalition –, dann sehe ich ein Lobbyregister ganz weit vorn. Denn dieses Thema ist einfach zu wichtig, um es unter den Tisch fallen zu lassen. Übrigens, Frau Haßelmann, wünsche auch ich mir, dass das Tabakwerbeverbot dazugehört.Ich würde mich also freuen, wenn wir das verpflichtende Lobbyregister im Rahmen eines gemeinsamen Antrags aller demokratischen Fraktionen im Bundestag anpacken würden. Damit könnten wir ein deutliches Zeichen setzen: für Transparenz, für Demokratie und gegen Hinterzimmerpolitik. Deshalb ist die Überweisung in den Ausschuss ganz richtig.Ich schließe die Aussprache.
\end{document}
